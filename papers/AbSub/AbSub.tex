% !TEX program = xelatex

% transformations:
%   \autoref -> ~\ref
%   _\... -> _{\...}
%   enumerate -> enumerate[format=\textup]
%   : -> \colon
%   \Ibar, \fbar, \ybar

\documentclass[twoside]{article}

\usepackage{geometry}
\geometry{
    paperwidth = 155mm,
    paperheight = 235mm,
    outer = 20mm,
    inner = 20mm,
    top = 25mm,
    bottom = 20mm
}

% fonts & unicode
\usepackage[PunctStyle=kaiming]{xeCJK}
\usepackage{amsmath}
\usepackage{unicode-math}

\setCJKmainfont{fzssk.ttf}[
    Path = ../../fonts/,
    BoldFont = fzhtk.ttf,
    ItalicFont = fzktk.ttf
]

\setCJKsansfont{SourceHanSansCN-Normal.otf}[
    Path = ../../fonts/,
    BoldFont = SourceHanSansCN-Bold.otf,
    Scale = .97
]

\setCJKmonofont{SourceHanSansCN-Normal.otf}[
    Path = ../../fonts/,
    BoldFont = SourceHanSansCN-Bold.otf,
    Scale = .9
]

\newCJKfontfamily{\KaiTi}{fzktk.ttf}[
    Path = ../../fonts/,
    BoldFont = fzhtk.ttf,
    ItalicFont = fzssk.ttf
]

\setmainfont{STIX2Text}[
    Path = ../../fonts/,
    Extension = .otf,
    UprightFont = *-Regular,
    BoldFont = *-Bold,
    ItalicFont = *-Italic,
    BoldItalicFont = *-BoldItalic
]

\setsansfont{Lato}[
    Path = ../../fonts/,
    Scale = MatchUppercase,
    Extension = .ttf,
    UprightFont = *-Regular,
    BoldFont = *-Bold,
    ItalicFont = *-Italic,
    BoldItalicFont = *-BoldItalic
]

\setmonofont{FiraMono}[
    Path = ../../fonts/,
    Scale = .9,
    Extension = .otf,
    UprightFont = *-Regular,
    BoldFont = *-Bold
]

\setmathfont{STIX2Math.otf}[
    Path = ../../fonts/,
    BoldFont = STIX2Math-Bold.otf
]

\setmathfont{latinmodern-math.otf}[
    Path = ../../fonts/,
    range = {frak, bffrak}
]

\setmathfont{latinmodern-math.otf}[
    Path = ../../fonts/,
    range = {frak -> bffrak, bffrak},
    version = bold
]

\setmathfont{LatoMath.otf}[
    Path = ../../fonts/,
    Scale = .95,
    BoldFont = LatoMath.otf,
    version = sf
]

\setmathfont{LatoMath.otf}[
    Path = ../../fonts/,
    Scale = .95,
    BoldFont = LatoMath.otf,
    range = {bb, sfup -> up, sfit -> it, bfsfup -> bfup, bfsfit -> bfit}
]


\Umathcode`/  =  "0 "0 "2215    % / -> U+2215 division slash

% patch 'text math' math alphabets in bold math
\setmathfontface\mathrm{STIX2Text-Bold.otf}[
    Path = ../../fonts/,
    version = bold
]

\setmathfontface\mathit{STIX2Text-BoldItalic.otf}[
    Path = ../../fonts/,
    version = bold
]

\setmathfontface\mathbf{STIX2Text-Bold.otf}[
    Path = ../../fonts/,
    version = bold
]

\setmathfontface\mathtt{FiraMono-Bold.otf}[
    Path = ../../fonts/,
    Scale = .9,
    version = bold
]

\setmathfontface\mathrm{Lato-Regular.ttf}[
    Path = ../../fonts/,
    Scale = MatchUppercase,
    version = sf
]

\setmathfontface\mathit{Lato-Italic.ttf}[
    Path = ../../fonts/,
    Scale = MatchUppercase,
    version = sf
]

\setmathfontface\mathbf{Lato-Bold.ttf}[
    Path = ../../fonts/,
    Scale = MatchUppercase,
    version = sf
]

\setmathfontface\mathtt{FiraMono-Regular.otf}[
    Path = ../../fonts/,
    Scale = .9,
    version = sf
]


% bold math in bold text https://tex.stackexchange.com/q/41379
\makeatletter
    \g@addto@macro\bfseries{\boldmath} 
\makeatother

% title & abstract
\def\title#1\author#2{%
    \headertitle{#1}
    \vspace*{0mm}
    \begin{center}
        {\sf\LARGE#1\par}
        \vspace{10mm}
        {\large#2}
    \end{center}
    \vspace{10mm}
}
\def\headertitle#1{
    \def\theheadertitle{#1}
}

\renewcommand{\abstractname}{ABSTRACT}
\makeatletter
    \let\endabstract@orig\endabstract
    \def\endabstract{\endabstract@orig\vspace{5mm}}
\makeatother

% section titles
\usepackage{titlesec}
\titleformat*{\section}{\Large\sffamily\mathversion{sf}}
\titleformat*{\subsection}{\large\sffamily\mathversion{sf}}

% headers and footers
\usepackage{fancyhdr}
\fancyhf{}
\fancyhead[CE]{\sf\mathversion{sf}\theheadertitle}
\fancyhead[CO]{\sf\mathversion{sf}\nouppercase{\leftmark}}
\fancyhead[LE,RO]{\textbf{\textsf{\thepage}}}
\headsep=8mm
\headheight=6mm

\AtBeginDocument{
    \pagestyle{fancy}\thispagestyle{empty}
}

% spacing
\AtBeginDocument{
    \hfuzz=2pt
    \emergencystretch 2em
    \setlength{\belowdisplayshortskip}{\belowdisplayskip}
}

% environments
\usepackage{amsthm}
\newtheorem{theorem}{Theorem}[section]
\newtheorem{lemma}[theorem]{Lemma}
\newtheorem{corollary}[theorem]{Corollary}
\newtheorem{proposition}[theorem]{Proposition}

\theoremstyle{definition}
\newtheorem{definition}[theorem]{Definition}
\newtheorem{example}[theorem]{Example}
\newtheorem{remark}[theorem]{Remark}
\theoremstyle{plain}

\def\qedsymbol{$◻$}
\def\thmqedhere{\pushQED{\qed}\qedhere\popQED}

\numberwithin{equation}{theorem}

% renew theorem: https://tex.stackexchange.com/q/103013/
\makeatletter
\def\renewtheorem#1{%
    \expandafter\let\csname#1\endcsname\relax
    \expandafter\let\csname c@#1\endcsname\relax
    \gdef\renewtheorem@envname{#1}
    \renewtheorem@secpar
}
\def\renewtheorem@secpar{\@ifnextchar[{\renewtheorem@numberedlike}{\renewtheorem@nonumberedlike}}
\def\renewtheorem@numberedlike[#1]#2{\newtheorem{\renewtheorem@envname}[#1]{#2}}
\def\renewtheorem@nonumberedlike#1{  
\def\renewtheorem@caption{#1}
\edef\renewtheorem@nowithin{\noexpand\newtheorem{\renewtheorem@envname}{\renewtheorem@caption}}
\renewtheorem@thirdpar
}
\def\renewtheorem@thirdpar{\@ifnextchar[{\renewtheorem@within}{\renewtheorem@nowithin}}
\def\renewtheorem@within[#1]{\renewtheorem@nowithin[#1]}
\makeatother

% ref & biblatex
\usepackage[colorlinks,allcolors=black,bookmarksnumbered,linktoc=all]{hyperref}

\def\thesection{\arabic{section}\texorpdfstring{}{.}} % pdf bookmark numbering
\setcounter{secnumdepth}{1} % suppress subsection numbering

\usepackage[style=alphabetic,sorting=anyvt,useprefix=true]{biblatex}
\usepackage{xpatch}
\renewcommand*{\bibfont}{\small}
\DeclareFieldFormat[article]{volume}{\mkbibbold{#1}}
\DeclareFieldFormat[book,inbook]{number}{\mkbibbold{#1}}
\DeclareFieldFormat[article]{number}{(#1)}
\DeclareFieldFormat*{year}{(#1)}
\DeclareFieldFormat{pages}{#1}
\renewbibmacro{in:}{}
\renewbibmacro*{volume+number+eid}{%
    \printfield{volume}%
    \setunit*{\addnbspace}% originally: \setunit*{\adddot}
    \printfield{number}%
    \setunit{\addcomma\space}%
    \printfield{eid}%
}
\xapptobibmacro{author/editor+others/translator+others}{%
    \setunit{\space}%
    \printfield{year}%
    \clearfield{year}%
}{}{}
\xapptobibmacro{author/translator+others}{%
    \setunit{\space}%
    \printfield{year}%
    \clearfield{year}%
}{}{}
    \renewbibmacro*{issue+date}{%
    \printfield{issue}%
    \newunit%
}
\AtBeginBibliography{
    \DeclareFieldFormat{labelalpha}{#1}
    \DeclareFieldFormat{extraalpha}{\mknumalph{#1}}
}
\AtEveryBibitem{
    \ifentrytype{online}{%
        \clearfield{year}%
    }{}
}

% tikz
\usepackage{tikz}
\usepackage{tikz-cd}
\tikzset{
    > = latex
}
\tikzcdset{
    arrow style = tikz,
    arrows = {
        /tikz/line width = .5pt
    },
    diagrams = {
        > = {Straight Barb[scale = 0.8]}
    },
    nodes = {
        inner xsep = 3pt,
        inner ysep = 3pt
    }
}

\newenvironment{itms}{\begin{itemize}\itemsep=0pt\parsep=0pt}{\end{itemize}}
\newenvironment{enum}{\begin{enumerate}[label=(\arabic*)]\itemsep=0pt\parsep=0pt}{\end{enumerate}}

\theoremstyle{definition}
\newtheorem{construction}[theorem]{Construction}
\newtheorem{notation}[theorem]{Notation}

\def\varqed{\nolinebreak\hfill$◃$}
\def\varqedhere{\eqno ◃}

% math commands
\renewcommand{\:}{\colon}
\renewcommand{\/}{{∕}}
\newcommand{\bfDelta}{{
  \mathchoice{
    \tikz{
      \draw[line width=.05em]
        (0,0)--(.6em,0)--(.37em,1.32ex)--(.23em,1.32ex)--cycle
        (.23em,1.32ex)--(.46em,0);
    }
  }{
    \tikz{
      \draw[line width=.05em]
        (0,0)--(.6em,0)--(.37em,1.32ex)--(.23em,1.32ex)--cycle
        (.23em,1.32ex)--(.46em,0);
    }
  }{
    \tikz[scale=.7]{
      \draw[line width=.035em]
        (0,0)--(.6em,0)--(.37em,1.32ex)--(.23em,1.32ex)--cycle
        (.23em,1.32ex)--(.46em,0);
    }
  }{
    \tikz[scale=.55]{
      \draw[line width=.025em]
        (0,0)--(.6em,0)--(.37em,1.32ex)--(.23em,1.32ex)--cycle
        (.23em,1.32ex)--(.46em,0);
    }
  }
}}
\newcommand{\bigmid}{\mathrel{\big|}}
\newcommand{\biggmid}{\mathrel{\bigg|}}
\newcommand{\cat}[1]{\ensuremath\textup{\textsf{#1}}} % using textsf to retain kerning & ligature
\newcommand{\comma}{,}
\newcommand{\Ho}{\operatorname{Ho}}
\newcommand{\Hom}{\operatorname{Hom}}
\newcommand{\KD}{\operatorname{\reflectbox{$\mathrm{DK}$}}}
\newcommand{\Ndg}{\mathfrak{N}_{\mathrm{dg}}}
\newcommand{\op}{^{\mathrm{op}}}
\newcommand{\sHom}{\mathscr{H}\mkern-3mu\mathit{om}}
\newcommand{\simto}{\mathrel{\rlap{\raisebox{.8ex}{$\mkern2mu\sim$}}{\to}}}
\newcommand{\square}{\mathbin{◻}}

% text commands
\newcommand{\term}[1]{\textbf{\textup{#1}}}


\begin{document}

\title{On Freyd--Mitchell Embedding Theorem}
\author{Yuchen Wu\footnote{吴雨宸,清华大学数学系数 63 班.}}
\headertitle{Freyd--Mitchell Embedding Theorem}

\begin{abstract}
The passage presents a detailed proof for the Freyd--Mitchell embedding theorem which captures the resemblance between generic abelian categories and categories of modules.
\end{abstract}


\section{Introduction}

In a standard course about homological algebra, the following folklore theorem is often assumed in order to simplify the proof for commutativity of certain diagrams in a generic abelian category:
\begin{thm}[Freyd--Mitchell embedding] \label{FE} For any small abelian category $\EA$, there exists a unital ring $R$ and a covariant fully faithful exact additive functor from $\EA$ to $\rmod$ (the category of all left $R$-modules).
\end{thm}

The passage aims to provide a detailed proof for this theorem following \cite{Fre64}. Before doing so, I will first illustrate how this result helps to deal with diagram commutativity issues in arbitrary abelian categories. For an abelian category $\EA$, suppose we want to show the commutativity of a diagram $D$ in $\EA$. First we take $\EA_D$ the abelian closure of $D$, i.e.\ the smallest abelian full sub-category of $\EA$ containing objects in $D$. This $\EA_D$ can be constructed from objects in $D$ via freely taking kernels, cokernels and binary products, thus can be realized as a directed union of a sequence of small categories, hence $\EA_D$ is also small. Now Freyd--Mitchell embedding implies there is a full faithful exact functor $\iota\colon \EA_D \rightarrow \rmod$ for some unital ring $R$, thus we can element-wise diagram chase to show commutativity of $\iota(D)$ in $\rmod$, then reflect this to commutativity of $D$ in $\EA$.

\section{Proof of the main theorem}

We first introduce several definitions needed to present the proof:

\begin{defn}
Let $\EC$ be a category. A \textbf{generator} in $\EC$ is an object $G \in \EC$ such that the functor $\EC(G, -)\colon \EC \rightarrow \set$ is faithful; dually, a \textbf{cogenerator} in $\EC$ is an object $H \in \EC$ such that the functor $\EC(-, H)\colon \EC^\op \rightarrow \set$ is faithful.
\end{defn}

\begin{expl}
In the category of abelian groups $\ab$, the rank $1$ free abelian group $\Zb$ is clearly a generator, while the divisible group $\Qb/\Zb$ is an injective cogenerator.
\end{expl}

\begin{defn}
\ 
\begin{itemize}
\item An \textbf{AB5 category} is a cocomplete abelian category in which filtered colimits of exact sequences are exact.
\item A \textbf{Grothendieck abelian category} is an AB5 category with a generator.
\end{itemize}
\end{defn}

\begin{expl}
For every unital ring $R$, $\rmod$ is a Grothendieck abelian category with generator $R$.
\end{expl}

Next we claim the validity of two technical propositions, delaying their proof to the next section:

\begin{prop}\label{preone}
Every small abelian category admits a fully faithful exact embedding into a Grothendieck abelian category.
\end{prop}

\begin{prop}\label{pretwo}
A Grothendieck abelian category is automatically complete, cocomplete, well-powered, well-copowered (meaning that every object has only a small set of subobjects and a small set of quotient objects), and equipped with an injective cogenerator.
\end{prop}

The following proposition is essentially the ``core'' in the proof of Freyd--Mitchell embedding:

\begin{prop}\label{main}
Let $\bea$ be a complete abelian category with injective cogenerator $\Ibar$, and $\EA$ one of its small abelian full subcategories. Then there exists a unital ring $R$ and a contravariant faithful exact additive  functor $$K\colon \bea \rightarrow \rmod$$ which is also full on $\EA$ (here $\rmod$ denotes the category of left $R$-modules).
\end{prop}
\pf
Since $\Ibar$ is a cogenerator, for each object $A \in \EA$, the morphism $$A \rightarrow \prod_{\bea(A, \Ibar)} \Ibar$$ defined component-wise for all $f \in \bea(A, \Ibar)$ by
$$A \rightarrow \prod_{\bea(A, \Ibar)} \Ibar \xrightarrow{\pi_f} \Ibar \quad = \quad A \xrightarrow{f} \Ibar$$
is a monomorphism. Therefore, setting $$I = \prod_{\bigcup_{A \in \EA} \bea(A, \Ibar)} \Ibar \ ,$$ we can easily see $I$ is also an injective cogenerator of $\bea$, such that for all $A \in \EA$ there is a monomorphism in $\bea$ from $A$ to $I$. \vspace{\topsep}

The set $\bea(I,I)$ has a canonical structure of unital ring, with the obvious addition and with multiplication induced by morphism composition: $g \cdot f = g \circ f$. Write $R = \bea(I,I)$ for this ring. It is not hard to see that for each $A \in \bea$, $\bea(A,I)$ has a canonical left $R$-module structure induced by post-composition of morphisms, and each morphism $f\colon A \rightarrow B$ in $\bea$ induces an $R$-equivariant map $f^\ast\colon \bea(B, I) \rightarrow \bea(A,I)$. Therefore the arrangement $A \mapsto \bea(A,I)$ extends naturally to a contravariant additive functor $K\colon \bea \rightarrow \rmod$, which is faithful and exact since $I$ is an injective cogenerator.\vspace{\topsep}

Now it suffices to show $K$ is full on $\EA$. For two objects $A, B \in \EA$ and $$\fbar\colon K(B) = \bea(B,I) \rightarrow K(A) = \bea(A, I)$$ we need to find $f\colon A \rightarrow B$ such that $\fbar = K(f)$. By the argument in the first paragraph, there exist two exact sequences
$$\begin{tikzcd}
    0 \ar[r] & A \ar[r] & I
\end{tikzcd}$$
and $$\begin{tikzcd}
    0 \ar[r] & B \ar[r] & I \ar[r] & T \ar[r] & 0
\end{tikzcd}$$
An application of $K$ to these sequences yields the commutative  diagram $$\begin{tikzcd}
    0& KB \ar[d, "\fbar"'] \ar[l] & R \ar[d, dotted, "\ybar"]\ar[l] & KT\ar[l] & 0 \ar[l]\\
0 & KA \ar[l] & R \ar[l]
\end{tikzcd}$$
Here the existence of $\ybar$ is guaranteed by the projectivity of $R$ in $\rmod$. Since all $R$-equivariant maps from $R$ to $R$ can be realized as right multiplication of certain elements, we can see there exists $y \in R = \bea(I,I)$ such that $K(y) = \ybar$. Now in $\bea$ we have the diagram
$$\begin{tikzcd}
    0 \ar[r] & B \ar[r] & I  \ar[r] & T \ar[r] & 0 \\
0 \ar[r] & A \ar[r] & I \ar[u, "y"']
\end{tikzcd}$$
Here we have $\begin{tikzcd}
    A \ar[r] & I \ar[r, "y"] & I \ar[r] & T
\end{tikzcd} = 0$ (since its image under $K$ is $0$, and $K$ is faithful), hence there exists a morphism $f\colon A \rightarrow B$ such that 
$$\begin{tikzcd}
    B \ar[r] & I  \\
A \ar[u, "f"] \ar[r] & I \ar[u, "y"']
\end{tikzcd}$$
commutes. Now since $KB \rightarrow KI$ is a monomorphism, an application of $K$ to the above commutative square yields $K(f) = \fbar$, hence completes the proof.
\epf

Combining the above results altogether, we can thus deduce the Freyd--Mitchell embedding theorem:

\pf[Proof of Theorem~\ref{FE}]
Since $\EA$ is abelian, its opposite category $\EA^\op$, is also abelian, hence by Proposition~\ref{preone} it can be realized as a small abelian full subcategory of some Grothendieck abelian category $\widehat{\EA^\op}$ via a covariant fully faithful exact additive functor $\iota_{\EA^\op}$. Now by Proposition~\ref{pretwo} and Proposition~\ref{main}, there exists a unital ring $R$ and a contravariant faithful exact additive functor $K$ from $\widehat{\EA^\op}$ to $\rmod$, which is also full on the image category of \ $\iota_{\EA^\op}$. Now the composition $$\widehat{K} = K \circ \iota_{\EA^\op}$$ is a contravariant fully faithful exact additive functor from $\EA^\op$ to $\rmod$, which is equivalently a covariant fully faithful exact additive functor from $\EA$ to $\rmod$. 
\epf


\section{Proof of technical propositions}

This section is devoted to the proof of two technical propositions claimed in the previous section, namely Proposition~\ref{preone} and Proposition~\ref{pretwo} (actually, this part is more fun and insightful than it seems to be). To finish the proof we need several auxiliary definitions:

\begin{defn}
For any ordinal $\alpha$, its \textbf{cofinality} $\mathrm{cf}(\alpha)$ is the smallest cardinal such that there exists a subset $S \subset \alpha$ with this cardinal and $\sup S = \alpha$.
\end{defn}

\begin{defn}
Fix a cocomplete category $\EC$, a class of morphisms $\EJ \subset \EuScript{A}\mathrm{rr}(\EC)$ and an ordinal $\alpha$. Then an object $C \in \EC$ is \textbf{$\alpha$-small with respect to $\EJ$} iff for any $\alpha$-transfinite $\EJ$-morphism $X$ in $\EC$ (this is just a nickname for the functor $X\colon \alpha  \rightarrow \EC$ for which every morphism in its image lies in $\EJ$), the natural map
$$\Theta\colon \colim_{\gamma < \alpha} \EC(C, X_\gamma) \rightarrow \EC(C, \colim_{\gamma < \alpha} X_\gamma)$$
is a surjection.
\end{defn}

\begin{rem}
Equivalently, this requires for every morphism $f$ from $C$ to $\colim_{\gamma < \alpha} X_\gamma$ there exists an ordinal $\gamma_0 < \alpha$ such that $f$ factors through $X_{\gamma_0} \rightarrow \colim_{\gamma < \alpha} X_\gamma$. Note that if the morphisms in $\EJ$ are all monomorphisms, then any two such factorizations $C \xrightarrow{f'} X_{\gamma'_0}$, $C \xrightarrow{f''} X_{\gamma''_0}$ (here we assume $\gamma'_0 < \gamma''_0$) must be essentially the same, in the sense that the diagram
$$\begin{tikzcd}
    C \ar[r, "f'"] \ar[rd, "f''"']& X_{\gamma'_0}\ar[d]\\
& X_{\gamma''_0}
\end{tikzcd}$$
commutes, i.e.\ the natural map $\Theta$ above is bijective.
\end{rem}


\begin{defn}
The \textbf{ind-category} (or equivalently \textbf{category of ind-objects / free filtered cocompletion}) of a small category $\EC$ is a pair $(\ind(\EC), \iota_\EC)$, where 
\begin{enumerate}
\item $\ind(\EC)$ is a category closed under filtered colimits.

\item $\iota_\EC\colon \EC \rightarrow  \ind(\EC)$ is a fully faithful functor (i.e.\ realizes $\EC$ as a full subcategory of $\ind(\EC)$).

\item Up to equivalence, the ind-category of $\EC$ is the unique pair such that 
$$\iota_\EC^\ast\colon \fun^{\mathrm{finitary}}(\ind(\EC), \ED) \rightarrow \fun(\EC, \ED)$$
is an equivalence for every small category $\ED$ (here $ \fun^{\mathrm{finitary}}(\EP, \EQ)$ is the category of all functors from $\EP$ to $\EQ$ that preserves filtered colimits).
\end{enumerate}
\end{defn}

\begin{rem}\label{rem1}
A filtered category, for which each object is associated with again a filtered category, can be glued compatibly again into a filtered category. Hence, the ind-category $\ind(\EC)$ can be realized as the full subcategory of $$\pshv(\EC) = \fun(\EC^\op, \set)$$ spanned by all objects that can be written as (i.e.\ isomorphic to) $\colim_{\EJ} \EC(-,c)$ for some small filtered category $\EJ$.
From here we can see $\EC$-objects are \textbf{compact} in $\ind(\EC)$ (that is, for any $C \in \EC$, the functor $\ind(\EC)(C, -): \ind(\EC) \rightarrow \set$ commutes with filtered colimits) via a direct application of Yoneda lemma.
\end{rem}

In this section we first prove the following refinement of Proposition~\ref{preone}:

\begin{prop}\label{preoneprime}
For a small abelian category $\EA$, there exists a canonical structure of a Grothendieck abelian category on $\ind(\EA)$ such that the embedding functor $\iota_\EA$ is fully faithful exact additive (i.e.\ exhibits $\EA$ as its abelian full subcategory). 
\end{prop}

%A sketchy proof of the two propositions above can be found in Akhil Mathew's blogs. For more details, see \cite{Mit65} which is perfectly readable.

%{\color{red}Warning: a reader unwilling to dive into lengthy discussion on abstract nonsense shall jump directly to the next subsection.} The remaining part of this subsection consists of a thorough proof for EVERY detail in the two propositions mentioned above. Behold!


\pf[Proof of Proposition~\ref{preoneprime}]
First we discuss some basic properties of $\ind(\EC)$ for an ordinary small category $\EC$. We have $\EC \subset \ind(\EC) \subset \pshv(\EC)$, where both inclusions are fully faithful. Moreover, fine (co)limit properties of $\EC$ are inherited to $\ind(\EC)$ and are preserved by the embedding:
\begin{fact}\label{fact1} For a small category $\EC$ and its ind-category $(\ind(\EC), \iota_\EC)$:
\begin{enumerate}[format=\textup]
\item The colimit of a filtered diagram in $\ind(\EC)$ is exactly the colimit of the same diagram in $\pshv(\EC)$.
\item For a small filtered category $\EQ$, we say a diagram $F\colon \EP \rightarrow \ind(\EC)$ is \textbf{$\EQ$-uniform} iff there exists a functor $\widehat{F}\colon \EP \times \EQ \rightarrow \EC$ such that $F$ is the $\EQ$-colimit of the diagram of functors $\{\widehat{F}_{-,q}\}_{q \in \EQ}$. More precisely:
\begin{itemize}
\item For each object $X = F_p$ in the diagram $F$ we have $X \simeq \colim_\EQ \widehat{F}_{p,q}$.
\item For each morphism $f\colon X \rightarrow Y$ in the diagram $F$ (here $X \simeq \colim_\EQ \widehat{F}_{p_0, q}$,  $Y \simeq \colim_\EQ \widehat{F}_{p_1, q}$) we have $f \simeq \colim_\EQ f_q$ where $f_q\colon \widehat{F}_{p_0,q} \rightarrow \widehat{F}_{p_1,q}$ is a morphism in $\EC$ for each $q \in \EQ$.
\end{itemize}  
Our claim is that every finite \textbf{primitive} diagram (i.e.\ diagrams not containing any two composable nontrivial arrows, such as diagram of finite products, coproducts, equalizers and coequalizers etc.) is $\EQ$-uniform for some small filtered category $\EQ$.
\item If $\EC$ admits all finite limits then so does $\ind(\EC)$, and the embedding $\iota_\EC\colon \EC \rightarrow \ind(\EC)$ preserves finite limits. More precisely, the limit of a finite diagram in $\ind(\EC)$ is exactly the limit of the same diagram in $\pshv(\EC)$, hence finite limits commute with filtered colimits in $\ind(\EC)$.
\item If $\EC$ admits all finite colimits then so does $\ind(\EC)$, and the embedding $\iota_\EC\colon \EC \rightarrow \ind(\EC)$ preserves finite colimits. In this case $\ind(\EC)$ is automatically cocomplete.
\end{enumerate}
\end{fact}
\pf[Proof of Fact~\ref{fact1}]
The first statement holds trivially because $\ind(\EC)$ is closed under filtered colimits as full subcategory of $\pshv(\EC)$ (cf.\ Remark~\ref{rem1}).\vspace{\topsep}

The second assertion is somehow interesting, and the following proof here is cited from \cite{KS06}: by the natural equivalence $\Theta\colon \ind(\EC^n) \rightarrow \ind(\EC)^n$ for finite $n$, and the fact that a finite product of filtered categories is still a filtered category, it suffices to consider the case where the diagram $F$ is simply of the form $X_0 \xrightarrow{f} X_1$, here $X_0 = \colim_{\EQ_0} X_{0,q}$ and $X_1 = \colim_{\EQ_1} X_{1, q}$ for two small filtered categories $\EQ_0$ and $\EQ_1$. Write $\EC \downarrow X_0$ and $\EC \downarrow X_1$ for the two obvious over categories,  $\widetilde{X_0}\colon \EQ_0 \rightarrow \EC\downarrow X_0$, $\widetilde{X_1}\colon \EQ_1 \rightarrow \EC\downarrow X_1$ for the two obvious extension functors, and $\widetilde{f}\colon \EC \downarrow X_0 \rightarrow \EC\downarrow X_1$ for the post-composition functor. Now a careful verification (utilizing compactness of $\EC$-objects in $\ind(\EC)$) shows that 
$\EQ = \widetilde{f} \circ \widetilde{X_0} \downarrow \widetilde{X_1}$
is the small filtered category that we desire, with corresponding functor $\widehat{F}\colon \{0,1\}\times \EQ \rightarrow \EC$ defined ``fiberwise'' by $\widetilde{X}_0 \circ p_0$ and $\widetilde{X}_1 \circ p_1$. Warning: the result of this statement will be frequently used in subsequent discussions without specific mention.\vspace{\topsep}

To prove the third statement, since we already know the embedding of $\EC$ into $\pshv(\EC)$ preserves all limits, it suffices to show $\ind(\EC)$ is closed under finite limits as a full subcategory of $\pshv(\EC)$. For a finite primitive diagram $F\colon \EP \rightarrow \ind(\EC)$, WLOG we assume $F$ is $\EQ$-uniform. By the fact that finite limits commute with filtered colimits in $\pshv(\EC)$, we have (in $\pshv(\EC)$)
$$\lim_\EP F_p = \lim_\EP \colim_\EQ F_{p,q} = \colim_\EQ \lim_\EP F_{p,q}$$
since $\EC$ admits all finite limits, each of $\lim_\EP F_{p,q}$ lies in $\EC$, hence $\lim_\EP F_p$ lies in $\ind(\EC)$, which implies the result we desire. The general existence and preservation of finite limits follows from this since finite products and equalizers are all finite primitive diagrams.  Now we see finite limits commute with filtered colimits in $\ind(\EC)$ since this happens in $\pshv(\EC)$.\vspace{\topsep}

As to the fourth statement, let $F\colon \EP \rightarrow \EC$ be a finite diagram and $C_F \simeq \colim_\EP F$ be its colimit in $\EC$. Easy to see $C_F$ is the colimit of the same diagram in $\ind(\EC)$ iff $\ind(\EC)(C_F, X)$ is the colimit of $\ind(\EC)(F-,X)$ for every $X \in \ind(\EC)$, and this can be easily checked by compactness of $\EC$-objects in $\ind(\EC)$ together with the fact that finite limits commute with filtered colimits in $\set$. Now existence of general finite colimits in $\ind(\EC)$ can be verified by two steps: first, finite coproducts and coequalizers are primitive diagrams, hence these diagrams can be assumed $\EQ$-uniform for some filtered $\EQ$, hence these colimits in $\ind(\EC)$ can be extracted (as a colimit) from corresponding coproducts/coequalizers of the components in $\EC$; for the next step we can construct general finite colimits in $\ind(\EC)$ from these finite coproducts and coequalizers. Finally, as a category admitting all finite and filtered colimits, $\ind(\EC)$ is necessarily cocomplete.
\epf

Back to Proposition~\ref{preoneprime}, since $\EA$ an abelian category, $\ind(\EA)$ is automatically pre-additive: for any two objects $X = \colim_\EP X_p$ and $Y = \colim_\EQ Y_q$ we have
$$\ind(\EA) (X, Y) = \lim_{\EP^\op} \colim_\EQ \EA(X_p, Y_q)$$
which naturally is an abelian group (since the forgetful functor $U\colon \ab \rightarrow \set$ preserves all limits and all filtered colimits), and the composition map is also automatically bilinear. Moreover, the zero object $0_\EA$ of $\EA$ continues to be the zero object of $\ind(\EA)$. The abelian category $\EA$ admits all finite limits and colimits, hence all assertions in Fact~\ref{fact1} holds for ($\ind(\EA), \iota_\EA$): $\ind(\EA)$ is cocomplete additive with exact filtered colimits, and the embedding functor $\iota_\EA$ is fully faithful exact. These can  furthermore imply it is abelian since for any morphism $f\colon X \rightarrow Y$ in $\ind(\EC)$, we can assume the diagram induced by $f$ (apparently finite primitive) to be $\EQ$-uniform, hence 
$$\coker \ker (f) \simeq \ker \coker (f)$$ 
follows from the fact that 
$$\coker \ker (f_q) \simeq \ker \coker (f_q), \ \forall q \in \EQ$$
and exactness of filtered colimits in $\ind(\EC)$. Also, from smallness of $\EA$ and compactness of its objects in $\ind(\EA)$ we can deduce that $\ind(\EA)$ is locally presentable, hence it is a Grothendieck abelian category (with generator being $U = \bigoplus_{A \in \EA} A$).
\epf

Next we turn to Proposition~\ref{pretwo}: 

\pf[Proof of Proposition~\ref{pretwo}]
Suppose $\EA$ is a Grothendieck abelian category with generator $U$. For any object $A \in \EA$, write $\sub(A)$ for the category of (isomorphism classes of) subobjects of $\EA$, and write $\quot(A)$ for the category of its quotient objects.
Note that $\sub(A)$ and $\quot(A)$ are equipped with canonical partial order structures, and there is an order-reversing bijection between the two categories via taking the kernel/cokernel. Since $U$ is a generator, the covariant functor $$\EA(U, -)\colon \sub(A) \rightarrow \mathcal{P}(\EA(U,A))$$
(here $\mathcal{P}(X)$ is the power set of $X$ with poset structure induced by inclusion) is an order-preserving injection, which implies $\sub(A)$ is small, i.e.\ $\EA$ is well-powered. Thus, the order-reversing bijection mentioned earlier shows that $\quot(A)$ is also small, i.e.\ $\EA$ is also well-copowered. Now for the cocomplete well-copowered category $\EA$ with a generator, the Special Adjoint Functor Theorem \cite[Theorem~4.6.10]{Rie17} asserts that for any small category $\EJ$, the (cocontinuous) diagonal functor $$\Delta\colon \EA \rightarrow \fun(\EJ, \EA)$$ has a right adjoint (which is automatically $\lim_\EJ$), i.e.\ $\EA$ is complete.\vspace{\topsep}

Now suppose $\EA$ has enough injective objects, i.e.\ every object can be realized as a subobject of an injective object (this always holds in a Grothendieck abelian category; we will postpone the proof for a while). We claim that $\EA$ has an injective cogenerator. Let $\widehat{U} = \bigoplus_{W \in \quot(U)} W$ and $\widehat{U} \rightarrowtail I$ be a monomorphism into some injective $I$. Our claim is that $I$ is an injective cogenerator.\vspace{\topsep}

We know that a morphism in an abelian category is zero iff its image is the zero object. Hence, an exact additive functor between abelian categories is faithful iff it preserves nonzero objects. In particular, an injective object is a cogenerator iff from any nonzero object there is a nonzero morphism to this injective object. Now pick an arbitrary nonzero object $A \in \EA$. There is a nonzero morphism $t\colon U \rightarrow A$ since $U$ is a generator. Thus, we have the following diagram (here we write $\rightarrowtail$ and $\twoheadrightarrow$ for monomorphisms and epimorphisms respectively)
$$\begin{tikzcd}
    U\ar[rd, "t"'] \ar[r, two heads, "\pi_t"] & \im(t) \  \ar[d, tail, "\iota_t"] \ar[r, tail]  & \widehat{U} \  \ar[r, tail] & I \\
& A \ar[rru, dotted, "s"']&&&
\end{tikzcd}$$
Here existence of the dotted morphism $s\colon A \rightarrow I$ is induced by injectivity of $I$, and it is nonzero since $s \circ \iota_t$ is a monomorphism starting from a nonzero object $\im(t)$. Hence, $I$ is indeed an injective cogenerator as claimed.\vspace{\topsep}

Now our final task is to prove
\begin{thm}\label{eninj}
A Grothendieck abelian category $\EA$ has enough injective objects. Actually, there exists a functor $I\colon \EA \rightarrow \EA$ and natural transformation $\iota\colon \id_\EA \Rightarrow I$, such that for each object $A \in \EA$:
\begin{itemize}
\item $I(A)$ is an injective object.
\item $\iota_A\colon A \rightarrow I(A)$ is a monomorphism.
\end{itemize}
 \end{thm}
To verify this assertion, we should first understand the poset structure of $\sub(Y)$ for an object $Y$ in a Grothendieck abelian category $\EA$. In fact, $\sub(Y)$ is not merely a poset, but also a complete lattice: that is, it admits every possible \textbf{meet} (infimum) and \textbf{join} (supremum).  For a family of subobjects $\{C_p\}_{p \in \EP}$ of $Y$, their meet (i.e.\ greatest lower bound, conventionally called \textbf{intersection}), denoted by $\bigcap_{p \in \EP} C_p$,  can be realized as the limit of the diagram of all monomorphisms $C_p \rightarrowtail Y$. Their join (i.e.\ least upper bound, conventionally called \textbf{sum}), denoted by $\sum_{p \in \EP} C_p$, is realized as the image of the canonical morphism $\bigoplus_{p \in \EP} C_p \rightarrow Y$. If furthermore the index set $\EP$ is directed (i.e.\ includes the binary join for each pair $(C_{p_0}, C_{p_1})$), then the sum $\sum_{p \in \EP} C_p$ can also be realized as $\colim_{p \in \EP} C_p$ (the induced morphism from this colimit to $Y$ is a monomorphism since filtered colimits are exact). In this case, exactness of filtered colimits also tells us for each morphism $f\colon X \rightarrow Y$, we have (here $f^{-1}\colon \sub(Y) \rightarrow \sub(X)$ is defined via pullback of  subobjects)
$$f^{-1} \left( \sum_{p \in \EP} C_p \right) \simeq \sum_{p \in \EP} f^{-1} \left( C_p\right).$$
We will also need the following facts:
\begin{fact}\label{fact2} For a Grothendieck abelian category $\EA$:
\begin{enumerate}[format=\textup]
\item If $B \subset A$, $V \subset U$ are objects in $\EA$ (here the $\subset$ symbol stands for subobject relation), and the following diagram is a pullback square (i.e.\ $V \simeq f^{-1}(B)$)
$$\begin{tikzcd}
    V \ \ar[r, tail] \ar[d] \pullback& U \ar[d, "f"]\\
B \ \ar[r, tail, "\iota"'] & A
\end{tikzcd}$$
then the following sequence 
$$\begin{tikzcd}
    0 \ar[r] & V \ar[r] & U \oplus B \ar[r, "f + \iota"] & A
\end{tikzcd}$$
is exact, here $V \rightarrowtail U \oplus B$ is the map $V \rightarrowtail U$ minus the map $V \rightarrow B$ in the above square diagram. Hence, if we form the pushout diagram
$$\begin{tikzcd}
    V \ \ar[r, tail] \ar[d] & U \ar[d]\\
B \ \ar[r, tail] & T \pushout
\end{tikzcd}$$
then the induced morphism $T \rightarrow A$ is a monomorphism.
\item If $\{i_p\colon V_p \rightarrowtail U_p\}_{p \in \EP}$ is a set of monomorphisms, then the coproduct $\bigoplus_{p \in \EP} i_p\colon \bigoplus_{p \in \EP} V_p \rightarrow \bigoplus_{p \in \EP} U_p$ is also a monomorphism.
\item If $A$ is an object of $\EA$ and $\sub(A)$ has cardinality $\kappa_A$, then for each limit ordinal $\alpha$ whose cofinality $\mathrm{cf}(\alpha) > \kappa_A$, $A$ is $\alpha$-small with respect to $\EA_{\mathrm{mon}}$ (the class of all monomorphisms in $\EA$).
\end{enumerate}
\end{fact}
\pf[Proof of Fact~\ref{fact2}]
The first statement seems a little scary, but actually it is merely a reformulation of definition of pullback in abelian category. The part after ``Hence'' is true because $T$ is the cokernel of $V \rightarrowtail U \oplus B$.\vspace{\topsep}

To prove the second statement it suffices to show the kernel of this morphism is $0$; this is trivial when $\EP$ is finite, since finite coproducts are automatically products which commutes with taking kernels; for infinite $\EP$, note that an infinite coproduct is a filtered colimit of finite coproducts,  and the result follows from exactness of filtered colimits.\vspace{\topsep}

For the third statement, pick an arbitrary $\alpha$-transfinite $\EA_{\mathrm{mon}}$-morphism $X$ and a morphism $f\colon A \rightarrow \colim_{\gamma < \alpha} X_\gamma$. Then for each ordinal $\gamma < \alpha$, we define $A_\gamma = f^{-1} (X_\gamma)$. By the fact that inverse images (being pullbacks) commute with sums (being filtered colimits of finite products), we see $A = \sum_{\gamma < \alpha} A_\gamma$ as a subobject of $A$. Now to prove this statement it suffices to show there exists some $\gamma_0 < \alpha$ such that $A_{\gamma_0} \simeq A$. Since $\left| \sub(A) \right| = \kappa_A$, we can pick a subset $S$ of $\alpha$ with cardinality $\leq \kappa_A$ for which $A = \sum_{\gamma \in S} A_\gamma$. Now let $\gamma_0 = \sup S + 1$. Since $\mathrm{cf}(\alpha) > \kappa_A$, it follows that $\gamma_0 < \alpha$, and by its definition we have $A = \sum_{\gamma < \gamma_0} A_\gamma = A_{\gamma_0}$, which implies this $\gamma_0$ is what we desire.
\epf

Using these facts we develop a criterion for injective objects:
\begin{lemma}[Baer criterion] \label{baer}
Let $\EA$ be a Grothendieck abelian category with generator $U$. Then an object $I$ in $\EA$ is injective iff for every subobject $V$ of $U$, any morphism $i\colon V \rightarrow I$ can be extended to a morphism $\widetilde{i}\colon U \rightarrow I$.
\end{lemma}
\pf[Proof of Lemma~\ref{baer}]
The ``only if'' direction is trivial. For the ``if'' direction, suppose $B$ is an object in $\EA$ and $B'$ is a proper subobject of $B$, together with a morphism $g\colon B' \rightarrow I$. We claim that $g$ can be extended to some $\widetilde{g}\colon B'' \rightarrow I$, where $B''$ is a subobject of $B$ and a proper superobject of $B'$.\vspace{\topsep}

Since the inclusion $B' \rightarrow B$ is proper, by definition of a generator there exists a morphism $t\colon U \rightarrow B$ whose image is not contained in $B'$. Now consider $V = t^{-1}(B')$ defined by the following pullback square
$$\begin{tikzcd}
    V \ \ar[r, tail] \ar[d] \pullback& U \ar[d, "t"]\\
B' \ \ar[r, tail] & B
\end{tikzcd}$$
and form the pushout square 
$$\begin{tikzcd}
    V \ \ar[r, tail] \ar[d]  & U \ar[d]\\
B' \ \ar[r, tail] & B'' \pushout
\end{tikzcd}$$
Now Fact~\ref{fact2} tells us the induced map $B'' \rightarrow B$ is monomorphism, i.e.\ $B''$ is a subobject of $B$, which is necessarily a proper superobject of $B'$ since its image contains $\im(t)$. Moreover, by assumption the morphism $V \rightarrow B' \xrightarrow{g} I$ factors through $V \rightarrowtail U$, hence we have the following diagram
$$
\begin{tikzcd}
    V \ \ar[d] \ar[r, tail] & U \ar[d] \ar[ddr, dashed, bend left=20] &\\
        B' \ \ar[r, tail] \ar[drr, bend right=20, "g"'] & B'' \pushout \ar[dr, dotted, "\widetilde{g}" description] & \\
        & & I\\
\end{tikzcd}
$$
Now the dotted morphism $\widetilde{g}\colon B'' \rightarrow I$ in the lower-right corner provides our desired extension of $g$.\vspace{\topsep}

Now we show that $I$ is injective. For a morphism $f\colon A \rightarrow I$ and a monomorphism $A \rightarrowtail B$, let $\EJ$ be the poset of pairs $(B', f')$ (where $B'$ is a subobject of $B$ containing $A$ and $f'\colon B' \rightarrow I$ is a extension of $f$), ordered by the extension relation, i.e.\ $(B_0, f_0) \preccurlyeq (B_1,f_1)$ iff $B_0$ is a subobject of $B_1$ and $f_1$ is an extension of $f_0$ onto $B_1$. Clearly, every chain in $\EJ$ admits an upper bound (their sum), hence by Zorn's lemma there is a maximal element $(B_m, f_m)$. By discussion of the previous paragraph, the only possibility for $B_m$ is $B_m = B$, hence $f_m = \widetilde{f}$ is our desired extension. Thus we obtain the injectivity of $I$.
\epf

Now we turn to the proof of Theorem~\ref{eninj} which is fortunately the last part of the technical proposition sequence. Note the similarity between the following proof and Quillen's small object argument (cf.\ \cite{Qui67}).

\pf [Proof of Theorem~\ref{eninj}] Write $U$ for the generator in $\EA$.
First we define a functor $M\colon \EA \rightarrow \EA$ and a natural transformation $\iota_{0,1}\colon \id_\EA \Rightarrow M$, such that for each object $A \in \EA$, $\iota_{0, 1, A}\colon A \rightarrow M(A)$ is a monomorphism and for each morphism $g\colon V \rightarrow A$ starting from some subobject of $U$, there exists a commutative square
$$\begin{tikzcd}
    V \ \ar[r, tail] \ar[d, "g"']  & U \ar[d, "g_{\mathrm{ext}}"]\\
A \ \ar[r, tail, "\iota_{0, 1, A}"'] &M(A)
\end{tikzcd}$$
The object $M(A)$ may be constructed as follows: let $\EP_A$ be the set of all possible pairs $(V_p, g_p)$ where $V_p$ is a subobject of $U$ and $g_p\colon V_p \rightarrow A$. With this we form the pushout square
$$\begin{tikzcd}
    \bigoplus_{\EP_A} V_p \ \ar[r, tail] \ar[d, "\bigoplus_{\EP_A} g_p"']  & U^{\oplus \EP_A} \ar[d]\\
A \ \ar[r, tail, "\iota_{0, 1, A}"'] & M(A) \pushout
\end{tikzcd}$$
and henceforth constructs $M(A)$. Here the lower morphism $\iota_{0,1, A}\colon A \rightarrow M(A)$ is a monomorphism, since the upper map is one (cf.\ Fact~\ref{fact2}), and pushouts in an abelian category preserve monomorphisms; the other required property holds obviously as well. Easy to see the arrangement $A \mapsto M(A)$ can be extended naturally to a functor such that the morphisms $\iota_{0,1, A}$ assemble together into a natural transformation $\iota_{0,1}$.\vspace{\topsep}

Now we define via transfinite induction a sequence of functors $M_\alpha$ for all ordinals:
\begin{enumerate}
\item For $\alpha = 0$, let $M_0 = \id_\EA$.
\item For $\alpha = \beta + 1$ a successor ordinal, let $M_\alpha = M \circ M_\beta$.
\item For $\alpha$ limit ordinal, let $M_\alpha = \colim_{\gamma < \alpha} M_\gamma$ be the colimit over the directed set $\alpha$.
\end{enumerate}
Easy to see for all pair $\gamma < \delta$, there is a canonical natural transformation $\iota_{\gamma, \delta}\colon M_\gamma \Rightarrow M_\delta$ which is component-wise monomorphism. Now let $\Omega$ be the smallest infinite ordinal whose cofinality $\mathrm{cf}(\Omega)$ is strictly greater than $\left|  \sub(U) \right|$, the cardinal of $\sub(U)$). We define $I = M_\Omega$, $\iota = \iota_{0, \Omega}\colon \id_\EA \Rightarrow I$. We claim this pair is the construction we desire.\vspace{\topsep}

 For an arbitrary object $A \in \EA$, $\iota_A$ is by definition a monomorphism; for the injectivity of $I(A)$, by Lemma~\ref{baer} it suffices to consider an arbitrary pair $V \subset U$. Suppose there is a morphism $f\colon V \rightarrow I(A) = M_\Omega (A)$. Since $V \subset U$, we have $\left|\sub(V) \right| \leq \left|\sub(U) \right|$, then it follows from Fact~\ref{fact2} that there exists $\gamma_0 < \Omega$ such that $f =  \iota_{\gamma_0, \Omega, A} \circ f'$, where $f'$ actually has codomain $M_{\gamma_0}(A)$. Thus, the following diagram
$$
 \begin{tikzcd}
    V \ \ar[d, "f'"']  \ar[rr, tail] & & U \ar[d, dashed, "f'_{\mathrm{ext}}"'] \ar[ddr, dotted, bend left=10, "\widetilde{f}"] &\\
        M_{\gamma_0}(A) \ \ar[rr, tail, "\iota_{\gamma_0, \gamma_0 + 1, A}"] \ar[drrr, tail, bend right=10, "\iota_{\gamma_0, \Omega, A}"'] && M_{\gamma_0 +1}(A)  \ar[dr, tail, "\iota_{\gamma_0 + 1, \Omega, A}" description] & \\
        && & M_\Omega (A) = I(A)
\end{tikzcd}
$$
tells us the dotted morphism $\widetilde{f}$ is our desired extension of $f$ onto $U$. Now the injectivity of $I(A)$ follows from Lemma~\ref{baer}.
\epf
Now we finally have the proof of Proposition~\ref{pretwo} completed.
\epf

% \small
% \bibliographystyle{alpha}

% \bibliography{cite0}

\printbibliography

\end{document}
