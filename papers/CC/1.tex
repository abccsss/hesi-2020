示性类的理论可以从多个观点出发而建立. 在本节中, 我们从分类空间的观点开始.
从拓扑的角度看, 这种观点是最简洁、最本质的.
但从几何的角度看, 这种观点有时难以应用到实际的计算中.
因此, 在几节之后, 我们会从另一种几何的观点出发, 来重新讨论流形上的示性类理论.

每个拓扑群 $G$ 都有一个\term{分类空间}, 记为 $\upB G$.
这是一个拓扑空间, 借助它, 我们就能知道任何空间 $X$ 上的 $G$-主丛的所有同构类.
通过这个空间的信息, 我们能给出主丛的一些拓扑不变量, 这就是示性类.

\subsection{带结构的纤维丛}

我们从一些基本的定义开始. 

\begin{definition}
    拓扑空间 $B$ 上的\term{纤维丛} (fibre bundle)
    是一个映射
    \[ p \: E \to B, \]
    它满足以下条件.
    \begin{itemize}
        \item 
            (\term{局部平凡性})
            对每个 $b \in B$, 存在 $b$ 的邻域 $U$,
            使得有交换图
            \[ \begin{tikzcd}[column sep={4em,between origins}]
                p^{-1}(U) \ar[rr, "\phi", "\simeq"'] \ar[dr, "p"'] &&
                U \times F \ar[dl, "\operatorname{pr}_1"] \\
                & U \rlap{\ ,}
            \end{tikzcd} \]
            其中 $F = p^{-1}(b)$ 是映射 $p$ 的纤维, $\phi$ 是同胚.
    \end{itemize}
    特别地, $B$ 的同一连通分支上的所有纤维都同胚. 
\end{definition}

当我们提到 $B$ 上的纤维丛时,
我们总是假定取好了 $B$ 一个开覆盖 $\{ U_\alpha \}$,
使得局部平凡性在每个开集上成立,
并且选好了上面的交换图中的映射
\[ \phi_\alpha \: p^{-1}(U_\alpha) \simeq U_\alpha \times F_\alpha. \]
对于 $b \in U_\alpha \cap U_\beta$,
以上信息给出了纤维 $F = p^{-1}(b)$ 的自同胚
\[ \phi_\beta^{-1} \circ \phi_\alpha \: F \simeq F. \]
这个自同胚称为\term{转移映射} (transition map).

\begin{definition}
    设 $B$ 是拓扑空间, 记 $\bbK$ 为 $\bbR, \bbC, \bbH$ 之一. 
    $B$ 上的 $\bbK$-\term{向量丛} (vector bundle) 是一个纤维丛
    \[ p \: E \to B, \]
    其纤维都是 $\bbK$-线性空间, 
    且转移映射是线性映射. 
\end{definition}

\begin{definition}
    设 $G$ 是拓扑群, $B$ 是拓扑空间. 
    $B$ 上的 $G$-\term{主丛} (principal bundle) 是一个纤维丛
    \[ p \: E \to B, \]
    其中 $E$ 带有一个保持纤维的右 $G$-作用, 
    使得每个纤维作为 $G$-空间同构于 $G$. 
\end{definition}

为了叙述方便, 我们将这一类定义统一起来. 

\begin{definition}
    设 $\cat{C}$ 是群胚, 带有函子 $\cat{C} \to \cat{Top}$. 
    设 $B$ 是拓扑空间. $B$ 上的 $\cat{C}$-\term{丛} 是一个纤维丛
    \[ p \: E \to B, \]
    其每个纤维来自一个 $\cat{C}$ 的对象, 其转移映射来自 $\cat{C}$ 中相应对象间的映射. 
\end{definition}

这一定义描述了具有额外结构的纤维丛. 
我们可以得到 $B$ 上所有 $\cat{C}$-丛构成的范畴, 记为
\[ \cat{Bund}_{\cat{C}} (B), \]
其中的态射是保持纤维的连续映射, 使得所有纤维之间的映射都来自 $\cat{C}$ 中相应对象间的映射.
这样的态射自动是可逆的.

例如, 对于纤维丛和主丛而言, 我们分别取了 $\cat{C} = \{ \mathbb{K}\text{-向量空间} \}$
和 $\cat{C} = \{ G\text{-齐性空间} \}$. 
后一种情况下, 我们使用记号 $\cat{Bund}_G (B)$. 

\begin{notation}
    对 $\bbK = \bbR, \bbC, \bbH$, 
    记 $\cat{Vect}(n, \bbK)$ 为所有 $n$ 维 $\bbK$-向量空间构成的群胚. 
    对拓扑群 $G$, 记 $\cat{Hmg}(G)$ 为所有 $G$-齐性空间构成的群胚. \varqed
\end{notation}

\begin{proposition} \label{thm-1-functor}
    设 $\cat{C}, \cat{D}$ 是满足上述条件的群胚, 
    $F \: \cat{C} \to \cat{D}$ 是任一函子. 
    则对任何空间 $B$, 有诱导的函子
    \[ F_* \: \cat{Bund}_{\cat{C}}(B) \to \cat{Bund}_{\cat{D}}(B), \]
    它把一个纤维为 $X$ 的丛映到一个纤维为 $F(X)$ 的丛. 
    并且, 这一构造关于 $F$ 是函子性的. \qed
\end{proposition}

\begin{remark}
    这意味着 $\cat{Bund}_{(-)}(B) \: \cat{Gpd}_{/\cat{Top}} \to \cat{Gpd}$
    是一个 $2$-函子.  \varqed
\end{remark}

\begin{example}
    在上述命题中, 如果令
    \[
        F = \oplus \: 
        \cat{Vect}(n, \bbK) \times \cat{Vect}(m, \bbK) \to 
        \cat{Vect}(n + m, \bbK),
    \]
    那么对空间 $B$ 上的两个向量丛 $E_1$ 和 $E_2$, 我们可以得到 $B$ 上新的向量丛
    \[ E_1 \oplus E_2 = \oplus_* (E_1, E_2). \]
    类似地, 我们可以定义向量丛的张量积、外代数、$\Hom$、对偶, 等等.  \varqed
\end{example}

给定 $n$ 维 $\bbK$-向量空间 $X$, 其所有\term{标架} (frame, 即有序基) 构成的空间是一个 $\GL(n, \bbK)$-齐性空间. 
这定义了一个函子
\[ F \: \cat{Vect}(n, \bbK) \to \cat{Hmg}(\GL(n, \bbK)), \]
它把向量空间变成对应的 $\GL(n, \bbK)$-齐性空间. 

\begin{definition}
    设 $E$ 是空间 $B$ 上的 $n$ 维 $\bbK$-向量丛, 其中 $\bbK = \bbR, \bbC, \bbH$. 
    设 $F$ 是上述的标架函子. 则 $\GL(n, \bbK)$-主丛
    \[ F_*(E) \to B \]
    称为 $E$ 的\term{标架丛} (frame bundle). 
\end{definition}

因为在 $F$ 的作用下, 向量空间的自同构恰好与 $\GL(n, \bbK)$-齐性空间的自同构相对应, 
所以 $F$ 是可逆的. 因此, 我们有自然的同构
\[ 
    \cat{Bund}_{\cat{Vect}(n, \bbK)} (B) 
    \simeq \cat{Bund}_{\GL(n, \bbK)} (B). 
\]
因此, 在术语上, 我们不再区分向量丛和它对应的 $\GL(n,\bbK)$-主丛. 

\subsection{同伦不变性}

\begin{definition}
    设 $f \: X \to Y$ 是连续映射, $E$ 是 $Y$ 上的 $\cat{C}$-丛. 
    $E$ 的\term{拉回} (pullback) 是拓扑空间的纤维积 $f^* E = X \times_Y E$, 
    它是 $X$ 上的 $\cat{C}$-丛. 
\end{definition}

直观地看, 向量丛 $f^* E$ 由下面的描述而决定:
它在 $x \in X$ 处的纤维 ``自然地'' 等于 $E$ 在 $f(x) \in Y$ 处的纤维.

\begin{definition}
    一个 Hausdorff 空间称为\term{仿紧} (paracompact) 的, 
    如果每个开覆盖都有局部有限的开加细. 
\end{definition}

仿紧性是较弱的点集拓扑性质, 例如, 所有流形、CW 复形都是仿紧的.
在本文中, 我们只对仿紧的空间感兴趣. 

\begin{theorem}[同伦不变性]
    设 $E \to Y$ 是 $\cat{C}$-丛. 
    如果 $X$ 是仿紧空间, 
    $f_0, f_1 \: X \to Y$ 是同伦的映射, 那么有 $X$ 上的 $\cat{C}$-丛的同构
    \[ f_0^* E \simeq f_1^* E. \]
\end{theorem}

\begin{proof}
    我们只对 $X$ 紧的情况给出证明. 设
    \[ h \: X \times I \to Y \]
    是 $f_0$ 到 $f_1$ 的同伦. 
    则 $h^* E$ 是 $X \times I$ 上的 $\cat{C}$-丛.
    我们想要构造 $X \times I$ 上两个 $\cat{C}$-丛之间的同构
    \[ \Phi \: f_0^* E \times I \simeq h^* E. \]

    选取 $X$ 的有限开覆盖 $\{ U_i \} _{i=1} ^n$, 选取 $\epsilon > 0$, 并选取
    \[ 0 = t_0 < t_1 < \cdots < t_r = 1, \]
    使得 $h^* E$ 在每个 ``方块'' $U_i \times I_j$ 上是平凡丛,
    其中 $I_j = I \cap (t_{j-1} - \epsilon, t_j + \epsilon)$.

    我们对 $j$ 归纳, 假设 $\Phi$ 已经在 $X \times [0, t_j]$ 上定义好了.

    对每个 $i$, 选取开集 $W_i \subset V_i \subset U_i$,
    其中每个开集的闭包都含于下一个开集, 且 $\{ W_i \}$ 仍覆盖 $X$.
    由 Урысон 引理, 对每个 $i$, 可选取函数 $u_i \: X \to [t_j, t_{j+1}]$, 使得 
    \[ u_i^{-1} (t_j) = X \setminus V_i, \quad u_i^{-1} (t_{j+1}) = \overline{W_i}. \]
    对 $x \in X$, 定义 $v_0 (x) = t_j$, 并定义
    \[ v_i (x) = \max (u_1 (x), \dotsc, u_i (x)). \]
    则有 $t_j = v_0 (x) \leq v_1 (x) \leq \cdots \leq v_n (x) = t_{j+1}$.
    定义 $B_i = \{ (x, t) \mid t_j \leq t \leq v_i (x) \}$. 则
    \[ X \times t_j = B_0 \subset B_1 \subset \cdots \subset B_n = X \times [t_j, t_{j+1}]. \]
    
    现在, 我们对 $i$ 归纳, 假设 $\Phi$ 已经在 $B_{i-1}$ 上定义好了.
    
    由于上面的构造, $B_i \setminus B_{i-1} \subset V_i \times [t_j, t_{j+1}]$,
    并且, $\overline{V_i} \times [t_j, t_{j+1}] \subset U_i \times I_j$.
    对 $(e, t) \in (f_0^* E \times I)|_{B_i \setminus B_{i-1}}$, 定义
    \[ \Phi (e, t) = \phi_i \bigl(x,\ t,\ \rho \circ \Phi (e, v_{i-1} (x))\bigr), \]
    其中 $x \in X$ 是 $e$ 对应的点,
    $\phi_i \: U_i \times I_j \times F \simeq h^* E |_{U_i \times I_j}$ 是局部平凡化映射, $F$ 表示纤维,
    $\rho \: h^* E |_{U_i \times I_j} \to F$ 是先局部平凡化, 再投影到 $F$. 
    我们就完成了构造. 请读者验证我们确实得到了 $\cat{C}$-丛的同构.
\end{proof}

\begin{corollary}
    如果 $X$ 是可缩的仿紧空间, 那么 $X$ 上的所有 $\cat{C}$-丛都是平凡的. \qed
\end{corollary}

\subsection{分类空间}

设 $E \to B$ 是 $\cat{C}$-丛, $X$ 是仿紧空间. 
则有良好定义的映射
\[ \begin{aligned}{}
    [X, B] & \to \mathrm{Bund}_{\cat{C}} (X) \\
    f & \mapsto f^*(E),
\end{aligned} \]
其中 $[X, B]$ 是映射的同伦类的集合, 
$\mathrm{Bund}_{\cat{C}} (X)$ 是 $\cat{C}$-丛的同构类的集合. 

\begin{definition}
    设 $G$ 是拓扑群. 一个 $G$-主丛
    \[ \upE G \to \upB G \]
    称为\term{万有} (universal) 的, 如果对任何仿紧空间 $X$, 映射
    \[ \begin{aligned}{}
        [X, \upB G] & \to \mathrm{Bund}_G (X) \\
        f & \mapsto f^*(\upE G)
    \end{aligned} \]
    是集合的同构. 
    此时, $\upB G$ 称为 $G$ 的\term{分类空间} (classifying space). 
\end{definition}

把 $\upB G$ 叫做分类空间的原因是, 通过它, 
任何仿紧空间 $X$ 上的 $G$-主丛都可以被完全分类. 

我们将会看到, $\upE G$ 是可缩的. 
因此, $\upB G$ 可以看作同伦意义下的商空间
\[ \upB G \simeq \text{``}\ {*} / G\ \text{''}. \]

\begin{example}
    我们会看到, 任何一个全空间可缩的主丛都是万有的 (\ref{thm-1-contractible}). 因此, 
    \[ \arraycolsep=.16em
    \begin{array}{rcl}
        \upB\bbZ \simeq & \bbR / \bbZ & \simeq S^1, \\
        \upB\bbZ_2 \simeq & S^\infty / \bbZ_2 & \simeq \mathbb{RP}^\infty, \\
        \upB\bbR \simeq & \bbR / \bbR & \simeq \{*\}.
    \end{array} \]
    这里 $S^\infty$ 是无穷维球面, 它是一个可缩的 CW 复形. \varqed
\end{example}

\begin{exercise}
    对任意有限群 $G$, 
    通过 $G$ 的一个有限维自由酉表示, 
    给出 $G$ 在 $S^\infty \subset \bbC^\infty$ 上的一个自由作用, 
    从而给出 $\upB G$ 的构造. \varqed
\end{exercise}

在说明分类空间的存在性之前, 我们先指出关于仿紧空间的两个事实. 

\begin{fact} \label{thm-1-paracompact}
    设 $X$ 是仿紧空间, 设 $\{ U_\alpha \}_{\alpha \in A}$
    是 $X$ 的一个开覆盖.
    \begin{itemize}
        \item 
            存在从属于此覆盖的\term{单位分解},
            即一族紧支连续函数
            \[ \rho_\alpha \: U_\alpha \to [0, 1], \]
            使得它们的支集局部有限, 并且对任意 $x \in X$, 有
            \[ \sum_{\alpha \in A} \rho_\alpha (x) = 1. \]
        \item
            存在 $\{ U_\alpha \}$ 一个开加细 $\{ V_\beta \}$,
            和 $X$ 的一个可数开覆盖 $\{ W_n \}$, 
            使得每个开集 $W_n$ 都是某些开集 $V_\beta$ 的不交并.
    \end{itemize}
\end{fact}

\begin{proof}
    \cite[定理~41.7]{munkres} 和 \cite[引理~5.9]{milnor-stasheff}. 
\end{proof}

下面, 我们通过 Milnor 给出的构造 \cite{milnor}, 说明 $\upB G$ 的存在性. 

\begin{theorem}
    每个拓扑群 $G$ 都有一个分类空间 $\upB G$, 
    并且这个构造关于 $G$ 具有函子性. 
\end{theorem}

\begin{proof}
    设 $\upE G$ 是所有序列
    \[ ( t_1 g_1,\ t_2 g_2,\ \dotsc ) \]
    的集合, 其中
    \begin{itemize}
        \item
            对每个 $i$, 有 $t_i \in [0, 1]$, 和 $g_i \in G$. 
        \item
            这些 $t_i$ 中只有有限个非零, 
            且满足 $\sum_{i = 1}^{\infty} t_i = 1$. 
        \item
            对 $g, h \in G$, 
            形式乘积 $0g$ 和 $0h$ 被认为是相同的. 
    \end{itemize}
    我们赋予 $\upE G$ 作为余极限
    \[ \colim _{n \to \infty} \ \bigl\{
        ( t_1 g_1, \ \dotsc, \ t_n g_n, \ 0, \ 0, \ \dotsc ) \in \upE G
    \bigr\} \]
    的余极限拓扑, 这里余极限的每一项可看作
    $\prod _{i=1} ^{n} (I \times G) / (0 \times G)$ 的子集.

    我们定义 $G$ 在 $\upE G$ 上的右作用如下:
    \[
        ( t_1 g_1,\ t_2 g_2,\ \dotsc ) \cdot g
        = ( t_1 (g_1 g),\ t_2 (g_2 g),\ \dotsc ).
    \]
    这是一个自由作用. 因此, 如果我们定义
    \[ \upB G = \upE G / G, \]
    则 $\upE G$ 是 $\upB G$ 上的 $G$-主丛. 

    我们验证这个主丛是万有的.
    设 $E \to B$ 是任一 $G$-主丛, 其中 $B$ 是仿紧空间. 
    则 $B$ 有一个局部有限的开覆盖 $\{ U_\alpha \}$, 
    使得 $E$ 限制在每个 ${U_\alpha}$ 上都是平凡丛. 
    由 (\ref{thm-1-paracompact}), 
    我们可以通过取这些 $U_\alpha$ 的子集的不交并,
    把开集的个数变成至多可数. 
    我们记之为 $\{ U_1, U_2, \dotsc \}$.

    由 (\ref{thm-1-paracompact}), 
    我们取一组单位分解 $\{ \rho_i \: U_i \to [0, 1] \}$,
    并定义
    \[ \begin{aligned}
        f \: E & \to \upE G, \\
        x & \mapsto ( \rho_1(x) \, g_1(x),\ \rho_2(x) \, g_2(x),\ \dotsc ),
    \end{aligned} \]
    其中 $g_i(x) \in G$ 是点 $x$ 在同胚
    \[ p^{-1}(U_i) \simeq U_i \times G \]
    下对应的 $G$ 方向的坐标.
    这个映射和 $G$ 的作用是相容的, 因此, 我们得到了诱导的映射
    \[ \widetilde{f} \: B \to \upB G, \]
    使得 $E$ 是 $\upE G$ 沿 $\widetilde{f}$ 的拉回.

    我们还需要验证, 映射 $\widetilde{f}$ 的同伦类与开覆盖和单位分解的选取无关.
    我们略过这个细节 (读者不妨自己尝试). 这样, 我们就定义了映射
    \[ \operatorname{Bund}_G (B) \to [B, \upB G], \]
    它是自然的映射 $[B, \upB G] \to \operatorname{Bund}_G (B)$
    的逆映射, 从而是同构.
\end{proof}

\begin{remark}
    这里给出的 $\upE G$ 的构造有一个几何解释. 
    对拓扑空间 $X,Y$, 我们定义它们的\term{连接} (join)
    为把 $X$ 的每个点和 $Y$ 的每个点连一条线段得到的空间. 
    准确地说, 我们定义 
    \[ X \star Y = (X \times Y \times I)/{\sim}, \]
    其中等价关系 $\sim$ 定义为
    \[ (x, y_1, 0) \sim (x, y_2, 0), \quad (x_1, y, 1) \sim (x_2, y, 1). \]

    连接具有结合律, 因为 $X \star Y \star Z$
    就是把 $X$ 的每个点、$Y$ 的每个点和 $Z$ 的每个点连一个实心三角形得到的空间.

    不难验证, 我们给出的构造实际上是一个无穷的连接
    \[ \upE G \simeq G \star G \star G \star \cdots . \varqedhere \]
\end{remark}

\begin{remark}
    对一般的 $\cat{C}$-丛而言, 也有分类空间的构造,
    这一构造就是拓扑群胚的\term{几何实现} (geometric realisation). \varqed
\end{remark}

\begin{exercise}
    验证 $\upE G$ 是可缩的.
    (这与证明 $S^\infty$ 可缩的方法相同.) \varqed
\end{exercise}

对带基点的拓扑空间 $X$, 
记 $\Sigma X$ 和 $\Omega X$
分别为 $X$ 的悬挂 (suspension) 和圈空间 (loop space). 
对 Hausdorff 空间而言, 它们满足伴随关系 $\Sigma \dashv \Omega$. 

\begin{proposition}
    如果 $G$ 是 Hausdorff 拓扑群, 那么有弱同伦等价
    \[ G \simeq \Omega \upB G. \]
    换言之, $\upB G$ 是 $G$ 的\term{消圈} (delooping). 
\end{proposition}

\begin{proof}
    纤维丛 $\upE G \to \upB G$ 诱导了 Puppe 正合列
    \[ 
        \cdots \to \Omega \upE G \to \Omega \upB G
        \to G \to \upE G \to \upB G.
    \]
    因为 $\upE G$ 可缩, 所以 $\Omega\upE G$ 也可缩. 
    因此, 对任一带基点的 CW 复形 $X$, 我们有
    \[ [X, \Omega \upB G] \simeq [X, G]. \qedhere \]
\end{proof}

\begin{proposition} \label{thm-1-contractible}
    如果 $G$ 是 Hausdorff 拓扑群, $E \to B$ 是 $G$-主丛, 
    且 $E$ 弱可缩, 那么有弱同伦等价
    \[ B \simeq \upB G. \]
\end{proposition}

\begin{proof}
    $E$ 对应的映射 $B \to \upB G$ 诱导了两个 $G$-主丛的 Puppe 正合列的映射. 
    作用函子 $[X, {-}]$, 其中 $X$ 是带基点的 CW 复形, 并应用五引理, 得到
    \[ [X, B] \simeq [X, \upB G]. \qedhere \]
\end{proof}

\begin{definition}
    一个关于 $G$-主丛的\term{示性类} (characteristic class) 是一个上同调类
    \[ c \in H^\bullet(\upB G; R), \]
    其中 $R$ 是系数环. 
    对于 $G$-主丛 $E \to X$,
    记 $f \: X \to \upB G$ 为对应的映射, 定义
    \[ c(E) = f^*(c) \in H^\bullet(X; R). \]
\end{definition}

\begin{exercise}
    示性类可以如下等价地定义: 一个示性类 $c$ 是一个自然变换
    \[ c \: \mathrm{Bund}_G(-) \Rightarrow H^\bullet(-; R), \]
    其中两个函子都视为从仿紧空间的范畴到 $\cat{Set}$ 的反变函子. \varqed
\end{exercise}

\subsection{结构群的约化}

设 $X$ 是仿紧空间, $\bbK = \bbR, \bbC, \bbH$. 
我们已经分类了 $X$ 上的向量丛:
\[ \mathrm{Bund}_{ \cat{Vect}(n, \bbK) } (X) \simeq [X, \upB \GL(n, \bbK)]. \]
由线性代数中的 Gram--Schmidt 过程, 
我们知道, $\GL(n, \bbR)$ 可以形变收缩到正交群 $\upO(n)$, 
而 $\GL(n, \bbC)$ 可以形变收缩到酉群 $\upU(n)$. 
同样的道理, $\GL(n, \bbH)$ 可以形变收缩到\term{四元数酉群}
\[ \Sp(n) = \{ A \in \GL(n, \bbH) \mid A^{-1} = (A\text{ 的共轭转置}) \, \} . \]
这里, 四元数 $a + b\upi + c\upj + d\upk$ 的共轭定义为
$a - b\upi - c\upj - d\upk$.

\begin{remark}
    这里的 $\Sp(n)$ 和辛群 $\Sp(n, \bbK)$ 不是同一个东西! 它们的关系是
    \[ \Sp(n) \simeq \Sp(2n, \bbC) \cap \upU(2n). \]
    这个同构通过嵌入 $\bbH \hookrightarrow \mathrm{M}_{2 \times 2}(\bbC)$ 诱导.
    \varqed
\end{remark}

\begin{proposition}
    如果 $f \: G \to H$ 是拓扑群的同态, 且是弱同伦等价, 
    那么 $\upB f \: \upB G \to \upB H$ 也是弱同伦等价. 
\end{proposition}

\begin{proof}
    对 $n\geq1$, 我们有交换图
    \[ \begin{tikzcd}[row sep=large]
        \pi_n (\upB G)
            \ar[d, "\pi_n (\upB f)"']
            \ar[r, "\simeq"] &
        \pi_{n-1} (\Omega \upB G)
            \ar[d, "\pi_{n-1} (\Omega\upB f)"']
            \ar[r, "\simeq"] &
        \pi_{n-1} (G)
            \ar[d, "\simeq", "\pi_{n-1}(f)"'] \\
        \pi_n (\upB H)
            \ar[r, "\simeq"] &
        \pi_{n-1} (\Omega \upB H)
            \ar[r, "\simeq"] &
        \pi_{n-1} (H) \rlap{\ .}
    \end{tikzcd} \]
    因此, $\pi_n (\upB f)$ 是同构. 
    而 $\upB G, \upB H$ 都是道路连通的, 所以 $\pi_0(f)$ 也是同构. 
\end{proof}

\begin{corollary}
    我们有弱同伦等价
    \[ \begin{aligned}[b]
        \upB \GL(n, \bbR) & \simeq \upB \upO(n), \\
        \upB \GL(n, \bbC) & \simeq \upB \upU(n), \\
        \upB \GL(n, \bbH) & \simeq \upB \Sp(n). \\
    \end{aligned} \thmqedhere \]
\end{corollary}

下一节, 我们将说明这些弱同伦等价实际上是同伦等价.
