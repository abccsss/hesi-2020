下面, 我们完成上一节中忽略的关于热核的计算,
然后在上一节的基础上, 引入 Lie 群的作用, 得到计算等变指标的公式.
和通常的指标定理不同, 在等变指标定理中, 我们不需要在整个流形上积分,
只需要在不动点集上积分.
当不动点的个数有限时, 这个积分化为求和,
我们就得到 Atiyah--Bott 的不动点公式, 它是 Lefschetz 不动点公式的推广.


\subsection{热核的计算}

首先, 我们完成 (\ref{thm-9-ahat-ch}) 的证明,
也就是通过热核 $k_t (x, y)$ 当 $t \to 0$ 时在对角线附近的渐进行为,
得到流形的 $\Ahat$ 示性类和向量丛的陈特征.

设 $x_0 \in M$, 取法坐标系 $x = \exp_{x_0} \bfx$, 其中 $\bfx \in U \subset \bbR^n$.
我们使用沿径向的平行移动来将 $E$ 平凡化,
得到 $U$ 上的平凡丛 $E \simeq V \times U$. 我们记
\[ L = D^2 \: C^\infty (U, V) \to C^\infty (U, V) \]
为相应的广义 Laplace 算子, 并记
\[ k (t, \bfx) = k_t (x, x_0) \quad \in \End (V). \]
它满足热方程
\[ (\partial_t + L) \, k (t, \bfx) = 0. \]

证明的关键想法是对 $t$ 和 $\bfx$ 做缩放. 定义
\[ k_\epsilon (t, \bfx) = \epsilon^{n/2} \, k(\epsilon t, \epsilon^{1/2} \bfx). \]
这里, $\bfx$ 的系数 $\epsilon^{1/2}$ 出现的原因是,
热方程对 $\bfx$ 求二阶导数, 而只对 $t$ 求一阶导数;
系数 $\epsilon^{n/2}$ 是为了保证 $t \to 0$ 时 
$k_{\epsilon} (t, \bfx) \to \delta (\bfx)$.
例如, 欧氏空间中 $\Delta$ 算子的热核 $q (t, \bfx)$ 在这个缩放下是不变的.

我们还引入 Clifford 代数上的缩放.
设 $V \simeq S \otimes W$, 并设
\[ \alpha (t, -) \in C^\infty (U, \End (V)) \simeq 
    \Gamma \bigl( \Cl (\bbR^n, g) \otimes \End (W) \bigr), \]
其中 $g = (g^{ij})$ 是 $M$ 的度量.
我们定义缩放算子 $\delta_\epsilon$ 如下:
\[ \delta_\epsilon \alpha (t, \bfx) =
    \sum_{p=0}^n \epsilon^{-p/2} \, \alpha_p (\epsilon t, \epsilon^{1/2} \bfx), \]
其中 $\alpha_p$ 是在 Clifford 代数中分次为 $p$ 的部分.

\begin{definition}
    设 $\epsilon > 0$. 我们定义\term{经过缩放的热核}
    \[ r_\epsilon (t, \bfx) = \epsilon^{n/2} \, \delta_\epsilon k (t, \bfx), \]
    它满足的热方程是
    \[ (\partial_t + L_\epsilon) \, r_\epsilon (t, \bfx) = 0, \]
    其中 $L_\epsilon = \epsilon \, \delta_\epsilon L \delta_\epsilon^{-1}$.
\end{definition}

这里, $L_\epsilon$ 的表达式可以从恒等式
$\partial_t \delta_\epsilon = \epsilon \, \delta_\epsilon \partial_t$ 得出.

\begin{proposition} \label{thm-10-lim-l-epsilon}
    作为 $\Omega^\bullet (U, \End (W)) \simeq C^\infty (U, \End (V))$
    上的算子, 我们有
    \[ \lim_{\epsilon \to 0} L_\epsilon = K
        = -\sum_i \Bigl( \partial_i - \frac14 \bfx^j \Omega_{ij} (0) \wedge {} \Bigr)^2
            + \Omega^W (0) \wedge {}, \]
    这里极限的意义是 $L_\epsilon = K + O(\epsilon^{1/2})$.
\end{proposition}

\begin{proof} [证明概要]
    记 $c^i$ 为 Clifford 作用. 我们有 \term{Lichnerowicz 公式} \cite[\S3.5]{bgv}
    \[ D^2 = \Delta^E + \frac12 \Omega^W (e_i, e_j) \, c^i c^j + \frac14 R, \]
    其中 $R$ 是标量曲率. 事实上,
    这个公式就是通过 Dirac 算子与 Clifford 联络的关系, 来计算 $D^2$ 得到的结果.
    它是 Weitzenböck 恒等式的一个推广形式.
    
    由这一表达式, 我们能立即得出 $L_\epsilon$ 的表达式.
    记 $\nabla^{\epsilon} = 
    \epsilon^{1/2} \delta_\epsilon \nabla^E \delta_\epsilon^{-1}$, 则
    \begin{multline*}
        L_\epsilon = -\sum_i \bigl( (\nabla_i^{\epsilon})^2 
        - \epsilon^{1/2} \nabla_{\nabla_i e_i}^{\epsilon} \bigr) \\
        + \frac12 \Omega^W (\epsilon^{1/2} \bfx) (e_i, e_j) \, 
        (e^i - \epsilon \iota^i) \, (e^j - \epsilon \iota^j)
        + \frac14 \epsilon R (\epsilon^{1/2} \bfx).
    \end{multline*}
    另外, 计算表明
    \[ \lim_{\epsilon \to 0} \nabla_i^{\epsilon} = \partial_i -
        \frac14 \bfx^j \Omega_{ij} (0) \wedge {}. \]
    现在, 我们可以直接计算 $\lim_{\epsilon \to 0} L_\epsilon$ 了.
    在 $L_\epsilon$ 的表达式中, 带 $\epsilon$ 的项通通消失,
    剩下的项就给出了要证的公式.
\end{proof}

我们回忆热核 $k_t$ 的渐进展开式 (\ref{thm-7-asymp-exp}).
它给出了热核 $r_\epsilon$ 的渐进展开
\begin{align*}
    r_\epsilon (t, \bfx)_p
    & \sim \epsilon^{-p/2} q (t, \bfx) \, \sum_{i=0}^\infty {} 
        (\epsilon t)^i \, \Phi_i (\epsilon^{1/2} \bfx)_p \\
    & = q (t, \bfx) \, \sum_{i=-n}^\infty {}
        \epsilon^{i/2} \, \gamma_i (t, \bfx)_p,
\end{align*}
其中下标 $p$ 表示在 Clifford 代数中分次为 $p$ 的部分,
$\gamma_i (t, \bfx)_p$ 是由这一等式定义的, 即定义为式中 $\epsilon^{i/2}$ 的系数.
对 $p$ 求和, 得到
\begin{equation} \label{eq-10-r-and-gamma}
    r_\epsilon (t, \bfx) \sim q (t, \bfx) \, \sum_{i=-n}^\infty {}
        \epsilon^{i/2} \, \gamma_i (t, \bfx),
\end{equation}
其中, 我们把 $\gamma_i$ 看成微分形式
\[ \gamma_i (t, -) \in C^\infty (U, \End (V)) 
    \simeq \Omega^\bullet (U, \End (W)). \]

注意到
\[ \gamma_i (0, \bfx) \equiv \biggl \{ \begin{array}{ll}
    1, & i = 0, \\
    0, & i \neq 0,
\end{array} \]
因为 $t = 0$ 时只有 $\Phi_0 (\bfx) = \mathrm{id}_V$ 一项有贡献.
我们对 (\ref{eq-10-r-and-gamma}) 作用算子 $\partial_t + L_\epsilon$.
则由 (\ref{thm-10-lim-l-epsilon}),
若 $\gamma_i$ 是 $i$ 最小的非零项, 那么它应满足
\[ (\partial_t + K) \, \gamma_i (t, \bfx) = 0. \]
因此, 当 $i < 0$ 时必有 $\gamma_i \equiv 0$, 因为若不然,
则使 $i$ 最小的那个 $\gamma_i$ 满足以 $0$ 为初值的热方程, 从而恒为 $0$, 矛盾.
作为推论, 我们有
\[ (\partial_t + K) \, \gamma_0 (t, \bfx) = 0. \]

\begin{proposition} \label{thm-10-lim-r-epsilon}
    我们有
    \begin{multline*}
        \lim_{\epsilon \to 0} r_\epsilon (t, \bfx) = \\
        \frac {1} {(4 \uppi t)^{n/2}} 
        \det^{1/2} \biggl( \frac {t \Omega / 2} {\sinh (t \Omega / 2)} \biggr)
        \exp \biggl( - \frac {1} {4t} 
        \Bigl( \frac{t \Omega}{2} \coth \frac{t \Omega}{2} \Bigr) (\bfx, \bfx)
        - t \Omega^W \biggr).
    \end{multline*}
\end{proposition}

\begin{proof}
    事实上,
    \[ \lim_{\epsilon \to 0} r_\epsilon (t, \bfx) = 
        q (t, \bfx) \, \gamma_0 (t, \bfx). \]
    因此, 剩下的工作就是解出热方程
    \[ (\partial_t + K) \, \gamma_0 (t, \bfx) = 0. \]
    实际上, 我们只需验证命题中的表达式确实给出了热方程的解.
    我们就不把计算过程写出来了, 读者可参见 \cite[\S4.2]{bgv}.
\end{proof}

\begin{proof} [定理 \ref{thm-9-ahat-ch} 的证明]
    在 (\ref{thm-10-lim-r-epsilon}) 中取 $t = 1$, $\bfx = 0$, 就得到了我们要的表达式.
\end{proof}


\subsection{等变指标}

接下来, 我们进入这一节的主题, 即等变指标定理.

设 $M$ 是偶数维可定向 Riemann 流形, $H$ 是紧拓扑群, 它作用在 $M$ 上.
我们要求这个作用保持 $M$ 的度量和定向.

设 $E \to M$ 是\term{等变 Clifford 模}, 带有一个 Hermite 度量.
也就是说, $E$ 也带有 $H$ 的作用,
每个元素 $\gamma \in H$ 把纤维 $E_x$ 映到 $E_{\gamma (x)}$.
我们要求这个作用保持 Clifford 作用, 也保持 $E$ 的度量.

\begin{proposition}
    在上述情况下, $H$-等变的 Dirac 算子与 $H$-等变的 Clifford 超联络一一对应. \qed
\end{proposition}

设 $D$ 是 $H$-等变的 Dirac 算子.
在以前, 我们把 $\bbZ_2$-分次空间 $\ker D$ 的维数定义成 $D$ 的指标.
但现在, $\ker D$ 是 $H$ 的一个 $\bbZ_2$-分次的酉表示.
这里, 元素 $\gamma \in H$ 作用在截面 $s \in \Gamma (E)$ 上的方式是
\[ (\gamma \cdot s) (x) = \gamma^E \, s (\gamma^{-1} x), \]
其中 $\gamma^E \: E_{\gamma^{-1} x} \to E_x$ 是 $\gamma$ 在 $E$ 上的作用.

\begin{definition}
    设 $\gamma \in H$. 我们定义\term{等变指标} (equivariant index)
    \[ \ind (\gamma, D) = \trs (\gamma |_{\ker D}). \]
\end{definition}

注意到 $\ind (1, D) = \ind (D)$.
和之前的情况类似, 我们通过算子 $D^2$ 的谱理论, 得到等变指标的 McKean--Singer 公式.

\begin{theorem}[McKean--Singer] \label{thm-10-equivar-mckean-singer}
    我们有
    \[ \ind (\gamma, D) = \trs (\gamma \, \upe^{-t D^2})
        = \int_M \trs k_{t, \gamma} (x, x) \, dx, \]
    其中 $k_{t, \gamma}$ 是算子 $\gamma \, \upe^{-t D^2}$ 的核. \qed
\end{theorem}

\begin{proof}
    和 (\ref{thm-9-mckean-singer}) 的证明一样,
    设 $V_\lambda = V_\lambda^+ \oplus V_\lambda^-$ 是算子 $D^2$ 的 $\lambda$-特征子空间.
    因为 $D$ 是 $\gamma$-等变的,
    所以 $V_\lambda^{\pm}$ 都是 $\gamma$ 的不变子空间.
    并且, 当 $\lambda \neq 0$ 时, 同构
    \[ D \: V_\lambda^+ \to V_\lambda^- \]
    保持 $\gamma$ 的作用. 这说明 $\trs (\gamma \, \upe^{-t D^2}|_{V_\lambda}) = 0$.
    我们就得到了第一个等号.
    
    第二个等号的证明和 (\ref{thm-9-mckean-singer}) 的证明同理.
\end{proof}


\subsection{Atiyah--Bott 不动点公式}

Atiyah--Bott 不动点公式是等变指标定理的一个特例,
但放宽了一些假设.
它也是 Lefschetz 不动点公式的一个推广.

下面, 我们设 $M$ 是紧可定向流形, 拓扑群 $H$ 作用在 $M$ 上.
我们不要求 $H$ 是紧的, 也不要求其作用保持度量.
设 $E \to M$ 是 $H$-等变复超向量丛, 
$d$ 是 $E$ 上的 $H$-等变微分算子, 满足
\begin{itemize}
    \item 
        $d^2 = 0$.
    \item
        上同调 $H (d) = H \bigl( \Gamma (E), d \bigr)$ 是有限维超向量空间.
    \item
        存在 $M$ 上的 Riemann 度量和 $E$ 上的 Hermite 度量 (不必等变),
        使得 $D = d + d^*$ 是 Dirac 算子.
\end{itemize}
对 $\gamma \in H$, 我们定义
\[ \ind (\gamma, d) = \trs \Bigl( H(\gamma) \: H(d) \to H(d) \Bigr). \]

\begin{theorem} [Atiyah--Bott 不动点公式] \label{thm-10-atiyah-bott-fixed}
    假设 $\gamma \in H$ 在 $M$ 上的作用具有离散的不动点集.
    则当下面和式中每一项的分母都不为零时, 有
    \[ \ind (\gamma, d) = \sum_{x: \text{不动点}}
        \frac {\trs \gamma_x^E} { \abs{\det (1 - \gamma_x^{-1})} }, \]
    其中 $\gamma_x \: T_x M \to T_x M$ 和 $\gamma_x^E \: E_x \to E_x$
    都是 $\gamma$ 的作用.
\end{theorem}

在证明定理之前, 我们先推出 Lefschetz 不动点公式的一个特殊情形.

\begin{corollary} [Lefschetz 不动点公式]
    在上述假设下,
    \[ \sum_{x: \text{不动点}} \pm1 = \sum_{i=0}^n (-1)^i \tr H^i(\gamma), \]
    其中 $H^i (\gamma) \: H^i (M) \to H^i (M)$ 是 $\gamma$ 诱导的映射,
    式中的 $\pm1$ 等于 $\operatorname{sgn} \det (1 - \gamma_x^{-1})$.
\end{corollary}

\begin{proof}
    取 $E = \wedge^\bullet \, T_x^* M \otimes \bbC$,
    则等式右边就是 $\ind (\gamma, d)$, 其中 $d$ 是 de~Rham 微分.
    我们再来计算 $\trs \gamma_x^E$. 此时 $E_x \simeq \wedge^\bullet \, T_x^* M$.
    通过考虑 $\gamma_x$ 的特征值, 不难证明
    \[ \trs \gamma_x^E = \det (1 - \gamma_x^{-1}). \qedhere \]
\end{proof}

\begin{proof} [定理 \ref{thm-10-atiyah-bott-fixed} 的证明]
    设 $x_1, \dotsc, x_m$ 是所有的不动点.
    由 (\ref{thm-10-equivar-mckean-singer}), 我们有
    \[ \ind (\gamma, d) = \int_M \trs k_{t, \gamma} (x, x) \, dx
        = \sum_{i=0}^m \int_M \psi_i (x) \trs k_{t, \gamma} (x, x) \, dx, \]
    其中函数 $\psi_i$ 取得使 $\sum_i \psi_i = 1$,
    并且当 $i > 0$ 时, $\psi_i$ 在 $x_i$ 附近恒等于 $1$.
    我们想要证明
    \[ \lim_{t \to 0} \psi_i (x) \, k_{t, \gamma} (x, x) = \left \{ \begin{array}{ll}
        \dfrac {\gamma_{x_i}^E} { \abs{\det(1 - \gamma_{x_i}^{-1})} } \,
            \delta_{x_i}, & i > 0, \\
        0, & i = 0,
    \end{array} \right. \]
    然后积分并取迹, 就得到了要证的等式.
    
    事实上, 热核 $k_{t, \gamma}$ 是能算出来的:
    \[ k_{t, \gamma} (x, y) = \gamma \cdot k_t (\gamma^{-1} x, y), \]
    其中 $\cdot$ 表示 $\gamma$ 在截面上的作用.
    设 $\phi \in \Gamma (E)$ 是测试函数.
    记 $V = T_{x_i} M$, 并将 $0 \in V$ 的一个邻域通过法坐标系与
    $x_i$ 的邻域等同起来, 以便计算积分. 则
    \begin{align*}
        & \lim_{t \to 0} \int_M \psi_i (x) \, k_{t, \gamma} (x, x) \, \phi (x) \, dx \\
        ={} & \lim_{t \to 0} \frac {1} {(4 \uppi t)^{n/2}}
        \int_V \upe^{-|\gamma_{x_i}^{-1} \bfx - \bfx|^2 / 4t} \, 
        \gamma \cdot \phi (\bfx) \, \bigl( 1 + O(\bfx) \bigr) \, d \bfx \\
        ={} & \frac {\gamma_{x_i}^E} {\abs{\det(1 - \gamma_{x_i}^{-1})}} \, \phi (0),
    \end{align*}
    其中最后一步是初等计算.
    % 而行列式 $\det(1 - \gamma_{x_i}^{-1})$ 总是正的,
    % 因为若不然, 则 $\gamma_{x_i}^{-1}$ 必有大于 $1$ 的实特征值,
    % 但由于 $\gamma$ 的作用保持度量, 这是不可能发生的.
    这一计算也说明 $\lim_{t \to 0} \psi_0 (x) \, k_{t, \gamma} (x, x) = 0$.
\end{proof}


\subsection{等变指标定理}

在上一小节中, 我们已经知道, 算子的指标可以通过不动点处几何量的求和来计算.
在等变指标定理中, 我们不再要求不动点是离散的,
并把算子的指标写成不动点集上几何量的积分.

下面, 设 $M$ 是偶数维紧可定向 Riemann 流形, 紧 Lie 群 $H$ 作用在 $M$ 上,
并保持其度量和定向. 对 $\gamma \in H$, 记 
\[ M^\gamma = \{ x \mid \gamma x = x \} \subset M. \]
由 Lie 群的理论, $M^\gamma$ 是 $M$ 的子流形,
但需要注意的是, 它可能是不同维数的子流形的不交并.

在 $M^\gamma$ 上, 我们有
\[ TM |_{M^\gamma} = TM^\gamma \oplus N, \]
其中 $N$ 是法丛. 因为 $\gamma$ 的作用是等距同构,
所以 $M^\gamma$ 是全测地子流形, 从而 $TM$ 的 Levi-Civita 联络
分解成两个子丛上的联络的直和, 我们记为
\[ \nabla = \nabla^0 \oplus \nabla^N, \]
其中 $\nabla^0$ 是 $TM^\gamma$ 上的 Levi-Civita 联络.

\begin{definition}
    若 $W \to M$ 是 $H$-等变的超向量丛, 我们定义\term{等变陈特征}
    \[ \ch (\gamma, W) = \trs^W \bigl( \gamma \cdot \upe^{-\Omega^W} \bigr)
        \quad \in \Omega^\bullet (M^\gamma), \]
    其中 $\Omega^W$ 是 $W$ 上的一个 $H$-等变的联络的曲率形式.
\end{definition}

\begin{theorem} [等变指标定理, Atiyah--Segal--Singer] \label{thm-10-equivar-index-thm}
    等变 Dirac 算子 $D$ 的指标等于
    \[ \ind (\gamma, D) = \int_{M^\gamma} 
        \frac {(-2 \upi)^{n_1 / 2}} {(2 \uppi \upi)^{n_0 / 2}} \, T_M \left( \frac
            {\Ahat (M^\gamma) \trs^W \bigl( (\gamma^E)_{n_1} \, \upe^{-\Omega^W} \bigr)}
            {\det^{1/2} (1 - \gamma^N) \det^{1/2} (1 - \gamma^N \upe^{-\Omega^N})}
        \right) \, dx, \]
    其中 $n_0 (x) = \dim_x M^\gamma$, \ $n_1 = n - n_0$, \ $T_M$ 是 Березин 积分,
    $(\gamma^E)_{n_1}$ 的下标表示 Clifford 分次.
    
    如果 $M$ 具有 $H$-等变的旋量结构, 并且 $M^\gamma$ 也具有旋量结构,
    那么这个公式可以简化成
    \[ \ind (\gamma, D) = \int_{M^\gamma} 
        \frac {(-1)^{n_1 / 2}} {(2 \uppi \upi)^{n_0 / 2}} \, T_M \left( \frac
            {\Ahat (M^\gamma) \ch (\gamma, W)}
            {\ch (\gamma, S(N))}
        \right) \, dx, \]
    其中, 旋量丛 $S(N)$ 定义为
    \[ S(N) = \Hom_{\Cl(M^\gamma)} \bigl( S (M^\gamma), S (M) |_{M^\gamma} \bigr). \]
\end{theorem}

定理的第一个公式看起来很吓人,
但它其实就是在第二个公式中, 把陈特征的计算公式写开来的结果.
另外, 如果我们在第二个公式中取 $\gamma = 1$,
我们就得到了非等变情形的指标定理 (\ref{thm-9-atiyah-singer}).

\begin{proof} [证明概要]
    我们在 (\ref{thm-10-equivar-mckean-singer}) 中已经知道,
    \[ \ind (\gamma, D) = \int_M \trs k_{t, \gamma} (x, x) \, dx. \]
    和之前的证明方法类似, 我们把等变热核在对角线上做渐进展开:
    \[ k_{t, \gamma} (x, x) \sim \frac {1} {(4 \uppi t)^{n_0/2}}
        \sum_{i=0}^\infty t^i \Phi_{i, \gamma} (x). \]
    这里, 系数中 $t$ 的次数是 $-n_0 / 2$,
    因为热方程实际上只沿着 $M^\gamma$ 的方向进行演化.
    注意, 这里的 $\Phi_{i, \gamma}$ 都是分布截面, 即允许例如 $\delta$-函数出现.
    
    为了研究这些 $\Phi_{i, \gamma}$, 我们固定一个 $x_0 \in M^\gamma$,
    并取法坐标系 $x = \exp_{x_0} \bfx$.
    这个指数映射诱导了 $x_0$ 的邻域 $V$ 到 $0$ 的邻域 $V'$ 的微分同胚.
    设 $\phi$ 是 $M$ 上的光滑函数, 其支集包含于 $V$.
    我们做渐进展开
    \[ \int_{N_{x_0}} k_{t, \gamma} (\bfx, \bfx) \, \phi (\bfx) \, d \bfx
        \sim \frac {1} {(4 \uppi t)^{n_0/2}}
        \sum_{i=0}^\infty t^i \Phi_{i, \gamma, \phi} (x_0). \]
    通过对等变热核的计算 \cite[\S\S6.5--6.7]{bgv}, 可以得到
    \begin{equation} \label{eq-10-equivar-asymp-exp}
        \Phi_{i, \gamma, \phi} (x_0)_{p} = \biggl\{ \begin{array}{ll}
            I_\gamma (x_0)_p \, \phi (x_0), & p = 2i + n_1, \\
            0 & p > 2i + n_1,
        \end{array}
    \end{equation}
    其中下标 $p$ 表示 Clifford 分次, 
    $I_\gamma\in \Gamma \bigl( M^\gamma, \ 
        {\wedge^\bullet \, T^* M} \otimes \End_{\Cl (M)} (E) \bigr)$ 定义为
    \[ I_\gamma = \frac
        {\Ahat (M^\gamma) \, (\gamma^E)_{n_1} \, \upe^{-\Omega^W}}
        {\det^{1/2} (1 - \gamma^N) \det^{1/2} (1 - \gamma^N \upe^{-\Omega^N})}. \]
    事实上, 当 $\dim M^\gamma = 0$ 时,
    这个等式的本质就是我们在证明 (\ref{thm-10-atiyah-bott-fixed}) 时计算的 Gauß 积分.
    
    这一计算结果蕴涵了
    \[ \Phi_{i, \gamma} (x_0) = I_\gamma \cdot \delta_{M^\gamma}, \]
    其中 $\delta$-函数 $\delta_{M^\gamma}$ 作用在测试函数 $\phi$ 上,
    给出的是 $\phi$ 在 $M^\gamma$ 上的积分.
    
    我们在 $M$ 上取一单位分解, 将热核的积分化为局部的计算:
    \[ \int_M \trs k_{t, \gamma} (x, x) \, dx \sim
        \int_U d x_0 \int_{N_{x_0}}
        \trs k_{t, \gamma} (\bfx, \bfx) \, \psi (\bfx) \, b_{x_0} (\bfx) \, d \bfx, \]
    其中我们省略了求和号, $U \subset M^\gamma$ 是开集,
    $\psi$ 是单位分解, $b_{x_0} (\bfx)$ 是坐标变换的系数.
    令 $\phi (\bfx) = \psi (\bfx) \, b_{x_0} (\bfx)$,
    则上述积分关于 $t$ 的渐进展开是
    \[ \frac {1} {(4 \uppi t)^{n_0/2}} \sum_{i=0}^\infty t^i
        \int_U \trs \Phi_{i, \gamma, \phi} (x_0)_n \, dx_0, \]
    这里, 我们可以只考虑分次 $n$ 的项, 因为只有这些项能有非零的超迹.
    令 $t \to 0$, 由 (\ref{eq-10-equivar-asymp-exp}), 我们得到
    \begin{align*}
        \lim_{t \to 0} \int_M \trs k_{t, \gamma} (x, x) \, dx
        &= \frac {1} {(4 \uppi)^{n_0 / 2}}
            \int_{M^\gamma} \trs \Phi_{n_0 / 2, \gamma, \phi} \, dx \\
        &= \frac {(-2 \upi)^{n/2}} {(4 \uppi)^{n_0 / 2}} \int_{M^\gamma}
            T_M \bigl( \trs^W I_\gamma \bigr) \, dx,
    \end{align*}
    其中最后一步利用了 (\ref{eq-9-tre-trw}).
    我们就证明了定理的第一个公式.
    
    为了证明第二个公式, 我们先来计算 $\ch (\gamma, W)$.
    因为对任意 $\alpha \in N_x^\vee$, 都有
    $\gamma^S \, c(\alpha) = c(\gamma^N \alpha) \, \gamma^S$,
    所以 $\gamma^S$ 作为 $\Cl (M)$ 的截面, 实际上是 $\Cl (N^\vee)$ 的截面.
    由于 $\gamma^E = \gamma^W \otimes \gamma^S$, 我们有
    \[ (\gamma^E)_{n_1} = T (\gamma^S) \, \gamma^W, \]
    其中 $T$ 是 $\Cl (N^\vee)$ 上的 Березин 积分. 
    对 $\alpha \in \Spin (2k) \subset \Cl (2k)$, 有如下公式:
    \begin{equation} \label{eq-10-berezin-half-det}
        T (\alpha) = \pm \det^{1/2} (1 - \tau (\alpha)) / 2^k,
    \end{equation}
    其中映射 $\tau$ 是到 $\SO (2k)$ 的覆叠.
    这个公式不难证明. 事实上, 只需验证 $k = 1$ 的情况即可.
    在式中取 $\alpha = \gamma^S$, 则 $\tau (\alpha) = \gamma^N$. 从而
    \begin{align*}
        \ch (\gamma, W)
        &= \frac {1} {T(\gamma^S)} 
            \trs^W \bigl( (\gamma^E)_{n_1} \, \upe^{-\Omega^W} \bigr)\\
        &= \pm 2^{n_1 / 2} \, \frac
            {\trs^W \bigl( (\gamma^E)_{n_1} \, \upe^{-\Omega^W} \bigr)}
            {\det^{1/2} (1 - \gamma^N)}.
    \end{align*}
    
    再来考虑 $S(N)$ 的陈特征. 可以证明, 对 $S(N)$ 来说,
    (\ref{eq-9-tre-trw}) 和 (\ref{eq-10-berezin-half-det})
    的相应版本也成立, 从而
    \begin{align*}
        \ch (\gamma, S(N))
        &= (-2 \upi)^{n_1 / 2} \,
            T \bigl( \gamma^{S(N)} \cdot \upe^{-\Omega^{S(N)}} \bigr) \\
        &= \pm (-\upi)^{n_1 / 2} \det^{1/2} (1 - \gamma^N \upe^{-\Omega^N}).
    \end{align*}
    通过更仔细的计算 \cite[\S6.4]{bgv}, 可知两处正负号是相同的.
    我们就导出了第二个公式.
\end{proof}

