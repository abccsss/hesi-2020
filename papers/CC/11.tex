从这一节开始, 我们介绍等变上同调的理论.
我们首先从代数拓扑的角度出发, 建立这一理论,
然后把流形的等变上同调也用微分形式表示出来.
这个理论的一个有名的结论是 Atiyah--Bott 局部化公式,
这个公式把等变微分形式的积分化为不动点集上的积分.


\subsection{等变上同调}

设 $X$ 是拓扑空间, 拓扑群 $G$ 左作用在 $X$ 上.
我们想要研究商空间 $X / G$, 也就是所有轨道构成的空间.

一般来说, 这个空间不具有好的性质.
但如果这个作用是\term{自由作用}, 也就是说, 它的轨道都同构于 $G$,
那么映射 $X \to X / G$ 就成为一个 $G$-主丛.
当 $G$ 是紧 Lie 群, $X$ 是紧流形时,
Lie 理论的一个著名定理说明, 商空间 $X / G$ 也是流形.

代数拓扑中的一个重要的方法是做替换, 以使坏的对象变成好的对象.
我们可以将一端有界的链复形替换成拟同构的投射或内射链复形 (即消解),
也可以将任何的空间替换成弱同伦等价的 CW 复形 (即 CW 逼近).
同样的方法可以应用到带群作用的空间上.

\begin{definition}
    定义\term{同伦商} (homotopy quotient)
    \[ X \hoq G = X' / G, \]
    其中 $X'$ 是满足以下条件的空间:
    \begin{itemize}
        \item
            $G$ 自由地作用在 $X'$ 上, 从而 $X' \to X'/G$ 是 $G$-主丛.
        \item
            存在 $G$-等变的同伦等价 $X' \simeq X$.
    \end{itemize}
\end{definition}

例如, 我们可以取
\[ X' = X \times \upE G, \]
并令 $G$ 分别作用在两个分量上.

可以证明, 同伦商在同伦等价的意义下是唯一的.

\begin{definition}
    设 $R$ 是系数环. 则 $X$ 的\term{等变上同调} (equivariant cohomology) 定义为
    \[ H_G^\bullet (X; R) = H^\bullet (X \hoq G; R)
        \simeq H^\bullet \bigl( (X \times \upE G) / G; R \bigr). \]
\end{definition}

例如, 我们有
\begin{align*}
    H_G^\bullet (G; R) & \simeq H^\bullet (\{*\}; R), \\
    H_G^\bullet (\{*\}; R) & \simeq H^\bullet (\upB G; R).
\end{align*}

下面, 再设 $E \to X$ 是 $G$-等变向量丛.
也就是说, $G$ 作用在全空间 $E$ 上,
元素 $g \in G$ 把纤维 $E_x$ 映到 $E_{g(x)}$, 且这个映射是线性的.

我们可以把 $E$ 拉回到 $X'$ 上, 得到 $G$-等变向量丛 $E' \to X'$.
例如, 若取 $X' = X \times \upE G$, 则 $E' \simeq E \times \upE G$.

我们取同伦商, 得到向量丛
\[ \widetilde{E} = E \hoq G \to X \hoq G. \]

\begin{definition}
    等变向量丛 $E \to X$ 的\term{等变示性类} (equivariant characteristic class)
    就是向量丛 $\widetilde{E}$ 的示性类, 它是 $H_G^\bullet (X; R)$ 的元素.
\end{definition}


\subsection{等变微分形式}

接下来, 我们打算用微分形式来描述流形的等变上同调.
事实上, 这不是一件容易的事, 因为对 Lie 群 $G$ 来说,
空间 $\upE G$ 通常不是流形, 所以我们需要合理地实现
``$\upE G$ 上的微分形式'' 这个不存在的概念.

设 $G$ 是紧 Lie 群, 作用在紧流形 $M$ 上. 设 $\frg$ 是 $G$ 的 Lie 代数.
每个元素 $X \in \frg$ 可以看作 $M$ 上的向量场.
这个操作定义了 Lie 代数同态
\[ \frg \to \bigl( \Gamma (TM), \ [ \ , \ ] \bigr). \]

另外, $G$ 还可以作用在 $\Omega^\bullet (M)$ 上, 每个元素 $g \in G$ 的作用是
\[ ( g^{-1} )^* \: \Omega^\bullet (M) \to \Omega^\bullet (M). \]
然而, 如果我们直接取在这一作用下不变的元素的集合
$\Omega^\bullet (M)^G \subset \Omega^\bullet (M)$,
并考虑它的上同调, 我们并不能得到等变上同调.
例如, 当 $M = \{*\}$ 时, 我们希望得到 $H^\bullet (\upB G)$,
但这个链复形仍然给出 $H^\bullet (\{*\})$.

因此, 我们的目标是定义 ``de~Rham 链复形''
\[ \text{``} \Omega^\bullet (M \times \upE G) \text{''} \quad \text{和} \quad
    \text{``} \Omega^\bullet \biggl( \frac {M \times \upE G} {G} \biggr) \text{''}. \]

即使 $G$ 在 $M$ 上的作用是自由的,
复形 $\Omega^\bullet (M/G)$ 也不等于 $\Omega^\bullet (M)^G$.
例如, 我们考虑 $M = \bbR^2 = \bbR x \oplus \bbR y$,
并令 $G = \bbR$ 通过 $x$ 方向的平移作用在 $M$ 上. 则
\begin{align*}
    \Omega^1 (M)^G &= \{ \, f(y) \, dx + g(y) \, dy \, \}, \\
    \Omega^1 (M/G) &= \{ \, g(y) \, dy \, \}.
\end{align*}
我们真正感兴趣的是第二种微分形式,
也就是下面定义中的 $\Omega^\bullet (M)_{\mathrm{bas}}$.

\begin{definition}
    微分形式 $\alpha \in \Omega^\bullet (M)$ 称为\term{水平} (horizontal) 的,
    如果对任意 $X \in \frg$, 都有 $\iota_X \alpha = 0$. 定义
    \begin{align*}
        \Omega^\bullet (M)_{\mathrm{hor}} &= \{ \text{水平的形式} \}, \\
        \Omega^\bullet (M)_{\mathrm{bas}} &= \Omega^\bullet (M)_{\mathrm{hor}}^G ,
    \end{align*}
    其中, 第二个空间中的微分形式叫做\term{基础} (basic) 的.
\end{definition}

我们还需要描述什么是 ``$\Omega^\bullet (\upE G)$''.
这一描述来自于 Lie 群的陈--Weil 理论.
我们回忆, 示性类对应着曲率形式的不变多项式. 对 Lie 群而言, 这个对应关系意味着
\[ H^\bullet (\upB G; \bbC) \simeq \bbC [\frg]^G, \]
这里 $\bbC [\frg] \simeq \operatorname{Sym}^\bullet \frg^\vee$
是 $(\dim \frg)$-元多项式环,
而 $G$ 通过伴随作用
\[ \operatorname{Ad}_g (X) = g X g^{-1} \quad (g \in G, \ X \in \frg) \]
作用在这个多项式环上.
也就是说, 我们把 $\bbC [\frg]$ 的元素看成是万有丛
$\upE G \to \upB G$ 的 ``曲率形式'' 的多项式.

\begin{definition}
    Lie 代数 $\frg$ 的 \term{Weyl 代数}定义为交换超代数
    \[ W^\bullet (\frg) = \operatorname{Sym}^\bullet \frg^\vee
        \otimes {\wedge^\bullet \, \frg^\vee}, \]
    其中, 在第一项中 $\frg^\vee$ 的分次为 $2$,
    在第二项中 $\frg^\vee$ 的分次为 $1$.
\end{definition}

这个定义的意义如下.
设 $X_1, \dotsc, X_n$ 是 $\frg$ 的一组基,
则 Weyl 代数的定义式中,
两个 $\frg^\vee$ 的生成元分别记为 $\Omega^i$ 和 $\omega^i$. 元素
\[ \begin{aligned}
    \omega &= \textstyle \sum_i \omega^i \otimes X_i
        && \in W^1 (\frg) \otimes \frg, \\
    \Omega &= \textstyle \sum_i \Omega^i \otimes X_i
        && \in W^2 (\frg) \otimes \frg,
\end{aligned} \]
分别称为\term{万有联络} (universal connection)
和\term{万有曲率} (universal curvature).
事实上, 对任意的 $G$-主丛 $E \to B$, 若选好一个联络,
则存在唯一的映射
\[ W^\bullet (\frg) \to \Omega^\bullet (E), \]
使得 $\omega$ 和 $\Omega$ 在
$\Omega^\bullet (E, \frg) \simeq \Omega^\bullet (E) \otimes \frg$
中的像就是 $E$ 的联络和曲率形式.
(注意, 主丛的联络形式是能在整体上定义的, 这与向量丛的联络形式不同.)

这个万有性质说明, Weyl 代数 $W^\bullet (\frg)$
就是我们要找的 ``$\Omega^\bullet (\upE G)$''.

为了真正地将这两者等同起来, 我们还需要在
$W^\bullet (\frg)$ 上定义外微分和缩并.
我们定义
\begin{align*}
    \iota_j \omega^i &= \delta_j^i, 
        & d \omega^i &= \Omega^i - \frac12 c^i_{jk} \, \omega^j \wedge \omega^k, \\
    \iota_j \Omega^i &= 0, 
        & d \Omega^i &= c^i_{jk} \Omega^j \wedge \omega^k,
\end{align*}
其中 $c^i_{jk}$ 是 $\frg$ 的结构常数.
这么定义是为了满足 $\Omega = d \omega + (1/2) [\omega, \omega]$,
以及 Bianchi 恒等式 $d \Omega = [\Omega, \omega]$. 
这样, 我们就有
\[ W^\bullet (\frg)_{\mathrm{hor}} = \bbC [\frg], \quad
    W^\bullet (\frg)_{\mathrm{bas}} = \bbC [\frg]^G
    \simeq H^\bullet (\upB G) . \]

\begin{definition}
    $M$ 上的\term{等变微分形式} (equivariant differential form) 是指
    \[ \Omega_G^\bullet (M) = \bigl(
        \Omega^\bullet (M) \otimes W^\bullet (\frg)
        \bigr)_{\mathrm{bas}} \]
    的元素. 这个链复形的上同调
    \[ H_G^\bullet (M) = H^\bullet \bigl( \Omega_G^\bullet (M), d \bigr) \]
    称为 $M$ 的\term{等变 de~Rham 上同调}.
\end{definition}

可以证明, 等变 de~Rham 上同调和通过代数拓扑定义的等变上同调是同构的.

等变微分形式还有另一种看上去不太相同的定义.

\begin{proposition}
    我们有
    \[ \Omega_G^\bullet (M) \simeq
        \bigl( \operatorname{Sym}^\bullet \frg^\vee \otimes 
        \Omega^\bullet (M) \bigr)^G , \]
    其中右边的 $\frg^\vee$ 的分次是 $2$.
    等号左边的外微分 $d$ 对应右边的算子
    \[ d - \iota \: \alpha (X) \mapsto d \alpha (X) - \iota_X \alpha (X), \]
    这里 $\alpha$ 看作从 $\frg$ 到 $\Omega^\bullet (M)$
    的多项式映射.
\end{proposition}

\begin{proof}
    只需要证明
    \[ \bigl( \wedge^\bullet \, \frg^\vee \otimes \Omega^\bullet (M)
        \bigr)_{\mathrm{hor}} \simeq \Omega^\bullet (M), \]
    这样, 在两边与 $\bbC [\frg]$ 做张量积, 并取 $G$-不变的部分,
    就得到了要证明的同构.
    
    我们定义从左边到右边的映射为取出 $1 \otimes \Omega^\bullet (M)$ 的部分,
    也就是说, 将所有 $\omega^i$ 都映到 $0$.
    我们把这个映射叫做 $\epsilon$.
    
    再来定义它的逆映射 $\epsilon^{-1}$.
    也就是说, 给定 $M$ 上的形式, 我们要找出它在 $\wedge^{>0} \, \frg^\vee$
    中缺失的部分, 使得它变成水平的. 我们定义
    \[ \epsilon^{-1} = \prod_i {} (1 - \omega^i \iota_i). \]
    因为 $\iota_i \omega^i = 1$, 所以 $\iota_i \circ (1 - \omega^i \iota_i) = 0$,
    从而 $\iota_i \circ \epsilon^{-1} = 0$.
    这说明我们确实给出了 $\epsilon$ 的逆映射.
    
    最后, 我们来计算微分 $d_{\frg} = \epsilon d \epsilon^{-1}$.
    若 $\alpha \in \Omega^\bullet (M)$, 我们有
    \begin{align*}
        d_{\frg} \alpha
        &= \epsilon d \bigl( \alpha - \omega^i \iota_i \alpha + 
            \omega^i \iota_i \omega^j \iota_j \alpha - \cdots \bigr) \\
        &= \epsilon \bigl( d \alpha - (d \omega^i) \, \iota_i \alpha
            + (\text{含 $\omega^i$ 的项}) \bigr) \\
        &= d \alpha - \Omega^i \iota_i \alpha.
    \end{align*}
    由于我们把 $\Omega^i$ 视为 $\alpha (X)$ 中的 $X$ 方向,
    我们就证明了要证的等式.
\end{proof}

通过这个命题, 我们可以写下等变微分形式的第二种定义.

\begin{definition}
    $M$ 上的\term{等变微分形式}是一个多项式映射
    \[ \alpha \: \frg \to \Omega^\bullet (M), \]
    它在 $G$ 的作用下不变. 换言之, 对任意 $g \in G$, $X \in \frg$, 有
    \[ \alpha (\operatorname{Ad}_g X) = g \cdot \alpha (X). \]
    $\alpha$ 的\term{等变外微分}是
    \[ d_{\frg} \alpha (X) = d \alpha (X) - \iota_X \alpha (X). \]
\end{definition}

当然, 这一定义的动机是之前的那一个定义.

\begin{example}
    如果 $M = \{ * \}$, 那么等变微分形式就是一个元素
    \[ \alpha \in \bbC[\frg]^G , \]
    也就是 $\frg$ 上的一个 $G$-不变多项式.
    这和我们的预期是一致的.

    再例如, 设 $G$ 是交换 Lie 群. 则伴随作用 $\operatorname{Ad}$ 是平凡的,
    从而, 一个等变微分形式就确实是一族 $G$-不变的微分形式. \varqed
\end{example}


\subsection{等变辛形式}

下面, 我们来了解等变微分形式的一个例子, 即等变辛形式.

设 $(M, \omega)$ 是辛流形. 也就是说, $M$ 是偶数维流形,
$\omega$ 是 $M$ 上的闭 $2$-形式, 它定义的切空间上的双线性型是非退化的.
一个典型的例子是
\[ \textstyle M = T^* N, \quad \omega = \sum_i dx^i \wedge dp^i, \]
其中 $N$ 是流形, $dx^i$ 是 $N$ 的切向,
而 $dp^i$ 是余切丛 $T^* N$ 中 $dx^i$ 所对应的方向.
这样定义的 $\omega$ 不依赖于坐标的选取.

在 Hamilton 力学中, 若 $N$ 是一个力学系统的位置空间,
那么余切丛 $T^* N$ 的纤维就是相应的动量空间.
力学系统的 \term{Hamilton 函数} (Hamiltonian)
\[ H \: M = T^* N \to \bbR \]
指定了每个状态的总能量, 即动能与势能的和. 
系统随时间的演化由 $M$ 上的 \term{Hamilton 向量场}
\[ X_H = \omega^{-1} (dH) \]
的流给出, 其中 $\omega^{-1}$ 表示每个点处的同构
$\omega_x \: T_x M \to T_x^* M$ 的逆映射.

设 Lie 群 $G$ 作用在 $M$ 上, 并保持其辛形式,
这个作用描述了力学系统的一些对称性.

\begin{definition}
    设 $(M, \omega)$ 是辛流形, $G$ 是 Lie 群.
    一个 \term{Hamilton 作用}是指二元组
    $(G \curvearrowright M, \ \mu)$, 满足以下条件:
    \begin{itemize}
        \item 
            辛形式 $\omega$ 在 $G$ 的作用下不变.
        \item
            $\mu \: \frg \to C^\infty (M)$ 是等变微分形式, 也就是说, 它满足
            \[ \mu (\operatorname{Ad}_g (X)) \, (x) = \mu (X) \, (g^{-1} x)
                \quad (g \in G, \ X \in \frg, \ x \in M). \]
            映射 $\mu$ 被称为\term{动量映射} (moment map).
        \item
            对任意 $X \in \frg$, 函数 $\mu (X)$ 的 Hamilton 向量场是 $X$,
            也就是说,
            \[ \omega^{-1} (d \mu(X)) = X. \]
    \end{itemize}
\end{definition}

我们通过一个例子来解释这个定义的意义.
这个例子也是 ``动量映射'' 这一称呼的来源.

\begin{example}
    我们仍考虑辛流形 $M = T^* N$, 其中 $N$ 是 Riemann 流形.
    我们把 $N$ 看作系统的位置空间.
    令 $G$ 通过等距自同构作用在 $N$ 上.
    例如, 可以取 $N = \bbR^3$, $G = \SO (3)$.
    
    定义动量映射
    \[ \mu (X) \, (\xi) = -\langle \xi, X \rangle_N \quad (\xi \in T^* M). \]
    每个点 $\xi$ 描述了系统的位置和动量,
    而 $\mu (X) \, (\xi)$ 就是这个动量沿 $X$ 方向的分量.
    通过动量映射, 我们也得到了一个元素
    \[ \mu (\xi) \in \frg^\vee . \]
    当 $N = \bbR^3$, $G = \SO (3)$ 时,
    这个 $\so (3)^\vee$ 的元素就是质点的\term{角动量}. \varqed
\end{example}

通过动量映射, 我们可以把辛形式 $\omega$ 变成等变微分形式.

\begin{definition} \label{def-11-equivar-symp-form}
    沿用上面的记号, 我们定义 $M$ 的\term{等变辛形式}为
    \[ \widetilde{\omega} = \mu + \omega, \]
    它将元素 $X \in \frg$ 映到微分形式 $\mu (X) + \omega$.
\end{definition}

形式 $\widetilde{\omega}$ 确实是等变闭形式, 因为
\[ d_{\frg} \widetilde{\omega}
    = d \mu (X) + d \omega - \iota_X \omega = 0. \]
特别地, 它确定了一个等变上同调类.

\begin{remark} \label{rmk-11-geo-quant}
    \term{几何量子化} (geometric quantisation) 是数学物理中的一个著名的方法.
    给定一个经典力学系统, 即辛流形, 我们想要构造对应的量子力学系统,
    这就是量子化问题.
    几何量子化的第一步就是寻找一个 Hermite 线丛, 使得其曲率等于 $\upi \omega$,
    其中 $\omega$ 是辛形式, 然后将量子态定义为这个线丛的某些截面.
    在 $G$-等变的情形下,
    要考虑的辛形式 $\omega$ 就是我们定义的等变辛形式 $\widetilde{\omega}$.
\end{remark}


\subsection{等变示性类}

下面, 设 $E \to M$ 是一个 $G$-等变的超向量丛.
我们想要通过等变微分形式, 来描述 $E$ 的等变示性类.

首先, 我们定义 $E$-取值的等变微分形式的代数:
\[ \Omega_G^\bullet (M, E) = \bigl( \operatorname{Sym}^\bullet \frg^\vee \otimes 
    \Omega^\bullet (M, E) \bigr)^G . \]
当然, 等变微分形式也可以看成 $\frg \to \Omega^\bullet (M, E)$ 的多项式.

我们也定义等变联络的概念, 使得它与等变微分形式的外微分 $d_{\frg}$ 相容.

\begin{definition}
    设 $\nabla$ 是 $E$ 上的一个 $G$-不变的超联络. 我们定义
    \begin{align*}
        \nabla_{\frg} \: \Omega_G^\bullet (M, E) &\to \Omega_G^\bullet (M, E), \\
        \alpha (X) &\mapsto \nabla \alpha (X) - \iota_X \alpha (X),
    \end{align*}
    从而, 它满足以下 Leibniz 法则: 对齐次元素
    $\omega \in \Omega_G^\bullet (M)$, \ $\alpha \in \Omega_G^\bullet (M, E)$,
    \[ \nabla_{\frg} (\omega \wedge \alpha)
        = d_{\frg} \omega \wedge \alpha
        + (-1)^{|\omega|} \, \omega \wedge \nabla_{\frg} \alpha. \]
    映射 $\nabla_{\frg}$ 称为一个 $G$-\term{等变超联络}.
\end{definition}

从等变联络出发, 我们可以定义等变曲率,
并通过等变曲率的不变多项式得到等变示性类.

\begin{definition}
    等变超联络 $\nabla_{\frg}$ 的\term{等变曲率}
    $\Omega_{\frg} \in \Omega_G^\bullet (M, \End (E))$ 定义为
    \[ \Omega_{\frg} \alpha (X) = \nabla_{\frg}^2 \alpha (X) + \scrL_X \alpha (X), \]
    其中 $\alpha \in \Omega_G^\bullet (M, E)$, 记号 $\scrL_X$ 表示 Lie 导数.
\end{definition}

这里, 加入 Lie 导数的项是为了使 $\Omega_{\frg}$ 成为一个张量 (即零阶微分算子).
事实上, 当原先的联络 $\nabla$ 是普通联络时, 我们有
\begin{align*}
    \Omega_{\frg} \alpha (X)
    &= \nabla^2 \alpha (X) - [\nabla, \iota_X] \, \alpha (X) + \scrL_X \alpha (X) \\
    &= \Omega \alpha (X) - \nabla_X \alpha (X) + \scrL_X \alpha (X) 
\end{align*}
其中 $\Omega$ 是 $\nabla$ 的曲率形式.
特别地, 当 $E = \wedge^\bullet \, T^* M$ 且 $\nabla = d$ 时,
Cartan 公式说明 $[d, \iota_X] = \scrL_X$, 从而 $\Omega_{\frg} = 0$.

\begin{remark} \label{rmk-11-moment-map}
    参照 (\ref{def-11-equivar-symp-form}), 我们可以定义\term{动量映射}
    $\mu \: \frg \to \Gamma(\End (E))$ 为
    \[ \mu (X) = \Omega_{\frg} (X) - \Omega. \]
    当 $\nabla$ 是普通联络时, 它就等于 $\scrL_X - \nabla_X$.
    我们已经在 (\ref{rmk-11-geo-quant}) 中提到了曲率形式与辛形式的关系. \varqed
\end{remark}

接下来, 我们来建立等变上同调的陈--Weil 理论.

\begin{theorem}
    设 $f \in \bbC [x]$ 是多项式.
    则等变微分形式 $\trs f(\Omega_{\frg})$ 是 $d_{\frg}$-闭的,
    并且, 它决定的等变上同调类不依赖于超联络的选取.
\end{theorem}

\begin{proof}
    和非等变的情形类似, 对 $\alpha \in \Omega_G^\bullet (M, \End(E))$, 我们有
    \[ d_{\frg} \trs \alpha = \trs (\nabla_{\frg} \alpha). \]
    由于 $\nabla_{\frg} \Omega_{\frg} = 0$, 我们有
    \[ d_{\frg} \trs f(\Omega_{\frg}) = \trs (\nabla_{\frg} f(\Omega_{\frg})) = 0. \]
    
    定理最后一部分的证明和 (\ref{thm-5-conn-indep}) 的证明同理.
\end{proof}

\begin{corollary} [等变陈--Weil 理论]
    等变曲率形式 $\Omega_{\frg}$ 的不变多项式给出了等变示性类,
    通过这种方式, 不变多项式能与等变示性类一一对应. \qed
\end{corollary}

例如, 我们可以定义\term{等变陈特征}
\[ \ch_{\frg} (E) \, (X) = \trs \upe^{- \Omega_{\frg} (X)}, \]
或者采用 (\ref{eq-9-ch-redef}) 之前的约定,
定义为 $\trs \upe^{\Omega_{\frg} (X) / 2 \uppi \upi}$.

我们也可以定义\term{等变 Euler 类}
\[ e_{\frg} (E) \, (X) = \operatorname{pf} (-\Omega_{\frg} (X)). \]
我们在 \S\ref{sect-6} 中介绍的 Mathai--Quillen 理论也有等变的版本,
也就是构造出等变 Thom 类. 这里, 我们就不详细介绍了.


\subsection{Кириллов 公式}

上一节, 我们概述了等变指标定理 (\ref{thm-10-equivar-index-thm})
的证明. 我们把等变指标表示成了不动点集上的积分:
\[ \ind (\gamma, D) = \int_{M^\gamma} 
    \frac {(-1)^{n_1 / 2}} {(2 \uppi \upi)^{n_0 / 2}} \, T_M \left( \frac
        {\Ahat (M^\gamma) \ch (\gamma, W)}
        {\ch (\gamma, S(N))}
    \right) \, dx. \]

事实上, 如果我们从等变示性类出发,
我们也能得到等变指标的一个表达式, 它将等变指标写成等变示性类在整个 $M$ 上的积分.

设紧 Lie 群 $G$ 作用在偶数维紧可定向 Riemann 流形 $M$ 上,
保持其度量和定向. 设 $\frg$ 是 $G$ 的 Lie 代数.
设 $D$ 是等变 Clifford 模 $E \to M$ 上的等变 Dirac 算子.

\begin{theorem} [Кириллов 公式\footnotemark{}] \label{thm-11-kirillov}
    \footnotetext{英文转录为 Kirillov.}
    设 $X \in \frg$ 是 $0$ 附近的元素. 则
    \[ \ind (\upe^{-X}, D) = \frac {1} {(2 \uppi \upi)^{n/2}}
        \int_M \Ahat_{\frg} (M) \, (X) \ch_{\frg} (W) \, (X). \]
\end{theorem}

注意到, 这个表达式其实就是在 (\ref{thm-9-atiyah-singer}) 中,
把示性类换成等变示性类得到的结果.

我们给出了等变指标的两个表达式,
它们是有联系的. 这一联系就是下一节中将要介绍的局部化公式,
这个公式把等变微分形式的积分转换为不动点集合上的积分.
通过局部化公式, 我们能从 (\ref{thm-10-equivar-index-thm})
推出 (\ref{thm-11-kirillov}). 详细过程可见 \cite[\S8.1]{bgv}.

Кириллов 公式还有另一个证明方法, 也就是模仿 (\ref{thm-9-atiyah-singer}) 的证明过程,
对等变热核做渐进展开, 但仅考虑 $G$ 的无穷小作用, 就能得到等变示性类.
详细的证明可见 \cite[\S8.3]{bgv}.

