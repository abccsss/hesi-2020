在这一节中, 我们从几何与代数两种观点出发,
证明等变微分形式的局部化公式.
这个公式能够把等变微分形式的积分写成不动点集上的积分.
特别地, 如果不动点集是离散的, 那么这个积分就化为一个求和.


\subsection{Berline--Vergne 局部化公式}

设紧 Lie 群 $G$ 作用在紧流形 $M$ 上, $n = \dim M$.
我们想要证明, 一个等变形式的积分只与它在不动点附近的取值有关.
证明的思路是利用 Stokes 公式, 将整个流形上的积分化为不动点附近的积分.

这个想法可以通过下面的命题得到验证.

\begin{proposition} \label{thm-12-exact}
    设 $\alpha \: \frg \to \Omega^\bullet (M)$ 是光滑映射, 不要求等变.
    假设 $d_{\frg} \alpha = 0$. 则对任意 $X \in \frg$, 形式
    \[ \alpha (X)_n \in \Omega^n (M) \]
    是 $M \setminus M_0 (X)$ 上的恰当 (exact) 形式,
    其中 $M_0 (X)$ 是向量场 $M$ 的零点集.
\end{proposition}

\begin{proof}
    在 $M$ 上取一个 $G$-不变的度量. 设 $\theta$ 是 $X$ 对应的 $1$-形式,
    即对任意向量场 $Y$, 有
    \[ \theta (Y) = \langle X, Y \rangle. \]
    虽然 $\theta$ 不是等变形式, 我们也还是记 $d_{\frg} \theta = (d - \iota_X) \theta$.
    因为 $X$ 的流保持 $M$ 的度量, 所以
    \[ d_{\frg}^2 \theta = - \scrL_X \theta = 0. \]
    另外, $d_{\frg}^2 \theta$ 在 $M \setminus M_0 (X)$ 上是可逆的偶数阶形式:
    \[ \frac {1} {d_{\frg} \theta} =
        - \frac {1} {\iota_X \theta} \biggl(
            1 + \frac {d \theta} {\iota_X \theta} + 
            \biggl( \frac {d \theta} {\iota_X \theta} \biggr)^2 + \cdots
        \biggr), \]
    其中, $\iota_X \theta = |X|^2$ 在 $M_0 (X)$ 以外都非零.
    因此,
    \[ 0 = d_{\frg} \biggl( d_{\frg} \theta \wedge \frac {1} {d_{\frg} \theta} \biggr)
        = d_{\frg} \theta \wedge d_{\frg} \frac {1} {d_{\frg} \theta}. \]
    由于 $d_{\frg} \theta$ 可逆, 知
    \[ d_{\frg} \frac {1} {d_{\frg} \theta} = 0. \]
    因此, 对命题叙述中的 $\alpha$, 我们有
    \[ \alpha (X) = d_{\frg} \frac {\theta \wedge \alpha (X)} {d_{\frg} \theta}, \]
    从而
    \[ \alpha (X)_n = d \biggl[ \biggl( 
        \frac {\theta \wedge \alpha (X)} {d_{\frg} \theta}
        \biggr)_{n-1} \biggr]. \qedhere \]
\end{proof}

下面, 我们假设 $M_0 (X)$ 是离散的集合.
对每个 $p \in M_0 (X)$, 子群
\[ H = \{ \upe^{tX} \mid t \in \bbR \} \subset G \]
作用在切空间 $T_p M$ 上. 从而, 这个群的无穷小生成元 $X$ 也作用在切空间上,
我们将这个作用记为
\[ L_p \: T_p M \to T_p M. \]
事实上, $L_p$ 就是 Lie 导数, 即向量场的 Lie 括号 $L_p(-) = [X, -]_p$.

\begin{lemma} \label{lem-12-lp-invert}
    $L_p$ 是可逆线性映射.
\end{lemma}

\begin{proof}
    若不然, 则存在 $0 \neq v \in T_p M$, 使得 $L_p v = 0$.
    这意味着 $X$ 的流将 $v$ 固定不动, 从而对任意 $t \in \bbR$,
    $\exp_p (tv)$ 都是 $H$ 作用的不动点, 其中 $H \subset G$ 是上面定义的子群
    (注意 $M$ 上的度量是 $G$-不变的).
\end{proof}

而 $G$ 通过等距同构作用在 $M$ 上, 从而 $L_p \in \so (T_p M)$.
因此, 引理说明 $M$ 一定是偶数维. 此时, 我们可以得到 Pfaff 值
\[ \operatorname{pf} (L_p) = \pm \det^{1/2} (L_p), \]
其中正负号取决于 $M$ 的定向.

\begin{theorem} [Berline--Vergne 公式]
    设 $\alpha \: \frg \to \Omega^\bullet (M)$ 是光滑映射, 不要求等变.
    假设 $d_{\frg} \alpha = 0$. 固定一个元素 $X \in \frg$. 
    如果 $M_0 (X)$ 是离散的集合, 那么
    \[ \int_M \alpha (X) = (-2 \uppi)^{n/2} \sum_{p \in M_0 (X)}
        \frac {\alpha(X)_0 (p)} {\operatorname{pf} (L_p)}, \]
    其中 $n = \dim M$, 下标 $0$ 表示取 $0$-形式的部分.
\end{theorem}

\begin{proof}
    取 $T_p M$ 的一组与定向相符的标准正交基, 使得 $L_p$ 表示为分块对角阵
    \[ \operatorname{diag} \biggl( \Bigl( \begin{smallmatrix}
        0 & a_1 \\ -a_1 & 0
    \end{smallmatrix} \Bigr), \ \dotsc, \ \Bigl( \begin{smallmatrix}
        0 & a_{n/2} \\ -a_{n/2} & 0
    \end{smallmatrix} \Bigr) \biggr). \]
    在 $p$ 附近建立法坐标系, 则 $X$ 的表达式是
    \[ X = a_1 (x^2 \, \partial_1 - x^1 \, \partial_2) + \cdots + 
        a_{n/2} (x^n \, \partial_{n-1} - x^{n-1} \, \partial_n). \]
    事实上, $X$ 是一个 Killing 向量场 (也就是说, 它的流保持 $M$ 的度量),
    它在整个流形上的值能由一个点处的信息确定.
    
    和 (\ref{thm-12-exact}) 的证明中一样, 我们令
    \[ \theta = \frac{1}{a_1} (x^2 \, dx^1 - x^1 \, dx^2) + \cdots +
        \frac{1}{a_{n/2}} (x^n \, dx^{n-1} - x^{n-1} \, dx^n). \]
    则 $\scrL_X \theta = 0$, 且 $\iota_X \theta = \sum_i (x^i)^2$.
    
    我们在每个 $p \in M_0 (X)$ 附近构造这样的 $\theta$,
    然后通过单位分解, 得到整个 $M$ 上的 $1$-形式 $\theta$, 满足
    \begin{itemize}
        \item
            $\scrL_X \theta = 0$, 并且在 $M_0 (X)$ 以外, 有 $\iota_X \theta \neq 0$.
        \item 
            在每个 $p \in M_0 (X)$ 附近, $\theta$ 与上面构造的 $\theta$ 相同.
    \end{itemize}
    借助 (\ref{thm-12-exact}) 中的计算, 我们有
    \begin{align*}
        \int_M \alpha (X)
        &= \lim_{\epsilon \to 0} \int_{M \setminus \bigcup_p B_\epsilon (p)} \alpha (X) \\
        &= \lim_{\epsilon \to 0} \int_{M \setminus \bigcup_p B_\epsilon (p)}
            d \frac {\theta \wedge \alpha (X)} {(d - \iota_X) \, \theta} \\
        &= - \lim_{\epsilon \to 0} \sum_{p \in M_0 (X)} \int_{\partial B_\epsilon (p)}
            \frac {\theta \wedge \alpha (X)} {(d - \iota_X) \, \theta}.
    \end{align*}
    此时, 式中的每个形式都有具体的坐标表达式.
    通过初等的计算, $\partial B_\epsilon (p)$ 上的积分等于
    \[ (-2 \uppi)^{n/2} \cdot \frac{\alpha (X)_0 (p)} {a_1 \cdots a_{n/2}}
        = (-2 \uppi)^{n/2} \cdot \frac{\alpha (X)_0 (p)} {\operatorname{pf} (L_p)}.
        \qedhere \]
\end{proof}

\subsection{几个应用}

下面, 我们介绍这个结论的两个应用.
第一个应用是计算示性数的 Bott 公式.
我们回忆, 若
\[ \Phi \in \bbC[\so (n)]^{\SO (n)} \simeq \bbC [ p_1, \dotsc, p_{n/2-1}, e ] \]
是不变多项式 (参见 \ref{thm-3-pon-class}),
则 $\Phi$ 所对应的 $M$ 的\term{示性数}是
\[ \Phi (M) = \Bigl( \frac{-1}{2 \uppi} \Bigr)^{n/2} \int_M \Phi (\Omega). \]

\begin{theorem} [Bott 公式]
    如果 $S^1$ 作用在 $M$ 上, 其不动点集是离散的, 那么
    \[ \Phi (M) = \sum_{p: \text{不动点}}
        \frac {\Phi (L_p)} {\operatorname{pf} (L_p)}. \]
\end{theorem}

这里我们说的 $L_p$ 是相对于 $S^1$ 的 Lie 代数的一个生成元 $X$ 而言的,
但事实上, 不难看出这个表达式的值不依赖于生成元的选取.

\begin{proof}
    选取 $M$ 上的 $S^1$-不变度量, 我们有等变曲率形式
    \[ \Omega_{\frg} (X) = \Omega + \mu (X), \]
    其中 $\mu$ 是动量映射 (\ref{rmk-11-moment-map}).
    由陈--Weil 理论, $d_{\frg} \Phi (\Omega_{\frg}) = 0$.
    利用局部化公式, 我们得到
    \begin{align*}
        \Phi (M) &= \Bigl( \frac{-1}{2 \uppi} \Bigr)^{n/2}
        \int_M \Phi (\Omega + \mu (X)) \\
        &= \sum_{p: \text{不动点}} \frac {\Phi (\mu (X))} {\operatorname{pf} (L_p)}.
    \end{align*}
    而由 (\ref{rmk-11-moment-map}), $\mu (X) = \scrL_X - \nabla_X$,
    后一项在不动点处消失, 而前一项就给出了我们需要的 $L_p$.
\end{proof}

我们要介绍的第二个应用是 Duistermaat--Heckman 公式,
它描述了辛流形的 Liouville 测度沿动量映射的前推.

\begin{theorem} [Duistermaat--Heckman 公式] \label{thm-12-duistermaat-heckman}
    设 $(M, \omega)$ 是紧辛流形, $(G \curvearrowright M, \ \mu)$ 是 Hamilton 作用.
    取一个元素 $X \in \frg$, 假定 $M_0 (X)$ 是离散的集合. 则
    \[ \int_M \upe^{\upi \mu (X)} \frac{\omega^{n/2}}{(n/2)!}
        = (2 \uppi \upi)^{n/2} \sum_{p \in M_0 (X)}
        \frac {\upe^{\upi \mu(X) (p)}} {\operatorname{pf} (L_p)}. \]
\end{theorem}

事实上, 这个公式就是使用\term{稳定相位法}近似计算左边的积分得到的公式,
但误差项消失了. 也就是说, 这个近似结果实际上是精确的.

定理中, $n$-形式 $\omega^{n/2} / (n/2)!$ 称为 \term{Liouville 测度},
它是辛流形上典范的测度. 通过动量映射 $\mu \: M \to \frg^\vee$,
我们可以定义 $\frg^\vee$ 上的测度 $\nu$, 使得
\[ \int_{\frg^\vee} f \, d \nu =
    \int_M (f \circ \mu) \, \frac{\omega^{n/2}}{(n/2)!}. \]
测度 $\nu$ 称为 \term{Duistermaat--Heckman 测度},
它是 Liouville 测度沿动量映射的前推. 这样, 定理中等式的左边就能写成
\[ \int_{\frg^\vee} \upe^{\upi \langle -, X \rangle} \, d \nu, \]
被积分的项其实是 Duistermaat--Heckman 测度的 Fourier 变换.
从这个角度出发, 可以证明 Duistermaat--Heckman 公式的一个等价形式,
即度量 $\nu$ 实际上是由 $\frg^\vee$ 中的一些多面体上的多项式函数拼起来而得到的.
读者可参见 \cite{atiyah-bott}.

\begin{proof} [定理 \ref{thm-12-duistermaat-heckman} 的证明]
    我们有
    \begin{align*}
        \int_M \upe^{\upi \mu (X)} \frac{\omega^{n/2}}{(n/2)!}
        &= (-\upi)^{n/2} \int_M \upe^{\upi \mu (X)} \, \upe^{\upi \omega} \\
        &= (-\upi)^{n/2} \int_M \upe^{\upi \widetilde{\omega} (X)},
    \end{align*}
    其中, 等变辛形式 $\widetilde{\omega} = \omega + \mu$ 是等变闭形式,
    因此, $d_{\frg} \upe^{\upi \widetilde{\omega}} = 0$.
    使用局部化公式就完成了证明.
\end{proof}


\subsection{一般的局部化公式}

下面, 我们不再要求 $M_0 (X)$ 是离散的.
和我们介绍等变指标定理时类似, 这时 $M_0 (X)$ 是 $M$ 的子流形,
但可能是不同维数的子流形的不交并. 设 $N$ 是法丛.
与之前不同, 现在 $M$ 可能是奇数维的, 但 $N$ 的秩一定是偶数,
其原因和 (\ref{lem-12-lp-invert}) 相同.

我们记
\[ \frg_0 = \{ Y \in \frg \mid [Y, X] = 0 \} \quad \subset \frg, \]
并设 $G_0 = \exp \frg_0 \subset G$.
则作用 $G_0 \curvearrowright M$ 保持向量场 $X$ 不变, 从而
$G_0$ 的作用也保持了子流形 $M_0 (X)$ (但并非将它固定不动).
这又诱导了 $G_0$ 在法丛 $N$ 上的作用.

在 $N$ 上取 $G_0$-不变的度量和联络, 我们能得到 $G_0$-等变的曲率形式
\[ \Omega_{\frg_0}^N (Y) = \Omega^N + \mu^N (Y), \]
其中 $Y \in \frg_0$.

\begin{theorem}[Berline--Vergne 公式] \label{thm-12-loc-formula}
    设 $\alpha \: \frg \to \Omega^\bullet (M)$ 是光滑映射, 不要求等变.
    假设 $d_{\frg} \alpha = 0$. 固定一个元素 $X \in \frg$. 
    则对足够靠近 $X$ 的 $Y \in \frg_0$, 有
    \[ \int_M \alpha (Y) = \int_{M_0 (X)} (2 \uppi)^{n_1/2} 
        \frac {\alpha(Y)} {e_{\frg_0} (N) \, (Y)}, \]
    其中 $n_1 (x) = n - \dim_x M_0$, 而 $e_{\frg_0}$ 是等变 Euler 类.
    特别地,
    \[ \int_M \alpha (X) = \int_{M_0 (X)} (-2 \uppi)^{n_1/2} 
        \frac {\alpha(X)} {\operatorname{pf} (\Omega^N + L^N)}, \]
    其中 $L^N$ 的定义和之前的 $L_p$ 同理.
\end{theorem}

定理的证明方法与之前类似, 但涉及冗长的曲率计算.
详细的证明可见 \cite[\S7.2]{bgv}.
不过, 我们在下一节将给出这一公式的另一证明路径.


\subsection{Atiyah--Bott 局部化定理}

设紧 Lie 群 $G$ 作用在流形 $M$ 上,
设 $T \subset G$ 是极大环面. 则有一个自然的映射
\[ i^\sharp \: H_G^\bullet (M; \bbC) \to H_T^\bullet (M; \bbC). \]
可以证明, 这一映射是单射, 且其像是 $H_T^\bullet (M; \bbC)^W$,
其中 $W$ 是 Weyl 群. 因此, 我们可以只研究 $H_T^\bullet (M)$,
然后将 $H_G^\bullet (M)$ 从其中恢复出来.

设 $F \subset M$ 是 $T$-不动点的集合. 我们将证明, 映射
\[ i^* \: H_T^\bullet (M; \bbC) \to H_T^\bullet (F; \bbC) \]
几乎是一个同构, 并从中得到局部化公式.

设 $X_1, \dotsc, X_{\ell}$ 是 $T$ 的 Lie 代数 $\frt$ 的一组基. 则
\[ H_T^\bullet (\{*\}; \bbC) \simeq \bbC[\frt]^T \simeq \bbC[\frt] \simeq \bbC[X_i]. \]
因此, 等变上同调环 $H_T^\bullet (M)$ 实际上是一个 $\bbC[X_i]$-代数.

从代数几何的角度看, $\bbC[X_i]$-代数 $H_T^\bullet (M; \bbC)$ 实际上是仿射空间
\[ \operatorname{Spec} \bbC[X_i] \simeq \bbC^{\ell} \simeq \frt \otimes \bbC \]
上的一个拟凝聚代数层. 我们将这个拟凝聚层记为 $\scrH$.
这个代数层的支集是
\[ \operatorname{supp} H
    = \bigcap_{\substack{f \in \bbC[X_i] \\ fH = 0}} V(f)
    = V(\operatorname{ann} H), \]
其中 $H = H_T^\bullet (M; \bbC)$, 记号 $V(f)$ 表示 $f$ 的零点集,
$\operatorname{ann} H$ 是 $H$ 的零化子.

\begin{lemma}
    设 $K \subset T$ 是闭子群.
    如果存在 $T$-等变的映射 $M \to T/K$
    (换言之, $M$ 中的 $T$-轨道至多被压缩成 $T/K$), 那么
    \[ \operatorname{supp} H \subset \frk \otimes \bbC, \]
    其中 $\frk$ 是 $K$ 的 Lie 代数.
    特别地, 如果 $K \neq T$, 那么 $H$ 是挠模.
\end{lemma}

\begin{proof}
    设 $f \in \bbC [\frt]$, 满足 $f|_{\frk \otimes \bbC} = 0$.
    我们只需要证明 $fH = 0$.
    
    我们有
    \[ H_T^\bullet (T/K)
        \simeq H^\bullet \biggl( \frac{T/K \times \upE T}{T} \biggr)
        \simeq H^\bullet (\upE T / K)
        \simeq H_K^\bullet (\{*\}) \simeq \bbC [\frk], \]
    其中第三个等号是因为 (\ref{thm-1-contractible}).
    而由假设, 我们有 $H_T^\bullet (T/K)$-模的交换图
    \[ \begin{tikzcd}
        H_T^\bullet (T/K) \ar[r] \ar[d, "f"'] & H_T^\bullet (M) \ar[d, "f"] \\
        H_T^\bullet (T/K) \ar[r] & H_T^\bullet (M) \rlap{ .}
    \end{tikzcd} \]
    因为 $f|_{\frk \otimes \bbC} = 0$, 所以左边的竖直映射为零,
    而右边的竖直映射由左边的映射诱导, 故也为零.
\end{proof}

我们回忆, 如果子群 $K \subset T$ 是某个点 $p \in M$ 的稳定子, 即
\[ K = \{ t \in T \mid tp = p \}, \]
就称 $K$ 是一个\term{迷向子群} (isotropy subgroup).

\begin{lemma} \label{lem-12-atiyah-bott}
    沿用上面的记号, 则作为 $\frt \otimes \bbC$ 的子集, 有
    \[ \operatorname{supp} H \subset
        \bigcup_{K \subset T \ \text{迷向}} \frk \otimes \bbC, \]
    并且, 这样的迷向子群只有有限个.
\end{lemma}

\begin{proof}
    设 $O \simeq T/K \subset M$ 是一个 $T$-轨道.
    通过不变度量的指数映射, 我们能得到 $O$ 的一个 $T$-不变的管状邻域 $U$,
    它有一个 $T$-等变的投影映射 $U \to T/K$.
    因此, 对任何 $T$-不变的开集 $U' \subset U$, 都有
    \[ \operatorname{supp} H_T^\bullet (U'; \bbC) \subset \frk \otimes \bbC. \]
    
    我们把 $U$ 和轨道 $O$ 的法丛中零截面的 $\epsilon$-邻域等同起来.
    则 $T$ 在 $U$ 上的作用可以如下描述: 首先把 $K$ 作用在法丛的纤维上,
    然后通过 $T/K$ 的作用移动到别的纤维上.
    并且, $K$ 在纤维上的作用是线性的, 且在每个纤维上作用的方式相同.
    因此, 这些纤维中的点的稳定子群, 实际上就是 $K$ 关于这个线性作用的迷向子群.
    
    因为 $\GL(n)$ 的交换子群的所有元素可同时对角化,
    所以 $T \curvearrowright U$ 的所有迷向子群就是 $K$ 的有限个子群.
    而紧流形 $M$ 可被有限个这样的 $U$ 覆盖,
    从而 $T \curvearrowright M$ 只有有限个迷向子群.
    
    最后, 关于支集的命题不难通过 Mayer--Vietoris 序列得到.
\end{proof}

设 $F \subset M$ 是 $T$-不动点的集合.
则 $F$ 以外的点的稳定子群都是 $T$ 的真子群.
上面的命题说明, $M$ 的等变上同调在这些地方都是 ``几乎为零'' 的.
下面的定理就是这个想法的严格表述.

\begin{theorem} [Atiyah--Bott 局部化定理] \label{thm-12-atiyah-bott}
    在上面的记号下, 映射
    \[ i^* \: H_T^\bullet (M; \bbC) \to H_T^\bullet (F; \bbC) \]
    的核与余核的支集都包含于
    \[ \bigcup_{K \subsetneq T \ \text{迷向}} \frk \otimes \bbC. \]
    特别地, 其核与余核作为 $\bbC [\frt]$-模都是挠模.
\end{theorem}

\begin{proof}
    我们有上同调的长正合列
    \[ \cdots \to H_T^\bullet (M, F)
        \to H_T^\bullet (M)
        \to H_T^\bullet (F)
        \to H_T^{\bullet+1} (M, F) \to \cdots, \]
    其中 $H_T^\bullet (M, F)$ 是相对等变上同调, 读者不难猜出其定义.
    
    为了证明定理, 我们只需要证明
    \[ \operatorname{supp} H_T^\bullet (M, F) \subset
        \bigcup_{K \subset T \ \text{迷向}} \frk \otimes \bbC, \]
    其中 $K$ 取遍 $T \curvearrowright M \setminus F$ 的所有 (有限个) 迷向子群.
    
    通过不变度量的指数映射, 我们可以取 $F$ 的 $T$-等变管状邻域 $U$.
    将 (\ref{lem-12-atiyah-bott}) 的证明稍作修改,
    也就是把这里的 $U$ 加入到证明中的开覆盖中, 我们实际上证明了
    \[ \operatorname{supp} H_T^\bullet (M, \oline2U) \subset
        \bigcup_{K \subsetneq T \ \text{迷向}} \frk \otimes \bbC, \]
    因为我们可以把 $\oline2U$ 中未被其它开集覆盖的部分用切除引理移走.
    
    最后, 因为 $\oline2U$ 可以 $T$-等变地形变收缩到 $F$,
    所以它们的等变上同调同构.
    利用 $(M, \oline2U)$ 和 $(M, F)$ 的上同调长正合列和五引理,
    我们就完成了证明.
\end{proof}

接下来, 我们通过这个方法, 推导出等变微分形式积分的局部化公式.

我们回忆 Thom 同构定理 (\ref{thm-2-thom}).
在非等变的情形下, 如果我们记 $c = \operatorname{codim}_M F$, 则有图表
\[ \begin{tikzcd}[column sep=1.5em]
    \cdots \ar[r] &
    H^{\bullet-1} (M \setminus F) \ar[r, "\delta"] &
    H^\bullet (M, M \setminus F) \ar[r] \ar[d, "\simeq"'] &
    H^\bullet (M) \ar[r] \ar[dl, dashed, 
        end anchor=north east, shift left=3, "i^*"] &
    H^\bullet (M \setminus F) \ar[r] & \cdots \\
    && H^{\bullet-c} (F) \ar[ur, 
        start anchor=north east, shift right, "i_*"] \rlap{ \ ,}
\end{tikzcd} \]
其中虚线表示其次数与标注的不符, 映射 $i_*$ 定义为它上方两个映射的复合.
我们回忆
\[ i^* \, i_* = e(N), \]
其中 $N$ 是 $F$ 的法丛.

类似地, 映射 $i_*$ 在等变的情形也可以定义,
此时 $i^* \, i_*$ 就等于等变 Euler 类 $e_{\frt}(N)$.
这一事实是等变示性类的 (代数拓扑) 定义的直接推论.

\begin{corollary} \label{thm-12-atiyah-bott-2}
    映射
    \[ i_* \: H_T^\bullet (F; \bbC) \to H_T^{\bullet + c} (M; \bbC) \]
    的核与余核的支集都包含于
    \[ \bigcup_{K \subsetneq T \ \text{迷向}} \frk \otimes \bbC. \]
\end{corollary}

\begin{proof}
    由 (\ref{thm-12-atiyah-bott}) 和上面的长正合列的等变版本, 命题立即得证.
\end{proof}

选一个 $f \in \bbC [\frt]$,
使得 (\ref{thm-12-atiyah-bott-2}) 中的支集包含于 $V(f)$.
我们将 $\bbC [\frt]$ 局部化为 $\bbC [\frt]_f$. 则
\[ (i_*)_f \: H_T^\bullet (F)_f \to H_T^{\bullet + c} (M)_f \]
是同构. 这说明, 对任何等变上同调类 $\alpha \in H_T^\bullet (M)$, 有
\[ \alpha = i_* \, \frac{i^*}{e_{\frt} (N)} \, \alpha + (\text{$f$-挠元}). \]
然而, 如果 $\beta$ 是 $f$-挠元, 也就是说, 
对任意 $X \in \frt$, 微分形式 $f (X) \, \beta (X)$ 都是 $d_{\frt}$-恰当形式,
那么, 对任意 $X \in \frt$ 都有 $\int_M \beta(X) = 0$. 从而,
\[ \int_M \alpha (X)
    = \int_M i_* \, \frac{i^* \alpha (X)}{e_{\frt} (N) \, (X)}
    = \int_F (2 \uppi)^{c/2} \, \frac{\alpha (X)}{e_{\frt} (N) \, (X)}, \]
其中 $(2 \uppi)^{c/2}$ 是 Thom 形式的积分值.
我们就重新得到了局部化公式 (\ref{thm-12-loc-formula}).

