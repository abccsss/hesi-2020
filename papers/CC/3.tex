上一节中, 我们提到了 Euler 类的乘积公式.
下面, 我们导出其它示性类的乘积公式, 
这些公式给出了示性类与向量丛的直和、张量积的关系.

\subsection{Whitney 乘积公式}

\begin{definition}
    设 $E \to B$ 是 $\bbK$-向量丛, $R$ 是 $\bbK$-可定向环. 
    当 $\bbK = \bbR, \bbC, \bbH$ 时, 我们分别定义
    \[ \left. \begin{aligned}
        w(E) & = 1 + w_1(E) + w_2(E) + \cdots \\
        c(E) & = 1 + c_1(E) + c_2(E) + \cdots \\
        p(E) & = 1 + p_1(E) + p_2(E) + \cdots
    \end{aligned} \quad \right\} \in H^\bullet (B; R) \]
    为 $E$ 的\term{全 Stiefel--Whitney 类}、
    \term{全陈类}和\term{全 Понтрягин 类}.
    注意, 这里的求和是有限的.
\end{definition}

\begin{theorem}[Whitney 乘积公式]
    如果 $E_1,E_2$ 是 $B$ 上的两个 $\bbK$-向量丛, 
    那么当 $\bbK = \bbR, \bbC, \bbH$ 时, 分别有
    \[ \begin{aligned}
        w (E_1 \oplus E_2) & = w (E_1) \, w (E_2), \\
        c (E_1 \oplus E_2) & = c (E_1) \, c (E_2), \\
        p (E_1 \oplus E_2) & = p (E_1) \, p (E_2).
    \end{aligned} \]
\end{theorem}

\begin{proof}
    先证明 $\bbK = \bbR$ 的情况. 
    记 $n_1, n_2$ 分别为 $E_1, E_2$ 的秩, 记 $N = n_1 + n_2$. 

    固定 $n$, 我们对 $N$ 归纳证明 $w_n$ 满足公式. 
    当 $N = n$ 时, 有
    \[ w_n (E_1 \oplus E_2) = e (E_1 \oplus E_2)
        = e(E_1) \, e(E_2) = w_{n_1} (E_1) \, w_{n_2} (E_2). \]
    当 $N > n$ 时, 假设公式对 $< N$ 的维数都成立. 注意到 $E_1 \oplus E_2$ 的分类映射是
    \[ B \xrightarrow{E_1,\ E_2} \upB \upO (n_1) \times \upB \upO (n_2)
        \overset{\mu}{\longrightarrow} \upB \upO (n_1 + n_2), \]
    其中 $\mu$ 是
    \[ \oplus \: \upO (n_1) \times \upO (n_2) \hookrightarrow \upO(n_1+n_2) \]
    诱导的映射, 也是向量丛 $\upE \upO(n_1) \times \upE \upO(n_2)$ 的分类映射. 
    考虑投影映射
    \[ \operatorname{pr}_k \:
        H^n (\upB \upO (n_1) \times \upB \upO (n_2)) \longrightarrow
        H^k (\upB \upO (n_1)) \otimes H^{n-k}(\upB\upO(n_2)). \]
    我们只需证明 $\operatorname{pr}_k (\mu^* (w_n)) = w_k \otimes w_{n-k}$. 
    此时要么 $k < n_1$, 要么 $n - k < n_2$. 不妨设 $k < n_1$. 
    我们有交换图
    \[ \begin{adjustbox}{scale=.95}
        \begin{tikzcd}[column sep=small]
            H^n (\upB \upO (n_1 + n_2)) \ar[d, "i^*"'] \ar[r, "\mu^*"] &
            H^n (\upB \upO (n_1) \times \upB \upO(n_2)) \ar[r, "\operatorname{pr}_k"] &
            H^k (\upB \upO (n_1)) \otimes H^{n-k}(\upB\upO(n_2)) \ar[d, "i^* \otimes 1", "\simeq"'] \\
            H^n (\upB \upO (k + n_2)) \ar[r, "\mu^*"] &
            H^n (\upB \upO (k) \times \upB \upO(n_2)) \ar[r, "\operatorname{pr}_k"] &
            H^k (\upB \upO (k))\otimes H^{n - k} (\upB \upO (n_2))\rlap{\ .}
        \end{tikzcd}
    \end{adjustbox} \]
    因为最右边的竖箭头是同构, 只需验证
    $w_n \in H^n (\upB \upO (n_1 + n_2))$
    沿着下面三个箭头走会得到 $w_k \otimes w_{n-k}$. 
    但这是由于 $i^* w_n = w_n$ 和归纳假设. 
\end{proof}

\begin{example}
    对球面 $S^n$ 的切丛 $TS^n$, 有
    \[ w (TS^n) = 1, \]
    因为如果把 $S^n$ 嵌入 $\bbR^{n + 1}$ 中,
    就能看出 $TS^n \oplus \bbR \simeq \bbR^{n + 1}$,
    这里 $\bbR$ 和 $\bbR^{n + 1}$ 表示平凡丛. \varqed
\end{example}

下面的结论说明,
示性类刻画了向量丛的扭曲程度.

\begin{proposition}
    设 $E \to B$ 是 $\bbK^n$-丛.
    如果它有 $k$ 个 $\bbK$-线性无关的截面,
    那么当 $i > n - k$ 时, 相应的示性类
    $w_i(E)$, $c_i(E)$ 或 $p_i(E)$ 等于零. \qed
\end{proposition}

\begin{exercise}
    当 $\bbK = \bbR, \bbC, \bbH$ 时, 
    记 $G(n) = \upO(n), \upU(n), \Sp(n)$.
    对 $0 \leq k \leq n$, 记
    \[ i \: \upB G (k) \to \upB G (n) \]
    为含入映射 $G(k) \hookrightarrow G(n)$ 诱导的映射.
    设 $1 \leq j \leq k$. 
    则当 $\bbK = \bbR, \bbC, \bbH$ 时, 分别有
    \[
        i^* w_j = w_j, \quad 
        i^* c_j = c_j, \quad 
        i^* p_j = p_j. \varqedhere
    \]
\end{exercise}

\subsection{实向量丛的 Понтрягин 类}

到目前为止, 我们已经知道了复向量丛和四元数向量丛的所有示性类.
但对实向量丛而言, 我们只在系数环是特征 $2$ 时解决了这个问题.
对于一般系数的环, 我们下面定义一种新的示性类, 即 Понтрягин 类.
(注意, 这和四元数向量丛的 Понтрягин 类是不同的.)

\begin{definition}
    在 (\ref{thm-1-functor}) 的意义下, 我们定义两种向量丛的操作.
    \begin{itemize}
        \item
            对实向量丛 $E$, 我们可以定义 $E$ 的\term{复化},
            即复向量丛 $E \otimes \bbC$.

        \item
            对复向量丛 $E$, 我们可以定义 $E$ 的\term{复共轭},
            即复向量丛 $\Ebar$.
    \end{itemize}
\end{definition}

\begin{lemma}
    设 $E$ 是复向量丛. 则对任意正整数 $i$, 有
    \[ c_i (\Ebar) = (-1)^i \, c_i (E). \]
\end{lemma}

\begin{proof}
    我们只需对万有复向量丛, 记为 $E(n) \to \upB \upU (n)$, 证明这一等式.
    因为 $\bbC^n$ 的定向在取共轭之后变成原来的 $(-1)^n$ 倍,
    所以由 (\ref{thm-2-top-equals-euler}),
    \[ 
        c_n \bigl( \overline{E\smash{(n)}} \bigr)
        = e \bigl( \overline{E\smash{(n)}} \bigr)
        = (-1)^n \, e (E(n))
        = (-1)^n \, c_n (E(n)).
    \]
    对 $1 \leq k \leq n$, 记
    $i \: \upB \upU (k) \to \upB \upU (n)$
    为映射 $\upU (k) \hookrightarrow \upU (n)$ 诱导的映射. 则
    \[
        i^* c_k \bigl( \overline{E\smash{(n)}} \bigr)
        = c_k \bigl( \overline{E\smash{(k)}} \bigr)
        = (-1)^k \, c_k (E(k))
        = (-1)^k \, i^* c_k (E(n)).
    \]
    而我们对 Graßmann 流形的上同调的计算结果表明,
    $i^* \: H^{2k} (\upB \upU (n)) \to H^{2k} (\upB \upU (k))$
    是同构. 这就得到了要证的等式.
\end{proof}

对于实向量丛 $E$, 有自然的复向量丛的同构
\[ E \otimes \bbC \simeq \overline{E \otimes \bbC}. \]
这个同构由 $x + \upi y \mapsto x - \upi y$ 给出.
由引理, 这说明当 $i$ 是奇数时,
\[ 2 c_i ( E \otimes \bbC ) = 0. \]
当系数环非特征 $2$ 时, 这些陈类一定为 $0$,
而偶数陈类能给出有用的信息.

\begin{definition}
    设 $E \to B$ 是实向量丛. 
    定义 $E$ 的第 $i$ 个 \term{Понтрягин 类}为
    \[ p_i (E) = (-1)^i \, c_{2i} (E \otimes \bbC) \quad \in H^{4i} (B). \]
    定义 $p(E) = 1 + p_1(E) + p_2(E) + \cdots$
    为 $E$ 的\term{全 Понтрягин 类}.
\end{definition}

这里, 加入系数 $(-1)^i$ 没有什么深刻的原因, 也不会影响下一个结论,
只是为了和公认的定义相符, 并使得下面的公式 (\ref{thm-3-pon-euler-square}) 不带符号.

\begin{theorem}[Whitney 乘积公式]
    如果 $E_1,E_2$ 是 $B$ 上的两个实向量丛, 
    且系数环 $R$ 中 $2$ 不是零因子. 那么
    \[ p (E_1 \oplus E_2) = p (E_1) \, p (E_2). \thmqedhere \]
\end{theorem}

定理的证明非常简单, 留给读者完成.

\begin{example} \label{eg-3-pon-sphere}
    与之前同理, 当 $R$ 非特征 $2$ 时, 我们有
    \[ p (TS^n) = 1. \varqedhere \]
\end{example}

与之前提到的示性类类似,
最高阶的 Понтрягин 类也与 Euler 类有关.

\begin{theorem} \label{thm-3-pon-euler-square}
    设 $E$ 是可定向的 $\bbR^{2n}$-丛. 则
    \[ p_n (E) = e (E) ^2. \]
\end{theorem}

\begin{proof}
    由 (\ref{thm-2-top-equals-euler}), 我们有
    \[ p_n (E) = (-1)^n \, c_n (E \otimes \bbC)
        = (-1)^n \, e (E \otimes \bbC). \]
    对任意带有定向的 $2n$ 维实向量空间 $V$, 同构
    \[ V \otimes \bbC \simeq V \oplus \upi V \]
    将定向变为原来的 $(-1)^n$ 倍,
    因为若取 $V$ 的一组符合定向的有序基 $v_1, \dotsc, v_{2n}$,
    那么上述两个空间的定向分别由排序
    \[ v_1,\ \upi v_1,\ \dotsc,\ v_{2n},\ \upi v_{2n}; \quad
        v_1, \dotsc, v_{2n}, \upi v_1, \dotsc, \upi v_{2n} \]
    确定.
    这两个排序通过 $(2n)(2n - 1)/2$ 次对换互相转换,
    因此它们的定向相差 $(-1) ^{(2n)(2n - 1)/2} = (-1)^n$ 倍.
    从而,
    \[ p_n (E) = (-1)^n \, e (E \otimes \bbC)
    = e (E \oplus E) = e (E) ^2. \qedhere \]
\end{proof}

于是, 对非特征 $2$ 的情形,
我们可以给出可定向实向量丛的所有示性类.

\begin{theorem} \label{thm-3-pon-class}
    设系数环 $R$ 中 $2$ 不是零因子. 则有分次环的同构
    \[ \begin{aligned}
        H^\bullet (\upB \SO (2n);\ R) & \simeq R[p_1, \dotsc, p_{n-1}, e], \\
        H^\bullet (\upB \SO (2n + 1);\ R) & \simeq R[p_1, \dotsc, p_n],
    \end{aligned} \]
    其中 $\deg p_i = 4i,\ \deg e = 2n,\ e^2 = p_n$.
\end{theorem}

定理的证明方法和 (\ref{thm-2-main}) 的证明相同.
读者可以自己尝试, 也可以参见 \cite[\S3.5.3]{cohen}.

\subsection{分裂原理}

为了研究示性类与向量丛的张量积的关系,
我们需要分裂原理的帮助.
分裂原理可以把一般的向量丛的问题简化为线丛的问题.

\begin{theorem}[分裂原理] \label{thm-3-splitting}
    设 $E \to B$ 是 $\bbK^n$-丛, 系数环 $R$ 是 $\bbK$-可定向的.
    则存在拓扑空间 $Y$ 和映射 $f \: Y \to B$,
    使得拉回的向量丛 $f^* E \to Y$ 满足以下条件:
    \begin{itemize}
        \item
            拉回的向量丛 $f^* E \to Y$
            分裂为 $n$ 个线丛 (即 $\bbK^1$-丛) 的直和:
            \[ f^* E \simeq L_1 \oplus \cdots \oplus L_n. \]
        \item
            上同调的映射 
            \[ f^* \: H^\bullet (B;\ R) \to H^\bullet (Y;\ R) \]
            是单射, 从而 $f^* E$ 的示性类满足的关系也被 $E$ 的示性类满足.
    \end{itemize}
\end{theorem}

例如, 我们将证明对复线丛 $L_1, L_2$,
有 $c_1 (L_1 \otimes L_2) = c_1 (L_1) + c_1 (L_2)$.
从而, 对一般的复向量丛的张量积 $E_1 \otimes E_2$, 
我们就可以直接假设 $E_1, E_2$ 都是线丛的直和, 给出计算张量积的陈类的公式.
下一小节将讨论这个问题.

在证明分裂原理之前, 我们先叙述一个重要的步骤.
这个结论与 Thom 同构定理相似, 证明方法是相同的.

\begin{theorem}[Leray--Hirsch] \label{thm-3-leray-hirsch}
    设 $E \to B$ 是纤维为 $F$ 的纤维丛. 
    设有有限个元素
    \[ \alpha_1, \dotsc, \alpha_m \in H^\bullet (E;\ R), \]
    使得
    \begin{itemize}
        \item
            这些元素限制在每个纤维 $F$ 上,
            都自由地生成 $R$-模 $H^\bullet (F;\ R)$.
    \end{itemize}
    则这些元素也自由地生成 $H^\bullet (B;\ R)$-模 $H^\bullet (E;\ R)$.
    换言之,
    \[ H^\bullet (B;\ R) \underset{R}{\otimes} H^\bullet (F;\ R)
        \simeq H^\bullet (E;\ R). \thmqedhere \]
\end{theorem}

下面, 我们通过一个直接的构造, 证明分裂原理.

\begin{definition}
    设 $V$ 是 $n$ 维 $\bbK$-向量空间. $V$ 的一面\term{旗} (flag) 是一列线性子空间
    \[ 0 = V_0 \subset V_1 \subset \cdots \subset V_n = V, \]
    其中 $\dim V_i = i$. 空间 $V$ 的所有旗构成的空间记为 $\operatorname{Fl}(V)$,
    称为\term{旗流形} (flag manifold), 因为它有一个自然的流形结构.
\end{definition}

如果 $V$ 带有一个 Euclid 或 Hermite 度量, 
那么一面旗等价于一组互相正交的直线
\[ \ell_1,\ \ell_2,\ \dotsc,\ \ell_n \subset V, \]
满足 $V_i = \operatorname{span}(\ell_1, \dotsc, \ell_i)$.

在 (\ref{thm-1-functor}) 的意义下, 函子
\[ \operatorname{Fl} \: \cat{Vect}(n,\bbK) \to \cat{Top} \]
可以把一个向量丛 $E$
变成纤维为旗流形的纤维丛, 即\term{旗丛} (flag bundle),
我们记为 $\operatorname{Fl}(E)$.

\begin{proof}[定理 \ref{thm-3-splitting} 的证明]
    我们证明, 映射 $f \: \operatorname{Fl}(E) \to B$
    满足定理的条件.

    先证明 $f^* E$ 分裂成线丛的直和.
    在 $E$ 上取一个 Euclid 或 Hermite 度量,
    则旗可以由一组互相正交的直线给出. 我们有
    \[ f^* E = \{ 
        (x,\ \ell_1,\ \dotsc,\ \ell_n,\ v) \mid 
        x \in B,\ v \in E_x,\ \ell_i \subset E_x \text{ 是互相正交的直线}
    \}. \]
    定义 $f^* E$ 的子线丛
    \[ L_i = \{
        (x,\ \ell_1,\ \dotsc,\ \ell_n,\ v) \mid v \in \ell_i
    \} \subset f^* E. \]
    则
    \[ f^* E = L_1 \oplus \cdots \oplus L_n. \]

    再证明上同调的映射
    $f^* \: H^\bullet (B) \to H^\bullet (\operatorname{Fl}(E))$ 是单射.
    我们有图表
    \[ \begin{tikzcd}
        p_n^* \cdots p_1^* E \ar[d] \ar[r] \ar[dr, phantom, pos=.25, "\ulcorner"] &
        \cdots \ar[r] &
        p_2^* p_1^* E \ar[d] \ar[r] \ar[dr, phantom, pos=.25, "\ulcorner"] &
        p_1^* E \ar[d] \ar[r] \ar[dr, phantom, pos=.25, "\ulcorner"] &
        E \ar[d, "p"] \\ 
        \upP^n (E) \ar[r] &
        \cdots \ar[r] &
        \upP^2 (E) \ar[r, "p_2"] &
        \upP (E) \ar[r, "p_1"] &
        B \rlap{\ ,}
    \end{tikzcd} \]
    其中每个 $\upP^i (E)$ 定义为它右边的竖直映射代表的向量丛的\term{射影化} (projectivisation),
    也就是把纤维换成其射影空间得到的纤维丛.
    容易验证, $\upP^n (E) \simeq \operatorname{Fl} (E)$. 因此, 只需验证
    \[ p_1^* \: H^\bullet (B) \to H^\bullet (\upP (E)) \]
    是单射.

    设 $x \in B$. 考虑纤维丛 $\upP (E) \to B$ 的纤维
    $\upP (E_x) \simeq \bbK \upP^{n - 1}$.
    则向量丛 $p_1^* E$ 限制在 $\upP (E_x)$ 上,
    实际上是 $\bbK \upP^{n - 1}$ 上的平凡丛 $\bbK \upP^{n - 1} \times \bbK^n$.
    这个平凡丛有一个著名的子线丛, 叫做\term{自言线丛} (tautological line bundle),
    记作 $\scrO (-1)$, 定义为
    \[ \scrO (-1) = \{ 
        (\ell, x) \in \bbK \upP^{n-1} \times \bbK^n \mid 
        x \in \ell
    \}. \]
    记 $L$ 为 $p_1^* E$ 的子线丛,
    它由每个 $\upP (E_x) \simeq \bbK \upP^{n - 1}$ 上的自言线丛拼起来得到.
    为记号方便, 我们记 $c_1 (L)$ 为 $w_1 (L)$, $c_1 (L)$ 或 $p_1 (L)$ 中合适的那一个.
    考虑 $\upP (E)$ 的上同调类
    \[ 1,\ c_1(L),\ c_1(L)^2,\ \dotsc,\ c_1(L)^{n - 1}. \]
    它们限制在每个 $\upP (E_x)$ 上,
    都自由地生成 $\upP (E_x) \simeq \bbK \upP^{n-1}$ 的上同调
    (因为自言线丛是万有线丛的拉回, 细节留给读者).
    由 Leray--Hirsch 定理 (\ref{thm-3-leray-hirsch}),
    $H^\bullet (\upP (E))$ 是自由的 $H^\bullet (B)$-模,
    而 $p_1^*$ 作为模同态, 把生成元 $1 \in H^\bullet (B)$
    映到生成元之一 $1 \in H^\bullet (\upP (E))$.
    这就说明 $p_1^*$ 是单射.
\end{proof}

分裂原理有一种更抽象的形式, 即下面的命题.
为方便叙述, 我们只叙述 $\bbK = \bbC$ 的情况,
但其它情况也相同.

\begin{corollary} \label{thm-3-splitting-cor}
    映射 $\oplus \: \upB \upU (1)^n \to \upB \upU (n)$
    诱导了单射
    \[ H^\bullet (\upB \upU (n)) 
        \hookrightarrow H^\bullet (\upB \upU (1))^{\otimes n} 
        \simeq R [c_1^{(1)}, \dotsc, c_1^{(n)}], \]
    其像是所有对称多项式构成的子环. \qed
\end{corollary}

这个推论的含义是, 如果要研究 $n$ 维向量丛的示性类,
即 $H^\bullet (\upB \upU (n))$ 的上同调类,
我们可以不妨将它视为 $n$ 个线丛的直和,
然后用 Whitney 乘积公式计算出这个直和的示性类,
即 $H^\bullet (\upB \upU (1))^{\otimes n}$ 的元素.

\subsection{陈特征}

我们知道, 全陈类对向量丛的直和有积性.
通过这个性质, 我们可以构造出一种新的示性类, 叫做陈特征,
它对向量丛的直和有加性, 对张量积有积性.

在这一小节中, 设 $R$ 是一个包含 $\bbQ$ 的环.

\begin{definition} \label{def-3-ch}
    设 $E \to B$ 是复向量丛,
    它是 $n$ 个线丛的直和: $E \simeq L_1 \oplus \cdots \oplus L_n$.
    我们定义 $E$ 的\term{陈特征}为
    \[ \ch E = \upe^{c_1 (L_1)} + \cdots + \upe^{c_1 (L_n)}
        \quad \in \quad H^{2\bullet} (B) = \prod_{k = 0}^{\infty} H^{2k} (B), \]
    这里指数的含义是形式幂级数.
    
    对一般的复向量丛 $E \to B$,
    我们通过分裂原理定义其陈特征.
    事实上, 由 (\ref{thm-3-splitting-cor}),
    上面的表达式可以写成 $E$ 的陈类的形式幂级数.
    当然, 我们也可以具体地写下来:
    \[
        \ch E =
        n + \sum_{k = 1}^{\infty} \frac{1}{k!}
        \left| \begin{matrix}
            1 & 0 & \cdots &   & c_1 \\
            c_1 & 1 & 0 & \cdots & 2c_2 \\
            c_2 & c_1 & 1 &   & 3c_3 \\
            \vdots & \ddots & \ddots & \ddots & \vdots \\
            c_{k-1} & \cdots & c_2 & c_1 & kc_k 
        \end{matrix} \right|,
    \]
    这里 $c_i$ 代表 $c_i(E)$.
\end{definition}

\begin{theorem}
    设 $E_1, E_2$ 是空间 $B$ 上的两个复向量丛. 则
    \[ \begin{aligned}
        \ch (E_1 \oplus E_2) & = \ch E_1 + \ch E_2, \\
        \ch (E_1 \otimes E_2) & = \ch E_1 \cdot \ch E_2.
    \end{aligned} \]
\end{theorem}

\begin{proof}
    不妨假设两个向量丛都分裂成线丛的直和.
    则第一个等式是显然的.
    要证明第二个等式,
    我们只需对线丛 $L_1, L_2$ 证明
    \[ c_1 (L_1 \otimes L_2) = c_1 (L_1) + c_1 (L_2). \]
    事实上, 只需对万有线丛 $\upE \upU (1) \to \bbC \upP^{\infty}$ 证明结论.
    而包含映射 $\bbC \upP^1 \hookrightarrow \bbC \upP^{\infty}$
    诱导了上同调 $H^2$ 的单射, 因此, 只需证明 $\bbC \upP^1$ 上的自言丛 $\scrO (-1)$
    满足
    \[ c_1 (\scrO (-2)) = 2 \, c_1 (\scrO (-1)), \]
    其中 $\scrO (-2) = \scrO (-1)^{\otimes 2}$.
    
    剩下的工作就是计算这两个线丛的陈类.
    但是, 我们还没有准备好计算的工具, 即陈--Weil 理论.
    在 \S\ref{sect-5} 的理论框架下, 这一命题几乎是显然的.
    因此, 我们把证明推迟到 \S\ref{sect-5}.
\end{proof}

\begin{remark}
    对实向量丛和四元数向量丛而言,
    通过对应的 Понтрягин 类, 可以定义 \term{Понтрягин 特征},
    这种特征也有加性和乘性.

    对于 Stiefel--Whitney 类而言, 相应的特征无法定义,
    因为系数环为特征 $2$ 时, 指数映射 (作为形式幂级数) 无法定义.
    \varqed
\end{remark}

\begin{remark}
    之所以把自言线丛叫做 $\scrO (-1)$ 而不是 $\scrO (1)$,
    是因为前者的第一陈类是负的. 我们将在 \S\ref{sect-5} 中证明这一点.
    \varqed
\end{remark}

\subsection{\texorpdfstring{$K$}{K}-理论}

陈特征的加性和乘性启发我们,
可以将映射 $\ch$ 看成一个 ``环同态''
\[ \ch \: (\text{$B$ 上的向量丛},\ \oplus,\ \otimes)
    \to (H^{2 \bullet} (B),\ +,\ \cdot). \]
这一想法可以在 $K$-理论的框架里实现.
事实上, 在合适的假设下,
这一映射是一个 ``环同构''.

\begin{definition}
    设 $A$ 是一个 Abel 半群 (带有交换、结合乘法的集合, 不一定有单位元和逆元).
    它的\term{群化} (groupification) 是 Abel 群
    \[ \operatorname{G}(A) = \ 
        \frac{\bigoplus_{a \in A} \bbZ \cdot [a]}
        {[a + b] \sim [a] + [b]} \ . \]
\end{definition}

例如 $\operatorname{G}(\bbN, +) \simeq (\bbZ, +)$, 又例如
$\operatorname{G}(\bbZ \setminus \{0\},\ \times) \simeq (\bbQ \setminus \{0\},\ \times)$.

\begin{definition}
    设 $X$ 是拓扑空间. 我们定义
    \[ K(X) = \operatorname{G} (\operatorname{Bund}_{\cat{Vect}(\bbC)}(X),\ \oplus). \]
    向量丛的张量积诱导了 $K(X)$ 上的乘法结构,
    使它成为一个交换环, 其乘法单位元是平凡丛 $X \times \bbC$.
    函子 $K \: \cat{Top}^{\mathrm{op}} \to \cat{Ring}$
    称为 \term{$K$-理论}.
\end{definition}

例如, 在 $K(S^n)$ 中, 切丛 $TS^n$ 和平凡丛 $S^n \times \bbR^n$ 是同一个元素,
因为它们与平凡丛 $S^n \times \bbR$ 的直和是同构的.

现在, 陈特征
\[ \ch \: K(X) \to H^{2\bullet}(X) \]
就是一个严格意义下的环同态了.

\begin{remark}
    类似地, 我们可以考虑实向量丛或四元数向量丛,
    得到的函子分别称为 \term{KO 理论}和 \term{KSp 理论}. \varqed
\end{remark}

\begin{remark}
    将函子 $K$ 称为 ``理论'' 的原因是,
    它是一个\term{广义上同调理论}.
    也就是说, 它是系数环为谱环 (而不是普通的环) 的上同调理论.
    在广义上同调理论中, 也有示性类的理论,
    参见 \cite[16.27~之后]{switzer}. \varqed
\end{remark}

\begin{theorem} \label{thm-3-bu-times-z}
    设 $X$ 是紧空间. 则有集合的自然同构
    \[ K(X) \simeq [X,\ \upB \upU \times \bbZ], \]
    其中 $\upU = \colim \upU (n)$,
    记号 $[\ ,\ ]$ 表示映射的同伦类的集合.
\end{theorem}

见 \cite[定理~II.1.33]{karoubi}.

关于 $K$-理论最著名的定理是 Bott 周期律.
上一个结论帮助我们叙述这个大定理.

\begin{theorem}[Bott 周期律]
    我们有
    \[ \begin{aligned}
        \Omega (\upB \upU \times \bbZ) & \simeq \upU, \\
        \Omega \upU & \simeq \upB \upU \times \bbZ.
    \end{aligned} \]
    作为推论, 对任何紧空间 $X$, 我们有
    \[ K(\Sigma^2 X) \simeq K(X). \]
\end{theorem}

见 \cite[定理~2.11]{vbkt}.

\begin{remark}
    对于实数和四元数的情况, 也有 Bott 周期律,
    但需要作用 8 次 $\Omega$ (而不是 2 次) 才能回到原来的空间. \varqed
\end{remark}

\begin{remark} \label{rmk-3-bott}
    对于紧空间 $X$, 映射
    \[ \otimes (H-1) \: K(X) \to K(\Sigma^2 X) \]
    是同构, 其中 $\Sigma^2 X$ 等同于 $X \wedge S^2$,
    而 $H$ 表示 $S^2 \simeq \bbC \upP^1$ 的线丛 $\scrO (1)$,
    这里 $\otimes$ 就表示向量丛的张量积.
    这个结论是 Bott 周期律的另一形式, 我们在这里就不证明它了. \varqed
\end{remark}

\begin{corollary} \label{thm-3-k-sphere}
    我们可以计算出球面的 $K$-理论:
    \[ \begin{aligned}
        K(S^{2n}) & \simeq \bbZ \oplus \bbZ, \\
        K(S^{2n + 1}) & \simeq \bbZ.
    \end{aligned} \]
\end{corollary}

\begin{proof}
    显然, $K(S^0) \simeq \bbZ \oplus \bbZ$.
    利用 Bott 周期律就得到第一个同构.

    因为 $\pi_1 (\upB \upU) \simeq \pi_0 (\upU) \simeq 0$,
    所以由 (\ref{thm-3-bu-times-z}), 得到 $K(S^1) \simeq \bbZ$.
    利用 Bott 周期律就得到第二个同构.
\end{proof}

\begin{theorem}
    设 $X$ 是紧 CW 复形. 则有环同构
    \[ \ch \: K(X) \underset{\bbZ}{\otimes} \bbQ \simeq H^{2\bullet}(X;\ \bbQ). \]
\end{theorem}

\begin{proof}
    我们对胞腔的个数归纳.
    当胞腔个数是 $0$ 或 $1$ 时, 命题是显然的.
    胞腔个数是 $2$ 时, (\ref{thm-3-k-sphere}) 说明结论正确.

    设 $X = A \sqcup_{S^{n-1}} D^n$,
    其中 $A$ 比 $X$ 胞腔个数少, 但至少有 $2$ 个胞腔. 我们有 Puppe 正合列
    \[ A \to X \to X/A \to 
        \Sigma A \to \Sigma X \to \Sigma X / \Sigma A \to 
        \Sigma^2 A \to \cdots. \]
    由 $K$-理论的可表性 (\ref{thm-3-bu-times-z})
    和上同调函子 $H^\bullet$ 的可表性,
    序列的第三项至第七项诱导了两个长正合列的图表
    \[ \begin{adjustbox}{scale=.8}
        \begin{tikzcd}[column sep=1em]
            \cdots \ar[r] &
            K (\Sigma^2 A) \otimes \bbQ \ar[d, "\ch", "\simeq"'] \ar[r] &
            K (\Sigma X / \Sigma A) \otimes \bbQ \ar[d, "\ch", "\simeq"'] \ar[r] &
            K (\Sigma X) \otimes \bbQ \ar[d, "\ch"] \ar[r] &
            K (\Sigma A) \otimes \bbQ \ar[d, "\ch", "\simeq"'] \ar[r] &
            K (X / A) \otimes \bbQ \ar[d, "\ch", "\simeq"'] \ar[r] & \cdots \\
            \cdots \ar[r] &
            H^{2\bullet} (\Sigma^2 A;\ \bbQ) \ar[r] &
            H^{2\bullet} (\Sigma X / \Sigma A;\ \bbQ) \ar[r] &
            H^{2\bullet} (\Sigma X;\ \bbQ) \ar[r] &
            H^{2\bullet} (\Sigma A;\ \bbQ) \ar[r] &
            H^{2\bullet} (X / A;\ \bbQ) \ar[r] & \cdots \rlap{.}
        \end{tikzcd}
    \end{adjustbox} \]
    注意到 $\Sigma X$ 和 $X$ 胞腔个数相同,
    而另外四个空间的胞腔个数都更少.
    由五引理, 知要证的结论对 $\Sigma X$ 成立.
    同样的论述表明, 结论对 $\Sigma^2 X$ 成立.
    由 Bott 周期律, 结论对 $X$ 成立.
\end{proof}

这一定理告诉我们, 在合适的假设下, $K$-理论给出的信息和陈类是一样的.
具体地说, 两个向量丛的所有陈类相等,
当且仅当它们在 $K(X) \otimes \bbQ$ 中相同.
例如, 两种理论都无法区分球面的切丛和平凡丛.
