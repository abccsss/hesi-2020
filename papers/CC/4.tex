在这一节中, 我们首先介绍著名的 Poincaré--Hopf 定理,
然后介绍几个运用示性类解决问题的例子.

\subsection{Poincaré--Hopf 定理}

这一节的目标是证明 Poincaré--Hopf 定理,
它把向量场的零点和流形的拓扑联系起来.
这个结论的另一个形式是,
紧流形的切丛的 Euler 类就等于它的 Euler 示性数.
这也是 Euler 类命名的原因.

在本节中, 所有流形都默认是光滑的.

\begin{theorem}[Poincaré--Hopf] \label{thm-4-poincare-hopf}
    设系数环 $R$ 是域, 设 $M$ 是 $R$-已定向的紧、连通的流形. 则
    \[ e (TM) = \chi (M) = [M \cap M]_{TM}, \]
    其中 $\chi (M)$ 表示 $M$ 的 Euler 示性数
    (这个整数看作 $M$ 的一个最高次上同调类),
    $[M \cap M]_{TM}$ 表示切丛 $TM$ 中 $M$ 和自己的相交数.
\end{theorem}

\begin{remark}
    相交数 $[M \cap M]_{TM}$ 可以被看作 $M$ 上的向量场的\term{指标} (index).
    具体地说, 我们把其中一个 $M$ 看作零截面, 另一个看作向量场对应的截面.
    如果这两个截面横截相交, 那么这个相交数就等于向量场的零点的 ``加权个数''.
    \varqed
\end{remark}

在证明之前, 我们先给出几个例子.

\begin{corollary}[毛球定理]
    球面 $S^2$ 上不存在处处非零的连续向量场. \qed
\end{corollary}

我们也可以用 Euler 类来计算其它的示性类.

\begin{corollary}
    \ 
    \begin{itemize}
        \item
            如果 $R$ 是特征 $2$ 域, $M$ 是紧、连通的 $n$ 维流形, 则
            \[ w_n (TM) \equiv \chi (M) \pmod{2}. \]
        \item
            如果 $M$ 是紧、连通的 $n$ 维近复流形 (这包括所有的复流形), 则
            \[ c_n (TM) = \chi (M). \]
            特别地, 亏格 $g$ 的 Riemann 面的切丛的 $c_1$ 类等于 $2 - 2g$.
        \item
            如果 $M$ 是紧、连通的 $n$ 维近四元数流形 (例如 $\bbH \upP^n$), 则
            \[ p_n (TM) = \chi (M). \thmqedhere \]
    \end{itemize}
\end{corollary}

下面, 我们开始做一些定理证明的准备工作.

\begin{lemma}
    设 $M$ 是流形. 将 $M$ 通过对角映射嵌入 $M \times M$.
    则 $M$ 在 $M \times M$ 中的法丛同构于切丛 $TM$.
\end{lemma}

\begin{proof}
    在点 $x \in M$ 处, 这个同构由
    \[ \begin{aligned}
        T_x M & \to T_{(x, x)} (M \times M), \\
        v & \mapsto (v, -v)
    \end{aligned} \]
    给出.
\end{proof}

\begin{corollary}
    将 $M$ 通过对角映射嵌入 $M \times M$. 则
    \[ \widetilde{H}^\bullet ( \Th (TM) ) \simeq
        H^\bullet ( M \times M,\ M \times M \setminus M ). \thmqedhere \]
\end{corollary}

通过这个同构, $TM$ 的 Thom 类对应了一个上同调类
\[ u \in H^\bullet ( M \times M,\ M \times M \setminus M ) . \]
它称为\term{对角类},
它也可以看成 $H^\bullet (M \times M)$ 的元素.

Poincaré--Hopf 定理描述了流形的 Euler 类.
由于 Euler 类的构造,
它其实就是对角类 $u$ 在子流形 $M \subset M \times M$
上的限制. 我们的证明就通过研究对角类来完成.

\begin{remark}
    由 Thom 类的刻画,
    如果用一个微分形式来代表对角类,
    我们可以取支集在 $M$ 的一个管状邻域内,
    且方向垂直于 $M$ 的一个形式.
    读过 \cite{bott-tu} 的读者会发现, 
    这说明对角类是子流形 $M \subset M \times M$ 的 Poincaré 对偶.
    \varqed
\end{remark}

我们通过紧流形的上同调的 Poincaré 对偶, 来描述对角类.

\begin{theorem}[上同调的 Poincaré 对偶] \label{thm-4-poincare-duality}
    设系数环 $R$ 是域, 设 $M$ 是 $R$-已定向的紧、连通的 $n$~维流形.
    取 $R$-向量空间 $H^\bullet (M;\ R)$ 的一组基
    $b_1, \dotsc, b_r$, 使得每个 $b_i$ 都是齐次的 (即落在某个 $H^k$ 中). 则
    \begin{itemize}
        \item
            存在 $H^\bullet (M;\ R)$ 的一组对偶基 $\{ b_i^* \}$, 满足
            \[ b_i \smallcup b_j^* = \delta_{ij} \in H^n (M;\ R), \]
            这里 $\delta_{ij} \in H^n (M;\ R)$ 的含义是,
            它作用在 $M$ 的基本类 $[M] \in H_n (M;\ R)$ 上等于 $\delta_{ij}$.
        \item
            记 $|b_i| = \deg b_i$. 则
            \[ u = \sum_{i = 1}^r (-1)^{|b_i|} \, b_i \times b_i^* \in H^n (M \times M;\ R). \]
    \end{itemize}
\end{theorem}

证明这个定理之前, 我们先假设它成立, 证明 Poincaré--Hopf 定理.

\begin{proof}[定理 \ref{thm-4-poincare-hopf} 的证明]
    \allowdisplaybreaks
    记 $\Delta \: M \hookrightarrow M \times M$ 是对角映射. 则
    \begin{align*}
        e (TM) & = \Delta^* u \\
        & = \Delta^* \biggl( {} \sum_{i = 1}^r
            (-1)^{|b_i|} \, b_i \times b_i^* \biggr) \\
        & = \sum_{i = 1}^r (-1)^{|b_i|} \, b_i \smallcup b_i^*
            & (\text{由杯积的定义}) \\
        & = \sum_{i = 1}^r (-1)^{|b_i|} \\
        & = \chi (M).
    \end{align*}
    另一方面, 因为 $u$ 是子流形 $M$ 的 Poincaré 对偶,
    所以 $M \cap M$ 的 Poincaré 对偶是 $u \smallcup u$.
    在表达式 $u = \sum (-1)^{|b_i|} \, b_i \times b_i^*$
    中交换 $M \times M$ 的两个因子 $M$ 的地位,
    我们得到 $u = \sum b_i^* \times b_i$,
    这里 $(-1)^{|b_i|}$ 消失的原因是, 交换两个 $M$ 的次序之后,
    奇异单形的定向 (或者说, 微分形式相乘的次序) 不同. 因此,
    \begin{align*}
        u \smallcup u
        & = \sum_{i, j = 1}^r (-1)^{|b_i|} \,
            (b_i \times b_i^*) \smallcup (b_j^* \times b_j) \\
        & = \sum_{i, j = 1}^r (-1)^{|b_i| + |b_i^*| |b_j^*|} \,
            (b_i \smallcup b_j^*) \times (b_i^* \smallcup b_j) \\
        & = \sum_{i = 1}^r (-1)^{|b_i| + |b_i^*|} \,
            (b_i \smallcup b_i^*) \times (b_i^* \smallcup b_i) \\
        & = \sum_{i = 1}^r (-1)^{|b_i| + |b_i^*| + |b_i| |b_i^*|} \\
        & = \chi (M),
    \end{align*}
    其中最后一步通过讨论 $\dim M$ 的奇偶性得到.
\end{proof}

下面, 我们来证明 (\ref{thm-4-poincare-duality}).
证明的概要如下:
首先把 $u$ 写成 $\sum_{i = 1}^r b_i \times c_i$ 的形式.
通过 $u$ 的性质, 我们实际上能证明, $\{ (-1)^{|b_i|} \, c_i \}$
构成 $\{ b_i \}$ 的一组对偶基.
这样, 我们就在描述对角类的同时, 重新证明了 Poincaré 对偶定理.

\begin{definition}
    设 $X, Y$ 是拓扑空间. 我们定义\term{除积} (slant product)
    \[ {/} \: H^{p + q} (X \times Y) \otimes H_q (Y) \to H^p (X), \]
    定义的方法是将 $Y$ 的上同调和下同调进行配对.
    具体地说, 它是由奇异链复形、奇异上链复形的映射
    \[ S^\bullet (X \times Y) \otimes S_\bullet (Y)
        \to S^\bullet (X) \otimes S^\bullet(Y) \otimes S_\bullet (Y)
        \xrightarrow{\text{配对}} S^\bullet (X) \]
    诱导的映射.
\end{definition}

由这个定义, 如果 $a \in H^p (X)$, $b \in H^q (Y)$, $c \in H_q (Y)$,
那么
\[ (a \times b) / c = \langle b, c \rangle \cdot a. \]

\begin{lemma}
    设 $[M] \in H_n (M)$ 是 $M$ 的基本类. 则
    \[ u / [M] = 1 \ \in \ H^0 (M), \]
    其中 $u \in H^n (M \times M)$ 是之前定义的对角类.
\end{lemma}

\begin{proof}
    只需对一个 $x \in M$, 证明 $(u/[M])\big|_x = 1 \in H^0 (x)$. 我们有交换图
    \[ \begin{tikzcd}
        H^n (M \times M) \ar[d, "i_x^*"] \ar[r, "/ \lbrack M \rbrack"] & H^0 (M) \ar[d] \\
        H^n (x \times M) \ar[r, "/ \lbrack M \rbrack"] & H^0 (x) \rlap{ ,}
    \end{tikzcd} \]
    其中 $i_x \: x \times M \hookrightarrow M \times M$ 是含入映射.
    对角类 $u$ 在图表的左上角, 它在右下角的像是 $u/[M]$ (先右后下),
    也是 $(1 \otimes i_x^* u) / [M] = \langle i_x^* u, [M] \rangle$ (先下后右).

    另一方面, 我们有四个拓扑空间对构成的交换图
    \[ \begin{tikzcd}
        ( x \times M,\ \emptyset ) \ar[d, hook, "i_x"] \ar[r] &
        ( x \times M,\ x \times M \setminus x \times x ) \ar[d, hook, "j_x"] \\ 
        ( M \times M,\ \emptyset ) \ar[r] &
        ( M \times M,\ M \times M \setminus M ) \rlap{ ,}
    \end{tikzcd} \]
    其中右下角的 $M$ 是 $M \times M$ 中的对角线.
    我们将 $u$ 视为 $H^n (M \times M,\ M \times M \setminus M)$ 的元素, 则
    \[ \langle i_x^* u, [M] \rangle = \langle j_x^* u, [M] \rangle = 1, \]
    其中第二个 $[M]$ 是 $H^n (M,\ M \setminus x)$ 的元素,
    第一个等号是由于上面的交换图,
    第二个等号是因为 $j_x^* u$ 本质上是 Thom 类在纤维上的限制.
\end{proof}

\begin{lemma}
    设 $a \in H^\bullet (M)$ 是任一上同调类. 则在 $M \times M$ 的上同调环中,
    \[ (a \otimes 1) \smallcup u = (1 \otimes a) \smallcup u. \]
\end{lemma}

\begin{proof}
    取对角线 $M \subset M \times M$ 的管状邻域 $N$,
    并记 $i \: N \hookrightarrow M \times M$ 为含入映射.
    通过选取合适的 $N$, 我们可以保证向两个分量的投影
    \[ p_1 |_N , \ p_2 |_N \: N \to M \]
    是同伦的 (请读者验证). 这说明
    \[ i^* (a \otimes 1) = i^* (1 \otimes a). \]
    通过交换图
    \[ \begin{tikzcd}
        H^\bullet (M \times M) \ar[d, "\smallcupcd u"] \ar[r, "i^*"] &
        H^\bullet (N) \ar[d, "\smallcupcd u"] \\
        H^{\bullet+n} (M \times M,\ M \times M \setminus M) \ar[r, "\simeq", "\text{切除}"'] &
        H^{\bullet+n} (N,\ N \setminus M) \rlap{ ,}
    \end{tikzcd} \]
    我们就得到了要证的等式.
\end{proof}

\begin{proof}[定理 \ref{thm-4-poincare-duality} 的证明]
    和上面一样, 记 $u \in H^n (M \times M)$ 为对角类. 我们将它写成
    \[ \textstyle u = \sum_i b_i \otimes c_i \]
    的形式 (这是因为 Künneth 公式中的 Tor 部分消失).

    设 $a \in H^\bullet (M)$ 是任一上同调类. 一方面, 我们有
    \[ \begin{aligned}
        & \bigl( (a \otimes 1) \smallcup u \bigr) \big/ [M] \\
        ={} & (a \otimes 1) / [M] \, \smallcup \, u / [M] \\
        ={} & a \smallcup 1 = a,
    \end{aligned} \]
    而另一方面, 我们有
    \[ \begin{aligned}
        & \bigl( (1 \otimes a) \smallcup u \bigr) \big/ [M] \\
        ={} & \textstyle \sum_i {} \bigl( (1 \otimes a) \smallcup (b_i \otimes c_i) \bigr) \big/ [M] \\
        ={} & \textstyle \sum_i {} (-1)^{|a| |b_i|} \bigl( (1 \smallcup b_i) \otimes (a \smallcup c_i) \bigr) \big/ [M] \\
        ={} & \textstyle \sum_i {} (-1)^{|a| |b_i|} \langle a \smallcup c_i, \ [M] \rangle \, b_i.
    \end{aligned} \]
    由引理, 上面两个式子是相等的. 我们取 $a = b_j$, 并比较式中 $b_i$ 的系数, 得到
    \[ (-1)^{|b_i| |b_j|} \langle b_j \smallcup c_i, \ [M] \rangle = \delta_{ij}. \]
    因此, 取 $b_i^* = (-1)^{|b_i|} c_i$, 就能满足定理的要求.
\end{proof}


\subsection{\texorpdfstring{$\bbR \upP^n$ 能否浸入 $\bbR^N$}{ℝPⁿ 能否浸入 ℝᴺ}}

\begin{theorem} \label{thm-4-immersion}
    设 $n = 2, 4, 8, 16, \dotsc$.
    则 $\bbR \upP^n$ 能够浸入 $\bbR^N$ 当且仅当
    $N \geq 2n - 1$.
\end{theorem}

在流形理论中, \term{Whitney 浸入定理}说明,
当 $n \geq 2$ 时, 每个 $n$ 维流形都能浸入 $\bbR^{2n - 1}$.
我们的定理说明, 这个线性界 $2n - 1$ 是最佳的.

定理的证明思路是, 如果这样的浸入存在,
那么切丛和法丛的直和就是平凡丛. 因此, 由 Whitney 乘积公式, 得到
\[ w (T \bbR \upP^n) \, w (\text{法丛}) = 1. \]
通过这个等式, 我们能给出法丛的秩的下界.

首先, 我们计算射影空间的切丛的示性类.

\begin{theorem} \label{thm-4-tcpn}
    设系数环是 $\bbK$-可定向的. 如果 $n \geq 1$,
    那么当 $\bbK = \bbR, \bbC, \bbH$ 时分别有
    \[ \begin{aligned}
        w (T \bbR \upP^n) & = (1 + x)^{n + 1}, \\
        c (T \bbC \upP^n) & = (1 + y)^{n + 1}, \\
        p (T \bbH \upP^n) & = (1 + z)^{n + 1}, \\
    \end{aligned} \]
    其中 $x, y, z$ 分别是
    $H^1 (\bbR \upP^n),\ H^2 (\bbC \upP^n),\ H^4 (\bbH \upP^n)$
    的生成元. 这里, 我们赋予 $T \bbH \upP^n$ 自然的四元数向量丛结构.
\end{theorem}

\begin{proof}
    我们只叙述 $\bbK = \bbR$ 情况的证明, 其它情况同理.

    考虑流形 $\bbR^{n + 1} \setminus \{0\}$ 的切丛
    \[ T = T(\bbR^{n + 1} \setminus \{0\})
        \simeq (\bbR^{n + 1} \setminus \{0\}) \times \bbR^{n + 1}. \]
    实数乘法群 $\bbR^\times$ 通过 (对所有分量) 数乘作用在 $T$ 上.
    作为 $\bbR \upP^n$ 上的向量丛, 有
    \[ T / \bbR^\times \simeq T \bbR \upP^n \oplus \bbR, \]
    其中最后一个 $\bbR$ 表示 $\bbR \upP^n$ 上的平凡丛,
    它来自于径向的切向量.
    另一方面, 作为 $\bbR \upP^n$ 上的向量丛, 有
    \[ \bigl( (\bbR^{n + 1} \setminus \{0\}) \times \bbR \bigr) / \bbR^\times
        \simeq \scrO (-1), \]
    其中 $\bbR^\times$ 的作用是数乘. 于是,
    \[ T \bbR \upP^n \oplus \bbR \simeq T / \bbR^\times
        \simeq \scrO (-1) ^{\oplus (n + 1)}. \]
    在这一等式中, 计算两边的全 Stiefel--Whitney 类, 得到
    \[ w (T \bbR \upP^n) = w (\scrO (-1))^{n + 1} = (1 + x)^{n + 1}. \qedhere \]
\end{proof}

\begin{proof}[定理 \ref{thm-4-immersion} 的证明]
    Whitney 浸入定理说明了后一条件的充分性.
    为证必要性, 假设存在这样的浸入.
    则由之前的讨论, 注意到 $n$ 是 $2$ 的幂, 有
    \[ \begin{aligned}
        w(\text{法丛})
        & = w^{-1} (T \bbR \upP^n) \\
        & = \bigl( (1 + x)^{n + 1} \bigr)^{-1} \\
        & = (1 + x + x^n + \cancel{x^{n+1}})^{-1} \\
        & = 1 + (x + x^n) + (x + x^n)^2 + \cdots \\
        & = 1 + x + x^2 + \cdots + x^{n-1}.
    \end{aligned} \]
    因此, 法丛的秩至少是 $n - 1$, 从而 $N \geq 2n - 1$.
\end{proof}


\subsection{球面的近复结构}

下面, 我们利用示性类的计算, 来判断哪些球面具有近复结构.
和之前一样, 本节中的流形都是指光滑流形.
本节的内容主要参考了 \cite{spheres}.

\begin{definition}
    设 $M$ 是 $2n$ 维流形. 它的一个\term{近复结构} (almost complex structure)
    是一个将切丛 $TM$ 看成 $n$ 维复向量丛的方法.
\end{definition}

根据这个定义, 复流形都有近复结构.
但是, 有近复结构的流形不一定有复结构.

我们先叙述这一节的主要定理:

\begin{theorem} \label{thm-4-sphere}
    球面 $S^n$ 有近复结构, 当且仅当 $n = 0, 2, 6$.
\end{theorem}

球面 $S^0$ 和 $S^2$ 事实上都有复结构, 后者的复结构由 $\bbC \upP^1$ 给出.
至于 $S^6$ 是否有复结构, 则是一个未解难题.
著名的 Michael Atiyah 在 2016 年宣称 $S^6$ 没有复结构, 但他的论证未被大多数人接受.

定理证明的概要是, 先利用示性类的理论, 排除 $n$ 的所有其它可能性,
然后再在 $S^6$ 上构造一个近复结构.

\begin{lemma} \label{lem-4-sphere-1}
    当 $k \geq 1$ 时, 球面 $S^{4k}$ 没有近复结构.
\end{lemma}

\begin{proof}
    \allowdisplaybreaks
    假设 $S^{4k}$ 有近复结构, 记 $T$ 为相应的复向量丛,
    使得 $T$ 对应的实向量丛 $T_{\bbR}$ 与 $TS^{4k}$ 同构. 则
    \begin{align*}
        1 &= 1 + (-1)^k \, p_k (T_{\bbR}) && \text{(\ref{eg-3-pon-sphere})} \\
        &= c (T_{\bbR} \otimes \bbC) && \text{(由 $p_k$ 的定义)} \\
        &= c (T \oplus \Tbar) && (T_{\bbR} \otimes \bbC \simeq T \oplus \Tbar) \\
        &= (1 + c_{2k} (T)) \, (1 + c_{2k} (\Tbar)) \\
        &= (1 + e (T)) \, (1 + e (\Tbar)) \\
        &= (1 + e (T))^2 && (T_{\bbR} \simeq (\Tbar)_{\bbR}) \\
        &= 1 + 2e (T).
    \end{align*}
    这说明 $e (T) = 0$, 这与 Poincaré--Hopf 定理矛盾.
\end{proof}

\begin{lemma} \label{lem-4-sphere-2}
    当 $k \geq 4$ 时, 球面 $S^{2k}$ 没有近复结构.
\end{lemma}

\begin{proof}
    设 $T$ 为 $S^{2k}$ 上的复向量丛, 使得 $T_{\bbR} \simeq TS^{2k}$ 同构.
    我们知道 $c (T) = 1 + c_k (T) = 1 + 2x$, 其中 $x$ 是 $H^{2k} (S^{2k}; \bbZ) \simeq \bbZ$ 的生成元 $1$.
    因此, 由 (\ref{def-3-ch}) 给出的公式, $T$ 的陈特征是
    \[ \ch (T) = k + \frac {2} {(k-1)!} x. \]
    我们断言系数 $2/(k-1)! \in \bbZ$, 从而必有 $k < 4$.

    我们证明更一般的结论: 陈特征映射
    \[ \ch \: K (S^{2k}) \to H^{2\bullet} (S^{2k}; \bbQ) \]
    的像都是整上同调类, 即 $H^{2\bullet} (S^{2k}; \bbZ)$ 的元素.

    为此, 我们引入\term{约化 $K$-理论}: 对带基点的拓扑空间 $X$, 定义
    \[ \widetilde{K} (X) := \ker \Bigl( \operatorname{rank} \: K(X) \to \bbZ \Bigr), \]
    其中映射 $\operatorname{rank}$ 定义为在 $X$ 的基点处取向量丛的秩.
    对于紧空间 $X$, 由 (\ref{thm-3-bu-times-z}), 我们有
    \[ \widetilde{K} (X) \simeq [X, \upB \upU \times \bbZ]_*, \]
    其中记号 $[\ ,\ ]_*$ 表示固定基点的同伦类的集合.

    由 Bott 周期律 (\ref{rmk-3-bott}), 对紧空间 $X$, 我们有交换图
    \[ \begin{tikzcd}[column sep=4em]
        \widetilde{K} (X) \ar[d, "\ch"] \ar[r, "\otimes \, (H-1)", "\simeq"'] &
        \widetilde{K} (\Sigma^2 X) \ar[d, "\ch"] \\
        \widetilde{H}^{2\bullet} (X; \bbQ) \ar[r, "\otimes \ch (H-1)", "\simeq"'] &
        \widetilde{H}^{2\bullet} (\Sigma^2 X; \bbQ) \rlap{ ,}
    \end{tikzcd} \]
    其中 $\Sigma^2 X$ 等同于 $X \wedge S^2$,
    而 $H$ 是 $S^2 \simeq \bbC \upP^1$ 的线丛 $\scrO (-1)$.
    注意 $\ch (H-1) = (1 + c_1(H)) - 1 = c_1 (H)$ 就是 $H^2 (S^2) \simeq \bbZ$ 的生成元,
    它是一个整上同调类.

    我们用归纳法, 在交换图中将 $X$ 取成 $S^0$, $S^2$, $S^4$, $\dotsc$, 就完成了证明.
\end{proof}

\begin{proof} [定理 \ref{thm-4-sphere} 的证明]
    \newcommand*{\ImO}{\operatorname{Im}\bbO}
    \newcommand*{\vbarshort}{\oline1{v \mkern-1mu} \mkern1mu}
    由 (\ref{lem-4-sphere-1}), (\ref{lem-4-sphere-2}),
    我们只需要在 $S^6$ 上构造一个近复结构.

    这个构造用到了八元数的集合 $\bbO$.
    我们记 $\ImO \simeq \bbR^7$ 为虚八元数的集合,
    然后考虑单位球面 $S^6 \subset \ImO$.
    对每个 $p \in S^6$, 切空间 $T_p S^6$ 可以看作 $\ImO$ 的子空间 (切空间的原点和 $\ImO$ 的原点对齐).
    为了给出一个近复结构, 我们只需在每个 $p \in S^6$ 定义一个线性变换
    \[ J_p \: T_p S^6 \to T_p S^6, \]
    作为复数 $\upi$ 的数乘, 满足 $J_p^2 = -1$.

    我们列举一些八元数的性质.
    八元数的乘法不满足交换律和结合律. 对 $u,v \in \bbO$,
    我们有 $\oline1{uv} = \vbarshort \oline2u$.
    并且, 它们的欧氏内积由 $2 \langle u, v \rangle = u \oline1v + v \oline1u$ 给出.
    因此, $u \perp v$ 当且仅当 $u \oline1v = -v \oline1u$.
    当 $u, v \in \ImO$ 时, 我们有 $\oline1u = -u$, 以及 $\oline1v = -v$,
    故此时 $u \perp v$ 等价于 $uv = -vu$.

    对 $v \in T_p S^6$, 我们就定义
    \[ J_p (v) = pv. \]
    我们需要验证这个构造满足要求. 首先, 我们有
    \[ \oline2{pv} = \vbarshort \oline3p = (-v) (-p) = v p = -p v, \]
    从而 $pv \in \ImO$. 又
    \[ p(pv) = -p(vp) = -(pv)p, \]
    其中第二个等号是由于, 三个八元数相乘时, 若其中两个相等, 则结合律成立.
    从而, $pv \perp p$. 最后,
    \[ J_p^2 v = p(pv) = (pp)v = -(p\oline3p) v = -|p|^2 v = -v. \]
    因此, $J_p^2 = -1$.
\end{proof}


\subsection{代数曲线的亏格}

设 $f (x, y, z)$ 是复系数 $n$ 次齐次多项式. 方程
\[ f (x, y, z) = 0 \]
在射影平面 $\bbC \upP^2$ 中画出一条曲线, 称为 $n$ 次\term{代数曲线} (algebraic curve).

我们假定这条代数曲线是\term{光滑}的, 也就是说,
它是一个紧一维复流形, 即紧 Riemann 面.
当 $n = 2$ 时, 我们得到的就是用来折磨高中生的圆锥曲线,
它对应的 Riemann 面是球面.
当 $n = 3$ 时, 我们得到的是三次曲线, 又称为\term{椭圆曲线} (elliptic curve).
它的拓扑是圆环面, 可以看作复平面 $\bbC$ 的商空间, 复数加法赋予了它一个群结构.

Riemann 面的拓扑由其亏格 $g$ 决定. 本节中, 我们将导出 $g$ 与 $n$ 的关系.

\begin{theorem} \label{thm-4-genus}
    光滑 $n$ 次代数曲线的亏格是
    \[ g = \frac12 (n-1) (n-2). \]
\end{theorem}

如果读者不熟悉代数几何或复几何, 我们介绍一个事实:
对每个 $d \in \bbZ$, 射影空间 $\bbC \upP^n$ 都有一个线丛 $\scrO (d)$,
它的所有截面就是 $\bbC^{n+1} \setminus \{0\}$ 上所有的 $d$~次齐次复值函数.
当 $d = -1$ 时, 我们就得到了自言线丛 $\scrO (-1)$.

\begin{lemma} \label{lem-4-normal-bundle}
    设 $X \subset \bbC \upP^2$ 是光滑 $n$ 次代数曲线.
    则法丛 $N$ 同构于 $\scrO (n) |_X$.
\end{lemma}

\begin{proof}
    设 $X$ 的方程是 $f (x, y, z) = 0$.
    将 $f$ 看作 $\bbC^3 \setminus \{0\}$ 上的复值函数,
    则其微分 $df$ 是 $n-1$ 次齐次 $1$-形式.

    我们知道, $\bbC \upP^2$ 上的 $1$-形式对应着
    $\bbC^3 \setminus \{0\}$ 上的 $-1$ 次齐次 $1$-形式.
    这说明 $df$ 是 $\bbC \upP^2$ 上的向量丛
    \[ T^* \bbC \upP^2 \otimes \scrO (n) \]
    的截面.

    我们将法丛 $N$ 看成商丛 $(T \bbC \upP^2 |_X) / TX$.
    则余法丛 $N^{\vee}$ (法丛的对偶) 是 $T^* \bbC \upP^2 |_X$ 的子丛,
    它由所有把 $X$ 的切向量变成 $0$ 的线性函数构成.
    而 $df$ 就满足这个条件. 因此, $df$ 是 $X$ 上的线丛
    \[ L = N^\vee \otimes \scrO (n) |_X \]
    的截面, 且在 $X$ 上处处非零.
    这说明 $L$ 是平凡丛. 在等式两边与 $\scrO (-n) |_X$ 做张量积, 我们得到
    \[ \scrO (-n) |_X \simeq N^\vee, \]
    也就是 $N \simeq \scrO (n) |_X$.
\end{proof}

\begin{proof} [定理 \ref{thm-4-genus} 的证明]
    \allowdisplaybreaks
    设 $X \subset \bbC \upP^2$ 是光滑 $n$ 次代数曲线. 
    由 (\ref{lem-4-normal-bundle}), 其法丛是 $\scrO (n) |_X$. 因此,
    \[ T \bbC \upP^2 |_X \simeq TX \oplus \scrO (n) |_X. \]
    取全陈类, 由 (\ref{thm-4-tcpn}), 有
    \begin{align*}
        c (TX) &= \frac {c (T \bbC \upP^2 |_X)} {c (\scrO (n) |_X)} \\
        &= \frac {(1 + x)^3} {1 + nx} \\
        &= (1 + 3x) (1 - nx) \\
        &= 1 + (3 - n) x,
    \end{align*}
    其中 $x$ 是 $H^2 (\bbC \upP^2)$ 的生成元在 $H^2 (X)$ 中的像.
    我们有
    \[ \langle x,\ [X] \rangle = n, \]
    这是因为, 这个配对等于 $X$ 与 $x$ 的 Poincaré 对偶 (即一条直线) 的相交数.
    最后, Poincaré--Hopf 定理告诉我们
    \[ 2 - 2g = \chi (X) = (3 - n) n, \]
    这蕴涵了计算 $g$ 的式子.
\end{proof}


\subsection{配边和示性数}

下面, 我们介绍一个深刻的结论, 即 Понтрягин--Thom 定理.
它将示性类和流形的配边联系起来, 从而, 我们可以通过示性类,
来判断一个流形是否是某个带边流形的边界.

和之前一样, 在本节中, 流形默认是光滑、不带边的.

\begin{definition}
    设 $M$ 是紧 $n$ 维 (i) 实, (ii) 复, (iii) 实流形.
    设 $I = (i_1, \dotsc, i_r)$ 是一组正整数, 满足 $\sum_k i_k =$ (i, ii) $n$, (iii) $n/4$. 则
    \begin{enumerate}
        \item
            $M$ 的第 $I$ 个 \term{Stiefel--Whitney 数}定义为模 $2$ 的整数
            \[ w_I (M) = \bigl\langle w_{i_1} (TM) \cdots w_{i_r} (TM) ,\ [M] \bigr\rangle \pmod {2}. \]
        \item
            $M$ 的第 $I$ 个\term{陈数}定义为整数
            \[ c_I (M) = \bigl\langle c_{i_1} (TM) \cdots c_{i_r} (TM) ,\ [M] \bigr\rangle. \]
        \item
            $M$ 的第 $I$ 个 \term{Понтрягин 数}定义为整数
            \[ p_I (M) = \bigl\langle p_{i_1} (TM) \cdots p_{i_r} (TM) ,\ [M] \bigr\rangle. \]
    \end{enumerate}
\end{definition}

\begin{definition}
    设 $M, N$ 是两个已定向的 $n$ 维实流形. 它们的一个\term{配边} (cobordism)
    是一个带边 $n+1$ 维流形 $W$, 满足 
    \[ \partial W = (-M) \sqcup N, \]
    其中 $-M$ 表示 $M$ 的相反定向, $\sqcup$ 表示不交并.
    此时, 我们说 $M$ 与 $N$ 是\term{可配边} (cobordant) 的.
\end{definition}

例如, 一个流形与 $\emptyset$ 可配边, 等价于它是某个带边流形的边界.

\begin{theorem}
    设实流形 $M$ 是某个带边流形的边界.
    则 $M$ 的所有 Stiefel--Whitney 数和所有 Понтрягин 数 (如果能定义的话) 都等于 $0$.

    特别地, 如果实流形 $M, N$ 可配边,
    那么它们的所有示性数都相等.
\end{theorem}

\begin{proof}
    \allowdisplaybreaks
    设 $M = \partial W$. 则
    \[ TW |_M \simeq TM \oplus \bbR, \]
    其中 $\bbR$ 是秩为 $1$ 的平凡丛,
    它由 $\partial W$ 上指向 $W$ 外侧的向量场给出.
    这说明 $TW |_M$ 和 $TM$ 的所有示性类都相等.

    考虑上同调的长正合列
    \[ \cdots \to H^n (W) 
        \overset{i^*}{\longrightarrow} H^n (M) 
        \overset{\delta}{\longrightarrow} H^{n+1} (W, M) \to \cdots , \]
    其中 $n = \dim M$. 由上面的讨论, $H^n (W)$ 的元素
    $w_{i_1} (TW) \cdots w_{i_r} (TW)$ 被 $i^*$ 映到
    $w_{i_1} (TM) \cdots w_{i_r} (TM)$, 它被 $\delta$ 映到 $0$. 这说明
    \begin{align*}
        0 &= \bigl\langle \delta \bigl( w_{i_1} (TM) \cdots w_{i_r} (TM) \bigr) ,\ [W] \bigr\rangle
            && \bigl( [W] \in H_{n+1} (W, M) \bigr) \\
        &= \bigl\langle w_{i_1} (TM) \cdots w_{i_r} (TM) ,\ \partial [W] \bigr\rangle
            && \bigl( \partial [W] \in H_n (M) \bigr) \\
        &= \bigl\langle w_{i_1} (TM) \cdots w_{i_r} (TM) ,\ [M] \bigr\rangle \\
        &= w_I (M).
    \end{align*}
    同样的讨论对 Понтрягин 数也成立.

    至于最后一个命题, 只需注意到 $M, N$ 可配边等价于 $(-M) \sqcup N$ 是某个流形的边界,
    而 $-M$ 的示性数与 $M$ 相差一个符号, 因为 $[M]$ 反号了.
\end{proof}

这个定理的逆向是一个更深刻的结论,
我们在这里就不证明了.

\begin{theorem}[Понтрягин--Thom]
    实流形 $M$ 是某个带边流形的边界,
    当且仅当 $M$ 的所有 Stiefel--Whitney 数都等于 $0$.
\end{theorem}

最后, 我们再介绍一个有趣的应用.

\begin{proposition}
    如果 $4n$ 维流形 $M$ 有一个反定向的自微分同胚,
    那么它的所有 Понтрягин 数都等于 $0$.
\end{proposition}

\begin{proof}
    我们已经知道, 如果将定向反过来, 那么示性数会反号.
    因此, $M$ 的所有示性数都和它的相反数相等.
\end{proof}

借助这个简单的结论, 我们可以找到一些 ``不能内外翻转'' 的流形.

\begin{exercise}
    通过计算示性数, 证明 $\bbC \upP^{2n}$ 没有反定向的自微分同胚,
    也不是另一个流形的边界.
\end{exercise}

