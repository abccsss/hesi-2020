从这一节开始, 我们从一个全新的角度出发, 来重新建造示性类的理论.
这个角度就是微分几何的观点.
我们将研究流形上的向量丛和主丛,
并通过它们的几何量, 即曲率形式, 来定义示性类.
这样定义出的示性类是一些闭微分形式, 其 de~Rham 上同调类就是我们之前研究的示性类.
这一套理论被称为\term{陈--Weil 理论}.

通过这种方式, 我们能通过流形的几何量, 来计算它的拓扑.
这种想法的一个经典的形式是 \term{Gauß--Bonnet 公式}:
对于二维可定向紧 Riemann 流形 $M$, 我们有
\[ \int_M K \, d S = 2 \uppi \, \chi (M), \]
其中 $K$ 是 Gauß 曲率, $dS$ 是面积元.
我们可以把陈--Weil 理论看作这个公式的一般形式.

从本节开始, 所有流形都默认是光滑、不带边的.


\subsection{联络和曲率}

\begin{definition}
    设 $M$ 是流形. $M$ 上的一个\term{光滑向量丛} (smooth vector bundle)
    是指一个 (实或复) 向量丛
    \[ E \to M, \]
    其转移映射都是 (从 $M$ 的开集出发的) 光滑映射.
    此时, $E$ 具有一个自然的光滑结构.
\end{definition}

在下面的讨论中, 设 $E \to M$ 是一个光滑 $\bbK$-向量丛, 其中 $\bbK = \bbR, \bbC$.
设 $M$ 的维数为 $n$, $E$ 的秩为 $m$.
我们用记号 $\Gamma (E)$ 表示 $E$ 的所有光滑截面的集合.

向量丛 $E$ 上的一个\term{联络}就是一个将
$E$ 的每个纤维与它周围的纤维对齐的方法.
如果我们知道如何对齐这些纤维, 我们就可以对 $E$ 的截面取方向导数.
这解释了下面的定义.

\begin{definition}
    $E$ 上的一个\term{联络} (connection) 是一个映射
    \[ \begin{aligned}
        \nabla \: \Gamma (TM) \otimes \Gamma (E) &\to \Gamma (E), \\
        (X, \ s) &\mapsto \nabla_X s, 
    \end{aligned} \]
    满足以下公理:
    \begin{itemize}
        \item
            $\nabla_X s$ 关于 $X$ 是 $C^\infty (M)$-线性的,
            关于 $s$ 是 $\bbK$-线性的.
        \item (\term{Leibniz 法则})
            如果 $X \in \Gamma (TM)$, $s \in \Gamma (E)$, $f \in C^\infty (M, \bbK)$, 那么
            \[ \nabla_X (fs) = (Xf) \, s + f \, \nabla_X s. \]
            换言之, 作为 $E$-取值的 $1$-形式 (即 $T^* M \otimes E$ 的截面), 有
            \[ \nabla (fs) = (df) \, s + f \, \nabla s. \]
    \end{itemize}
    如果 $E$ 的纤维带有 (Euclid 或 Hermite) 度量, 我们还可以要求:
    \begin{itemize}
        \item (\term{Leibniz 法则})
            如果 $X \in \Gamma (TM)$, $s_1, s_2 \in \Gamma (E)$, 那么
            \[ X \, \langle s_1, s_2 \rangle =
                \langle \nabla_X s_1, s_2 \rangle + \langle s_1, \nabla_X s_2 \rangle. \]
    \end{itemize}
    此时, $\nabla$ 称为一个\term{度量联络} (metric connection).
\end{definition}

在实际的计算中, 上面的定义常常是不好用的.
我们将联络的局部信息提取出来, 成为一些微分形式, 以便于直接计算截面的方向导数.

\begin{definition}
    设开集 $U \subset M$ 上 $E$ 平凡, 即 $E |_U \simeq U \times \bbK^m$.
    设 $e_1, \dotsc, e_m$ 是由 $\bbK^m$ 的基给出的 $E |_U$ 的 $m$ 个截面. 设
    \[ \nabla e_j = \omega_j^i e_i, \]
    其中 $\omega_j^i \in \Omega^1 (U, \bbK)$.
    这 $m^2$ 个 $1$-形式称为联络 $\nabla$ 的\term{联络形式} (connection form).
    我们可以把 $\omega$ 看作
    \[ \Omega^1 \bigl( U, \ \End (E) \bigr) =
        \Gamma \bigl( T^* U \otimes \End (E) \bigr) \]
    的截面, 因为它是由 $m^2$ 个 $1$-形式构成的矩阵, 也就是一个取值为矩阵的 $1$-形式.
\end{definition}

需要注意的是, $\omega$ 并不能定义出整个 $M$ 上的 $1$-形式,
因为它不符合 $1$-形式的坐标变换法则.

\begin{remark}
    为了直观地理解联络形式, 我们把 $\End (E)$ 与 Lie 代数 $\gl (E)$ 等同起来.
    对任意向量场 $X \in \Gamma (TM)$, 局部截面
    \[ \omega (X) \in \Gamma \bigl( U, \gl (E) \bigr) \]
    给出了从某一点 $x \in U$ 出发, 沿着 $X$ 的方向走无穷小的距离时,
    得到 $E$ 的纤维的无穷小自同构, 即 $\GL (E_x)$ 的一个无穷小元素.
    这个自同构就是由联络所指定的将相邻纤维等同起来的方法.
    
    事实上, 对于一般的 Lie 群 $G$, 我们也可以谈论 $G$-主丛上的联络,
    其联络形式是取值于 Lie 代数 $\mathfrak{g}$ 的形式.
    为了对用户友好, 我们在行文中总是取 $G = \GL (n)$,
    但一般的 Lie 群的情况其实大同小异. \varqed
\end{remark}

现在, 我们就可以在局部上计算出截面的方向导数了:
\[ \nabla_X (s^j e_j) = X^\alpha (\partial_\alpha s^j) \, e_j 
    + s^j X^\alpha (\omega_j^i)_\alpha \, e_i, \]
其中希腊字母表示流形方向的坐标, 罗马字母表示纤维方向的坐标.
记号 $(\omega_j^i)_\alpha$ 表示 $1$-形式 $\omega_j^i$ 的 $\alpha$-分量.
注意到, $\omega$ 描述了联络给出的方向导数和局部坐标下的导数的差异.

\begin{definition}
    联络 $\nabla$ 的\term{曲率形式} (curvature form)
    是 $m^2$ 个 $2$-形式, 定义为
    \[ \Omega_j^i = d \omega_j^i + \omega_k^i \wedge \omega_j^k. \]
    这个公式可以简写成
    \[ \Omega = d \omega + \omega \wedge \omega, \]
    其中 $\Omega$ 是 $\End (E)$ 取值的 $2$-形式,
    记号 $\omega \wedge \omega$ 的意思是把两个矩阵相乘,
    但矩阵的元素都是 $1$-形式, 这些元素相乘的方法是做外积.
\end{definition}

由于记号约定不同, 一些作者会把这个公式写成
\[ \Omega = d \omega - \omega \wedge \omega. \]
出现这种现象的原因是, 这些作者的上下标与我们相反,
所以我们的 $\omega_k^i \wedge \omega_j^k$
相当于这些作者的 $-\omega_k^j \wedge \omega_i^k$.

之所以将 $\Omega$ 称为曲率形式, 是因为对 $X, Y \in \Gamma (TM)$, 有
\begin{equation} \label{eq-5-omega-equals-r}
    \Omega (X, Y) = R (X, Y) \quad \in \Gamma (\End (E)),
\end{equation} 
其中 $R (X, Y) = \nabla_X \nabla_Y - \nabla_Y \nabla_X - \nabla_{[X, Y]}$
是 Riemann 曲率张量. 读者可以验证这个公式.

实际上, 这个公式还说明, 虽然 $\Omega$ 是局部定义的,
但它是整个 $M$ 上的一个张量. 这个性质是联络形式所不具有的.

\begin{proposition}
    $\Omega$ 定义了整个 $M$ 上的一个 $2$-形式:
    \[ \Omega \in \Omega^2 \bigl( M, \End (E) \bigr). \thmqedhere \]
\end{proposition}

曲率形式还有另一种诠释.
给定 $E$ 上的一个联络, 我们不仅能对 $E$ 的截面求方向导数,
也能对 $E$-取值的微分形式求方向导数:

\begin{definition}
    设 $s \in \Gamma (E)$, $\alpha \in \Omega^\bullet (M)$.
    通过 Leibniz 法则, 我们定义
    \[ \nabla ( s \otimes \alpha ) = \nabla s \wedge \alpha + s \otimes d \alpha. \]
    这样就定义了 $E$-取值的微分形式的方向导数.
\end{definition}

于是, 我们有一系列映射
\[ \Omega^0 (M, E) \overset{\nabla}{\longrightarrow}
    \Omega^1 (M, E) \overset{\nabla}{\longrightarrow} 
    \Omega^2 (M, E) \overset{\nabla}{\longrightarrow} \cdots. \]
对于 $s \in \Gamma (E)$, 我们有
\begin{equation} \label{eq-5-curv-nabla-sq}
    \nabla^2 s = \Omega s \quad \in \Omega^2 (M, E).
\end{equation}
这个性质常常被简写为 $\Omega = \nabla^2$. 请读者验证这个等式.

最后, 我们有一个著名的关于曲率形式的等式, 其验证留给读者.

\begin{proposition} [Bianchi 恒等式] \label{thm-5-bianchi}
    在局部上, 我们有
    \[ d \Omega = [ \Omega, \omega ], \]
    其中 $[ \Omega, \omega ] = \Omega \wedge \omega - \omega \wedge \Omega$. \qed
\end{proposition}


\subsection{曲率不变量}

下面, 我们从一个联络的曲率 $2$-形式中提取出一些不变量,
这些不变量不依赖于联络的选取.

\begin{definition}
    $n$ 阶矩阵的\term{不变多项式} (invariant polynomial)
    是指一个函数
    \[ P \: \operatorname{M}_{n \times n} (\bbC) \to \bbC, \]
    它是 $n^2$ 个矩阵元的多项式, 并满足对任意矩阵 $A$ 和可逆矩阵 $T$, 有
    \[ P (TAT^{-1}) = P(A). \]
\end{definition}

\begin{exercise} \label{ex-5-inv-poly}
    设 $P$ 是 $n$ 阶矩阵的不变多项式. 
    则 $P$ 一定是矩阵的 $n$ 个特征值的对称多项式,
    由此推出 $P$ 一定具有
    \[ P (A) = p ( \tr A, \ \tr A^2, \ \dotsc, \ \tr A^n ) \]
    的形式, 其中 $p$ 是多项式. \varqed
\end{exercise}

设 $E \to M$ 是光滑复向量丛, 带有联络 $\nabla$.
一个不变多项式给出了向量丛 $\End (E)$ 的每个纤维到 $\bbC$ 的函数,
因为其取值不依赖于基的选取, 只依赖于线性算子的特征值.
这个函数可以把 $\End (E)$-取值的微分形式变成普通的微分形式.

设 $P$ 是不变多项式. 我们将它作用在曲率 $2$-形式 $\Omega$ 上, 得到一个形式
\[ P (\Omega) \in \Omega^{2 \bullet} (M, \bbC). \]
我们将证明, 这个微分形式是闭形式, 且它给出的 de~Rham 上同调类不依赖于联络的选取.

\begin{proposition}
    $P (\Omega)$ 是闭形式.
\end{proposition}

\begin{proof}
    将 $P$ 写成 (\ref{ex-5-inv-poly}) 的形式.
    因为闭形式的外积仍然是闭形式, 所以, 我们只需证明
    \[ d \tr \Omega^k = 0 \quad (k = 0, 1, \dotsc, m). \]
    而由 Bianchi 恒等式 (\ref{thm-5-bianchi}), 由归纳法不难证明
    \[ d \Omega^k = [ \Omega^k, \omega ] \quad (k = 0, 1, \dotsc). \]
    因此,
    \[ d \tr \Omega^k = \tr d \Omega^k
        = \tr {} [ \Omega^k, \omega ] = 0. \qedhere \]
\end{proof}

\begin{proposition} \label{thm-5-conn-indep}
    上同调类 $[P (\Omega)] \in H^{2 \bullet} (M; \bbC)$
    不依赖于联络 $\nabla$ 的选取.
\end{proposition}

\begin{proof}
    设 $\nabla^0, \nabla^1$ 是 $E$ 上的两个联络.
    考虑向量丛 $E \times \bbR \to M \times \bbR$, 带有联络
    \[ \nabla = t \widetilde{\nabla} ^1 + (1-t) \widetilde{\nabla} ^0, \]
    其中 $\widetilde{\nabla} ^i$ ($i = 1, 2$)
    是 $E \times \bbR$ 上的联络,
    使得截面沿 $E$ 方向的导数由 $\nabla^i$ 给出, 
    沿 $\bbR$ 方向的导数由普通的导数给出.
    换言之, $\widetilde{\nabla} ^i$ 是将
    $\nabla^i$ 的联络 $1$-形式沿着投影映射拉回,
    所定义的 $E \times \bbR$ 上的联络.
    
    记 $i_0, i_1 \: M \hookrightarrow M \times \bbR$ 为含入映射,
    其 $\bbR$-分量分别为 $0$ 和 $1$. 记 $\Omega^0, \Omega^1$
    分别为 $\nabla^0, \nabla^1$ 的曲率形式. 则
    \[ [ P (\Omega^0) ] = i_0^* \, [ P (\Omega) ]
        = i_1^* \, [ P (\Omega) ] = [ P (\Omega^1) ], \]
    因为 $i_0, i_1$ 是同伦的映射.
\end{proof}

这一系列结论表明, 对每个不变多项式 $P$,
上同调类 $[ P (\Omega) ]$ 给出了复向量丛的拓扑不变量.
事实上, 这些不变量就是我们熟悉的陈类, 以及陈类的多项式.
在下面几个小节中, 我们将导出这些不变量与陈类的对应关系.

\begin{definition}
    设 $1 \leq k \leq n$ 为整数. 对 $n$ 阶复矩阵 $A$, 我们记
    \[ \sigma_k (A) = \sum_{1 \leq i_1 < \dotsc < i_k \leq n}
        \lambda_{i_1} \cdots \lambda_{i_k} \]
    为 $A$ 的 $n$ 个特征值的第 $k$ 个初等对称多项式,
    它是一个不变多项式.
\end{definition}

我们即将证明, $E$ 的陈类由下面的公式给出:
\[ c_k (E) = \frac {1} {(2 \uppi \upi)^k} \, [ \sigma_k (\Omega) ]. \]


\subsection{第一陈类}

我们首先考虑最简单的情况, 即线丛的第一陈类.

\begin{theorem} \label{thm-5-c1}
    设 $E \to M$ 是光滑复线丛, 带有联络 $\nabla$ 和曲率形式 $\Omega$. 则
    \[ c_1 (E) = \frac {1} {2 \uppi \upi} \, [ \Omega ]. \]
\end{theorem}

\begin{proof}
    设 $E$ 的分类映射是
    \[ \phi \: M \to \bbC \upP^\infty, \]
    其中 $\bbC \upP^\infty$ 是复线丛的分类空间.
    由胞腔近似定理, $\phi$ 的像可以同伦到 $\bbC \upP^\infty$ 的一个有限维子 CW 复形中.
    因此, 在同伦意义下, 我们可以将 $\phi$ 视为一个映射
    \[ \phi \: M \to \bbC \upP^N \quad (N > 0). \]
    在同伦意义下, 我们还可以假定 $\phi$ 是光滑映射 (例如利用磨光).
    
    我们知道 $H^2 ( \bbC \upP^N; \bbC ) \simeq \bbC \cdot c_1$,
    其中 $c_1$ 是自言线丛 $\scrO (-1)$ 的第一陈类. 
    因此, 存在常数 $\alpha \in \bbC$, 使得 
    \[ [ \Omega ] = \alpha \, c_1, \]
    其中 $\Omega$ 是 $\scrO (-1)$ 的曲率形式.
    
    我们有 $E \simeq \phi^* \scrO (-1)$. 
    我们可以把 $\scrO (-1)$ 的联络 $1$-形式通过 $\phi$ 拉回,
    得到 $E$ 上的联络, 其曲率形式是 $\Omega^E = \phi^* \Omega$. 从而
    \[ [ \Omega^E ] = \alpha \, \phi^* c_1 = \alpha \, c_1 (E). \]
    
    最后, 我们来确定常数 $\alpha$.
    
    我们改变记号: 设 $M$ 是二维可定向 Riemann 流形,
    $E \to M$ 是 $M$ 的切丛. 则 $E$ 可以自然地被看作一个复线丛,
    因为乘以 $\upi$ 的操作就是逆时针旋转 $90^\circ$.
    在 $E$ (作为实向量丛) 上有一个自然的度量联络, 
    即 Levi-Civita 联络, 我们记作 $\nabla^{\bbR}$.
    
    取 $E$ 的局部截面 $e_1$, 满足 $|e_1| \equiv 1$, 并记 $e_2 = \upi e_1$. 
    以它们为局部坐标, 则 $\omega_1^1 = 0$, 因为对任何向量场 $X$,
    有 $\nabla^{\bbR}_X e_1 \perp e_1$. 这说明
    \[ \nabla^{\bbR} e_1 = \omega_1^2 \otimes e_2
        = \upi \omega_1^2 \otimes e_1. \]
    我们定义 $E$ (作为复向量丛) 上的联络 $\nabla^{\bbC}$ 如下:
    对任何截面 $s$, 定义
    \[ \nabla^{\bbC} s = \nabla^{\bbR} s \quad \in \Omega^1 (M, E). \]
    则 $\nabla^{\bbC}$ 确实是联络,
    因为联络的本质就是把邻近的纤维对齐的方法,
    而 $\nabla^{\bbR}$ 保持了纤维上的内积, 从而也是 $\bbC$-线性的. 
    
    现在, 我们有 $\nabla^{\bbC} e_1 = \upi \omega_1^2 \otimes e_1$.
    因此, $\nabla^{\bbC}$ 的联络形式是
    \[ \omega^{\bbC} = \upi \omega_1^2, \]
    这里我们省去了 $\omega^{\bbC}$ 的上下标 $1$. 从而, $\nabla^{\bbC}$ 的曲率形式是
    \[ \Omega^{\bbC} = \upi \Omega_1^2. \]
    
    为了确定 $\alpha$ 的值, 我们取 $M = S^2$ 为标准的单位球面.
    一方面, 我们有
    \[ \int_{S^2} \Omega^{\bbC} = \alpha \, \langle c_1 (TM), \ [M] \rangle
        = \alpha \, \langle e(TM), \ [M] \rangle = 2 \alpha, \]
    而另一方面, 由 (\ref{eq-5-omega-equals-r}),
    在开集 $U \subset S^2$ 上的局部坐标中, 有
    \[ \int_U \Omega^{\bbC} = \upi \int_U \Omega_1^2
        = \upi \int_U K \, dS = \upi \cdot (U \text{ 的面积}), \]
    其中 $K$ 是 Gauß 曲率, $S$ 是面积元. 因此,
    \[ 2 \alpha = \upi \cdot (S^2 \text{ 的面积}) = 4 \uppi \upi, \]
    从而 $\alpha = 2 \uppi \upi$.
\end{proof}

现在, 我们已经知道了 $\alpha$ 的值.
因此, 在上面的证明最后, 如果我们把 $S^2$ 换成别的流形,
就可以导出 Gauß--Bonnet 公式.

\begin{corollary} [Gauß--Bonnet 公式]
    设 $M$ 是二维可定向紧 Riemann 流形. 则
    \[ \int_M K \, d S = 2 \uppi \, \chi (M), \]
    其中 $K$ 是 Gauß 曲率, $dS$ 是面积元. \qed
\end{corollary}


\subsection{陈形式}

\begin{theorem}
    设 $E \to M$ 是光滑复向量丛, 带有联络 $\nabla$ 和曲率形式 $\Omega$. 则
    \[ c_k (E) = \frac {1} {(2 \uppi \upi)^k} \, [ \sigma_k (\Omega) ]. \]
    因此, $2k$-形式 $(2 \uppi \upi)^{-k} \sigma_k (\Omega)$
    称为 $E$ 的第 $k$ 个\term{陈形式}.
\end{theorem}

\begin{proof}
    \allowdisplaybreaks
    考虑 $m$ 阶矩阵的不变多项式 $P$, 定义为
    \[ P (A) = \sum _{k=0} ^m \frac {1} {(2 \uppi \upi)^k} \, \sigma_k (A)
        = \det {} \Bigl( I + \frac {A} {2 \uppi \upi} \Bigr), \]
    其中 $I$ 表示单位矩阵. 这个不变多项式对应着向量丛的全陈类.
    
    我们先考虑 $E = L_1 \oplus \cdots \oplus L_m$ 是若干线丛的直和的情况.
    我们在每个 $L_i$ 上取联络 $\nabla^i$, 然后定义 $E$ 上的联络
    $\nabla = \bigoplus_i \nabla^i$. 换言之, $\nabla$ 的联络 $1$-形式是
    \[ \omega = \operatorname{diag} (\omega^1, \dotsc, \omega^m), \]
    即由各 $\nabla^i$ 的联络形式构成的对角矩阵. 从而, $\nabla$ 的曲率形式是
    \[ \Omega = \operatorname{diag} (\Omega^1, \dotsc, \Omega^m). \]
    因此,
    \begin{align*}
        P (\Omega) 
        &= \det {} \biggl[ \operatorname{diag} {} \biggl(
            1 + \frac {\Omega^1} {2 \uppi \upi}, \ \dotsc, \ 
            1 + \frac {\Omega^m} {2 \uppi \upi} \biggr) \biggr] \\
        &= P (\Omega^1) \cdots P (\Omega^m) \\
        &= c (L_1) \cdots c (L_m) \\
        &= c (E),
    \end{align*} 
    其中第三个等号使用了 (\ref{thm-5-c1}).
    
    对于一般的情况, 使用分裂原理不难完成证明.
    当然, 我们需要的是光滑版本的分裂原理,
    但其证明与拓扑版本的分裂原理是完全相同的.
\end{proof}

在上面的证明中, 我们也得到了计算全陈类的公式. 
我们将这个公式和计算陈特征的公式一并列出:

\begin{corollary} \label{thm-5-total-chern}
    设 $E \to M$ 是光滑复向量丛, 带有联络 $\nabla$ 和曲率形式 $\Omega$. 则
    \[ \begin{aligned}[b]
        c (E) &= \biggl[ \det {} \Bigl( I + \frac {\Omega} {2 \uppi \upi} \Bigr) \biggr]
            && \in H^{2 \bullet} (M; \bbC), \\
        \ch (E) &= [ \tr \upe^{\Omega / 2 \uppi \upi} ]
            && \in H^{2 \bullet} (M; \bbC).
    \end{aligned} \thmqedhere \]
\end{corollary}

实向量丛的 Понтрягин 类是通过陈类定义的.
因此, 我们也能将 Понтрягин 类用微分形式表示出来.

\begin{corollary}
    设 $E \to M$ 是光滑实向量丛, 带有联络 $\nabla$ 和曲率形式 $\Omega$. 则
    \[ p_k (E) = \frac {1} {(2 \uppi)^{2k}} \, [ \sigma_{2k} (\Omega) ]. \]
    因此, $4k$-形式 $(2 \uppi)^{-2k} \sigma_{2k} (\Omega)$
    称为 $E$ 的第 $k$ 个 \term{Понтрягин 形式}.
\end{corollary}

\begin{proof}
    证明中唯一不太显然的地方是:
    $E$ 上的联络可以自然地诱导 $E \otimes \bbC$ 上的联络,
    二者的联络形式相等, 从而曲率形式也相等.
\end{proof}


\subsection{陈--Gauß--Bonnet 公式}

利用上面的理论, 我们可以证明 Gauß--Bonnet 公式向高维流形的推广,
它把流形的 Euler 数表示为曲率不变量的积分.
事实上, 奇数维紧流形的 Euler 数总是 $0$,
而对偶数维流形来说, 切丛的 Euler 类是最高阶 Понтрягин 类的平方根.
因此, 我们只要描述清楚这个平方根, 就能获得计算 Euler 数的公式.

\begin{definition}
    设 $V$ 是已定向的 $n$ 维欧氏向量空间, 
    $e_1, \dotsc, e_n$ 为一组与定向相符的标准正交基. 线性映射
    \[ T \: {\wedge^\bullet \, V} \to \bbR \]
    定义如下: $T (e_1 \wedge \cdots \wedge e_n) = 1$, $T (\wedge^{<n} \, V) = 0$.
    映射 $T$ 称为 \term{Березин 积分}.\footnote{英文转录为 Berezin.}
\end{definition}

\begin{definition}
    沿用上面的记号, 设 $A \in \wedge^2 \, V$, 可以看作一个反对称矩阵. 定义
    \[ \exp^{\wedge} A = \sum _{k=0} ^\infty \frac {A^{\wedge k}} {k!}
        \quad \in \wedge^\bullet \, V. \]
    事实上, 这个求和是有限和.
    我们定义 $A$ 的 \term{Pfaff 值} (Pfaffian) 为
    \[ \operatorname{pf} A = T ( \exp^{\wedge} A ). \]
\end{definition}

Pfaff 值的重要性质是, 它是行列式的平方根, 
同时也是矩阵元素的多项式.

\begin{proposition}
    沿用上面的记号, 将 $A$ 看成反对称矩阵, 我们有
    \[ (\operatorname{pf} A)^2 = \det A. \]
\end{proposition}

\begin{proof}
    通过正交相似变换, $A$ 可以化为标准型, 即分块对角矩阵
    \[ A = \operatorname{diag} (B_1, \dotsc, B_k, 0, \dotsc, 0), \quad
        B_i = \Bigl( \begin{smallmatrix} 0 & a_i \\ -a_i & 0 \end{smallmatrix} \Bigr). \]
    我们记 $A = A_1 + \cdots + A_k$, 
    其中 $A_i$ 是将 $A$ 中除了 $B_i$ 外的部分都清零得到的矩阵. 则
    \[ A_i = a_i \, \bigl( e_{2i-1} \otimes e_{2i} - e_{2i} \otimes e_{2i-1} \bigr)
        = a_i \, (e_{2i-1} \wedge e_{2i}), \]
    其中 $e_i$ 表示第 $i$ 个基向量. 
    因此, 若 $k = n/2$, 则
    \begin{align*}
        T (\exp^\wedge A) 
        &= \frac {1} {k!} \, T (A^{\wedge k}) \\
        &= \frac {1} {k!} \, T ( k! \, A_1 \wedge \cdots \wedge A_k ) \\
        &= a_1 \cdots a_k \, T ( e_1 \wedge \cdots \wedge e_n ) \\
        &= a_1 \cdots a_k.
    \end{align*}
    而 $\det A = a_1^2 \cdots a_k^2$.
    若 $k < n/2$, 则 $T (\exp^\wedge A) = 0$.
\end{proof}

有了 Pfaff 值, 我们可以对反对称矩阵的行列式开平方根.
对于度量联络而言, 曲率形式确实是反对称的:

\begin{proposition} \label{thm-5-curv-antisym}
    设 $E \to M$ 是光滑实向量丛, 具有 Euclid 度量. 设 $\nabla$ 是度量联络.
    则曲率形式 $\Omega$ 是取值于 $\mathfrak{so}(E)$ 的 $2$-形式,
    即在标准正交基下表示为反对称矩阵: $\Omega_i^j = -\Omega_j^i$.
\end{proposition}

\begin{proof}
    事实上, 在局部坐标系中, 联络形式也是反对称的, 即满足若 $e_1, \dotsc, e_m$
    是 $E$ 的局部截面, 它们构成 $E$ 的每个纤维的标准正交基, 则以它们为局部坐标,
    有 $\omega_i^j = -\omega_j^i$. 这是因为对任何向量场 $X$, 有
    \[ 0 = X \langle e_i, \ e_j \rangle
        = \langle \nabla_X e_i, \ e_j \rangle + \langle e_i, \ \nabla_X e_j \rangle 
        = \omega_i^j (X) + \omega_j^i (X). \qedhere \]
\end{proof}

因此, 我们可以对得到偶数维流形切丛的最高阶 Понтрягин 类开平方, 得到 Euler 类的表达式.

\begin{theorem} [陈--Gauß--Bonnet 公式]
    设 $E \to M$ 是光滑实向量丛, 其秩为 $2n$. 则
    \[ e (E) = \biggl[ \Bigl( \frac{-1}{2\uppi} \Bigr)^n \operatorname{pf} (\Omega) \biggr]
        \quad \in H^{2n} (M). \]
    特别地, 如果 $M$ 是 $2n$ 维紧可定向流形, 那么
    \[ \chi (M) = \Bigl( \frac{-1}{2\uppi} \Bigr)^n \int_M \operatorname{pf} (\Omega), \]
    其中 $\Omega$ 是切丛的曲率形式.
\end{theorem}

\begin{proof}
    和 (\ref{thm-5-c1}) 的证明一样, 我们只需要对足够大的 $N$,
    证明有限维 Graßmann 流形 $G_{2n} (\bbR^N)$ 上的 ``万有向量丛'' $E$ 满足这一公式.%
    \footnote{事实上, 这一步还用到了如下事实: $G_{2n} (\bbR^\infty)$ 有一个 CW 结构,
    其每个 $k$-骨架都包含于某个子复形 $G_{2n} (\bbR^N)$. 
    对这样的 CW 结构的描述可参见 \cite[p.\,194ff.]{griffiths-harris}.}
    
    此时, $e (E)^2 = p_n (E)$, 且 
    $\operatorname{pf} (\Omega)^2 = \det (\Omega) = \sigma_{2n} (\Omega)$. 故必有
    \[ e (E) = \epsilon \, \biggl[ \Bigl( \frac{1}{2\uppi} \Bigr)^n \operatorname{pf} (\Omega) \biggr], \quad \epsilon = \pm 1. \]
    
    下面, 我们来确定 $\epsilon$ 的值.
    为此, 我们只需考虑一个特例即可.
    
    当 $n = 1$ 时, 由我们的记号约定, 局部上有 $\operatorname{pf} (\Omega) = -\Omega_1^2$.
    而在 (\ref{thm-5-c1}) 的证明中, 我们知道对球面 $S^2$, 有
    \[ [\operatorname{pf} (\Omega)] = \text{``} [ -\Omega_1^2 ] \text{''}
        = -4 \uppi = -2 \uppi \, e (TS^2). \]
    这就说明 $n = 1$ 时, $\epsilon = -1$.
    
    对一般的 $n$, 我们考虑 $n$ 个平面丛的直和 $E = E_1 \oplus \cdots \oplus E_n$. 则
    \begin{align*}
        [ \operatorname{pf} (\Omega) ]
        &= [ \operatorname{pf} (\Omega_1) ] \times \cdots \times [ \operatorname{pf} (\Omega_n) ] \\
        &= (-2 \uppi)^n \, e (E_1) \times \cdots \times e (E_n) \\
        &= (-2 \uppi)^n \, e (E),
    \end{align*}
    其中记号的意义是明显的. 因此, $\epsilon = (-1)^n$.
\end{proof}

