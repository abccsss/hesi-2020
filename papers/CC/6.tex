在上一节中, 我们已经基本完成了陈--Weil 理论的介绍.
这一节, 我们介绍陈--Weil 理论的几个延伸和推广.


\subsection{Thom 形式和 Euler 形式}

在这一小节中, 我们打算把 Thom 类也用曲率不变量表示出来.
因为 Euler 类是 Thom 类的拉回, 所以,
我们也能重新证明 Euler 类的陈--Gauß--Bonnet 公式.

这一小节的内容也被称为 \term{Mathai--Quillen 理论},
原始文献是 \cite{mathai-quillen}.

\begin{definition}
    设 $E \to M$ 是光滑向量丛. 
    $E$ 上的一个微分形式称为\term{垂直紧支} (with compact vertical support) 的,
    如果它在每个纤维上的限制都是紧支的. 记
    \[ \Omega_{\mathrm{cv}}^\bullet (E) \subset \Omega^\bullet (E) \]
    为 $E$ 的所有垂直紧支微分形式构成的微分分次代数.
    这个微分分次代数定义的 de~Rham 上同调称为\term{垂直紧支上同调}
    (compact vertical cohomology), 记作 $H_{\mathrm{cv}}^\bullet (E)$.
    
    若 $E$ 已定向, 我们还有\term{沿纤维积分} (integration along the fibre) 的操作:
    \[ \int_{E/M} \: \Omega_{\mathrm{cv}}^\bullet (E) \to \Omega^{\bullet - n} (M), \]
    其中 $n$ 是 $E$ 的秩, 定义为
    \begin{multline*}
        \int_{E/M} f(x, t) \, dx^{\alpha_1} \wedge \cdots \wedge dx^{\alpha_p}
        \wedge dt^{i_1} \wedge \cdots \wedge dt^{i_q} \\
        = \ \left \{ \begin{array}{ll}
            \displaystyle \biggl( \int_{\bbR^n} f(x, t) \,
                dt^1 \wedge \cdots \wedge dt^n \biggr) \,
                dx^{\alpha_1} \wedge \cdots \wedge dx^{\alpha_p}, & q = n, \\
            0, & \text{其它},
        \end{array} \right.
    \end{multline*}
    其中 $x$ 是底空间方向的变量, $t$ 是纤维方向的变量, 且要求 $i_1 < \cdots < i_q$.
\end{definition}

我们有自然的同构
\[ H_{\mathrm{cv}}^\bullet (E) \simeq \widetilde{H}^\bullet (\Th (E)). \]
直观地看, 这是因为流形的紧支上同调同构于其一点紧化的约化上同调,
故垂直紧支上同调应该对应于给每个纤维增加一个公共的无穷远点.
我们略过严格的证明.

有了这个同构, 我们可以将 Thom 类的定义用微分形式的语言重新写下来.

\begin{definition}
    设 $E \to M$ 是已定向的光滑向量丛, 其秩为 $n$.
    微分形式 $u \in \Omega_{\mathrm{cv}}^n (E)$ 称为一个 \term{Thom 形式}, 如果
    \begin{itemize}
        \item
            $du = 0$.
        \item
            $\int_{E/M} u = 1 \in \Omega^0 (M)$.
    \end{itemize}
    则 $u$ 确定了 $H_{\mathrm{cv}}^n (E)$ 中的一个上同调类, 即 Thom 类.
    
    相应地, 若 $i \: M \to E$ 是通过零截面含入, 那么闭形式
    \[ e = i^* u \quad \in H^n (M) \]
    称为一个 \term{Euler 形式}, 其上同调类就是 Euler 类.
\end{definition}

我们接下来的目标, 就是把 Thom 形式和 Euler 形式用曲率不变量表示出来.

首先, 我们考虑一个最简单的情况, 即 $M = \{ * \}$. 此时 $E$ 就是一个欧氏向量空间 $V$.
我们可以取
\[ u (x) = \frac {1} {(2 \uppi)^{n/2}} \, \upe^{-|x|^2 / 2} \, 
    dx^1 \wedge \cdots \wedge dx^n \quad (x \in V). \]
则 $u$ 满足定义 Thom 形式的公式, 但它不是紧支的.
我们先允许这个漏洞存在, 一会再来解决紧支的问题.

Mathai--Quillen 理论的一个独特见解,
就在于它把一些公式写成了 Березин 积分的样子. 例如,
\[ dx^1 \wedge \cdots \wedge dx^n = (-1)^{n (n-1) / 2} \, T ( \exp ( dx ) ), \]
其中 
\[ dx = e_1 \otimes dx^1 + \cdots + e_n \otimes dx^n
    \quad \in \Omega^1 (V, \ {\wedge^\bullet \, V}) \]
是 $V$ 上的取值于平凡丛 $V \times {\wedge^\bullet \, V} \to V$ 的 $1$-形式.
对它作用 $\exp$ 时, 纤维方向的向量 $e_i$ 在 $\wedge^\bullet \, V$ 中相乘.
Березин 积分取出这个外代数中 $e_1 \wedge \cdots \wedge e_n$ 的部分,
并扔掉其它部分. 而底空间方向的微分形式 $dx^i$ 随着它的伙伴 $e_i$ 一同被取舍,
当 $e_1 \wedge \cdots \wedge e_n$ 被留下来时,
对应被留下来的形式就是 $dx^1 \wedge \cdots \wedge dx^n$.
至于系数 $(-1)^{n (n-1) / 2}$, 它来自于
\[ e_1 \, dx^1 \wedge \cdots \wedge e_n \, dx^n
    = (-1)^{n (n-1) / 2} \, ( e_1 \wedge \cdots \wedge e_n ) \,
    (dx^1 \wedge \cdots \wedge dx^n), \]
这是因为由定义,
\[ \Omega^\bullet (V, \ {\wedge^\bullet \, V}) =
    \Omega^\bullet (V) \otimes (\wedge^\bullet \, V), \]
这是两个反交换分次代数的张量积,
它自身的乘法满足 \term{Koszul 符号法则} (Koszul sign rule):
若 $v, w \in \wedge^\bullet \, V$, $\alpha, \beta \in \Omega^\bullet (V)$, 则
\[ (v \otimes \alpha) \, (w \otimes \beta) =
    (-1)^{|w|\,|\alpha|} \, (v w) \otimes (\alpha \beta). \]

上面的讨论总结为下面的命题.

\begin{proposition}
    沿用上面的记号, 我们有
    \[ u (x) = \frac {\upi^{n^2}} {(2 \uppi)^{n/2}} \, T \, \biggl( \exp {} \biggl(
        - \frac {|x|^2} {2} - \upi \, dx
    \biggr) \biggr). \thmqedhere \]
\end{proposition}

这里, 给 $dx$ 加上系数 $-\upi$ 并不是必要的,
但这样可以方便和之后的结论联系起来.

接下来, 我们试着将这个方法推广到一般情形:
设 $E \to M$ 是已定向的光滑向量丛, 秩为 $n$.
我们取一个 Euclid 度量, 并取一个度量联络 $\nabla$.

\begin{lemma}
    设 $\alpha \in \Omega^\bullet (M, \ {\wedge^\bullet \, E})$. 则
    \[ d T(\alpha) = T (\nabla \alpha). \]
\end{lemma}

\begin{proof}
    我们把 Березин 积分看作向量丛 $\wedge^n \, E^\vee$ 的截面.
    则 Leibniz 法则表明
    \[ d T(\alpha) = T (\nabla \alpha) + (\nabla T) \, \alpha. \]
    但 $\nabla T = 0$, 因为 $\nabla$ 是度量联络,
    而 $T$ 作为一个线丛的截面, 对应着恒定的有向体积. (读者可以试着严格化这个说法.)
\end{proof}

我们把之前的做法照搬过来. 设
\[ \widetilde{E} = p^* E \]
是 $E$ 上的向量丛, 其中 $p \: E \to M$ 是投影.
则之前的 $\Omega^\bullet (V, \ {\wedge^\bullet \, V})$
对应着现在的 $\Omega^\bullet (E, \ {\wedge^\bullet \, \widetilde{E}})$.

向量丛 $\widetilde{E}$ 有一个\term{自言截面} (tautological section), 记为 $x$,
它在每个点 $e \in E$ 处, 在纤维 $\widetilde{E}_e$ 上选出 $e$ 对应的点. 
这样, 我们可以定义
\[ |x|^2 \in C^\infty (E). \]
将 $E$ 的联络拉回到 $\widetilde{E}$ 上,
可以得到 $\widetilde{E}$ 上的度量联络. 我们可以定义
\[ \nabla x \quad \in \Omega^1 (E, \ \widetilde{E}) 
    = \Omega^1 (E, \ {\wedge^1 \, \widetilde{E}}). \]
当 $M = \{*\}$ 时, $\nabla x$ 就是我们之前定义的 $dx$. 另外, 我们记
$\widetilde{\Omega}$ 是 $\widetilde{E}$ 上的度量联络的曲率形式,
它可以看作 $\Omega^2 (E, \ {\wedge^2 \, \widetilde{E}})$ 的元素 (\ref{thm-5-curv-antisym}).

\begin{definition}
    使用上面的记号, 我们记
    \[ A = \frac {|x|^2} {2} + \upi \, \nabla x + \widetilde{\Omega}
        \quad \in \Omega^\bullet (E, \ {\wedge^\bullet \, \widetilde{E}}). \]
\end{definition}

我们下面的目标, 就是证明 Thom 形式由
\[ u = \frac {\upi^{n^2}} {(2 \uppi)^{n/2}} \, T(\upe^{-A}) \]
给出. 在此之前, 我们先对 $A$ 的各项做一些注记.
$A$ 的前两项和之前的情况相似, 在作用 Березин 积分之后,
能给出纤维上的一个积分为 $1$ 的 $n$-形式.
而最后一项是曲率, 它在 $M = \{*\}$ 的时候是看不出来的.
它存在的原因会在下面的计算中显示出来.

这里, 我们引入沿纤维方向的缩并操作
\[ \iota_x \: \Omega^\bullet (E, \ {\wedge^\bullet \, \widetilde{E}})
    \to \Omega^\bullet (E, \ {\wedge^{\bullet - 1} \, \widetilde{E}}), \]
定义为
\[ \iota_x ( \alpha \otimes ( s_1 \wedge \cdots \wedge s_j ) )
    = \sum _{k=1} ^j (-1)^{|\alpha| + k - 1} \,
    \langle x, s_k \rangle \, \alpha \otimes
    ( s_1 \wedge \cdots \wedge \widehat{s_k} \wedge \cdots \wedge s_j ). \]

\begin{lemma}
    我们有
    \begin{itemize}
        \item
            $(\nabla + \upi \, \iota_x) \, A = 0$.
        \item
            $(\nabla + \upi \, \iota_x) \, \upe^{-A} = 0$.
    \end{itemize}
\end{lemma}

这里, $\upi \, \iota_x$ 是一个修正项, 它在之后的计算中是没有作用的.
这个引理的结论实际上是 ``$\nabla \upe^{-A} \approx 0$''.

\begin{proof}
    先证明第一个等式, 即
    \[ (\nabla + \upi \, \iota_x) \, 
        \biggl( \frac {|x|^2} {2} + \upi \, \nabla x + \widetilde{\Omega} \biggr) = 0. \]
    我们计算它的每一项:
    \begin{itemize}
        \item
            由 Leibniz 法则, 
            $\nabla ( |x|^2 / 2 ) = \langle x, \nabla x \rangle
            = \iota_x \nabla x$.
        \item
            由 (\ref{eq-5-curv-nabla-sq}), 
            $\nabla (\nabla x) = \widetilde{\Omega} x
            = -\iota_x \widetilde{\Omega}$.
        \item
            $\nabla \widetilde{\Omega} = 0$, 因为对任何截面 $s$, 有
            $(\nabla \widetilde{\Omega}) s 
            = \nabla (\widetilde{\Omega} s) - \widetilde{\Omega} (\nabla s)
            = \nabla^3 s - \nabla^3 s = 0$.
        \item
            由于次数原因, $\iota_x |x|^2 = 0$.
    \end{itemize}
    故等式得证.
    
    第二个等式可以从第一个等式推导出来, 细节留给读者.
\end{proof}

\begin{corollary}
    $T (\upe^{-A})$ 是闭形式.
\end{corollary}

\begin{proof}
    我们有 $d T (\upe^{-A}) = T (\nabla \upe^{-A}) = 0$.
    第二个等号是由引理, 以及由于次数原因, $T \circ \iota_x = 0$.
\end{proof}

\begin{corollary}
    定义
    \[ u = \frac {\upi^{n^2}} {(2 \uppi)^{n/2}} \, T(\upe^{-A})
        \quad \in \Omega^n (E). \]
    则如果忽略紧支的问题, 那么 $u$ 是 Thom 形式.
\end{corollary}

\begin{proof}
    这个形式限制在每个纤维上,
    就是我们之前讨论过的 $M = \{*\}$ 的情况,
    从而它沿纤维积分的结果是 $1$.
\end{proof}

通过这个公式, 我们可以重新证明陈--Gauß--Bonnet 公式.

\begin{corollary}[陈--Gauß--Bonnet 公式]
    $E$ 的一个 Euler 形式是
    \[ e = i^* u = \Biggl \{ \begin{array}{ll}
        \Bigl( \dfrac{-1}{2\uppi} \Bigr)^{n/2} \operatorname{pf} (\Omega), & n \text{ 偶数}, \\
        0, & n \text{ 奇数}.
    \end{array} \]
\end{corollary}

\begin{proof}
    只需证明第二个等号. 在 $E$ 的零截面上, $x = 0$, 且 $\widetilde{\Omega} = \Omega$.
    因此, 当 $n$ 为偶数时, 由 Pfaff 值的定义,
    \begin{align*}
        i^* u &= \frac {1} {(2 \uppi)^{n/2}} \, T(\upe^{-\Omega}) \\
        &= \frac {1} {(2 \uppi)^{n/2}} \operatorname{pf} (-\Omega) \\
        &= \Bigl( \frac{-1}{2\uppi} \Bigr)^{n/2} \operatorname{pf} (\Omega).
    \end{align*}
    当 $n$ 为奇数时, 没有任何东西能被 Березин 积分保留.
\end{proof}

最后, 我们来解决紧支的问题.
这可以通过选取一个合适的从开球到 $\bbR^n$ 的微分同胚来完成. 这个微分同胚是
\[ x \mapsto \frac {x} {\sqrt{1 - |x|^2}}. \]
沿这个微分同胚将我们的 Thom 形式拉回, 就能得到一个垂直紧支的 Thom 形式.


\subsection{超几何}

接下来, 我们打算把陈--Weil 理论推广到\term{超几何} (supergeometry) 的情况,
也就是构造超向量丛上的超联络的曲率不变量.

超几何起源于物理学中对超对称的描述, 术语中的 ``超'' 字也来源于此.
超几何的观点是我们证明指标定理的基础.

\begin{definition}
    一个\term{超空间} (superspace) 是一个有限维 $\bbZ_2$-分次的向量空间,
    它的两个部分记作
    \[ V = V^+ \oplus V^-. \]
\end{definition}

\begin{definition}
    一个\term{超代数} (superalgebra) 是一个有限维 $\bbZ_2$-分次的代数 $A$.
    如果 $a, b \in A$ 是齐次元素, 那么它们的\term{交换子} (commutator) 定义为
    \[ [a, b] = ab - (-1)^{|a|\,|b|} \, ba. \]
    这一定义可以双线性地延拓到整个 $A$ 上. 如果所有 $a, b \in A$ 都满足
    \[ ab = (-1)^{|a|\,|b|} \, ba, \]
    就说 $A$ 是\term{交换} (commutative) 的.
\end{definition}

例如, 如果 $V$ 是普通向量空间, 那么 $\wedge^\bullet \, V$
就是一个交换的超代数. 又如, 如果 $V$ 是超空间, $\dim V > 1$, 那么
$\End (V)$ 就是一个不交换的超代数.

\begin{definition}
    流形 $M$ 上的\term{超向量丛} (superbundle) 是一个 $\bbZ_2$-分次的向量丛
    \[ E = E^+ \oplus E^-. \]
\end{definition}

\begin{definition}
    设 $V$ 是超空间, $a \in \End (V)$. 我们将 $a$ 写成分块矩阵
    \[ a = \biggl( \, \begin{matrix}
        a^{++} & a^{-+} \\ a^{+-} & a^{--}
    \end{matrix} \, \biggr). \]
    则 $a$ 的\term{超迹} (supertrace) 定义为
    \[ \trs a = \tr a^{++} - \tr a^{--}. \]
\end{definition}

超迹的定义是为了满足下面的性质.

\begin{exercise}
    设 $V$ 是超空间, $a, b \in \End (V)$. 则
    \[ \trs {} [a, b] = 0. \varqedhere \]
\end{exercise}

下面, 我们还可以定义超向量丛的行列式丛.

\begin{definition}
    设 $E \to M$ 是超向量丛. 它的\term{行列式丛} (determinant bundle) 是
    \[ \det (E) = (\wedge^{n^+} \, E^+)^\vee \otimes (\wedge^{n^-} \, E^-), \]
    其中 $n^+, n^-$ 分别是 $E^+, E^-$ 的秩.
\end{definition}

例如, 若 $M$ 是流形, 则 $\det (TM) = \wedge^n \, T^* M$
是最高阶微分形式的线丛.


\subsection{超联络}

超联络的概念由 Quillen \cite{quillen-superconnection} 首先提出,
它是联络的概念向超几何的推广.

\begin{definition}
    设 $E_1, E_2 \to M$ 是两个光滑向量丛. 一个\term{微分算子} (differential operator)
    \[ D \: \Gamma (E_1) \to \Gamma (E_2) \]
    是在局部坐标系下, 具有如下表达式的算子:
    \[ D (s^i e_i^{(1)}) = \sum_{\alpha, i, j} 
        a_i^{j \alpha} \, ( \partial_{\alpha} s^i ) \, e_j^{(2)}, \]
    其中, 我们对 $M$ 上的所有多重指标 $\alpha$ 求和;
    每个 $a_i^{j \alpha}$ 是 $M$ 上的光滑函数, 它们中只有有限个非零;
    $e_i^{(k)} (x)$ 是 $E_k$ 的截面, 它们在每个纤维上构成一组基.
    使 $a_i^{j \alpha}$ 非零的 $|\alpha|$ 的最大值称为 $D$ 的\term{次数} (order).
\end{definition}

例如, Laplace 算子 
\[ \Delta = -\sum_{i=1}^n {} \Bigl( \frac {\partial} {\partial x^i} \Bigr)^2
    \: C^\infty (\bbR^n) \to C^\infty (\bbR^n) \]
是一个二阶微分算子, 这里 $E_1, E_2$ 都取成 $\bbR^n$ 上的平凡线丛.

在定义超联络之前, 我们引入一个记号,
设 $E \to M$ 是超向量丛. 我们记
\begin{align*}
    \Omega^+ (M, E) &= \Omega^{2\bullet} (M, E^+) 
        \oplus \Omega^{2\bullet + 1} (M, E^-), \\
    \Omega^- (M, E) &= \Omega^{2\bullet + 1} (M, E^+) 
        \oplus \Omega^{2\bullet} (M, E^-).
\end{align*}

\begin{definition}
    设 $E \to M$ 是超向量丛. $E$ 上的一个\term{超联络} (superconnection) 是一个算子
    \[ \nabla \: \Omega^\bullet (M, E) \to \Omega^\bullet (M, E), \]
    满足下列条件:
    \begin{itemize}
        \item
            $\nabla$ 是奇算子, 即它把子空间 $\Omega^+$ 映到 $\Omega^-$, 
            把 $\Omega^-$ 映到 $\Omega^+$.
        \item
            $\nabla$ 是一阶微分算子.
        \item
            $\nabla$ 满足 Leibniz 法则: 
            对 $\alpha \in \Omega^\bullet (M)$, $\omega \in \Omega^\bullet (M, E)$, 有
            \[ \nabla (\alpha \wedge \omega)
                = d \alpha \wedge \omega + (-1)^{|\alpha|} \alpha \wedge \nabla \omega. \]
    \end{itemize}
\end{definition}

如果 $E$ 是普通向量丛, 那么, 容易看出普通的联络也满足这三个条件.
因此, 超联络是普通联络的推广.

另外, 第三个条件说明, 一个超联络 $\nabla$ 由它在 
$\Omega^0 (M, E) = \Gamma (E)$ 上的取值确定. 我们记
\[ \nabla = \nabla^{(0)} + \nabla^{(1)} + \cdots, \]
其中 $\nabla^{(k)}$ 将 $\Omega^0 (M, E)$ 映到 $\Omega^k (M, E)$.
我们注意到以下事实.

\begin{remark}
    沿用上面的记号, 则
    \begin{itemize}
        \item 
            算子
            \[ \nabla^{(1)} \: \Omega^\bullet (M, E^{\pm}) 
                \to \Omega^{\bullet+1} (M, E^{\pm}) \]
            实际上是向量丛 $E^+$, $E^-$ 上的两个普通联络.
        \item
            每个算子 $\nabla^{(k)}$ 都是一个超联络.
            在局部上, 它可以由联络形式
            \[ \omega^{(k)} \in \Omega^k (M, \End(E)) \]
            描述. \varqed
    \end{itemize}
\end{remark}

和往常一样, 对于超联络 $\nabla$ 而言,
算子 $\nabla^2$ 具有很好的张量性质. 我们将它定义为曲率.

\begin{definition}
    设 $\nabla$ 是超向量丛 $E \to M$ 上的超联络. 它的\term{曲率}定义为算子
    \[ \nabla^2 \: \Omega^\bullet (M, E) \to \Omega^\bullet (M, E). \]
\end{definition}

\begin{proposition}
    设 $\nabla$ 是超向量丛 $E \to M$ 上的超联络. 则存在一个\term{曲率形式}
    \[ \Omega \in \Omega^{2\bullet} (M, \End (E)), \]
    使得曲率算子 $\nabla^2$ 由它的作用给出. 这里, $\Omega$ 的作用定义为
    \[ \Omega (s \otimes \alpha) = \Omega s \wedge \alpha, \]
    其中 $s \in \Gamma (E)$, $\alpha \in \Omega^\bullet (M)$.
\end{proposition}

\begin{proof}
    设 $s \in \Gamma (E)$, $\alpha \in \Omega^\bullet (M)$. 则
    \begin{align*}
        \nabla^2 (s \otimes \alpha)
        &= \nabla (\nabla s \wedge \alpha + s \otimes d \alpha) \\
        &= \nabla^2 s \wedge \alpha - \nabla s \wedge d \alpha
            + \nabla s \wedge d \alpha + s \otimes d^2 \alpha \\
        &= \nabla^2 s \wedge \alpha.
    \end{align*}
    因此, 只需证明存在 $\Omega$, 使得 $\Omega s = \nabla^2 s$.
    这等价于 $\nabla^2$ 作用在 $\Gamma (E)$ 上时是 $C^\infty (M)$-线性的.
    设 $f \in C^\infty (M)$. 我们有
    \begin{align*}
        \nabla^2 (fs)
        &= \nabla ( s \otimes df + f \, \nabla s ) \\
        &= \nabla s \wedge df + s \otimes d^2f 
            + df \wedge \nabla s + f \, \nabla^2 s \\
        &= f \, \nabla^2 s. \qedhere
    \end{align*}
\end{proof}


\subsection{超几何的陈--Weil 理论}

下面, 我们简述如何通过超联络的曲率形式, 得到超向量丛的示性类.

在这一小节中, 设 $E \to M$ 是光滑超向量丛, 其秩为 $n = n^+ + n^-$.
设 $\nabla$ 是 $E$ 上的超联络, 其曲率形式为 $\Omega$.

\begin{proposition}
    对每个 $k$, 微分形式
    \[ \trs \Omega^k \]
    是闭形式, 且它决定的上同调类和超联络 $\nabla$ 的选取无关.
    从而, 对任何 $N > 0$ 和 $N$ 元多项式 $p$, 闭形式
    \[ p ( \trs \Omega, \ \trs \Omega^2, \ \dotsc, \ \trs \Omega^N ) \]
    都决定了 $E$ 的一个拓扑不变量.
\end{proposition}

\begin{proof}
    证明和 \S\ref{sect-5} 中相同, 我们不再重复.
\end{proof}

回忆 (\ref{thm-5-total-chern}) 给出的公式.
在超几何的情况下, 全陈类没有合适的类比,
但计算陈特征的公式可以直接搬过来.

\begin{definition}
    $E$ 的\term{陈特征}定义为
    \[ \ch (E) = [ \trs \upe^{\Omega / 2 \uppi \upi} ] \quad \in H^{2\bullet} (M; \bbC). \]
\end{definition}

陈特征不依赖于超联络的选取. 我们可以在 $E^+$ 和 $E^-$ 上分别选取普通的联络,
从而得到 $E$ 上的普通联络, 它也是一个超联络. 从而由超迹的定义,
\[ \ch (E) = \ch (E^+) - \ch (E^-), \]
这里等号右边的 $\ch$ 表示普通意义下的陈特征.
作为推论, 超几何的陈特征也满足加性和乘性:
\begin{align*}
    \ch (E_1 \oplus E_2) &= \ch (E_1) + \ch (E_2), \\
    \ch (E_1 \otimes E_2) &= \ch (E_1) \ch (E_2).
\end{align*}


\subsection{指标定理前瞻}

从现在开始, 我们的目标就是证明著名的 Atiyah--Singer 指标定理.
在这一小节中, 我们介绍定理的内容, 以及几个著名的推论.

首先, 我们给出定理的大致叙述, 读者不必深究其中的细节.

\begin{theorem*} [Atiyah--Singer, 大致版本]
    设 $M$ 是偶数维紧可定向流形, $E \to M$ 是复超向量丛.
    设 $D \: \Gamma (E) \to \Gamma (E)$ 是一个自伴随 Dirac 微分算子. 则
    \[ \ind (D) = \int_M {} ( \text{一些示性类} ), \]
    其中
    \[ \ind (D) = \dim \ker D^+ - \dim \coker D^+ \]
    称为 $D$ 的\term{指标} (index), 其中 $D^+ = D|_{E^+}$.
\end{theorem*}

下面的三个结论都可以看作指标定理的推论.

\begin{theorem*} [陈--Gauß--Bonnet]
    设 $M$ 是 $2n$ 维紧可定向流形, 那么
    \[ \chi (M) = \Bigl( \frac{-1}{2\uppi} \Bigr)^n \int_M \operatorname{pf} (\Omega), \]
    其中 $\Omega$ 是切丛的曲率形式.
\end{theorem*}

\begin{theorem*} [Hirzebruch--Riemann--Roch]
    设 $M$ 是紧复流形, $E \to M$ 是全纯向量丛. 则
    \[ \chi (M, E) = \int_M \ch (E) \operatorname{td} (M), \]
    其中 $\chi (M, E)$ 是 $\dbar$-链复形的 Euler 数,
    $\operatorname{td}$ 是后面将引入的 Todd 类.
\end{theorem*}

\begin{theorem*} [Hirzebruch 符号差定理]
    设 $M$ 是 $4n$ 维流形, 则在 $H^{2n} (M)$ 上, 杯积给出了一个对称双线性型.
    设 $\sigma (M)$ 是这个双线性型的\term{符号差} (signature),
    即正特征值的个数减去负特征值的个数. 则
    \[ \sigma (M) = \frac {1} {(\uppi \upi)^{n/2}}
        \int_M \det^{1/2} \frac {\Omega / 2} {\tanh (\Omega / 2)}, \]
    其中右边的含义是一个形式幂级数.
\end{theorem*}

指标定理把等号左边的拓扑量和等号右边的几何量联系起来.
这种联系是通过一个超向量丛上的热核 $k_t (-, -)$ 来建立的.
我们将说明, 热核的超迹不依赖于时间 $t$, 而分别令 $t \to 0$ 和 $t \to +\infty$,
就能分别得到示性类的积分和算子的指标.
这一过程将在接下来的几节中完成.

