\subsection{广义 Laplace 算子}

我们回忆, 若 $E_1, E_2 \to M$ 是两个光滑向量丛,
则一个 $k$ 阶微分算子 $D \: \Gamma (E_1) \to \Gamma (E_2)$ 局部上具有如下形式:
\[ D (s^i e_i^{(1)}) = \sum_{ |\alpha| \leq k } 
    a_{i}^{j \alpha} \, ( \partial_{\alpha} s^i ) \, e_j^{(2)}. \]

\begin{definition}
    设 $D \: \Gamma (E_1) \to \Gamma (E_2)$ 是 $k$ 阶微分算子.
    它的\term{符号} (symbol) 定义为向量丛的映射
    \begin{align*}
        \sigma_D \: \operatorname{Sym}^k (T^* M) \otimes E_1 &\to E_2, \\
        e^\alpha \otimes e_i^{(1)} &\mapsto a_i^{j \alpha} e_j^{(2)}.
    \end{align*}
    换言之, $\sigma_D$ 的分量由 $(\sigma_D)_i^{j \alpha} = a_i^{j \alpha}$ 给出.
\end{definition}

例如, $\bbR^n$ 上的 Laplace 算子 $\Delta = -\sum_i ( \partial / \partial x^i )^2$ 的符号
\[ \sigma_{\Delta} \: \operatorname{Sym}^2 (\bbR^n) \to \bbR \]
就是 Euclid 内积的相反数.

\begin{definition}
    微分算子 $D \: \Gamma (E) \to \Gamma (E)$ 称为\term{椭圆} (elliptic) 的,
    如果对任意 $x \in M$ 以及 $0 \neq v \in T_x^* M$, 映射
    \[ \sigma_D (v, \dotsc, v) \: E_x \to E_x \]
    都是向量空间的自同构.
\end{definition}

例如, $\bbR^n$ 上的 Laplace 算子 $\Delta$ 就是椭圆微分算子,
因为 $\bbR^n$ 中非零向量和自身的内积不会等于 $0$.

\begin{definition}
    设 $M$ 是 Riemann 流形, $E \to M$ 是光滑向量丛.
    微分算子 $H \: \Gamma (E) \to \Gamma (E)$ 
    称为一个\term{广义 Laplace 算子} (generalised Laplacian), 如果
    \begin{itemize}
        \item 
            $H$ 是二阶算子.
        \item
            对任意 $x \in M$ 以及 $v, w \in T_x^* M$, 映射
            \[ \sigma_H (v, w) \: E_x \to E_x \]
            等于和 $-\langle v, w \rangle$ 的数乘.
    \end{itemize}
    等价地说, 广义 Laplace 算子在局部坐标系下表示为如下形式:
    \[ H = -g^{ij} \, \partial_i \, \partial_j + (\text{一阶微分算子}), \]
    其中 $g^{ij}$ 是 Riemann 度量.
\end{definition}

\begin{example}
    设 $M$ 是 Riemann 流形, $E \to M$ 是光滑向量丛.
    给定 $E$ 上的联络 $\nabla$, 我们就能定义对应的 Laplace 算子 $\Delta$,
    也叫做 \term{Laplace--Beltrami 算子}. 它在局部坐标系中定义为
    \[ \Delta s = -g^{ij} \, \bigl( \nabla_i \nabla_j \, s 
        - \Gamma_{ij}^k \nabla_k \, s \bigl), \]
    其中 $\nabla_i$ 是沿坐标向量场 $e_i$ 的方向导数.
    则每个联络 $\nabla$ 给出的 Laplace 算子都是广义 Laplace 算子. \varqed
\end{example}

\begin{proposition} \label{thm-7-gen-lap}
    如果 $H$ 是 $E$ 上的广义 Laplace 算子, 那么存在 $E$ 上的联络 $\nabla$, 使得
    \[ H = \Delta + f, \]
    其中 $\Delta$ 是联络 $\nabla$ 的 Laplace 算子, $f \in C^\infty (M)$.
\end{proposition}

这一结论的证明见 \cite[命题~2.5]{bgv}.


\subsection{\texorpdfstring{$\bbR^n$}{ℝⁿ} 中的热核}

在这一小节中, 我们回忆 $\bbR^n$ 中热方程的解法.

\term{热方程} (heat equation) 是偏微分方程中的一类基础的方程,
其初值问题叙述如下: 给定 $\bbR^n$ 上的函数 $f$, 求一个函数
\[ u (x, t) \quad (x \in \bbR^n , \ t \in \bbR_{\geq 0}), \]
满足
\[ \Biggl \{ \begin{array}{l}
    \partial_t u = - \Delta_x u, \\
    u |_{t=0} = f,
\end{array} \]
其中 $\Delta_x$ 表示 $\bbR^n$ 方向的 Laplace 算子.
热方程可以看作是描述了 $\bbR^n$ 中温度 $u$ 随着时间 $t$ 的演化过程.

\begin{definition}
    $\bbR^n$ 的\term{热核} (heat kernel) 是函数
    \[ k_t (x, y) = \frac {1} {(4 \uppi t)^{n/2}} \, \upe^{-|x-y|^2 / 4t}
        \quad (x, y \in \bbR^n, \ t > 0). \]
\end{definition}

热核 $k_t (x, y)$ 满足以下性质:
\[ \Biggl \{ \begin{array}{l}
    (\partial_t + \Delta_x) \, k_t (x, y) = 0, \\
    k_t (x, y) |_{t=0} = \delta (x, y),
\end{array} \]
其中 $\delta (x, y)$ 是 $\delta$ 函数. 
因此, 为了解出热方程的初值问题, 只需令
\[ u (x, t) = \int_{\bbR^n} k_t (x, y) \, f (y) \, dy, \]
就能满足
\[ \Biggl \{ \begin{array}{l}
    (\partial_t + \Delta_x) \, u = 0, \\
    u |_{t = 0} = \int_{\bbR^n} \delta (x, y) \, f (y) \, dy = f (x).
\end{array} \]
当然, 这里的推导不是完全严格的, 我们在此忽略有关的细节.

我们把由上面的卷积给出的 $u (x, t)$ 记作
\[ u = K_t (f) = \upe^{-t \Delta} f, \]
其中 $\upe^{-t \Delta}$ 是形式的记号, 在计算中能给人以直观.


\subsection{广义 Laplace 算子的热核}

下面, 设 $M$ 是紧 Riemann 流形, $E \to M$ 是光滑向量丛.

我们回忆, 在分析学中, $\bbR$ 上的一个\term{分布} (distribution)
是指一个连续线性泛函
\[ \phi \: C_{\mathrm{c}}^\infty (\bbR) \to \bbR, \]
其中左边是 $\bbR$ 上紧支光滑函数的空间, 带有 \term{$C^\infty$ 拓扑},
也就是说, 一列函数的收敛性等价于它们的支集包含于一个公共的紧集,
并且对每个 $n$, 它们的 $n$ 阶导数在这个紧集上一致收敛.

例如, 任何一个连续函数 $g \: \bbR \to \bbR$ 都可以看作 $\bbR$ 上的分布:
对 $f \in C_{\mathrm{c}}^\infty (\bbR)$, 定义
\[ g (f) = \int_{\bbR} f(x) \, g(x) \, dx. \]
再例如, $\delta$ 函数
\[ \delta (f) = f (0) \]
也是分布, 它可以看作在 $0$ 处取值 $+\infty$, 在别处取值 $0$, 但总积分为 $1$ 的函数.

分布的概念可以推广到向量丛的截面.
因为我们假设了 $M$ 是紧流形, 所以我们不再需要担心紧支的问题.

\begin{definition}
    在本节开头的假设下, $E$ 的\term{分布截面} (distributional section) 的集合定义为
    \[ \Gamma_{\mathrm{dist}} (E) = \Gamma (E^\vee) ^\vee, \]
    其中 $\Gamma (E^\vee)$ 带有 $C^\infty$ 拓扑,
    $\Gamma (E^\vee) ^\vee$ 是它的连续对偶空间.
\end{definition}

\begin{remark}
    这个定义带来了一个问题: 我们可以把普通的截面看成分布截面,
    但这个过程其实依赖于 $M$ 的度量: 若 $s$ 是普通截面, $f \in \Gamma(E^{\vee})$,
    我们需要定义
    \[ s(f) = \int_M \langle f, s \rangle \, \mathrm{vol} , \]
    其中 $\mathrm{vol}$ 是 $M$ 的体积形式, 
    但这个表达式的值依赖于 $\mathrm{vol}$ 的选取.
    
    为了做到和度量无关, 我们可以将 $\Gamma_{\mathrm{dist}} (E)$ 定义为
    \[ \Gamma \bigl( E^\vee \otimes (\wedge^n \, T^* M) \bigr) ^\vee, \]
    其中 $n = \dim M$. 因为 $\wedge^n \, T^* M$ 是平凡线丛,
    所以这和之前的定义是同构的, 但同构的选取和度量有关.
    为了简洁性, 我们不采用这种做法. \varqed
\end{remark}

$\bbR^n$ 中的热方程可以以 $\bbR^n$ 上的一个分布为初值.
事实上, 允许以分布作为初值是有好处的,
因为热核实际上就是以 $\delta$ 函数为初值的热方程的解.
通过同样的方法, 我们可以定义广义 Laplace 算子的热核.

\begin{definition}
    设 $E_1, E_2 \to M$ 是两个光滑向量丛. 设
    \[ K \: \Gamma_{\mathrm{dist}} (E_1) \to \Gamma (E_2) \]
    是一个连续线性映射. 那么, $K$ 的\term{核} (kernel) 是指一个截面
    \[ k \in \Gamma \bigl( M \times M, \ \Hom (p_2^* E_1, p_1^* E_2) \bigr), \]
    其中 $p_1, p_2 \: M \times M \to M$ 是投影,
    满足对任意的 $s \in \Gamma_{\mathrm{dist}} (E_1)$, 都有
    \begin{equation} \label{eq-7-kernel}
        (K s) (x) = \int_M k (x, y) \, s (y) \, dy,
    \end{equation}
    其中右边的含义是配对 $\langle s, k(x, -) \rangle$,
    这里 $k (x, -) \in \Gamma (E_1^\vee) \otimes (E_2)_x$.
\end{definition}

核的存在性是分布理论的一个重要的结论.

\begin{theorem} [Schwartz 核定理]
    沿用上面的记号, 我们有向量空间的同构
    \begin{align*}
        L \bigl( \Gamma_{\mathrm{dist}} (E_1), \ \Gamma (E_2) \bigr) \ 
            & \simeq \ \Gamma \bigl( M \times M, \ \Hom (p_2^* E_1, p_1^* E_2) \bigr), \\
        K \ & \leftrightarrow \ k,
    \end{align*} 
    其中 $L$ 表示连续线性映射的空间.
\end{theorem}

对于广义 Laplace 算子的热方程而言,
定理说明热方程的解的存在性等价于热核的存在性,
但它们的存在性需要我们稍后证明.

\begin{definition}
    设 $H \: \Gamma (E) \to \Gamma (E)$ 是广义 Laplace 算子.
    它的\term{热核} (heat kernel) 是一族截面
    \[ k_t \in \Gamma \bigl( M \times M, \ \Hom (p_2^* E, p_1^* E) \bigr)
        \quad (t > 0), \]
    关于 $t$ 有一阶导数, 并满足
    \[ \Biggl \{ \begin{array}{l}
        (\partial_t + H_x) \, k_t (x, y) = 0, \\
        k_t (x, y) |_{t=0} = \delta (x, y).
    \end{array} \]
    这里, 第二个式子的严格意义是, 对任意 $s \in \Gamma(E)$, 有一致收敛性
    \[ \lim _{t \to 0} K_t (s) = s, \]
    这里 $K_t$ 由 (\ref{eq-7-kernel}) 给出.
\end{definition}

热核的存在性将在后面证明, 但我们先叙述这一结论.

\begin{theorem} \label{thm-7-heat-ker-exists}
    每个广义 Laplace 算子都有热核.
\end{theorem}


\subsection{渐进展开}

我们先假定热核的存在性, 然后研究它在 $t \to 0$ 时的渐进行为.
我们将从中得到关于曲率的信息, 并在以后把它和示性类联系起来.

和前面一样, 我们设 $M$ 是紧 Riemann 流形, $E \to M$ 是光滑向量丛,
$H \: \Gamma (E) \to \Gamma (E)$ 是一个广义 Laplace 算子.

我们固定 $y \in M$, 然后取法坐标系
\[ x = \exp_y X. \]
我们记
\[ q_t (x, y) = \frac {1} {(4 \uppi t)^{n/2}} \, \upe^{-|X|^2 / 4t}, \]
并将 $H$ 的热核 $k_t$ 与它进行比较. 我们希望得到形如
\[ k_t = q_t \cdot \sum_{i=0}^\infty t^i \, \Phi_i \]
的表达式,
其中 $\Phi_i \in \Gamma \bigl( M \times M, \ \Hom (p_2^* E, p_1^* E) \bigr)$.
我们试着算出这些 $\Phi_i$.

我们考虑 $E$ 上由 (\ref{thm-7-gen-lap}) 给出的联络 $\nabla$.

\begin{lemma}
    设 $s_t \in \Gamma(E)$ ($t > 0$) 是一族截面, 关于 $t$ 有一阶导数. 则
    \[ (\partial_t + H) (q_t s_t) =
        q_t \, (\partial_t + t^{-1} \, \nabla_R + B) \, s_t, \]
    其中 $R = X^i e_i$ 是径向向量场, 
    微分算子 $B = \sqrt{g}{}^{1/2} \circ H \circ \sqrt{g}{}^{-1/2}$,
    其中 $\sqrt{g}$ 是 $\sqrt{\det \smash{(g_{ij})}}$ 的缩写.
\end{lemma}

\begin{proof}
    我们按 Leibniz 法则展开式子左边:
    \[ \text{左边} = (\partial_t q_t) \, s_t + q_t \, (\partial_t s_t) +
        (\Delta q_t) \, s_t + q_t \, (Hs_t) -2 \langle \nabla q_t, \nabla s_t \rangle, \]
    其中 $\Delta$ 是 $M$ 上的 Laplace 算子.
    而由热核 $q_t$ 的性质, 以及法坐标系中计算协变导数和 Laplace 算子的公式, 我们有
    \begin{align*}
        -2 \nabla q_t &= q_t \, (t^{-1} \, R + d \log \sqrt{g}), \\
        (\partial_t + \Delta) \, q_t &= \sqrt{g}{}^{1/2} \, 
            (\Delta \sqrt{g}{}^{-1/2}) \, q_t,
    \end{align*}
    故只需证
    \[ B = H + \nabla_{d \log \sqrt{g}} + 
        \sqrt{g}{}^{1/2} \, (\Delta \sqrt{g}{}^{-1/2}), \]
    直接计算即可验证这一等式.
\end{proof}

\begin{definition}
    如果给定一族截面
    \[ \Phi_i \in \Gamma \bigl( M \times M, \ \Hom (p_2^* E, p_1^* E) \bigr)
        \quad (i = 0, 1, \dotsc), \]
    使得对每个 $y \in M$, 都有
    \begin{numcases}{}
        \label{eq-7-formal-sol-1} 
        \Phi_0 (y,y) = I \qquad \in \End (E_y), \\
        \label{eq-7-formal-sol-2}
        (\partial_t + t^{-1} \, \nabla_R + B) \, 
            \sum_{i=0}^\infty t^i \, \Phi_i (-, y) = 0,
    \end{numcases}
    那么形式幂级数
    \[ k_t (x, y) = q_t (x, y) \cdot \sum_{i=0}^\infty t^i \, \Phi_i (x, y) \]
    称为热方程的一个\term{形式解} (formal solution).
\end{definition}

在引理中取 $s_t = \sum_i t^i \, \Phi_i (x, y) \, v$, 其中 $v \in E_y$,
我们看出, 如果定义中给出的 $k_t$ 确实是热核, 那么 (\ref{eq-7-formal-sol-2}) 一定成立.
事实上, 这个定义描述了热核 $k_t$ 在对角线 $M \subset M \times M$ 附近的渐进展开.

\begin{theorem}
    存在唯一的形式解. 并且, $\Phi_i$ 满足递归公式
    \[ \Phi_i (x, y) = - \int_0^1 s^{i-1} B_x \, \Phi_{i-1} (x_s, y) \, ds, \]
    其中 $B_x$ 表示 $B$ 作用于 $x$ 分量, $x_s = \exp_y (sX)$.
\end{theorem}

\begin{proof}
    实际上, (\ref{eq-7-formal-sol-2}) 等价于
    \[ \Biggl\{ \begin{array}{l}
        \nabla_R \, \Phi_0 = 0, \\
        (\nabla_R + i) \, \Phi_i = -B_x \, \Phi_{i-1}.
    \end{array} \]
    第一个式子说明, $\Phi_0 (x, y) \: E_y \to E_x$ 一定是沿径向的平行移动.
    为了从第二个式子导出递归公式, 考虑
    \[ \phi_i (s) = s^i \, \Phi_i (x_s, y). \]
    则 $\phi_i (0) = 0$, 且
    \[ \phi'_i (s) = (i s^{i-1} + s^i \nabla_{R/s}) \, \Phi_i (x_s, y) 
        = -s^{i-1} B_x \, \Phi_{i-1} (x_s, y). \qedhere \]
\end{proof}

\begin{example}
    设 $H = \Delta$ 是联络给出的 Laplace 算子. 由递归公式,
    \begin{align*}
        \Phi_1 (y, y) &= -B_x \, \Phi_0 (x, y) |_{x=y} \\
        &= - \Delta \bigl( \sqrt{g}{}^{-1/2} \, \Phi_0 (x, y) \bigr) \big|_{x=y}
            \hspace{3em} ( \text{因 $\sqrt{g} \big|_{x=y} = 1$} ) \\
        &= - \bigl( \Delta \sqrt{g}{}^{-1/2} \bigr) \, \Phi_0 (y, y)
            - \sqrt{g}{}^{-1/2} \, \cancel{\Delta \Phi_0 (x, y)} |_{x=y} \\
            & \hspace{3em} {} + 2 \bigl\langle \nabla \sqrt{g}{}^{-1/2}, 
            \cancel{\nabla \Phi_0 (x, y)} \bigr\rangle \big|_{x=y} \\
        &= \frac{1}{3} R \cdot \mathrm{id},
    \end{align*}
    其中 $R$ 是 $M$ 的标量曲率, 被划掉的部分等于 $0$, 因为对任何 $v \in E_y$,
    向量场 $\Phi_0 (x, y) \, v$ 都是平行移动得到的,
    故在 $x = y$ 处的所有协变导数都为 $0$. 最后一步是通过公式
    \[ g_{ij} = \delta_{ij} + \frac{1}{3} R_{ikjl} x^k x^l + \text{高阶项} \]
    得到的. 这个例子说明, $\Phi_1$ 蕴涵了关于流形曲率的信息. \varqed
\end{example}

我们将上面的讨论总结如下.

\begin{theorem} \label{thm-7-asymp-exp}
    在对角线 $M \subset M \times M$ 的一个邻域内,
    $H$ 的热核 $k_t$ 在 $t \to 0$ 时具有如下渐进展开式:
    \[ k_t (x, y) \sim \frac {1} {(4 \uppi t)^{n/2}} \, \upe^{-d(x, y)^2 / 4t} \cdot
        \sum_{i=0}^\infty t^i \, \Phi_i (x, y), \]
    其前 $N$ 项部分和在 $C^k$-范数下与 $k_t$ 的误差是 $O (t^{N - n/2 - k/2 + 1})$.
\end{theorem}

这里, 最后的误差估计是由下一小节的构造过程给出的.


\subsection{热核的构造}

这一小节, 我们简要概述如何证明广义 Laplace 算子的热核的存在性 (\ref{thm-7-heat-ker-exists}).

为了演示我们构造的方法, 我们先从一个简化的模型开始.
在这个简化模型中, 我们将无限维空间上的算子 $H \: \Gamma (E) \to \Gamma (E)$
换成有限维空间上的算子.

下面, 设 $V$ 是有限维向量空间, $H \in \End (V)$. 设
\[ K_t \in \End (V) \quad (t \geq 0) \]
是一族算子, 它是热方程 $(\partial_t + H) \, K_t = 0$ 的 ``近似解''.
准确地说, 它满足
\[ \Biggl\{ \begin{array}{ll}
    R_t = (\partial_t + H) \, K_t = O (t^\alpha) \quad (t \to 0), \\
    K_0 = \mathrm{id}_V,
\end{array} \]
其中 $\alpha \geq 0$ 是某个常数. 我们打算对 $K_t$ 进行调整,
从而它成为一个真正的解.

为此, 我们定义 
{\allowdisplaybreaks \begin{flalign*}
    && Q_t^0 &= K_t, & \\
    & \text{从而} &
        (\partial_t + H) \, Q_t^0 &= R_t; \\
    & \text{再定义} &
        Q_t^1 &= \int_0^t K_{t-t_1} \, R_{t_1} \, dt_1, \\
    & \text{从而} &
        (\partial_t + H) \, Q_t^1 &= R_t + \int_0^t R_{t-t_1} \, R_{t_1} \, dt_1; \\
    & \text{再定义} &
        Q_t^2 &= \int \limits _{0 \leq t_1 \leq t_2 \leq t}
        K_{t-t_2} \, R_{t_2-t_1} \, R_{t_1} \, dt_1 \, dt_2, \\
    & \text{从而} &
        (\partial_t + H) \, Q_t^2 &= R_t + \int_0^t R_{t-t_1} \, R_{t_1} \, dt_1 \\
        &&& \hspace{3em} {} + \int \limits _{0 \leq t_1 \leq t_2 \leq t} 
        R_{t-t_2} \, R_{t_2-t_1} \, R_{t_1} \, dt_1 \, dt_2,
\end{flalign*}}%
并以此类推.

\begin{proposition}
    级数 
    \[ Q_t = \sum_{k=0}^{\infty} (-1)^k Q_t^k \]
    收敛, 且给出了热方程的解. 并且,
    \[ Q_t = K_t + O(t^{1+\alpha}). \]
\end{proposition}

\begin{proof}
    对每个 $t > 0$, 存在常数 $C$, 使得对任意 $t' \leq t$, 有
    \[ |K_{t'}| < C, \quad |R_{t'}| < C. \]
    从而
    \begin{align*}
        |Q_t^k| &< C^{k+1} \int \limits _{0 \leq t_1 \leq \cdots \leq t_k \leq t}
            dt_1 \, \cdots \, dt_k \\
        &= C^{k+1} \, \frac{t^k}{k!}.
    \end{align*}
    这蕴涵了级数的收敛性. 用同样的方法, 借助上面的计算, 不难验证 $Q_t$ 是热方程的解.
\end{proof}

下面, 我们来考虑一般的情况, 即 $H \: \Gamma (E) \to \Gamma (E)$
是广义 Laplace 算子的情况.

我们注意到一个事实: 如果算子 $P, Q$ 的核分别是 $p, q$,
那么算子 $PQ$ 的核是卷积 $p * q$, 定义为
\[ (p * q)(x, y) = \int_M p(x, z) \circ q(z, y) \, dz. \]

和在简化模型中一样, 我们首先取定一个近似解 $K_t$.
对 $s \in \Gamma (E)$, 我们令
\[ (K_t s) (x) = \int_M k_t^N (x, y) \, s(y) \, dy, \]
其中 $N$ 是取定的正整数, $k_t^N$ 定义为
\[ k_t^N (x, y) = \psi(d(x,y)) \cdot 
    \bigl( \text{(\ref{thm-7-asymp-exp}) 中的部分和 $\textstyle \sum_0^N$} \bigr), \]
其中 $\psi$ 是一个光滑函数, 在 $0$ 附近取值恒为 $1$, 且 $\psi([\epsilon, +\infty)) = 0$,
其中 $\epsilon > 0$ 取得使 $k_t^N$ 在整个 $M \times M$ 上有定义.

为使记号简洁, 下面我们就记
\[ k_t = k_t^N. \]

我们和之前一样定义算子 $Q_t^k$. 它的核是
\[ q_t^k = \int \limits_{0 \leq t_1 \leq \cdots \leq t_k \leq t}
    k_{t-t_k} * r_{t_k-t_{k-1}} * \cdots * r_{t_1} \, dt_1 \, \cdots \, dt_k, \]
其中 $r_t = (\partial_t + H_x) \, k_t$ 是算子 $R_t$ 的核.

通过直接计算, 我们能给出估计
\[ \| r_t \|_{C^\ell} \leq C(\ell) \, t^{N - n/2 - \ell/2}. \]
计算过程不在这里给出, 见 \cite[定理~2.29]{bgv}.
注意到这里 $t^{N - n/2}$ 就是 $k_t^N$ 的最高次项.
由 $q_t^k$ 的卷积表达式, 我们有
\[ \| q_t^k \|_{C^\ell} \leq C(\ell) \, t^{k(N - n/2) - \ell/2} \, t^k
    \operatorname{vol} (M)^{k-1}, \]
这里 $C(\ell)$ 是和上面不同的常数; $\operatorname{vol} (M)^{k-1}$ 来自于卷积的定义.
卷积中 $k_t$ 一项没有出现在估计中, 因为算子 $K_t$ 关于 $t$ 一致有界
(参见上述引用的定理).

这一估计给出了级数 $\sum_k (-1)^k q_t^k$ 的收敛性.

\begin{theorem}
    当 $N > n/2 + 1$ 时, 级数
    \[ q_t = \sum_{k=0}^\infty (-1)^k q_t^k \]
    在 $C^2$-范数的意义下收敛.
    这个公式给出了算子 $H$ 的热核. \qed
\end{theorem}

作为推论, (\ref{thm-7-heat-ker-exists}) 和 (\ref{thm-7-asymp-exp})
都获得了证明.

