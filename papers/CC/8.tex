Dirac 算子的理论起源于量子力学.
为了在量子力学中引入狭义相对论,
需要对波函数的 Laplace 算子开平方而得到 Hamilton 算子.
如果波函数是普通的复值函数, 那么开平方的操作是不可能做到的.
Dirac 因而提出, 波函数应该是取值于某个 $\bbC^n$ 的函数.
这个空间 $\bbC^n$ 被称为\term{旋量空间} (spinor space).
在这个框架下, Dirac 的 Hamilton 算子自然地给出了 Clifford 代数在旋量空间中的表示.


\subsection{Dirac 算子与 Clifford 代数}

在这一节中, 设 $M$ 是 $n$ 维 Riemann 流形, $E \to M$ 是光滑向量丛.

\begin{definition}
    $E$ 上的一个 \term{Dirac 算子} (Dirac operator) 是一个一阶微分算子
    $D \: \Gamma (E) \to \Gamma (E)$, 使得 $D^2$ 是广义 Laplace 算子.
\end{definition}

\begin{example} \label{eg-8-dirac-d-dstar}
    若 $E = \wedge^\bullet \, T^* M$ 是微分形式的向量丛,
    则 $d + d^*$ 是一个 Dirac 算子. 它的平方是
    \[ (d + d^*)^2 = d d^* + d^* d, \]
    称为\term{微分形式的 Laplace 算子}.
    它与 Levi-Civita 联络给出的 Laplace 算子相差一个曲率项.
    这一关系称为 Weitzenböck 恒等式, 这里就不详述了.
    
    再例如, 若 $M$ 是 Kähler 流形, $E$ 是全纯微分形式的向量丛,
    则 $\dbar + \dbar^*$ 是一个 Dirac 算子. 此时, 相应的
    Weitzenböck 恒等式被称为 Bochner--小平 (Kodaira) 恒等式. \varqed
\end{example}

下面, 设 $D$ 是 $E$ 上的一阶微分算子. 则在局部坐标系中, 它具有
\[ D = \sum_{k=1}^n a^k \partial_k + b \]
的形式, 其中 $a^k, b \in \Gamma( \End (E) )$. 从而
\[ D^2 = \frac12 (a^i a^j + a^j a^i) \, \partial_i \partial_j + (\text{一阶算子}), \]
其中系数 $1/2$ 是因为 $\partial_i \partial_j$ 和 $\partial_j \partial_i$
在求和时被重复计算了. 因此, $D$ 是 Dirac 算子等价于
\[ a^i a^j + a^j a^i = -2 g^{ij}. \]
这个关系启发了 Clifford 代数的定义.

\begin{definition}
    设 $V$ 是 $n$ 维实向量空间, $Q$ 是 $V$ 上的二次型, 等同地看作对称双线性型.
    \term{Clifford 代数}定义为商代数
    \[ \Cl (V, Q) = \frac {T(V)} {v \cdot w + w \cdot v = -2 Q(v, w)}, \]
    其中 $T(V) = \bigoplus_{k \geq 0} V^{\otimes k}$ 是 $V$ 上的张量代数.
    
    如果 $Q$ 非退化, 其标准型中有 $p$ 个正项和 $q$ 个负项, 我们就记 
    \[ \Cl_{p,q} (V) = \Cl (V, Q). \]
    我们记 $\Cl (V) = \Cl_{n,0} (V)$.
\end{definition}

不难看出, $\Cl (V, Q)$ 的维数是 $2^n$, 它的一组基是
\[ \{ e_{i_1} \cdots e_{i_k} \mid i_1 < \cdots < i_k \}, \]
其中 $\{ e_i \}$ 是 $V$ 的一组基.

\begin{example}
    我们有 $\Cl (\bbR^2) \simeq \bbH$.
    事实上, 记 $\upi, \upj$ 为 $\bbR^2$ 的标准正交基, 
    则 Clifford 代数的定义给出的关系是
    \[ \upi^2 = \upj^2 = -1, \quad \upi \, \upj + \upj \, \upi = 0. \varqedhere \]
\end{example}

由于张量代数的 $\bbZ$-分次在 $\Cl (V, Q)$ 中得到保留,
后者也是 $\bbZ$-分次代数. 特别地, 我们有:

\begin{proposition}
    Clifford 代数是超代数. \qed
\end{proposition}

\begin{definition}
    一个 \term{Clifford 模}是指 Clifford 代数 $\Cl (V, Q)$ 在一个超向量空间 $E$ 中的表示.
    也就是说, 我们有 (保持 $\bbZ_2$-分次的) 超代数同态
    \[ \Cl (V, Q) \to \End (E). \]
\end{definition}

\begin{example} \label{eg-8-symbol-map}
    超向量空间 $\wedge^\bullet \, V$ 具有自然的 $\Cl (V, Q)$-模的结构, 定义为
    \[ v \cdot \alpha = v \wedge \alpha - Q (v, -) \contr \alpha \quad
        (v \in V, \ \alpha \in \wedge^\bullet \, V), \]
    其中 $\contr$ 表示缩并. 
    若取 $V$ 关于 $Q$ 的一组正交基 $\{ e_1, \dotsc, e_n \}$,
    则上式可以写成
    \[ e_i \cdot (e_{i_1} \wedge \cdots \wedge e_{i_k}) = \Biggl\{
        \begin{array}{ll}
            e_i \wedge e_{i_1} \wedge \cdots \wedge e_{i_k}, & i \notin \{ i_1, \dotsc, i_k \}, \\
            -Q(e_i) \, e_{i_2} \wedge \cdots \wedge e_{i_k}, & i = i_1.
        \end{array} \]
    映射
    \[ \sigma \: \Cl (V, Q) \overset{\cdot 1}{\longrightarrow} \wedge^\bullet \, V \]
    是超向量空间的同构, 称为\term{符号映射} (symbol map). \varqed
\end{example}


\subsection{旋量群}

旋量群 $\Spin (n)$ 的一个著名的性质是,
它是 $\SO (n)$ 的双重覆叠 (在 $n \geq 3$ 时是万有覆叠).
也就是说, 我们有 Lie 群的正合列
\[ 0 \to \bbZ_2 \to \Spin (n) \to \SO (n) \to 0. \]
通过 Clifford 代数, 我们可以得到旋量群的一个具体的构造.

\begin{definition}
    设 $V$ 是 $n$ 维欧氏空间. 我们定义 Lie 代数
    \[ \spin (V) = \Cl (V)_2 \]
    为 Clifford 代数中分次为 $2$ 的部分, 其 Lie 括号就是 Clifford 代数中的交换子.
\end{definition}

换言之, 如果 $e_1, \dotsc, e_n$ 是 $V$ 的一组标准正交基, 那么
\[ \spin (V) = \operatorname{span} {} \{ e_i e_j \mid i \neq j \}, \]
这里, 注意到我们有反交换关系 
\[ e_i e_j = -e_j e_i. \]

\begin{proposition} \label{thm-8-spin-alg-action}
    Lie 代数 $\spin (V)$ 可以作用在 $V \subset \Cl (V)$ 上,
    作用的方式是 $a \cdot v = [a, v]$. 这个作用诱导了 Lie 代数同构
    \[ \spin (V) \simeq \so (V). \]
\end{proposition}

\begin{proof}
    首先, 我们说明对应的映射 $\phi \: \spin (V) \to \gl (V)$ 的像落在 $\so (V)$ 中.
    取 $V$ 的标准正交基 $e_1, \dotsc, e_n$. 则元素 $e_i e_j \in \spin (V)$ 的作用是
    \[ \left \{ \begin{array}{l}
        e_i \mapsto 2e_j, \\
        e_j \mapsto -2e_i, \\
        e_k \mapsto 0 \qquad (k \neq i, j).
    \end{array} \right . \]
    这一线性映射的矩阵是反对称矩阵. 这也说明 $\phi$ 是单射.
    但 $\dim \spin (V) = \dim {\wedge^2 \, V} = \dim \so (V)$,
    从而 $\phi$ 是到 $\so (V)$ 的同构.
\end{proof}

\begin{definition}
    $V$ 的\term{旋量群} (spin group) 定义为
    \[ \Spin (V) = \exp \bigl( \spin (V) \bigr), \]
    其中 $\exp$ 是指在 $\Cl (V)$ 中的幂级数.
\end{definition}

事实上, Lie 理论告诉我们,
$\Spin (V)$ 是一个以 $\spin (V)$ 为 Lie 代数的连通 Lie 群,
它是 $\Cl (V)$ 的偶数阶元素的乘法群的 Lie 子群.

\begin{example}
    设 $\{ e_i \}$ 是 $V$ 的标准正交基. 则 $i \neq j$ 时, $(e_i e_j)^2 = -1$. 因此
    \[ \exp ( t e_i e_j ) = \sum _{k=0} ^\infty \frac {t^k} {k!} \, (e_i e_j)^k
        = \cos t + (\sin t) \, e_i e_j \quad (t \in \bbR). \]
    特别地, 当 $\dim V = 2$ 时, 我们得到了圆圈群
    \[ \Spin (2) \simeq S^1. \]
    当 $\dim V = 3$ 时, 我们可以将 $\Cl (V)$ 的子代数
    $\operatorname{span} \, \{ 1, e_1 e_2, e_2 e_3, e_3 e_1 \}$ 和 $\bbH$ 等同起来.
    此时, 旋量群就是单位四元数的乘法群:
    \[ \Spin (3) \simeq S^3 \subset \bbH, \]
    因为指数映射会把虚四元数映到单位四元数. \varqed
\end{example}

\begin{proposition}
    当 $\dim V \geq 2$ 时, $\Spin (V)$ 是 $\SO (V)$ 的双重覆叠.
\end{proposition}

\begin{proof}
    考虑 $\Spin (V)$ 在 $V$ 上的作用:
    \[ g \cdot v = g v g^{-1} \quad (g \in \Spin (V), \ v \in V). \]
    我们说明, 这个作用诱导了到 $\SO (V)$ 的双重覆叠.
    
    首先, 这个 Lie 群表示对应的 Lie 代数表示就是 (\ref{thm-8-spin-alg-action}) 给出的同构.
    因此, 对应的映射 $\Spin (V) \to \SO (V)$ 是覆叠映射.
    因为元素 $-1 \in \Spin (V)$ 的作用是平凡的, 所以这个覆叠至少是二重的.
    但 $\dim V \geq 3$ 时, $\SO (V)$ 的基本群是 $\bbZ_2$, 所以这个覆叠至多是二重的;
    $\dim V = 2$ 的情况可以直接计算而得出.
\end{proof}

当 $\dim V = 1$ 时, 为了和上述结论保持一致, 我们可以重新定义
\[ \Spin (1) = \Spin (V) = \{ \pm 1 \} \subset \Cl (V). \]


\subsection{旋量}

\term{旋量} (spinor) 的名字是仿照向量 (vector) 的名字而造出来的.
这样命名的原因是, 旋量和向量有类似的性质, 它们的区别在于,
向量的旋转是在 $\SO (n)$ 中完成的, 而旋量的旋转是在 $\Spin (n)$ 中完成的,
因此, 旋量有时需要转两圈, 才能回到原来的位置.

例如, 手的角度 (旋转位置) 可以由一个单位旋量描述.
伸出右手, 掌心向上, 然后将手掌逆时针旋转一周.
这时, 手臂会处于一个不自然的位置.
接下来, 将手举过头顶, 不改变手的朝向.
这样, 手掌可以继续逆时针旋转一周 (从下向上看是顺时针),
然后恢复到正常的位置.
这一过程说明, 手需要旋转两周才能回到原来的位置.

事实上, 在三维空间中, 旋量构成的空间就是 $\bbH$,
而旋量群 $\Spin (3)$ 是单位四元数的乘法群, 它通过乘法作用在旋量空间上.

一般地说, \term{旋量空间} (spinor space), 即所有旋量构成的空间,
是旋量群 $\Spin (n)$ 的一个表示, 并且, 这个表示无法通过 $\SO (n)$ 的表示得到.

\begin{theorem} \label{thm-8-def-spinor}
    设 $V$ 是 $n$ 维欧氏空间.
    \begin{itemize}
        \item
            若 $n$ 为偶数, 则存在一个 $2^{n/2}$ 维复 Clifford 模 $S$, 使得
            \[ \Cl (V) \otimes \bbC \simeq \End (S), \]
            这个同构与二者在 $S$ 上的作用相容.
            并且, $\dim S^+ = \dim S^- = 2^{n/2-1}$.
        \item 
            若 $n$ 为奇数, 则存在一个 $2^{(n-1)/2}$ 维复 Clifford 模 $S$,
            它不带有 $\bbZ_2$-分次, 使得作为普通 (非 $\bbZ_2$-分次) 的代数, 有
            \[ \Cl (V) \otimes \bbC \simeq \End (S) \oplus \End (S), \]
            这个同构与二者在 $S$ 上的作用相容.
    \end{itemize}
    在两种情况下, 超向量空间 $S$ 称为\term{旋量空间} (spinor space).
\end{theorem}

在证明定理之前, 我们先给出这一结果的一些推论.

我们知道, 矩阵代数 $\End (S)$ 是单代数,
它唯一的不可约表示就是它在 $S$ 上的作用. 因此, 定理蕴涵了以下结论.

\begin{corollary}
    若 $n$ 为偶数, 则 $\Cl (V)$ 的所有复表示都具有
    \[ W \otimes S \]
    的形式, 其中 $W$ 是超向量空间, 带有平凡的 Clifford 作用. \qed
\end{corollary}

因为旋量群 $\Spin (V)$ 是 Clifford 代数的乘法子群,
所以, 定理也蕴涵了关于旋量群的表示的结论.

\begin{corollary}
    旋量空间 $S$ 是旋量群 $\Spin (V)$ 的一个表示.
    当 $n$ 为偶数时, 它分裂成旋量群的两个不可约表示
    \[ S = S^+ \oplus S^-. \]
    当 $n$ 为奇数时, $S$ 本身就是不可约的.
\end{corollary}

\begin{proof}
    记 $\Cl^+ (V) \subset \Cl (V)$ 为偶数阶元素构成的子代数, 
    则 $\Spin (V) \subset \Cl^+ (V)$.
    因为 $\Spin (V)$ 包含 $\Cl^+ (V)$ 作为向量空间的一组基,
    所以 $\Cl^+ (V)$ 的不可约表示一定诱导 $\Spin (V)$ 的不可约表示.
    
    当 $n$ 为偶数时,
    \[ \Cl^+ (V) \otimes \bbC \simeq \End^+ (S) \simeq \End (S^+) \oplus \End (S^-), \]
    从而 $S$ 分裂为 $\Cl^+ (V)$ 的两个不可约表示, 即 $S^+$ 和 $S^-$.
    
    当 $n$ 为奇数时, 定理的证明 (在下方) 表明
    \[ \Cl^+ (V) \otimes \bbC \simeq \End (S). \qedhere \]
\end{proof}

\begin{proof} [定理 \ref{thm-8-def-spinor} 的证明]
    取 $V$ 的标准正交基 $e_1, \dotsc, e_n$.
    
    若 $n$ 是偶数, 我们令
    \begin{align*}
        P &= \operatorname{span}
            ( e_1 - \upi e_2, \ e_3 - \upi e_4, \dotsc, e_{n-1} - \upi e_n ), \\
        S &= \wedge^\bullet \, P.
    \end{align*}
    则 $V \otimes \bbC = P \oplus \oline2P$.
    我们定义 $\Cl (V) \otimes \bbC$ 在 $S$ 上的作用如下:
    对 $v \in V \otimes \bbC$, $s \in S$, 定义
    \[ v \cdot s = \Biggl\{ \begin{array}{ll}
        \sqrt2 \ v \wedge s, & v \in P, \\
        -\sqrt2 \ Q(v, -) \contr s, & v \in \oline2P,
    \end{array} \]
    其中记号的意义同 (\ref{eg-8-symbol-map}).
    因为 $\dim \Cl (V) = 2^n = (\dim S)^2$, 并且,
    通过计算可以验证, $\Cl (V)$ 非零元素的作用都非零.
    因此, 这个作用诱导了同构
    \[ \Cl (V) \otimes \bbC \simeq \End (S). \]
    
    若 $n$ 是奇数, 记 $W = \operatorname{span} (e_1, \dotsc, e_{n-1})$. 
    我们有 (非 $\bbZ_2$-分次的) 同构
    \begin{align*}
        ( \Cl (W) \oplus \Cl (W) ) \otimes \bbC 
            &\simeq \Cl (V) \otimes \bbC, \\
        ( e_{i_1} \cdots e_{i_k} , \ 0 )
            &\mapsto \frac12 \bigl( e_{i_1} \cdots e_{i_k}
            + (-1)^\sigma \, \upi^{(n+1)/2} \, e_{j_1} \cdots e_{j_{n-k}} \bigr), \\
        ( 0 , \ e_{i_1} \cdots e_{i_k} )
            &\mapsto \frac12 \bigl( e_{i_1} \cdots e_{i_k}
            - (-1)^\sigma \, \upi^{(n+1)/2} \, e_{j_1} \cdots e_{j_{n-k}} \bigr),
    \end{align*}
    其中 $j_1, \dotsc, j_{n-k}$ 表示 $1, \dotsc, n$ 中除了 $i_1, \dotsc, i_k$ 外的指标,
    $\sigma$ 是 $1, \dotsc, n$ 的排列 $i_1, \dotsc, i_k, j_1, \dotsc, j_{n-k}$ 的奇偶性.
\end{proof}


\subsection{流形上的 Clifford 模}

为了将 Clifford 代数的理论应用到 Dirac 算子上,
我们需要在流形上考虑余切空间的 Clifford 代数, 它们构成一个向量丛.

\begin{definition}
    设 $M$ 是流形. $M$ 上的 \term{Clifford 丛} 
    \[ \Cl(M) \to M \]
    是在点 $x \in M$ 处的纤维为 $\Cl (T_x^* M)$
    的代数丛. $M$ 上的一个 \term{Clifford 模}是一个超向量丛
    \[ E \to M, \]
    带有 Clifford 丛的作用.
\end{definition}

为了将上一小节的理论应用到这一情形,
我们还需要将旋量空间也变成 $M$ 上的向量丛.
这需要 $M$ 具有一个\term{旋量结构}.

\begin{definition}
    设 $E \to M$ 是秩为 $n$ 的可定向光滑实向量丛.
    $E$ 的一个\term{旋量结构} (spin structure)
    是一个 $\Spin (n)$-主丛
    \[ \Spin (E) \to M, \]
    使得它对应的 $\SO (n)$-主丛与 $E$ 对应的 $\SO (n)$-主丛同构.
    
    流形 $M$ 的一个\term{旋量结构}是指余切丛 $T^* M$ 上的旋量结构.
    带有一个旋量结构的流形称为\term{旋量流形} (spin manifold).
\end{definition}

旋量结构的存在性可以通过示性类来判断.

\begin{theorem}
    设 $E \to M$ 是秩为 $n$ 的可定向光滑实向量丛.
    则 $E$ 上存在旋量结构, 当且仅当其第二 Stiefel--Whitney 类消失, 即
    \[ w_2 (E) = 0. \]
\end{theorem}

这一结论的证明可参见 \cite[定理 II.1.7]{lawson-michelsohn}.

\begin{definition}
    设 $M$ 是旋量流形. 则 $M$ 上的\term{旋量丛} (spinor bundle)
    \[ S(M) \to M \]
    是一个复向量丛, 其纤维是由 $M$ 的旋量结构给出的旋量空间.
\end{definition}

具体地说, 这一定义是这样完成的:
$\Spin (n)$-主丛的转移映射是到 $\Spin (n)$ 的映射,
而 $\Spin (n)$ 可以映到 $\End (S)$, 给出旋量丛的转移映射.

当 $\dim M$ 是偶数时, 旋量丛是一个超向量丛 $S^+ (M) \oplus S^- (M)$,
也是 $M$ 上的 Clifford 模. 由前几个小节的讨论, 我们得到下面的结论.

\begin{proposition} \label{thm-8-cl-mod-can-form}
    设 $M$ 是偶数维旋量流形. 则 $M$ 上的每个复 Clifford 模都具有
    \[ W \otimes S(M) \]
    的形式, 其中 $W$ 是超向量丛, 带有 Clifford 丛的平凡作用.
\end{proposition}

\begin{proof}
    我们只指出下面的事实: 如果 $E \simeq W \otimes S$ 是 Clifford 代数 $\Cl (V)$ 上的模,
    那么 $W \simeq \Hom_{\Cl (V)} (S, E)$.
\end{proof}


\subsection{Dirac 算子与 Clifford 超联络}

现在, 我们可以让 Dirac 算子回到我们的视野了.

\begin{definition}
    设 $E \to M$ 是光滑的超向量丛. $E$ 上的 \term{Dirac 算子}的定义和之前一样,
    但我们额外要求 $D$ 是\term{奇算子}, 它把 $E^\pm$ 的截面分别变成 $E^\mp$ 的截面.
\end{definition}

设 $D \: \Gamma (E) \to \Gamma (E)$ 是 Dirac 算子. 则局部坐标下, $D$ 具有如下形式:
\[ D = \sum_k a^k \partial_k + b, \]
其中系数 $a^k \in \Gamma (\End^- (E))$ 满足 Clifford 代数的关系.
因此, Clifford 丛 $\Cl (M)$ 可以通过这些 $a^k$ 作用在 $E$ 上,
这赋予了 $E$ 一个 Clifford 模的结构.

反过来, 在每个 Clifford 模上, 我们可以构造出相应的 Dirac 算子,
但常数项 $b$ 可以随意改变. 因此, 这些 Dirac 算子构成一个 $\Gamma (\End^- (E))$-齐性空间.

下面, 我们打算把一个 Clifford 模上的 Dirac 算子与 Clifford 超联络对应起来.

\begin{definition}
    Clifford 模 $E$ 上的 \term{Clifford 超联络}是指一个超联络 $\nabla^E$,
    满足如下的 Leibniz 法则: 若
    $a \in \Gamma ( \Cl (M) )$,\ $X \in \Gamma ( TM ),$\ $s \in \Gamma (E)$, 则
    \[ \nabla^E_X (a \cdot s) = (\nabla_X a) \cdot s + (-1)^{|a|} \, a \cdot \nabla^E_X s, \]
    其中 $\nabla$ 是 Levi-Civita 联络. 这个关系可以更简洁地写成
    \[ [ \nabla^E, \ a \cdot {} ] = \nabla a \cdot {} \ . \]
\end{definition}

\begin{proposition} \label{thm-8-nabla-s}
    设 $M$ 是偶数维旋量流形. 则 Levi-Civita 联络诱导了旋量丛
    $S = S(M)$ 上的 Clifford 超联络 $\nabla^S$.
\end{proposition}
    
\begin{proof}
    $\nabla^S$ 的构造如下: 标架丛 $\SO (T^* M)$ 上的 Levi-Civita
    联络诱导了主丛 $\Spin (M)$ 上的联络, 而由旋量丛 $S$ 的构造,
    这个联络诱导了 $S$ 上的联络 $\nabla^S$.
    
    为了验证 $\nabla^S$ 是 Clifford 超联络,
    只需验证它诱导的 $\End (S)$ 上的联络与 Clifford 丛的 Levi-Civita 联络相同.
    
    我们知道, $\Spin (n) \subset \Cl^+ (n)$ 包含了后者的一组基. 定义
    \[ \mathrm{Pin} (n) = \{ 1, e_1 \} \cdot \Spin (n) \subset \Cl (n), \]
    则 $\mathrm{Pin} (n)$ 包含 $\Cl (N)$ 的一组基.
    事实上, Lie 群 $\mathrm{Pin} (n)$ 有两个连通分支,
    其单位元所在的分支就是 $\Spin (n)$, 另一个分支包含于 $\Cl^- (n)$.
    主丛 $\Spin (M)$ 诱导了 $\mathrm{Pin} (n)$-主丛 $\mathrm{Pin} (M)$,
    它的联络就同时诱导了 $\End (S)$ 的联络和 Clifford 丛的联络.
\end{proof}

事实上, 超联络 $\nabla^S$ 在以下意义上是最本质的 Clifford 超联络.

\begin{proposition} \label{thm-8-cliff-conn}
    设 $M$ 是偶数维旋量流形, $W \to M$ 是复超向量丛. 则有一一对应
    \begin{align*}
        \{ \text{$W$ 上的超联络} \} &\leftrightarrow
            \{ \text{$W \otimes S$ 上的 Clifford 超联络} \}, \\
        \nabla^W &\leftrightarrow \nabla^W \otimes 1 + 1 \otimes \nabla^S.
    \end{align*}
\end{proposition}

\begin{proof}
    留给读者.
\end{proof}

下面, 我们构造 Clifford 超联络所对应的 Dirac 算子.

设 $M$ 是偶数维旋量流形, $E \to M$ 是 Clifford 模,
带有 Clifford 超联络 $\nabla$. 则在局部坐标下, 
这个联络作用在 $E$ 的截面上时具有如下形式:
\[ \nabla = \sum_{i=1}^n dx^i \otimes \partial_i
    + \sum_{\alpha} dx^\alpha \otimes b_\alpha, \]
其中 $\alpha$ 是 $M$ 方向的多重指标
(我们不妨只考虑递增排列、无重复的多重指标),
$b_\alpha$ 是 $\End (E)$ 的截面, 其奇偶性与 $|\alpha|$ 相反.
这里, 联络的一阶导数项是 $dx^i \otimes \partial_i$, 因为任何联络都满足这一点.

我们对应地定义 Dirac 算子
\[ D_{\nabla} = \sum_{i=1}^n (dx^i \cdot {}) \circ \partial_i
    + \sum_{\alpha} (dx^\alpha \cdot {}) \circ b_\alpha, \]
这里 $\cdot$ 表示 Clifford 作用,
例如 $dx^i$ 的作用就是我们之前的记号 $a^i$.

这一定义可以写成坐标无关的形式:

\begin{definition}
    设 $M$ 是偶数维旋量流形, $E \to M$ 是 Clifford 模,
    带有 Clifford 超联络 $\nabla$. 则 Dirac 算子 $D_{\nabla}$ 定义为
    \[ D_{\nabla} \: \Gamma (E)
        \overset{\nabla}{\longrightarrow} \Omega^\bullet (M, E)
        \simeq \Gamma (\wedge^\bullet \, T^* M \otimes E)
        \overset{\sigma^{-1}}{ \underset{\simeq}{\longrightarrow} }
            \Gamma (\Cl (M) \otimes E)
        \overset{\text{作用}}{\longrightarrow} \Gamma (E), \]
    其中 $\sigma$ 是 (\ref{eg-8-symbol-map}) 中的符号映射.
\end{definition}

\begin{theorem} \label{thm-8-dirac-cliff-conn}
    设 $M$ 是偶数维旋量流形, $E \to M$ 是复 Clifford 模.
    则上述对应关系是一一对应:
    \begin{align*}
        \{ \text{$E$ 上的 Clifford 超联络} \} &\leftrightarrow
            \{ \text{$E$ 上的 Dirac 算子} \}, \\
        \nabla &\leftrightarrow D_{\nabla}.
    \end{align*}
    这里, 我们当然要求 Dirac 算子是与 $E$ 的 Clifford 模结构相容的.
\end{theorem}

\begin{proof}
    我们已经给出了从左边集合到右边集合的映射.
    我们还需要证明这个映射是双射.
    
    我们回忆, 右边的集合是 $\Gamma (\End^- (E))$-齐性空间.
    而 (\ref{thm-8-cl-mod-can-form}) 说明
    $E$ 一定能写成 $E \simeq W \otimes S$ 的形式,
    因此, 有超向量丛的同构
    \[ \End (E) \simeq \End (S) \otimes \End (W)
        \simeq \Cl (M) \otimes \End_{\Cl (M)} (E), \]
    其中 $\End_{\Cl (M)} (E)$ 表示 $E$ 作为 Clifford 模的自同态,
    不要求保持 $\bbZ_2$-分次. 从而, 我们有同构
    \begin{multline*}
        \Gamma (\End^- (E))
        \simeq \Gamma^- \bigl( \Cl (M) \otimes \End_{\Cl (M)} (E) \bigr) \\
        \overset{\sigma}{ \underset{\simeq}{\longrightarrow} }
            \Gamma^- \bigl( \wedge^\bullet \, T^* M \otimes \End_{\Cl (M)} (E) \bigr)
        \simeq \Omega^- \bigl( M, \End_{\Cl (M)} (E) \bigr).
    \end{multline*}
    如果这个映射将截面 $s$ 映到形式 $\omega$, 那么
    \[ D_{\nabla} + s = D_{\nabla + \omega}. \]
    因此, 我们只需要验证, $\nabla + \omega$ 是 Clifford 超联络当且仅当
    \[ \omega \in \Omega^- \bigl( M, \End_{\Cl (M)} (E) \bigr). \]
    我们将这一验证留给读者.
\end{proof}

