在这一节中, 我们将前两节的内容整合起来, 证明 Atiyah--Singer 指标定理.


\subsection{Dirac 算子的指标}

在这一小节中, 设 $M$ 是紧 Riemann 流形,
$E \to M$ 是带有 Riemann 或 Hermite 度量的超向量丛
(我们要求 $E^+ \perp E^-$), 并考虑 $E$ 上的\term{自伴随} Dirac 算子 $D$.
也就是说, 我们有
\[ D = \biggl( \, \begin{matrix}
    0 & D^- \\ D^+ & 0
\end{matrix} \, \biggr), \quad D^- = (D^+)^*, \]
其中 $D^{\pm} \: E^{\pm} \to E^{\mp}$.
例如, (\ref{eg-8-dirac-d-dstar}) 给出的例子 $d + d^*$ 和 $\dbar + \dbar^*$
都是自伴随 Dirac 算子.

\begin{definition}
    自伴随 Dirac 算子 $D$ 的\term{指标} (index) 是
    \begin{align*}
        \ind (D) &= \dims \ker D \\
        &= \dim \ker D^+ - \dim \ker D^- \\
        &= \dim \ker D^+ - \dim \coker D^+.
    \end{align*}
\end{definition}

$D$ 的指标是有限的, 因为
\[ D^2 = D^+ D^- + D^- D^+ \]
是广义 Laplace 算子, 它是二阶椭圆算子,
而分析学告诉我们如下事实:

\begin{fact}
    算子 $D^2$ 定义了 Hilbert 空间 $\Gamma_{L^2} (E)$ 上的一个无界算子.
    它将整个空间分成无限个特征子空间的 Hilbert 直和,
    每个特征子空间都是有限维的, 且由光滑截面构成.
    特别地, 它的核是有限维的.
    通过特征子空间分解, 我们能定义算子
    \[ K_t = \upe^{-t D^2}. \]
    它是一个\term{迹类算子} (trace-class operator),
    也就是说, 它的迹是有限的. 特别地, 它是紧算子.
\end{fact}

定理的证明可参见 \cite[定理~III.3.8]{lawson-michelsohn}.
这里的 $K_t$ 与 \S\ref{sect-7} 中一样, 给出了热方程初值问题的解.

我们取 $\Gamma_{L^2} (E)$ 的标准正交基 $\{ e_i \}_{i=1}^\infty$, 使得
\[ D^2 e_i = \lambda_i \, e_i, \quad \text{从而} \quad 
    \upe^{-t D^2} \, e_i = \upe^{-t \lambda_i} \, e_i. \]
则每个 $e_i$ 都是光滑截面. 不难算出, 算子 $D^2$ 的热核是
\begin{equation} \label{eq-9-heat-ker}
    k_t (x, y) = \sum_i \upe^{-t \lambda_i} \, e_i (x) \otimes e_i^* (y),
\end{equation}
这里 $e_i^*$ 是 $E^\vee$ 的截面,
它等于 $e_i$ 的对偶基乘以 $\mathrm{vol}(M)^{-1}$,
从而它与 $e_i$ 的配对的积分等于 $1$.

\begin{theorem} [McKean--Singer] \label{thm-9-mckean-singer}
    设 $k_t$ 是算子 $D^2$ 的热核. 则对任意 $t > 0$, 都有
    \[ \ind (D) = \trs \upe^{-t D^2} = \int_M \trs k_t (x, x) \, dx. \]
\end{theorem}

\begin{proof}
    注意到 $D^2$ 是偶算子, 即它分别作用在两个空间 $V^{\pm} = \Gamma_{L^2} (E^{\pm})$ 上.
    
    记 $V_\lambda^{\pm} \subset V^{\pm}$ 是 $D^2$ 的 $\lambda$-特征子空间.
    则复合映射
    \[ V_\lambda^+ \overset{D^+}{\longrightarrow}
        V_\lambda^- \overset{D^-}{\longrightarrow} V_\lambda^+ \]
    是乘以 $\lambda$. 因此, 当 $\lambda \neq 0$ 时,
    $V_\lambda^+ \simeq V_\lambda^-$, 从而
    \[ \trs {} \bigl( \upe^{-t D^2} |_{V_\lambda} \bigr)
        = \upe^{-t \lambda^2} \dims V_\lambda = 0. \]
    这就说明
    \[ \trs \upe^{-t D^2} = \trs {} \bigl( \upe^{-t D^2} |_{\ker D} \bigr )
        = \dims \ker D = \ind (D), \]
    这就证明了第一个等号.
    
    第二个等号是因为, 由热核的表达式 (\ref{eq-9-heat-ker}), 有
    \begin{align*}
        \trs \upe^{-t D^2}
        &= \sum_i \pm \upe^{-t \lambda_i} \\
        &= \sum_i \int_M \upe^{-t \lambda_i} 
            \trs \bigl( e_i (x) \otimes e_i^\vee (x) \bigr) \, dx \\
        &= \int_M \trs k_t (x, x) \, dx. \qedhere
    \end{align*}
\end{proof}

也就是说, Dirac 算子的指标等于热核的迹.
我们将在表达式中令 $t \to 0$, 这样, \S\ref{sect-7} 中所做的渐进展开就能派上用场.


\subsection{Atiyah--Singer 指标定理}

我们回到上一节的设定.
设 $M$ 是偶数维紧旋量流形, $E \to M$ 是复 Clifford 模.
定义 $W = \Hom_{\Cl (M)} (S, E)$, 从而
\[ E \simeq S \otimes W. \]
设 $D$ 是 $E$ 上的 Dirac 算子, 
它对应的 Clifford 超联络 (\ref{thm-8-dirac-cliff-conn}) 记为 $\nabla^E$.
由 (\ref{thm-8-cliff-conn}), 可以将 $\nabla^E$ 写成
\[ \nabla^E = \nabla^S + \nabla^W \]
的形式, 其中 $\nabla^S, \nabla^W$ 分别是 $S$ 和 $W$ 上的联络. 则 $\nabla^E$ 的曲率是
\begin{equation} \label{eq-9-curv-sum}
    \Omega^E = (\nabla^S + \nabla^W)^2 = \Omega^S + \Omega^W,
\end{equation}
这是因为在 $S \otimes W$ 上,
$\nabla^S \nabla^W + \nabla^W \nabla^S = [\nabla^S, \nabla^W] = 0$.

我们考虑算子 $D^2$ 的热核 $k_t$ 在对角线上的幂级数展开
\[ k_t (x, x) \sim \frac {1} {(4 \uppi t)^{n/2}} \sum_{i=0}^\infty t^i k_i (x), \]
其中 $k_i \in \Gamma (\End (E))$. 我们回忆上一节定义的同构
\begin{multline*}
    \Gamma (\End (E))
    \simeq \Gamma \bigl( \End (S) \otimes \End (W) \bigr)
    \simeq \Gamma \bigl( \Cl (M) \otimes \End_{\Cl (M)} (E) \bigr) \\
    \overset{\sigma}{ \underset{\simeq}{\longrightarrow} }
        \Gamma \bigl( \wedge^\bullet \, T^* M \otimes \End_{\Cl (M)} (E) \bigr)
    \simeq \Omega^\bullet \bigl( M, \End_{\Cl (M)} (E) \bigr),
\end{multline*}
其中 $\sigma$ 是 (\ref{eg-8-symbol-map}) 中的映射.
因此, 我们得到相应的微分形式
\[ \sigma (k_i) \in \Omega^\bullet \bigl( M, \End_{\Cl (M)} (E) \bigr). \]

\begin{theorem} \label{thm-9-ahat-ch}
    沿用上面的记号, 假设 $\nabla^W$ 是普通的联络.
    则微分形式 $\sigma (k_i)$ 的最高次数是 $2i$,
    且其 $2i$ 次的部分与 
    \[ \Ahat (M) \, \upe^{-\Omega^W} \]
    的 $2i$ 次部分相同, 其中 $\Ahat (M)$ 称为 \term{$\Ahat$ 类} ($\Ahat$-genus),
    定义为
    \[ \Ahat (M) = \det^{1/2} \frac {\Omega/2} {\sinh \Omega/2}, \]
    这里 $\det^{1/2} (-)$ 的意义是形式幂级数
    $\exp \tr \bigl( 2^{-1} \log (-) \bigr)$.
\end{theorem}

这个定理将在下一节证明, 其证明方法就是利用 \S\ref{sect-7} 中的 $\Phi_i$ 进行计算.

利用这个公式计算 $\ind (D)$, 就能得到指标定理.
事实上, 由 (\ref{thm-9-mckean-singer}), 我们有
\[ \ind (D) = \int_M \trs k_t (x, x) \, dx
    \sim \frac {1} {(4 \uppi t)^{n/2}} \sum_{i \geq n/2} t^i \int_M \trs k_i (x) \, dx, \]
这里, 注意 $i < n/2$ 时 $k_i (x)$ 不含 $n$-形式, 故积分为 $0$.
在式中令 $t \to 0$, 得到
\begin{equation} \label{eq-9-trs-kn2}
    \ind (D) = \frac {1} {(4 \uppi)^{n/2}} \int_M \trs k_{n/2} (x) \, dx.
\end{equation}

定理中的 $\upe^{-\Omega^W}$ 一项, 我们将它的迹定义为\term{陈特征}
\begin{equation} \label{eq-9-ch-redef}
    \ch (W) = \trs \upe^{-\Omega^W}.
\end{equation}
这一定义与 \S\ref{sect-6} 中不同,
我们当时的定义是 $\trs \upe^{\Omega^W / 2 \uppi \upi}$.
从现在开始, 我们采用新的定义.
这一修改不会影响陈特征的加性和乘性

将 (\ref{thm-9-ahat-ch}) 与 (\ref{eq-9-trs-kn2}) 结合起来,
我们就证明了指标定理.

\begin{theorem} [Atiyah--Singer 指标定理] \label{thm-9-atiyah-singer}
    设 $M$ 是偶数维紧旋量流形, $E \simeq S \otimes W$ 是 $M$ 上的复 Clifford 模,
    $D$ 是 $E$ 上的 Dirac 算子. 则
    \[ \ind (D) = \frac {1} {(2 \uppi \upi)^{n/2}}
        \int_M \Ahat (M) \ch (W). \]
\end{theorem}

\begin{proof}
    我们只需解释系数的变化. 这是因为, $k_{n/2}$ 的迹是在 $E$ 上取的,
    但 $\upe^{-\Omega^W}$ 的迹, 即陈特征, 却是在 $W$ 上取的.
    这两种迹满足如下关系: 若 $V$ 是偶数维欧氏空间,
    $a \in \Cl (V) \otimes \bbC \simeq \End (S)$, 那么
    \[ \trs^S (a) = (-2 \upi)^{n/2} \, T( \sigma(a) ), \]
    这里 $T$ 是 Березин 积分, $n = \dim V$.
    这一公式不难验证 \cite[命题~3.21]{bgv}.
    因此, 若 $E$ 是 $\Cl (V)$-模, $k \in \End (E)$, 那么
    \begin{equation} \label{eq-9-tre-trw}
        \trs^E (k) = (-2 \upi)^{n/2} \trs^W ( T \sigma (k) ). \qedhere
    \end{equation}
\end{proof}

\begin{remark}
    定理的条件可以放宽: 事实上, $M$ 不必具有旋量结构, 只需可定向即可.
    这是因为, $M$ 局部上总是具有旋量结构, 而热核的迹的积分可以在局部上计算.
    此时, $W$ 也只能局部地定义, 但 $\ch (W)$ 在局部上定义好之后,
    就自动变成了整体的微分形式. \varqed
\end{remark}

下面, 我们介绍指标定理的几个有名的应用.


\subsection{de Rham 算子}

在这一小节中, 我们对 de~Rham 算子 $D = d + d^*$ 应用指标定理,
来重新证明陈--Gauß--Bonnet 公式.

\begin{notation} \label{def-9-b-and-c}
    设 $V$ 是 $n$ 维欧氏空间, $v \in V$, 
    $\alpha \in \wedge^\bullet \, V$. 记
    \[ b (v) \, \alpha = v \wedge \alpha + \iota_v \alpha, \quad
        c (v) \, \alpha = v \wedge \alpha - \iota_v \alpha, \]
    其中 $\iota_v = Q(v, -) \contr {}$ 是缩并.
    将它们延拓到整个 Clifford 代数上,
    我们得到了 $\wedge^\bullet \, V$ 的两个 Clifford 模结构
    \[ b, c \: \Cl (V) \to \End (\wedge^\bullet \, V). \]
    我们将 $c$ 给出的模结构视为默认的. 这样, 如果记
    如果 $\wedge^\bullet \, V \otimes \bbC = E \simeq S \otimes W$,
    那么对任意 $a \in \Cl (V)$, 都有
    \[ b(a) \in \End_{\Cl (V)} (E) \simeq \End (W), \quad
        c(a) \in \End (S) \subset \End (E). \varqedhere \]
\end{notation}

\begin{lemma} \label{thm-9-tr-exp-b}
    设 $V$ 是 $n$ 维欧氏空间, $n$ 为偶数.
    则如果 $a \in \Cl (V)_2$ 为 $2$ 阶元素, 那么
    \begin{align*}
        \trs^W \bigl( \exp b(a) \bigr)
        &= (-2 \upi)^{n/2} \det^{1/2} \biggl( \frac{\sinh a/2}{a/2} \biggr)
            \operatorname{pf} \Bigl( \frac{a}{2} \Bigr), \\
        \tr^W \bigl( \exp b (a) \bigr)
        &= 2^{n/2} \det^{1/2} \Bigl( \cosh \frac{a}{2} \Bigr).
    \end{align*}
\end{lemma}

\begin{proof}
    引理的证明是纯粹的计算 \cite[引理~4.4]{bgv}.
    我们忽略计算过程.
\end{proof}

\begin{lemma} \label{thm-9-omega-s}
    在 (\ref{eq-9-curv-sum}) 中, 在局部坐标下, 我们有
    \[ \Omega^S = -\frac14 \Omega_{ij} \, c^i c^j, \]
    其中 $\Omega_{ij} = \Omega_i^k g_{jk}$ 是 $M$ 的曲率形式,
    $c^i$ 是 $\Cl (M) \otimes \bbC \simeq \End (S)$ 的截面.
\end{lemma}

\begin{proof}
    我们回忆 $\nabla^S$ 的定义 (\ref{thm-8-nabla-s}).
    标架丛 $\SO (T^* M)$ 的联络形式是
    \[ \omega^{\SO (T^* M)} = - \frac12 \omega_{ij} \, dx^i \wedge dx^j
        \quad \in \Omega^1 (M, \so (T^* M)). \]
    我们将 $\so (T^* M)$ 与 $\spin (T^* M) \subset \Cl (T^* M)$
    等同起来 (\ref{thm-8-spin-alg-action}), 这个对应关系是
    \[ \so (T^* M) \ni \quad dx^i \wedge dx^j \mapsto
        \frac12 c^i c^j \quad \in \Cl (T^* M). \]
    由 $\nabla^S$ 的定义, 其联络形式就是
    $\omega^{\SO (T^* M)}$ 在这个对应下的像
    \[ \omega^S = - \frac14 \omega_{ij} \, c^i c^j
        \quad \in \Omega^1 (M, \End (S)). \]
    如果我们取法坐标, 那么在原点处就有
    $d \omega_{ij} = g_{jk} \, d \omega_i^k$, 从而
    \[ \Omega^S = - \frac14 \Omega_{ij} \, c^i c^j
        \quad \in \Omega^2 (M, \End (S)). \qedhere \]
\end{proof}

下面, 设 $M$ 是 $n$ 维紧可定向流形, $n$ 为偶数.
设 $E = \wedge^{\bullet} \, T^* M \otimes \bbC$, 它具有自然的超向量丛结构.
设 $D = d + d^*$ 是 $E$ 上的 Dirac 算子, 参见 (\ref{eg-8-dirac-d-dstar}).

我们回忆 Hodge 分解
\[ \Omega^\bullet (M) = \im d \oplus \im d^* \oplus \scrH^\bullet, \]
其中 $\scrH^\bullet = \ker D^2$ 是所有调和微分形式的空间,
并且有 $\scrH^\bullet \simeq H^\bullet (M; \bbC)$.

\begin{proposition}
    $D$ 的指标等于 $M$ 的 Euler 数:
    \[ \ind (D) = \chi (M). \]
\end{proposition}

\begin{proof}
    通过 (\ref{thm-9-mckean-singer}) 证明中的特征空间分解, 不难看出
    \[ \ker D = \ker D^2 = \scrH^\bullet. \]
    从而 $\ind (D) = \dims \ker D = \dims \scrH^\bullet = \chi (M)$.
\end{proof}

\begin{theorem} [陈--Gauß--Bonnet 公式] \label{thm-9-chern-gauss-bonnet}
    我们有
    \[ \chi (M) = \Bigl( \frac{-1}{2\uppi} \Bigr)^{n/2} 
        \int_M \operatorname{pf} (\Omega). \]
\end{theorem}

\begin{proof}
    采用之前的记号, 并记 $b^i = b(dx^i)$ 等等,
    则 $E$ 的曲率形式是
    \[ \Omega^E = \Omega_i^j \, (e^i \wedge) \circ \iota_j, \]
    其中 $e^i$ 是 $dx^i$ 所对应的 $E$ 的截面,
    映射 $(e^i \wedge) \circ \iota_j$ 的作用就是把 $E$ 的截面的 $dx^j$ 项取出,
    并换成 $dx^i$. 继续计算下去, 我们有
    \begin{align*}
        \Omega^E &= \frac14 \Omega_{ij} \, (b^i + c^i) \, (b^j - c^j) \\
        &= \frac14 \Omega_{ij} \, (b^i b^j - c^i c^j),
    \end{align*}
    其中交叉项消失是由于 $\Omega_{ij}$ 关于 $i, j$ 的反对称性.
    由 (\ref{thm-9-omega-s}), 我们得到
    \[ \Omega^W = \frac14 \Omega_{ij} \, b^i b^j. \]
    
    因此,
    \begin{align*}
        \Ahat (M) \ch (W)
        &= \Ahat (M) \trs^W \bigl( \exp (- \Omega^W) \bigr) \\
        &= \Ahat (M) \trs^W \bigl( \exp b (\Omega) \bigr) \\
        &= (-2 \upi)^{n/2} \operatorname{pf} \Bigl( \frac{\Omega}{2} \Bigr)
            \qquad ( \text{由 \ref{thm-9-tr-exp-b}} ) \\
        &= (-\upi)^{n/2} \operatorname{pf} (\Omega).
    \end{align*}
    从而由指标定理,
    \[ \ind (D) = \chi (M)
        = \frac{1}{(2 \uppi \upi)^{n/2}} \int_M (-\upi)^{n/2} \operatorname{pf} (\Omega)
        = \Bigl( \frac{-1}{2\uppi} \Bigr)^{n/2} \int_M \operatorname{pf} (\Omega).
        \qedhere \]
\end{proof}

当然, 这一证明涉及的计算比以前的证明要复杂得多.


\subsection{符号差算子}

下面, 我们通过指标定理来证明 Hirzebruch 符号差定理.

我们仍令 $E = \wedge^\bullet \, T^* M$, $D = d + d^*$,
但我们给 $E$ 一个不同的 $\bbZ_2$-分次.
为了定义这个分次, 我们需要做一些准备.

\begin{definition}
    设 $V$ 是已定向的 $n$ 维欧氏空间, $e_1, \dotsc, e_n$ 是一组符合定向的标准正交基.
    我们定义\term{手性算子} (chirality operator)
    \[ \Gamma = \upi^{\lceil n/2 \rceil} \, e_1 \cdots e_n 
        \quad \in \Cl (V) \otimes \bbC. \]
    其定义不依赖于基的选取. 注意到 $\Gamma^2 = 1$, 并且,
    对任何 $v \in V$, 有反交换关系 $v \Gamma + (-1)^n \, \Gamma v = 0$.
\end{definition}

这个算子的记号来源于物理学家的 $\gamma^5$ 矩阵,
但物理学家考虑的 $V$ 是带有 $(1, 3)$ 符号的内积的空间.

对我们而言, 这个算子可以用来描述 Hodge 对偶.

\begin{definition}
    沿用上面的记号, 定义 \term{Hodge $*$-算子}
    \[ {*} = \Gamma \cdot {} \: {\wedge^\bullet \, V} \to \wedge^\bullet \, V, \]
    其中 $\cdot$ 表示 Clifford 作用. 我们可以写下它在基上的作用:
    \[ {*} \, \bigl( e_{i_1} \wedge \cdots \wedge e_{i_k} \bigr)
        = (-1)^\sigma \, \upi^{\lceil n/2 \rceil} \,
        e_{j_1} \wedge \cdots \wedge e_{j_{n-k}}, \]
    其中 $j_1, \dotsc, j_{n-k}$ 表示 $1, \dotsc, n$ 中除了 $i_1, \dotsc, i_k$ 外的指标,
    $\sigma$ 是 $1, \dotsc, n$ 的排列 $i_1, \dotsc, i_k, j_1, \dotsc, j_{n-k}$ 的奇偶性.
    
    这个算子能作用在流形 $M$ 的微分形式上, 给出一个同构
    \[ {*} \: \Omega^k (M, \bbC) \to \Omega^{n-k} (M, \bbC). \]
    它是同构的原因是 ${*}^2 = 1$.
\end{definition}

注意, 我们的 $*$-算子和通常的定义相差 $\upi$ 的一个幂.

通过计算, 容易证明 $*$-算子满足以下公式.

\begin{lemma}
    设 $M$ 是 $n$ 维已定向流形.
    \begin{itemize}
        \item 
            若 $\alpha, \beta \in \Omega^k (M, \bbC)$, 则
            \[ \int_M \alpha \wedge {*}\beta =
                (-1)^{k (k \pm 1) / 2} \, \upi^{\lceil n/2 \rceil}
                \int_M \langle \alpha, \beta \rangle \cdot \mathrm{vol}, \]
            其中 $\pm$ 号与 $n$ 的奇偶性相反.
        \item
            我们有 
            \[ d^* = (-1)^{n+1} \, {*} d {*}. \thmqedhere \]
    \end{itemize}
\end{lemma}

\begin{definition}
    我们定义 $E = \wedge^\bullet \, T^* M$ 上的 $\bbZ_2$-分次如下:
    \[ \biggl \{ \begin{array}{l}
        E^+ = \{ \alpha \mid {*} \alpha = \alpha \}, \\
        E^- = \{ \alpha \mid {*} \alpha = -\alpha \},
    \end{array} \]
    这里 $\alpha$ 表示 $E$ 的纤维中的元素, 不一定是齐次的.
\end{definition}

在这个分次下, 对算子 $D = d + d^*$ 应用指标定理,
我们将得到 Hirzebruch 符号差定理.

我们回忆, 若 $M$ 是 $n$ 维流形, $4 \mid n$,
则杯积给出了 $H^{n/2} (M; \bbR)$ 上的双线性型,
其符号差 (正特征值的个数减去负特征值的个数) 定义为 $M$ 的符号差 $\sigma (M)$.

\begin{proposition}
    设 $M$ 是 $n$ 维已定向流形, $4 \mid n$. 在上面的记号下,
    \[ \ind (D) = \sigma (M). \]
\end{proposition}

\begin{proof}
    与之前一样, $\ker D = \scrH^\bullet$ 是调和形式的空间.
    对 $k = 0, 1, \dotsc, n/2 - 1$, 如果 $\{ \alpha_i \}$ 是 $\scrH^k$ 的一组基,
    那么 $\{ \alpha_i \pm {*} \alpha_i \}$ 就构成 $\scrH^k \oplus \scrH^{n-k}$
    的一组基. 这说明 $\dims (\scrH^k \oplus \scrH^{n-k}) = 0$, 从而
    \[ \dims \ker D = \dims \scrH^{n/2}. \]
    而如果 $\alpha \in \scrH^{n/2}$ 满足 ${*} \alpha = \pm \alpha$,
    那么上同调类 $[\alpha] \smallcup [\alpha]$ 与 $[M]$ 的配对是
    \[ \int_M \alpha \wedge \alpha
        = \int_M \langle \alpha, {*} \alpha \rangle \cdot \mathrm{vol}
        = \pm \int_M |\alpha|^2 \cdot \mathrm{vol}, \]
    其正负号由 $\alpha$ 的分次决定, 从而 $\dims \scrH^{n/2} = \sigma (M)$.
\end{proof}

\begin{theorem} [Hirzebruch 符号差定理]
    设 $M$ 是 $n$ 维已定向流形, $4 \mid n$. 则
    \[ \sigma (M) = \frac{1}{(\uppi \upi)^{n/2}}
        \int_M \det^{1/2} \frac{\Omega/2}{\tanh \Omega/2}. \]
    这里, 积分号里的项也被称为 \term{$L$ 类} ($L$-genus), 记作 $L(M)$.
\end{theorem}

\begin{proof}
    注意到, $E$ 的分次实际上是由 $S$ 的分次诱导的,
    因为 $S^{\pm}$ 实际上就是 $\Gamma$ 的作用的 $\pm1$-特征子空间.
    因此, $W$ 只有偶数次部分. 由 (\ref{thm-9-tr-exp-b}), 我们有
    \begin{align*}
        \Ahat (M) \ch (W)
        &= \Ahat (M) \tr^W \bigl( \exp (-\Omega^W) \bigr) \\
        &= \det^{1/2} \biggl( \frac{\Omega/2}{\sinh \Omega/2} \biggr) \ 
            2^{n/2} \det^{1/2} \biggl( \cosh \frac{\Omega}{2} \biggr) \\
        &= 2^{n/2} \, L(M). \qedhere
    \end{align*}
\end{proof}


\subsection{Dolbeault 算子}

下面, 我们通过指标定理, 来计算全纯向量丛的 Euler 示性数.

在开始之前, 我们回忆复几何中的一些构造.
设 $M$ 是 Kähler 流形, 也就是一个复流形, 其切丛带有一个相容的 Hermite 度量. 我们有
\[ TM \underset{\bbR}{\otimes} \bbC \simeq T^{1,0} M \oplus T^{0,1} M, \]
这两个子丛是 $TM$ 上 ``乘以 $\upi$'' 的操作的 $(\pm \upi)$-特征子空间.
我们有 $M$ 上 $(p, q)$-微分形式的空间
\[ \Omega^{p,q} (M) = \Gamma \Bigl( M, \ 
    \bigl( \wedge^p (T^{1,0} M)^\vee \bigr) \wedge 
    \bigl( \wedge^q (T^{0,1} M)^\vee \bigr) \Bigr)
    \subset \Omega^{p+q} (M_{\bbR}, \bbC), \]
其中 $M_{\bbR}$ 是 $M$ 对应的实流形. 在局部坐标下, 我们将一个 $(p, q)$-形式写成
\[ \alpha = \alpha_{i_1 \cdots i_p \oline1j {}_1 \cdots \oline1j {}_q} \,
    d z^{i_1} \wedge \cdots \wedge d z^{i_p} \wedge 
    d \oline1z{}^{j_1} \wedge \cdots \wedge d \oline1z{}^{j_q}\]
的形式. 此时, de Rham 算子 $d$ 分成次数 $(1, 0)$ 和 $(0, 1)$ 部分, 记为
\[ \begin{array}{l}
    \partial \: \Omega^{p,q} (M) \to \Omega^{p+1,q} (M), \\
    \dbar \: \Omega^{p,q} (M) \to \Omega^{p,q+1} (M), \\ 
    d = \partial + \dbar.
\end{array} \]

下面, 设 $E \to M$ 是一个\term{全纯向量丛},
即转移映射是全纯映射的复向量丛. 和上面同理, 我们定义
\[ \Omega^{p,q} (M, E) = \Gamma \Bigl( M, \ 
    \bigl( \wedge^p (T^{1,0} M)^\vee \bigr) \wedge 
    \bigl( \wedge^q (T^{0,1} M)^\vee \bigr) \otimes E \Bigr) \]
为 $E$-取值的 $(p, q)$-形式的空间.
此时, 只有 $\dbar$ 算子能够良好地定义. 链复形
\[ \bigl( \Omega^{0, \bullet} (M, E), \ \dbar \bigr) \]
称为 $E$ 的 \term{Dolbeault 链复形}, 其上同调
\[ H_{\dbar}^\bullet (M, E) =
    H^\bullet \bigl( \Omega^{0, \bullet} (M, E), \ \dbar \bigr) \]
称为 $E$ 的 \term{Dolbeault 上同调}. 事实上,
它与层上同调 $H^\bullet (M, \scrO (E))$ 同构, 这一结论称为 Dolbeault 定理.

和 de Rham 算子的情形类似, 我们定义
\[ \Delta_{\dbar} = (\dbar + \dbar^*)^2 = \dbar \dbar^* + \dbar^* \dbar. \]
则
\[ \ker (\dbar + \dbar^*) = \ker \Delta_{\dbar} \simeq H_{\dbar}^\bullet (M, E). \]
另外, 我们还有
\[ \Delta_{\dbar} = \frac12 \Delta, \]
其中 $\Delta$ 是 de Rham 算子定义的 Laplace 算子. 我们定义
\[ \textstyle D = \smallsqrt2 \ (\dbar + \dbar^*), \]
则 $D^2 = \Delta$, 从而 $D$ 是向量丛 $\wedge^\bullet (T^{0,1} M)^\vee \otimes E$
上的 Dirac 算子. 相应地, 我们得到 Clifford 作用的坐标表达式
\[ c(dz^i) = - \smallsqrt2 \ \iota^i, \quad 
    c(d \oline1z{}^i) = \smallsqrt2 \ (\oline2e {}^i \wedge {}). \]

和 de Rham 算子的情况一样,
$D$ 的指标就是 Euler 数 $\chi (M, E)$.

\begin{proposition}
    沿用上面的记号, 我们有
    \[ \ind (D) = \chi (M, E) = \sum_{i=0}^n (-1)^i \dim H_{\dbar}^i (M, E). \thmqedhere \]
\end{proposition}

\begin{theorem} [Hirzebruch--Riemann--Roch]
    设 $M$ 是 $n$ 维 Kähler 流形, $E \to M$ 是全纯向量丛. 则
    \[ \chi (M, E) = \frac{1}{(-2 \uppi \upi)^n}
        \int_M \operatorname{td} (M) \ch (E), \]
    其中 $\operatorname{td} (M)$ 是 \term{Todd 类}, 定义为
    \[ \operatorname{td} (M) = \det \frac{\Omega}{\upe^{\Omega} - 1}. \]
\end{theorem}

\begin{proof}
    记 $\Lambda = \wedge^\bullet (T^{0,1} M)^\vee$. 
    与 (\ref{thm-9-chern-gauss-bonnet}) 的证明同理, 我们有
    \[ \Omega^{\Lambda}
        = \Omega {}_{\oline1i}^{j} \, (\oline2e {}^i \wedge) \circ \iota_j \\
        = -\frac12 \Omega_{\oline1i j} \, c^{\oline1i} c^j. \]
    由 (\ref{thm-9-omega-s}), 知
    \begin{align*}
        \Omega^{\Lambda} - \Omega^S
        &= -\frac14 \sum_i \Omega_{\, \oline1i i} \, 
            (c^{\oline1i} c^i + c^i c^{\oline1i}) \\
        &= \frac12 \sum_i \Omega_{\, \oline1i i} = \frac12 \tr \Omega,
    \end{align*}
    其中最后一步来自于 (\ref{thm-5-c1}) 的证明. 从而
    \[ \Omega^W = \Omega^{\Lambda \otimes E} - \Omega^S
        = \Omega^E + \frac12 \tr \Omega. \]
    由于 $TM \otimes_{\bbR} \bbC \simeq TM \oplus \oline1{TM}$, 我们有
    \begin{align*}
        \Ahat (M_{\bbR}) &= (-1)^* \, \Ahat (TM)^2 \\
        &= (-1)^* \det \frac{\Omega / 2}{\sinh \Omega / 2} \\
        &= (-1)^* \operatorname{td} (M) \det (\upe^{\Omega/2}),
    \end{align*}
    其中 $(-1)^*$ 的意思是将 $2k$-形式的部分乘以 $(-1)^{nk}$. 从而
    \[ \Ahat (M) \trs \upe^{-\Omega^W}
        = (-1)^* \operatorname{td} (M) \trs \upe^{-\Omega^E}. \qedhere \]
\end{proof}

\begin{remark}
    这一公式的更常见的形式, 也是我们之前叙述的版本, 是
    \[ \chi (M, E) = \int_M \operatorname{Td} (M) \operatorname{Ch} (E), \]
    其中 $\operatorname{Td}$ 和 $\operatorname{Ch}$
    是我们修改定义 (\ref{eq-9-ch-redef}) 之前的示性类, 也就是说,
    把 $\operatorname{td}$ 和 $\operatorname{ch}$ 的定义中的
    $\Omega$ 都换成 $-\Omega / 2 \uppi \upi$ 所得到的示性类.
    由于只有 $2n$-形式能被积分, 所以积分的系数相差了 $(-2\uppi\upi)^n$. \varqed
\end{remark}

