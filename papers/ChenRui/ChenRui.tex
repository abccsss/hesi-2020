% !TEX program = xelatex
% !BIB program = biber

\documentclass[twoside]{article}

\usepackage{geometry}
\geometry{
    paperwidth = 155mm,
    paperheight = 235mm,
    outer = 20mm,
    inner = 20mm,
    top = 25mm,
    bottom = 20mm
}

% fonts & unicode
\usepackage[PunctStyle=kaiming]{xeCJK}
\usepackage{amsmath}
\usepackage{unicode-math}

\setCJKmainfont{fzssk.ttf}[
    Path = ../../fonts/,
    BoldFont = fzhtk.ttf,
    ItalicFont = fzktk.ttf
]

\setCJKsansfont{SourceHanSansCN-Normal.otf}[
    Path = ../../fonts/,
    BoldFont = SourceHanSansCN-Bold.otf,
    Scale = .97
]

\setCJKmonofont{SourceHanSansCN-Normal.otf}[
    Path = ../../fonts/,
    BoldFont = SourceHanSansCN-Bold.otf,
    Scale = .9
]

\newCJKfontfamily{\KaiTi}{fzktk.ttf}[
    Path = ../../fonts/,
    BoldFont = fzhtk.ttf,
    ItalicFont = fzssk.ttf
]

\setmainfont{STIX2Text}[
    Path = ../../fonts/,
    Extension = .otf,
    UprightFont = *-Regular,
    BoldFont = *-Bold,
    ItalicFont = *-Italic,
    BoldItalicFont = *-BoldItalic
]

\setsansfont{Lato}[
    Path = ../../fonts/,
    Scale = MatchUppercase,
    Extension = .ttf,
    UprightFont = *-Regular,
    BoldFont = *-Bold,
    ItalicFont = *-Italic,
    BoldItalicFont = *-BoldItalic
]

\setmonofont{FiraMono}[
    Path = ../../fonts/,
    Scale = .9,
    Extension = .otf,
    UprightFont = *-Regular,
    BoldFont = *-Bold
]

\setmathfont{STIX2Math.otf}[
    Path = ../../fonts/,
    BoldFont = STIX2Math-Bold.otf
]

\setmathfont{latinmodern-math.otf}[
    Path = ../../fonts/,
    range = {frak, bffrak}
]

\setmathfont{latinmodern-math.otf}[
    Path = ../../fonts/,
    range = {frak -> bffrak, bffrak},
    version = bold
]

\setmathfont{LatoMath.otf}[
    Path = ../../fonts/,
    Scale = .95,
    BoldFont = LatoMath.otf,
    version = sf
]

\setmathfont{LatoMath.otf}[
    Path = ../../fonts/,
    Scale = .95,
    BoldFont = LatoMath.otf,
    range = {bb, sfup -> up, sfit -> it, bfsfup -> bfup, bfsfit -> bfit}
]


\Umathcode`/  =  "0 "0 "2215    % / -> U+2215 division slash

% patch 'text math' math alphabets in bold math
\setmathfontface\mathrm{STIX2Text-Bold.otf}[
    Path = ../../fonts/,
    version = bold
]

\setmathfontface\mathit{STIX2Text-BoldItalic.otf}[
    Path = ../../fonts/,
    version = bold
]

\setmathfontface\mathbf{STIX2Text-Bold.otf}[
    Path = ../../fonts/,
    version = bold
]

\setmathfontface\mathtt{FiraMono-Bold.otf}[
    Path = ../../fonts/,
    Scale = .9,
    version = bold
]

\setmathfontface\mathrm{Lato-Regular.ttf}[
    Path = ../../fonts/,
    Scale = MatchUppercase,
    version = sf
]

\setmathfontface\mathit{Lato-Italic.ttf}[
    Path = ../../fonts/,
    Scale = MatchUppercase,
    version = sf
]

\setmathfontface\mathbf{Lato-Bold.ttf}[
    Path = ../../fonts/,
    Scale = MatchUppercase,
    version = sf
]

\setmathfontface\mathtt{FiraMono-Regular.otf}[
    Path = ../../fonts/,
    Scale = .9,
    version = sf
]


% bold math in bold text https://tex.stackexchange.com/q/41379
\makeatletter
    \g@addto@macro\bfseries{\boldmath} 
\makeatother

% title & abstract
\def\title#1\author#2{%
    \headertitle{#1}
    \vspace*{0mm}
    \begin{center}
        {\sf\LARGE#1\par}
        \vspace{10mm}
        {\large#2}
    \end{center}
    \vspace{10mm}
}
\def\headertitle#1{
    \def\theheadertitle{#1}
}

\renewcommand{\abstractname}{ABSTRACT}
\makeatletter
    \let\endabstract@orig\endabstract
    \def\endabstract{\endabstract@orig\vspace{5mm}}
\makeatother

% section titles
\usepackage{titlesec}
\titleformat*{\section}{\Large\sffamily\mathversion{sf}}
\titleformat*{\subsection}{\large\sffamily\mathversion{sf}}

% headers and footers
\usepackage{fancyhdr}
\fancyhf{}
\fancyhead[CE]{\sf\mathversion{sf}\theheadertitle}
\fancyhead[CO]{\sf\mathversion{sf}\nouppercase{\leftmark}}
\fancyhead[LE,RO]{\textbf{\textsf{\thepage}}}
\headsep=8mm
\headheight=6mm

\AtBeginDocument{
    \pagestyle{fancy}\thispagestyle{empty}
}

% spacing
\AtBeginDocument{
    \hfuzz=2pt
    \emergencystretch 2em
    \setlength{\belowdisplayshortskip}{\belowdisplayskip}
}

% environments
\usepackage{amsthm}
\newtheorem{theorem}{Theorem}[section]
\newtheorem{lemma}[theorem]{Lemma}
\newtheorem{corollary}[theorem]{Corollary}
\newtheorem{proposition}[theorem]{Proposition}

\theoremstyle{definition}
\newtheorem{definition}[theorem]{Definition}
\newtheorem{example}[theorem]{Example}
\newtheorem{remark}[theorem]{Remark}
\theoremstyle{plain}

\def\qedsymbol{$◻$}
\def\thmqedhere{\pushQED{\qed}\qedhere\popQED}

\numberwithin{equation}{theorem}

% renew theorem: https://tex.stackexchange.com/q/103013/
\makeatletter
\def\renewtheorem#1{%
    \expandafter\let\csname#1\endcsname\relax
    \expandafter\let\csname c@#1\endcsname\relax
    \gdef\renewtheorem@envname{#1}
    \renewtheorem@secpar
}
\def\renewtheorem@secpar{\@ifnextchar[{\renewtheorem@numberedlike}{\renewtheorem@nonumberedlike}}
\def\renewtheorem@numberedlike[#1]#2{\newtheorem{\renewtheorem@envname}[#1]{#2}}
\def\renewtheorem@nonumberedlike#1{  
\def\renewtheorem@caption{#1}
\edef\renewtheorem@nowithin{\noexpand\newtheorem{\renewtheorem@envname}{\renewtheorem@caption}}
\renewtheorem@thirdpar
}
\def\renewtheorem@thirdpar{\@ifnextchar[{\renewtheorem@within}{\renewtheorem@nowithin}}
\def\renewtheorem@within[#1]{\renewtheorem@nowithin[#1]}
\makeatother

% ref & biblatex
\usepackage[colorlinks,allcolors=black,bookmarksnumbered,linktoc=all]{hyperref}

\def\thesection{\arabic{section}\texorpdfstring{}{.}} % pdf bookmark numbering
\setcounter{secnumdepth}{1} % suppress subsection numbering

\usepackage[style=alphabetic,sorting=anyvt,useprefix=true]{biblatex}
\usepackage{xpatch}
\renewcommand*{\bibfont}{\small}
\DeclareFieldFormat[article]{volume}{\mkbibbold{#1}}
\DeclareFieldFormat[book,inbook]{number}{\mkbibbold{#1}}
\DeclareFieldFormat[article]{number}{(#1)}
\DeclareFieldFormat*{year}{(#1)}
\DeclareFieldFormat{pages}{#1}
\renewbibmacro{in:}{}
\renewbibmacro*{volume+number+eid}{%
    \printfield{volume}%
    \setunit*{\addnbspace}% originally: \setunit*{\adddot}
    \printfield{number}%
    \setunit{\addcomma\space}%
    \printfield{eid}%
}
\xapptobibmacro{author/editor+others/translator+others}{%
    \setunit{\space}%
    \printfield{year}%
    \clearfield{year}%
}{}{}
\xapptobibmacro{author/translator+others}{%
    \setunit{\space}%
    \printfield{year}%
    \clearfield{year}%
}{}{}
    \renewbibmacro*{issue+date}{%
    \printfield{issue}%
    \newunit%
}
\AtBeginBibliography{
    \DeclareFieldFormat{labelalpha}{#1}
    \DeclareFieldFormat{extraalpha}{\mknumalph{#1}}
}
\AtEveryBibitem{
    \ifentrytype{online}{%
        \clearfield{year}%
    }{}
}

% tikz
\usepackage{tikz}
\usepackage{tikz-cd}
\tikzset{
    > = latex
}
\tikzcdset{
    arrow style = tikz,
    arrows = {
        /tikz/line width = .5pt
    },
    diagrams = {
        > = {Straight Barb[scale = 0.8]}
    },
    nodes = {
        inner xsep = 3pt,
        inner ysep = 3pt
    }
}

\newenvironment{itms}{\begin{itemize}\itemsep=0pt\parsep=0pt}{\end{itemize}}
\newenvironment{enum}{\begin{enumerate}[label=(\arabic*)]\itemsep=0pt\parsep=0pt}{\end{enumerate}}

\theoremstyle{definition}
\newtheorem{construction}[theorem]{Construction}
\newtheorem{notation}[theorem]{Notation}

\def\varqed{\nolinebreak\hfill$◃$}
\def\varqedhere{\eqno ◃}

% math commands
\renewcommand{\:}{\colon}
\renewcommand{\/}{{∕}}
\newcommand{\bfDelta}{{
  \mathchoice{
    \tikz{
      \draw[line width=.05em]
        (0,0)--(.6em,0)--(.37em,1.32ex)--(.23em,1.32ex)--cycle
        (.23em,1.32ex)--(.46em,0);
    }
  }{
    \tikz{
      \draw[line width=.05em]
        (0,0)--(.6em,0)--(.37em,1.32ex)--(.23em,1.32ex)--cycle
        (.23em,1.32ex)--(.46em,0);
    }
  }{
    \tikz[scale=.7]{
      \draw[line width=.035em]
        (0,0)--(.6em,0)--(.37em,1.32ex)--(.23em,1.32ex)--cycle
        (.23em,1.32ex)--(.46em,0);
    }
  }{
    \tikz[scale=.55]{
      \draw[line width=.025em]
        (0,0)--(.6em,0)--(.37em,1.32ex)--(.23em,1.32ex)--cycle
        (.23em,1.32ex)--(.46em,0);
    }
  }
}}
\newcommand{\bigmid}{\mathrel{\big|}}
\newcommand{\biggmid}{\mathrel{\bigg|}}
\newcommand{\cat}[1]{\ensuremath\textup{\textsf{#1}}} % using textsf to retain kerning & ligature
\newcommand{\comma}{,}
\newcommand{\Ho}{\operatorname{Ho}}
\newcommand{\Hom}{\operatorname{Hom}}
\newcommand{\KD}{\operatorname{\reflectbox{$\mathrm{DK}$}}}
\newcommand{\Ndg}{\mathfrak{N}_{\mathrm{dg}}}
\newcommand{\op}{^{\mathrm{op}}}
\newcommand{\sHom}{\mathscr{H}\mkern-3mu\mathit{om}}
\newcommand{\simto}{\mathrel{\rlap{\raisebox{.8ex}{$\mkern2mu\sim$}}{\to}}}
\newcommand{\square}{\mathbin{◻}}

% text commands
\newcommand{\term}[1]{\textbf{\textup{#1}}}


\addbibresource{ChenRui.bib}

\begin{document}

\title{Deformation Theory of Elliptic Curves}
\author{Chen Rui\footnote{陈睿,清华大学数学系数 61 班.}}

\begin{abstract}
In this article,
we will study the deformation theory of ordinary elliptic curves following \cite{Hida}.
In particular, we relate the deformation theory of ordinary elliptic curves to their Barsotti--Tate groups.
\end{abstract}

\section{Motivation}

As you may ask: why study ``deformation theory of elliptic curves''?

One important reason is that in the study of relatively representable moduli problem $\PP$
(that is a contravariant functor from the category of elliptic curves to sets),
the moduli scheme $\PP_{\,\mathbb{E}/W[[T]]}$ associated to the universal object
gives us the local information of a given elliptic curve
$E/\bar{\Fb}_{p}$ ($\mathbb{E}/W[[T]]$ is the universal deformation
of $E/\bar{\Fb}_{p}$ to complete local $W(\bar{\Fb}_{p})$-algebras),
which is the compete local ring of the ``geometric point'' of the moduli problem $\PP$.

\section{Drinfeld's lemma and Serre--Tate theorem}

In this section, following \cite{Katz} we present the proof of Serre--Tate theorem which is based on a lemma of Drinfeld.

\begin{definition}
    We call an abelian fppf sheaf $G$ a \term{$p$-divisible} sheaf if for any $x\in G(R)$,
    there exists an fppf extension $R_{1}$ of $R$ and a point $y\in G(R_{1})$ such that $x=py$.
    We call a $p$-divisible fppf sheaf $G_{/S}$ a $p$-divisible group 
    \textup{(\term{Barsotti--Tate Group})} if the following three conditions are satisfied:
    \begin{enumerate}
        \item $G=\liminv G[p^{n}]$ for  $G[p^{n}]=\Ker(p^{n}\colon G \to G)$.
        \item $G[p^{n}]$ is a finite flat group scheme over $B$ 
        with closed immersions $G[p^{n}] \hookrightarrow G[p^{m}]$ for $m\ge n$.
        \item The multiplication $[p^{m-n}]\colon G[p^{m}]\to G[p^{n}]$ is 
        an epimorphism in the category of finite flat group schemes.
    \end{enumerate}
\end{definition}

Assume $B$ is a local ring with residue field $k$ of characteristic $p$.
We denote $\Cl_{/B}$ the category of complete local $B$-algebras 
with residue field $k$.
And by $\Cofun(\Cl_{/B},\mathscr{A}b)$ the covariant functors from $\Cl_{/B}$ to the category of abelian groups.

\begin{definition}\label{def-csfg}
    A group functor $\hat{G}$ in $\Cofun (\Cl_{/B},\mathscr{A}b)$ is called a
    \term{connected smooth formal group} if it is a connected smooth formal scheme
    with multiplication $p$ inducing a finite flat morphism $\hat{G}\to \hat{G}$.
    
    In particular, $\hat{G}$ is an fppf $p$-divisible abelian sheaf over $B$.
\end{definition}

We now construct the connected-\'etale sequence for $p$-divisible groups.
First, let us recall the connected-\'etale sequence for a finite flat group scheme.

\begin{proposition}
    For a locally free group scheme $G_{/S}$ of finite rank 
    over $S=\Spec(A)$ with a henselian local ring $A$,
    we have a canonical exact sequence of locally free group schemes:
    \[ 0\to G^{\circ}\to G\to G^{\et} \to 0, \]
    where $G^{\circ}$ is connected and $G^{\et}$ is \'etale.
    We call this sequence the connected-\'etale sequence of $G$.
    If $G$ is a Barsotti--Tate group,
    then $G^{\circ},G^{\et}$ are also Barsotti--Tate groups.
\end{proposition}

\begin{definition} 
    A Barsotti--Tate group $G$ over a scheme $S$ is called ordinary 
    if there exists a faithfully-flat quasi-compact covering $S'\to S$
    such that over $S'$, $G_{S'}$ sits into the exact sequence:
    \[ 0\to \mu_{p^{\infty}/S'}^{s}\to G_{/S'} \to (\mathbb{Q}_{p}/\Zb_{p})^{r}_{/S'} \to 0. \]
    The integer $r$ is called the $p$-rank of $G$.
    We write the image of $\mu_{\smash{p^{\infty}/S'}}^{s}$ in $G$ as $G^{\mathrm{mult}}$ 
    which is the maximal multiplicative connected subgroup and denote $G^{\et}$ the maximal \'etale quotient.
    We get 
    \[ 0\to G^{\mathrm{mult}} \to G\to G^{\et} \to 0. \]
\end{definition}

We denote by $\operatorname{Lie}_{/B}$ the subcategory of 
connected smooth formal groups over $B$ in $\Cofun(\Cl_{/B},\mathscr{A}b)$.

Since $\hat{G}$ is smooth, we will have $\hat{G}=\Spf_{B}(B[[X_{1},\dotsc,X_{r}]])$ 
as formal schemes, where the identity element $0$ is given by $X_{i}\to 0$ for all $i$.

We now give the proof of the Serre--Tate theorem.

In the followings, $R$ will be a commutative ring, $p$ a prime nilpotent in $R$,
and $I$ an ideal nilpotent in $R$ say $I^{v+1}=0$, and we denote $R_{0}=R/I$.
Also we adapt the notations used before.

Let $G$ be a group functor over $S=\Spec(R)$. Let $A$ be an $R$-algebra.

We denote $G_{I}(A)=\Ker(G(A)\to G(A/IA))$, and $\hat{G}(A)=\Ker(G(A)\to G(A^{\mathrm{red}}))$.

\begin{lemma} \label{lem-a}
    If $G$ is a commutative formal smooth group
    (in the sense of Definition \ref{def-csfg}).
    Then the subgroup functor $G_{I}$ is killed by $N^{v}$.
\end{lemma}

\pf
    In terms of the coordinates $X_{1},\dotsc,X_{n}$ of $G$, we have 
    \[    ([N](X))_{i}=N\ X_{i} + \text{higher order terms} .  \]
    As a point of $G_{I}(A)$ has coordinates in $IA$, and $N$ kills $R$, we have 
    \[    [N](G_{I}) \subset G_{I^{2}},   \]
    and by induction, we get 
    \[    [N](G_{I^{a}}) \subset G_{I^{a+1}}  \]
    for integers $a\ge 1$. Since $G_{I^{v+1}}=0 $. We get $N^{v}[G_{I}]=0$.
\epf

\begin{lemma}[Drinfeld]
    Let $G$ and $H$ be two fppf abelian sheaves over $R$. Suppose that 
    \begin{enumerate}
        \item $G$ is $N$-divisible.
        \item $\hat{H}$ is locally representable by a formal Lie group.
        \item $H$ is formally smooth.
    \end{enumerate}
    Let $G_{0}$, $H_{0}$ be the restriction of $G$ and $H$ to $R_{0}$. Then
    \begin{enumerate}
        \item the group $\Hom_{R\mathhyphen\mathrm{gp}}(G,H)$ and
        $\Hom_{R_{0}\mathhyphen\mathrm{gp}}(G_{0},H_{0})$ have no $N$-torsion.
        \item the natural ``reduction mod $I$'' map 
        \[ \Hom(G,H)\to\Hom(G_{0},H_{0})\]
        is injective.
        \item for any homomorphism $f_{0}\colon G_{0} \to H_{0} $,
        there exists a unique homomorphism $N^{v}f \colon   G\to H$,
        which lifts $N^{v}f_{0}$.
        \item In order that a homomorphism $f_{0} \colon  G_{0} \to H_{0}$ 
        lifts to a (necessarily unique) homomorphism $f\colon G\to H$,
        it is necessary and sufficient that the homomorphism 
        $N^{v}f\colon  G\to H$ annihilate the subgroup $G[N^{v}]=\Ker([N^{v}] \colon G\to G )$ of $G$.
    \end{enumerate}
\end{lemma}

\pf
    (1)
    This follows from the $N$-divisibility of $G$:
    if $pf=0$, take $py=x$, then $f(x)=pf(y)=0$.

    (2)
    Notice that the kernel is $\Hom(G,H_{I})$,
    and since $N^{v}$ kills $\Hom(G,H_{I})$ by previous lemma 
    and $\Hom(G,H)$ is $N$-divisible,
    we have $\Hom(G,H_{I})=0$, that is the map is injective.

    (3)
    We will write down the canonical lifting explicitly.

    Once it exists, then it is unique by part (2).
    \[ \begin{tikzcd}
        G(A)\ar[r, "\text{``}N^{v}f\text{''}"] \ar[d, "\mod I"'] & H(A) \\ 
        G(A/IA) \ar[r, "f_{0}"]& H(A/IA) \ar[u, "N^{v} \times (\text{any lifting})"']
    \end{tikzcd} \]
    The final homomorphism is defined since we have $H$ formally smooth by assumption,
    then $H(R)\to H(R/IR)$ is surjective for any nilpotent ideal $I$, and it is well defined,
    since $N^{v}$ kills $H_{I}(R)=\Ker(H(R) \to H(R/IR) $.
    
    (4) 
    Note that if $f$ lifts $f_{0}$,
    then by the uniqueness of lifting,
    we must have $N^{v}f=\text{``}N^{v}f\text{''}$.
    Therefore we have $\text{``}N^{v}f\text{''}$ kills $G[N^{v}]$.
    Conversely, suppose $\text{``}N^{v}f\text{''}$ kills $G[N^{v}]$.
    Since $G$ is $N$-divisible, we will have the following exact sequence 
    \[ \begin{tikzcd}
        0\ar[r] &G[N^{v}] \ar[r]&G \ar[r, "N^{v}"]& G\ar[r] & 0
    \end{tikzcd} \]
    from which we can deduce $\text{``}N^{v}f\text{''}=N^{v}F$ for some homomorphism $F\colon G \to H$.

    Let $F_{0}$ be the reduction of $F$ mod $I$.
    Notice that $N^{v}F_{0}=N^{v}f_{0}$,
    and since $\Hom(G_{0},H_{0})$ has no $N$-torsion,
    we get $F_{0} =f_{0}$, that is $F$ lifts $f_{0}$.
\epf

We now turn to the special case $N=p^{n}$.
Let $\mathscr{A}$  be the category of abelian schemes over $R$,
and denote by $\operatorname{Def}(R,R_{0})$ the category of triples $(A_{0},G,i)$,
consisting of an abelian scheme $A_{0}$ over $R_{0}$,
a $p$-divisible group $G$ over $R$,
and an isomorphism $i$ of $p$-divisible groups over $R_{0}$:
\[    i\colon     G_{0} \cong A_{0}[p^{\infty}] .  \]

\begin{theorem}[Serre--Tate]
    Let $R$ be a ring in which a prime $p$ is nilpotent,
    $I \subset R$ a nilpotent ideal, and $R_{0}=R/I$.
    Then the functor
    \[ \begin{aligned}
        \mathscr{A} &\to \operatorname{Def}(R,R_{0}), \\
        A &\mapsto (A_{0},A[p^{\infty}],i)
    \end{aligned} \]
    is an equivalence of categories.
\end{theorem}

\pf
    First, we show it is fully faithful.
    Let $A,B$ be two abelian schemes over $R$.
    We suppose given a homomorphism 
    \[   f[p^{\infty}] \colon    A[p^{\infty}] \to B[p^{\infty}]    \]
    of $p$-divisible groups over $R$, and a homomorphism 
    \[   f_{0}\colon    A_{0} \to B_{0}    \]
    of abelian schemes over $R_{0}$,
    such that $f$ and $f_{0}$ are compatible with the natural isomorphism $i$.

    We must show there exists a unique homomorphism $f\colon  A\to B$, which induces $f[p^{\infty}]$ and $f_{0}$.

    Because both abelian schemes and $p$-divisible groups 
    satisfy the assumptions of our previous lemma,
    we can make use of its conclusions.
    The uniqueness of $f$ follows from the injectivity of 
    \[    \Hom(A,B) \to \Hom(A_{0},B_{0}).   \]
    For the existence, let $\text{``}N^{v}f_{0}\text{''}$ be the lifting of $N^{v}f_{0}$.
    We need to show $\text{``}N^{v}f\text{''}$ kills $A[N^{v}]$.
    But because $\text{``}N^{v}f\text{''}$ lifts $N^{v}f_{0}$, 
    then we have $\text{``}N^{v}f\text{''}[p^{\infty}]$ lifts $N^{v}f_{0}[p^{\infty}]$.
    By the uniqueness, we have 
    \[     \text{``}N^{v}f\text{''}[p^{\infty}]=N^{v}(f[p^{\infty}]).  \]
    Therefore $\text{``}N^{v}f\text{''}$ kills $A[N^{v}]$,
    and we find $\text{``}N^{v}f\text{''}=N^{v}F$,
    with $F$ a lifting of $f_{0}$.
    Therefore $F[p^{\infty}]$ lifts $f_{0}[p^{\infty}]$,
    and again by the uniqueness, we have $F[p^{\infty}]=f[p^{\infty}]$.

    It remains to  prove the essential surjectivity.
    The basic idea is first to lift the abelian scheme $A_{0}$ to an $R$ abelian scheme,
    and then  quotient a finite flat subscheme, to make the corresponding $p$-divisible groups match.

    Suppose we are given a triple $(A_{0},G,i)$.
    We must show the existence of an abelian scheme $A$ over $R$,
    which can produce this triple.

    Because $R$ is a nilpotent thickening of $R_0$,
    we can find an abelian scheme $B$ over $R$ which lifts $A_{0}$.
    This follows from the deformation theory of Grothendieck:
    If $S_{0}\to S$ is a closed immersion of affine schemes,
    defined by an ideal of square $0$, and if $X_{0}$ is an abelian scheme over $S_{0}$,
    then there exists an abelian scheme $X$ over $S$ which lifts $X_{0}$.

    That is to say we have an isomorphism of abelian schemes over $R_{0}$:
    \[  \alpha_{0}\colon   B_{0} \cong  A_{0} ,  \]
    which induces an isomorphism of $p$-divisible groups over $R_{0}$:
    \[  \alpha_{0}[p^{\infty}] \colon    B_{0}[p^{\infty}]     \to A_{0}[p^{\infty}] .  \]
    And $N^{v}\alpha_{0}[p^{\infty}]$ has a unique lifting to morphism of $p$-divisible groups over $R$:
    \[  \text{``}N^{v}\alpha\text{''}[p^{\infty}] \colon B[p^{\infty}]  \to G . \]
    This morphism is an isogeny, since it has an inverse up to isogeny,
    which is given by the canonical lifting of $N^{v}\times (\alpha_{0}[p^{\infty}])^{-1}$.

    And we have a short exact sequence 
    \[   0\to K \to B[p^{\infty}]\to G \to 0, \]
    with $K\subset B[N^{2v}] $ a finite flat subgroup scheme of $B[N^{2v}]$.
    So we may form the  quotient abelian scheme of $B$ by $K$, $A=B/K$.

    Because $K$ lifts $B_{0}[N^{v}]$,
    this quotient lifts $B_{0}/B_{0}[N^{v}]\cong B_{0}\cong A_{0}$.

    And the exact sequence 
    \[   0\to K \to B[p^{\infty}]\to G \to 0 \]
    induce a compatible isomorphism $A[p^{\infty}] \cong B[p^{\infty}]/K \cong G $.
\epf

\section{Deformation theory of ordinary elliptic curves}

With the Serre--Tate theorem,
we now study the deformation theory of  ordinary elliptic curves.

Fix $\Fb$ the algebraic closure of $\Fb_{p}$.
Pick a complete local ring $B$ with residue field $\Fb$.
Denote $\Cl_{/B}$ the category of complete local $B$-algebras with residue field $\Fb$,
$\mathrm{ART}_{/B}$ the full subcategory of $\Cl_{/B}$ made up of artinian local $B$-algebras with residue field $\Fb$.

Let $(E_{0},\w_{0})_{/\Fb}$ be an ordinary elliptic curve,
with $\w_{0}\in H^{0}(E_{0},\Omega_{E_{0}/\Fb})$. We consider the following deformation functor:
\[ \begin{aligned}
    &\hat{\Eb} \colon \Cl_{/B} \to \mathrm{Sets}, \\
    &\hat{\Eb}_{E_{0}}(R) = [(E_{/R},\iota_{E})\mid 
    E_{/R}\text{ is an elliptic curve and }\iota_{E}\colon E\times_{\Spec R}\Spec\Fb \cong E_{0}].
\end{aligned} \]
Here ``$[\enspace]$'' means the isomorphism classes of objects and we say
$f\colon (E,\iota_{E})_{/S}\cong (E',\iota_{E'})_{/S}$ is an isomorphism if we have the following commutative diagram:
\[ \begin{tikzcd}
    E\otimes_{B}\Fb \ar[r, "f_{0}"]\ar[d, "\iota_{E}", "\cong"']&E'\otimes_{B}\Fb \ar[d, "\iota_{E'}", "\cong"'] \\
    E_{0}\ar[r,equal]&E_{0}
\end{tikzcd} \]

One may ask, whether this functor is representable? The answer is yes!

We need a preliminary lemma.

\begin{lemma}
    Assume $E_{0}$ is ordinary.
    Take $(E,\iota_{E})_{/R}$ in $\hat{\Eb}_{E_{0}}(R)$ 
    for an artinian local $B$-algebra $R$ with residue field $\Fb$.
    Let $\hat{E}$ be the formal completion of $E$ along the origin of $E$.
    Then there exists a canonical $\Zb_{p}$-linear homomorphism $\phi_{E/R}\colon T_{p}E[p^{\infty}](\Fb)\to \hat{E}(R)$.
    Here $T_{p}E[p^{\infty}]^{\et}=\liminv E[p^{n}](\Fb)$.
\end{lemma}

\pf
    Since $R$ is an artinian local ring with maximal ideal $I$,
    we may find $v$ such that $I^{v+1}=0$, hence $p^{v+1}I=0$.
    Now by Lemma \ref{lem-a}, $p^{v}$ kills $\hat{E}$ since $\hat{E}$ is a formal Lie group over $R$.

    Since $E$ is ordinary, for the identity component of $E[p^{\infty}]$,
    we have $E[p^{\infty}]^{\circ}\cong \mu_{p^{\infty}}$ over $R$ 
    since it is a deformation of $E_{0}[p^{\infty}]^{\circ}\cong \mu_{p^{\infty}}$ over $\Fb$ 
    thus the Barsotti--Tate group $E[p^{\infty}]$ is ordinary,
    and we have the \'etale extension of Barsotti--Tate groups over $\mathrm{ART}_{/R}$
    \[ 0\to E[p^{\infty}]^{\mathrm{mult}}\to E[p^{\infty}] \to E[p^{\infty}]^{\et} \to 0 . \]
    Note that $E[p^{\infty}]^{\mathrm{mult}}$ over $\mathrm{ART}_{/R}$ is prorepresented by $\hat{E}$ since 
    \[  \hat{E}(R)=\hat{E}[p^{\infty}](R)\cong \mu_{p^{\infty}}(R)=E[p^{\infty}]^{\mathrm{mult}}(R) . \]
    We may rewrite the above sequence as an extension of fppf abelian sheaves:
    \[ 0\to \hat{E}\to E[p^{\infty}] \to E[p^{\infty}]^{\et} \to 0 . \]
    Taking the prime-to-$p$ part into account, we have the exact sequence 
    \[ 0\to \hat{E}(R) \to E(R) \to E(\Fb) \to 0 . \]
    Pick $x\in E(\Fb)$, and lift it to $\tilde{x}\in E(R)$.
    If $\tilde{x}'$ is another lift, we have $\tilde{x}-\tilde{x}'\in \hat{E}(R)$,
    so $p^{v}(\tilde{x}-\tilde{x}')=0$, that is $p^{v}\tilde{x}$ is uniquely determined by $x$, so we get a homomorphism 
    \[ \text{``}p^{v}\text{''}\colon  E(\Fb)[p^{v}]=E[p^{v}](\Fb)=E[p^{v}]^{\et}(\Fb)\to \hat{E} .  \]
    Note that this works for any $n\geq v$, we get a commutative diagram for any $n\geq v$:
    \[ \begin{tikzcd}
        E[p^{n+1}]^{\et}(\Fb)\ar[r, "\text{``}p^{n+1}\text{''}"]\ar[d, "p"]&\hat{E}(R) \ar[d] \\
        E[p^{n}]^{\et}(\Fb)\ar[r, "\text{``}p^{n}\text{''}"]&\hat{E}(R)
    \end{tikzcd} \]
    Taking the limit, we get $\phi_{E/R}\colon T_{p}E[p^{\infty}]^{\et}(\Fb)\to \hat{E}(R)$ 
    by sending $\liminv x_{n}$ to $\underset{n \to \infty}{\lim}\text{``}p^{n}\text{''}(x_{n})$.
\epf

\begin{theorem}
    Let $T_{p}E_{0}=\liminv E_{0}[p^{n}](\Fb)$ for a given ordinary elliptic curve $E_{0/\Fb}$,
    and regard as it as a constant group scheme over $R$
    (for any given artinian local $B$-algebra $R$ with residue field $\Fb$).
    The functor $\hat{\Eb}_{E_{0}}$ is represented by the formal torus 
    \[ R\to \Hom_{\Zb_{p}}(T_{p}E_{0}\otimes T_{p}E_{0},\hat{\mathbb{G}}_{m} (R))\cong \hat{\Gb}_{m}(R) . \]
    And each deformation $(E_{/R},\iota_{E}) \in \hat{\Eb}_{E_{0}}(R)$ gives rise a Serre--Tate coordinate 
    \[  t_{E/R}\colon T_{p}E_{0}\times T_{p}E_{0}\to \hat{\Gb}_{m}(R).   \]
    The isomorphism is induced by $\eta^{0}\colon \mu_{p^{\infty}}\cong E_{0}[p^{\infty}]^{\mathrm{mult}}$ 
    and its dual inverse $\eta^{\et}=(\eta^{0})^{*}\colon \mathbb{Q}_{p}/\Zb_{p} \cong E[p^{\infty}](\Fb)$.
\end{theorem}

\pf
    This theorem follows from the Serre--Tate theorem and the previous lemma.

    By the Serre--Tate theorem,
    an elliptic curve $E_{/R}\in \hat{\Eb}_{E_{0}}(R)$ which deforms $E_{0}/\Fb$ 
    is determined by the the extension 
    (the other deformation data are included in the definition of our deformation functors)
    \[ 0\to E[p^{\infty}]^{\mathrm{mult}}\to E[p^{\infty}]\to E[p^{\infty}]^{\et} \to 0 . \]
    Note that we have an isomorphism $\hat{E}\cong E[p^{\infty}]^{\mathrm{mult}} $.
    Then we get 
    \[ 0\to \hat{E} \to E[p^{\infty}] \to E[p^{\infty}]^{\et} \to 0. \]
    Thus the extension is determined by the extension class:
    \[   \operatorname{Ext}_{S(R_{\mathrm{fppf}})}( E[p^{\infty}]^{\et},\hat{E}). \eqno(*)  \]
    We are now going to describe $E[p^{\infty}]^{\et}$ and $\hat{E}$ use the data comes from $E_{0}$.

    By Cartier Duality theory of $E[N]$:
    \[ \begin{aligned}
        \hat{E}[p^{n}]
        &=E[p^{n}]^{\mathrm{mult}}=\Hom_{\mathrm{GSCH}_{/R}}\bigl(E[p^{n}]^{\et},\ \mu_{p^{n}}\bigr) \\ 
        &\cong \Hom_{\mathrm{GSCH}_{/R}}\bigl((T_{p}E[p^{\infty}]^{\et})\otimes_{\Zb}p^{-n}\Zb/\Zb,\ \mu_{p^{n}}\bigr) \\
        &\cong \Hom_{\mathrm{GSCH}_{/R}}\bigl(T_{p}E_{0}\otimes_{\Zb}p^{-n}\Zb/\Zb,\ \mu_{p^{n}}\bigr). 
    \end{aligned}      \]
    Passing to the limit, we get 
    \[ \hat{E}(R)\cong E[p^{\infty}]^{\mathrm{mult}}=
    \Hom_{\mathrm{GSCH}_{/R}}(T_{p}E_{0/R},\hat{\Gb}_{m/R})=\Hom_{\Zb_{p}}(T_{p}E_{0},\hat{\Gb}_{m}(R)) .  \]
    Also note that $E[p^{\infty}]^{\et}\cong T_{p}E_{0}\otimes \mathbb{Q}_{p}/\Zb_{p}$.
    We get that the extension group $(*)$ is isomorphic to 
    \[    \operatorname{Ext}_{S(R_{\mathrm{fppf}})}\bigl(
        T_{p}E_{0}\otimes \mathbb{Q}_{p}/\Zb_{p},\ \Hom_{\Zb_{p}}(T_{p}E_{0},\hat{\Gb}_{m}(R))\bigr) \]
    Now from the previous lemma, we can associate each extension a homomorphism 
    \[   T_{p}E_{0}\to \hat{E}=\Hom_{\Zb_{p}}(T_{p}E_{0},\hat{\Gb}_{m} ) . \]
    That is we have 
    \begin{multline*}
        \tau\colon  \operatorname{Ext}_{S(R_{\mathrm{fppf}})}\bigl(
            T_{p}E_{0}\otimes \mathbb{Q}_{p}/\Zb_{p},\ 
            \Hom_{\Zb_{p}}(T_{p}E_{0},\hat{\Gb}_{m})\bigr) \\
            \longrightarrow \Hom_{\Zb_{p}}\bigl(T_{p}E_{0},\ 
            \Hom_{\Zb_{p}}(T_{p}E_{0},\hat{\Gb}_{m})\bigr).
    \end{multline*}
    We are going to show this is in fact a bijection.
    First note that $T_{p}E_{0}\cong \Zb_{p}$. We get the following exact sequence 
    \[ 0\to T_{p}E_{0} \to (T_{p}E_{0})\otimes_{\Zb_{p}} \mathbb{Q}_{p} 
        \to (T_{p}E_{0})\otimes_{\Zb_{p}} \mathbb{Q}_{p}/\Zb_{p} \to 0 . \]
    From the morphism 
    $\phi \in \Hom_{\Zb_{p}}\bigl(T_{p}E_{0},\ \Hom_{\Zb_{p}}(T_{p}E_{0},\hat{\Gb}_{m})\bigr)$,
    we may form the push out of the above sequence 
    \[ \begin{tikzcd}[column sep=1em]
        0\ar[r]&T_{p}E_{0} \ar[r, "i"]\ar[d, "\phi"] &
        (T_{p}E_{0})\otimes_{\Zb_{p}} \mathbb{Q}_{p} \ar[r]\ar[d] & 
        (T_{p}E_{0})\otimes_{\Zb_{p}} \mathbb{Q}_{p}/\Zb_{p} \ar[r] \ar[d, "\mathrm{id}"]& 0 \\
        0 \ar[r]& \Hom_{\Zb_{p}}(T_{p}E_{0},\hat{\Gb}_{m}) \ar[r]& 
        E_{\phi}[p^{\infty}] \ar[r] & 
        (T_{p}E_{0})\otimes_{\Zb_{p}} \mathbb{Q}_{p}/\Zb_{p} \ar[r] & 0
    \end{tikzcd} \]
    where
    \[  E_{\phi}[p^{\infty}]:= \Coker \bigl( i-\phi\colon 
        T_{p}E_{0} \to ((T_{p}E_{0})\otimes_{\Zb_{p}} \mathbb{Q}_{p}) \times 
        \Hom_{\Zb_{p}}(T_{p}E_{0},\hat{\Gb}_{m}) \bigr). \]
    It is a $p$-divisible group.
    By construction $\tau(E_{\phi}[p^{\infty}])=\phi$. 
    And we recover the original extension of the Barsotti--Tate groups.
    Thus $\tau$ is bijective. We get 
    \[ \hat{\Eb}_{E_{0}}(R)=\Hom_{\Zb_{p}}\bigl(T_{p}E_{0},\ \Hom_{\Zb_{p}}(T_{p}E_{0},\hat{\Gb}_{m})\bigr)  \]
    Also note that we have 
    \[ \Hom_{\Zb_{p}}\bigl(T_{p}E_{0},\ \Hom_{\Zb_{p}}(T_{p}E_{0},\hat{\Gb}_{m} )\bigr)
        \cong \Hom_{\Zb_{p}} \bigl( T_{p}E_{0}\otimes_{\Zb_{p}} T_{p}E_{0},\ \hat{\Gb}_{m}\bigr) \]
    Sending $\phi_{E/R} \colon  T_{p}E_{0} \to \Hom_{\Zb_{p}}(T_{p}E_{0},\hat{\Gb}_{m} )$ 
    to $t_{E/R}\colon  T_{p}E_{0}\otimes_{\Zb_{p}} T_{p}E_{0}\to \hat{\Gb}_{m} $, given by 
    \[     t_{E/R}(a\otimes b)=\phi_{E/R}(a)(b)           \]
    Now note that $T_{p}E_{0} \cong \Zb_{p}$ since $E_{0}[p^{\infty}](\Fb)\cong \mathbb{Q}_{p}/\Zb_{p}$.
    We get 
    \[ \Hom_{\Zb_{p}}( T_{p}E_{0} \otimes_{\Zb_{p}} T_{p}E_{0},\hat{\Gb}_{m} ) 
    \cong \Hom_{\Zb_{p}}(\Zb_{p},\hat{\Gb}_{m} )\cong \hat{\Gb}_{m}. \qedhere  \]
\epf

The map $t_{E/R}\colon  T_{p}E_{0} \times T_{p}E_{0} \to \hat{\Gb}_{m}(R)$ 
is called the Serre--Tate coordinate. We are going to describe it explicitly.

Since $R\in \Cl_{/B}$ is a projective limit 
of local $B$-algebras $R_{n}$ with nilpotent maximal ideal,
we have $\hat{\Gb}_{m}(R)=\liminv \hat{\Gb}_{m}(R_{n})$; so 
\[   t_{E/R}=\liminv t_{E_{n}/R_{n}}    \]
for $E_{n}=E\times_{\Spec(R)} \Spec(R_{n})$. Therefore we may assume that 
$R$ is a local artinian $B$-algebra with nilpotent maximal ideal $m_{R}$.
Then by a previous lemma, we know $\hat{E}(R)$ is killed by $p^{v}$ for $v$ large enough.
Taking a lift $\tilde{x} \in E(R)$ of $x\in E(\Fb)=E_{0}(\Fb) $,
$\tilde{x}$ is determined modulo 
$\Ker(E(R)\to E(\Fb))=\hat{E}(R)$ which is a subgroup of $E[p^{n}]$ if $n\geq v$.
By the smoothness of $E_{/R}$, such a lift $\tilde{x}$ always exists.
$p^{n}\tilde{x} \in E(R)$ is uniquely determined by $x\in E_{0}(\Fb)$.
We get a morphism $T_{p}E_{0} \to \hat{E}(R)$. Note that we have the following exact sequence 
\[ 0\to \hat{E}[p^{\infty}]_{/R}\to E[p^{\infty}]_{/R} \to E[p^{\infty}]^{\et}_{/R}\to 0 \]
Take $x=\liminv x_{n} \in T_{p}E_{0}=T_{p}E[p^{\infty}]^{\et}$ 
with $x_{n}\in E_{0}[p^{n}](\Fb)$.
Lift $x_{n}$ to $v_{n}\in E(R)$ so that $\pi(v_{n})=x_{n}$.
Then for $\text{``}p^{n}\text{''}\colon E_{0}[p^{n}](\Fb)\to \hat{E}(R)$
\[    t_{n}(x)=\text{``}p^{n}\text{''}v_{n}\in \hat{E}(R)   \]
The values $t_{n}(x_{n})$ becomes stationary if $n\geq n_{0}$,
taking the limit as $n\to \infty$,
we get $t(x)\in \hat{E}(R) \cong \Hom_{\Zb_{p}}(T_{p}E_{0},\hat{\Gb}_{m}(R))$. 
We get $t(x)\in \hat{E}(R)\cong \Hom_{\Zb_{p}}(T_{p}E_{0},\hat{\Gb}_{m}(R))$.

Then we define $t_{E/R}(x,y)=t(x)(y)$.
Fix the isomorphism $\Zb_{p}\cong T_{p}E_{0}$,
we may define $t_{E/R}\in \hat{\Gb}_{m}(R)$ by $t_{E/R}(1)(1)$ for $1\in \Zb_{p}$.

\begin{theorem}
    Fix a level structure $\eta_{0}\colon  \mu_{p^{\infty}}\cong E_{0}[p^{\infty}]^{\mathrm{mult}}$.
    Then we have 
    \begin{enumerate}
        \item The functor $\hat{\Eb}_{E_{0}}$ is represented by the formal scheme
        \[   \Hom_{\Zb_{p}}(T_{p}E_{0}\otimes_{\Zb_{p}} T_{p}E_{0},\hat{\Gb}_{m})\cong \hat{\Gb}_{m}   \]
        by $t_{E/R}(x,y)\to t_{E/R}=t_{E/R}(1,1)$.

        \item $t_{E/R}(x,y)=t_{E/R}(y,x)$.

        \item Let $f_{0}\colon  E_{0/\Fb} \to E'_{0/\Fb}$ be a morphism 
        of two ordinary elliptic curves with the dual map, 
        $f^{t}_{0}\colon  E'_{0} \to E_{0}$. Then $f_{0}$ is induced by a homomorphism 
        $f\colon  E_{/R} \to E'_{/R}$ for $E\in \hat{\Eb}_{E_{0}}(R)$ and 
        $E'\in \hat{\Eb}_{E'_{0}}(R)$ if and only if 
        $t_{E/R}(x,f^{t}_{0}(y))=t_{E/R}(f_{0}(x),y)$.
        
        \item $E\in \hat{\Eb}_{E_{0}}(W(\Fb))$ has a complex multiplication 
        by an order $\mathscr{O}$ of an imaginary quadratic field with 
        $\mathscr{O}_{p}:=\mathscr{O}\otimes_{\Zb}\Zb_{p}$ isomorphic to 
        $\Zb_{p}\oplus \Zb_{p}$ if and only if $t_{E/W(\Fb)}=1$,
        where $W(\Fb)$ is the ring of Witt vectors of $\Fb$.    
    \end{enumerate}
\end{theorem}

\pf
    (1), (2) are clear from the proof of the Serre--Tate theorem.

    By the Serre--Tate theorem, we have such a lifting exists 
    if and only if we have a morphism $\phi\colon  E[p^{\infty}]\to E'[p^{\infty}]$ 
    such that the following diagram commutes:
    \[ \begin{tikzcd}
        \Hom_{\Zb_{p}}(T_{p}E_{0},\hat{\Gb}_{m})\ar[r]\ar[d, "\circ f^{t}"]&
        E[p^{\infty}] \ar[r]\ar[d, "\phi"] &
        T_{p}E_{0}\otimes_{\Zb_{p}} (\mathbb{Q}_{p}/\Zb_{p}) \ar[d, "f_{0}\otimes 1"'] \\
        \Hom_{\Zb_{p}}(T_{p}E_{0},\hat{\Gb}_{m})\ar[r]&
        E'[p^{\infty}]\ar[r]&
        T_{p}E_{0}\otimes_{\Zb_{p}}(\mathbb{Q}_{p}/\Zb_{p})
    \end{tikzcd} \]
    We have two ways to construct such a homomorphism $\phi$,
    either by push-out of the left square,
    or by the pull-back of the right square.
    And the condition for two such map coincide is 
    \[  t_{E/R}(\cdot,f^{t}(\cdot))=t_{E'/R}(f(\cdot),\cdot)  \]

    We now prove (4). By (3), if $t_{E/W}=1$ for $W=W(\Fb)$,
    then every endomorphism of $E_{0}$ lifts to an endomorphism of $E/W$.
    Thus $E$ has complex multiplication by the order $\operatorname{End}(E_{0/\Fb})$,
    then since $\mathscr{O}_{p}\cong \operatorname{End}(E_{0}[p^{\infty}]_{/\Fb})$,
    we conclude $\mathscr{O}_{p}\cong \Zb_{p} \oplus \Zb_{p}$.
    Conversely, if $E$ has complex multiplication by $\mathscr{O}$
    with $\mathscr{O}_{p}=\Zb_{p}\oplus \Zb_{p}$,
    the connected-\'etale sequence of $E[p^{\infty}]$ split over $W$,
    and this implies $t_{E/W}=1$.
\epf

Let $E_{/R/W(\bar{\Fb}_{p})}\in \hat{\Eb}_{E_{0}}(R)$.
We have the fundamental exact sequence 
\[ 0\to \mathscr{O}_{E}\otimes_{E}\Omega^{1}_{R/W}\to \Omega^{1}_{E/W}\to \Omega^{1}_{E/R}\to 0. \]

\begin{proposition}
    Let $\underline{\w}_{E/R}:=H^{0}(E,\Omega^{1}_{E/R})$, and 
    \[ \mathrm{Kod}\colon  \underline{\w}_{E/R} 
        \to H^{1}(E,\mathscr{O}_{E}\otimes_{R}\Omega^{1}_{R/W})
        \cong \operatorname{Lie}(E/R)\otimes_{R}\Omega_{R/W} \]
    the connection homomorphism. 
    Note that $\operatorname{Lie}(E/R)\cong \underline{\w}_{E/R}^{*}$ 
    by Grothendieck--Serre duality.
    Tensoring with $\underline{\w}_{E/R}$,
    we get the Kodaira--Spencer morphism:
    \[ \underline{\w}^{2}_{E/R}\to \Omega_{R/W}. \]
    If $E$ is the universal elliptic curve over $\hat{\mathbb{G}}_{m}$,
    then this map is an isomorphism.
\end{proposition}

\pf
    See \cite[Theorem~4.4.1]{Katz}.
\epf

\begin{remark}
    Please compare this proposition 
    with the complex Kodaira--Spencer morphism,
    especially \cite[Theorem~6.2.12]{Huybrechts}.
\end{remark}

\printbibliography

\end{document}
