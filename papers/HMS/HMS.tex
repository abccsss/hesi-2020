% !TEX program = xelatex
% !BIB program = biber

\documentclass[twoside]{article}

\usepackage{geometry}
\geometry{
    paperwidth = 155mm,
    paperheight = 235mm,
    outer = 20mm,
    inner = 20mm,
    top = 25mm,
    bottom = 20mm
}

% fonts & unicode
\usepackage[PunctStyle=kaiming]{xeCJK}
\usepackage{amsmath}
\usepackage{unicode-math}

\setCJKmainfont{fzssk.ttf}[
    Path = ../../fonts/,
    BoldFont = fzhtk.ttf,
    ItalicFont = fzktk.ttf
]

\setCJKsansfont{SourceHanSansCN-Normal.otf}[
    Path = ../../fonts/,
    BoldFont = SourceHanSansCN-Bold.otf,
    Scale = .97
]

\setCJKmonofont{SourceHanSansCN-Normal.otf}[
    Path = ../../fonts/,
    BoldFont = SourceHanSansCN-Bold.otf,
    Scale = .9
]

\newCJKfontfamily{\KaiTi}{fzktk.ttf}[
    Path = ../../fonts/,
    BoldFont = fzhtk.ttf,
    ItalicFont = fzssk.ttf
]

\setmainfont{STIX2Text}[
    Path = ../../fonts/,
    Extension = .otf,
    UprightFont = *-Regular,
    BoldFont = *-Bold,
    ItalicFont = *-Italic,
    BoldItalicFont = *-BoldItalic
]

\setsansfont{Lato}[
    Path = ../../fonts/,
    Scale = MatchUppercase,
    Extension = .ttf,
    UprightFont = *-Regular,
    BoldFont = *-Bold,
    ItalicFont = *-Italic,
    BoldItalicFont = *-BoldItalic
]

\setmonofont{FiraMono}[
    Path = ../../fonts/,
    Scale = .9,
    Extension = .otf,
    UprightFont = *-Regular,
    BoldFont = *-Bold
]

\setmathfont{STIX2Math.otf}[
    Path = ../../fonts/,
    BoldFont = STIX2Math-Bold.otf
]

\setmathfont{latinmodern-math.otf}[
    Path = ../../fonts/,
    range = {frak, bffrak}
]

\setmathfont{latinmodern-math.otf}[
    Path = ../../fonts/,
    range = {frak -> bffrak, bffrak},
    version = bold
]

\setmathfont{LatoMath.otf}[
    Path = ../../fonts/,
    Scale = .95,
    BoldFont = LatoMath.otf,
    version = sf
]

\setmathfont{LatoMath.otf}[
    Path = ../../fonts/,
    Scale = .95,
    BoldFont = LatoMath.otf,
    range = {bb, sfup -> up, sfit -> it, bfsfup -> bfup, bfsfit -> bfit}
]


\Umathcode`/  =  "0 "0 "2215    % / -> U+2215 division slash

% patch 'text math' math alphabets in bold math
\setmathfontface\mathrm{STIX2Text-Bold.otf}[
    Path = ../../fonts/,
    version = bold
]

\setmathfontface\mathit{STIX2Text-BoldItalic.otf}[
    Path = ../../fonts/,
    version = bold
]

\setmathfontface\mathbf{STIX2Text-Bold.otf}[
    Path = ../../fonts/,
    version = bold
]

\setmathfontface\mathtt{FiraMono-Bold.otf}[
    Path = ../../fonts/,
    Scale = .9,
    version = bold
]

\setmathfontface\mathrm{Lato-Regular.ttf}[
    Path = ../../fonts/,
    Scale = MatchUppercase,
    version = sf
]

\setmathfontface\mathit{Lato-Italic.ttf}[
    Path = ../../fonts/,
    Scale = MatchUppercase,
    version = sf
]

\setmathfontface\mathbf{Lato-Bold.ttf}[
    Path = ../../fonts/,
    Scale = MatchUppercase,
    version = sf
]

\setmathfontface\mathtt{FiraMono-Regular.otf}[
    Path = ../../fonts/,
    Scale = .9,
    version = sf
]


% bold math in bold text https://tex.stackexchange.com/q/41379
\makeatletter
    \g@addto@macro\bfseries{\boldmath} 
\makeatother

% title & abstract
\def\title#1\author#2{%
    \headertitle{#1}
    \vspace*{0mm}
    \begin{center}
        {\sf\LARGE#1\par}
        \vspace{10mm}
        {\large#2}
    \end{center}
    \vspace{10mm}
}
\def\headertitle#1{
    \def\theheadertitle{#1}
}

\renewcommand{\abstractname}{ABSTRACT}
\makeatletter
    \let\endabstract@orig\endabstract
    \def\endabstract{\endabstract@orig\vspace{5mm}}
\makeatother

% section titles
\usepackage{titlesec}
\titleformat*{\section}{\Large\sffamily\mathversion{sf}}
\titleformat*{\subsection}{\large\sffamily\mathversion{sf}}

% headers and footers
\usepackage{fancyhdr}
\fancyhf{}
\fancyhead[CE]{\sf\mathversion{sf}\theheadertitle}
\fancyhead[CO]{\sf\mathversion{sf}\nouppercase{\leftmark}}
\fancyhead[LE,RO]{\textbf{\textsf{\thepage}}}
\headsep=8mm
\headheight=6mm

\AtBeginDocument{
    \pagestyle{fancy}\thispagestyle{empty}
}

% spacing
\AtBeginDocument{
    \hfuzz=2pt
    \emergencystretch 2em
    \setlength{\belowdisplayshortskip}{\belowdisplayskip}
}

% environments
\usepackage{amsthm}
\newtheorem{theorem}{Theorem}[section]
\newtheorem{lemma}[theorem]{Lemma}
\newtheorem{corollary}[theorem]{Corollary}
\newtheorem{proposition}[theorem]{Proposition}

\theoremstyle{definition}
\newtheorem{definition}[theorem]{Definition}
\newtheorem{example}[theorem]{Example}
\newtheorem{remark}[theorem]{Remark}
\theoremstyle{plain}

\def\qedsymbol{$◻$}
\def\thmqedhere{\pushQED{\qed}\qedhere\popQED}

\numberwithin{equation}{theorem}

% renew theorem: https://tex.stackexchange.com/q/103013/
\makeatletter
\def\renewtheorem#1{%
    \expandafter\let\csname#1\endcsname\relax
    \expandafter\let\csname c@#1\endcsname\relax
    \gdef\renewtheorem@envname{#1}
    \renewtheorem@secpar
}
\def\renewtheorem@secpar{\@ifnextchar[{\renewtheorem@numberedlike}{\renewtheorem@nonumberedlike}}
\def\renewtheorem@numberedlike[#1]#2{\newtheorem{\renewtheorem@envname}[#1]{#2}}
\def\renewtheorem@nonumberedlike#1{  
\def\renewtheorem@caption{#1}
\edef\renewtheorem@nowithin{\noexpand\newtheorem{\renewtheorem@envname}{\renewtheorem@caption}}
\renewtheorem@thirdpar
}
\def\renewtheorem@thirdpar{\@ifnextchar[{\renewtheorem@within}{\renewtheorem@nowithin}}
\def\renewtheorem@within[#1]{\renewtheorem@nowithin[#1]}
\makeatother

% ref & biblatex
\usepackage[colorlinks,allcolors=black,bookmarksnumbered,linktoc=all]{hyperref}

\def\thesection{\arabic{section}\texorpdfstring{}{.}} % pdf bookmark numbering
\setcounter{secnumdepth}{1} % suppress subsection numbering

\usepackage[style=alphabetic,sorting=anyvt,useprefix=true]{biblatex}
\usepackage{xpatch}
\renewcommand*{\bibfont}{\small}
\DeclareFieldFormat[article]{volume}{\mkbibbold{#1}}
\DeclareFieldFormat[book,inbook]{number}{\mkbibbold{#1}}
\DeclareFieldFormat[article]{number}{(#1)}
\DeclareFieldFormat*{year}{(#1)}
\DeclareFieldFormat{pages}{#1}
\renewbibmacro{in:}{}
\renewbibmacro*{volume+number+eid}{%
    \printfield{volume}%
    \setunit*{\addnbspace}% originally: \setunit*{\adddot}
    \printfield{number}%
    \setunit{\addcomma\space}%
    \printfield{eid}%
}
\xapptobibmacro{author/editor+others/translator+others}{%
    \setunit{\space}%
    \printfield{year}%
    \clearfield{year}%
}{}{}
\xapptobibmacro{author/translator+others}{%
    \setunit{\space}%
    \printfield{year}%
    \clearfield{year}%
}{}{}
    \renewbibmacro*{issue+date}{%
    \printfield{issue}%
    \newunit%
}
\AtBeginBibliography{
    \DeclareFieldFormat{labelalpha}{#1}
    \DeclareFieldFormat{extraalpha}{\mknumalph{#1}}
}
\AtEveryBibitem{
    \ifentrytype{online}{%
        \clearfield{year}%
    }{}
}

% tikz
\usepackage{tikz}
\usepackage{tikz-cd}
\tikzset{
    > = latex
}
\tikzcdset{
    arrow style = tikz,
    arrows = {
        /tikz/line width = .5pt
    },
    diagrams = {
        > = {Straight Barb[scale = 0.8]}
    },
    nodes = {
        inner xsep = 3pt,
        inner ysep = 3pt
    }
}

\newenvironment{itms}{\begin{itemize}\itemsep=0pt\parsep=0pt}{\end{itemize}}
\newenvironment{enum}{\begin{enumerate}[label=(\arabic*)]\itemsep=0pt\parsep=0pt}{\end{enumerate}}

\theoremstyle{definition}
\newtheorem{construction}[theorem]{Construction}
\newtheorem{notation}[theorem]{Notation}

\def\varqed{\nolinebreak\hfill$◃$}
\def\varqedhere{\eqno ◃}

% math commands
\renewcommand{\:}{\colon}
\renewcommand{\/}{{∕}}
\newcommand{\bfDelta}{{
  \mathchoice{
    \tikz{
      \draw[line width=.05em]
        (0,0)--(.6em,0)--(.37em,1.32ex)--(.23em,1.32ex)--cycle
        (.23em,1.32ex)--(.46em,0);
    }
  }{
    \tikz{
      \draw[line width=.05em]
        (0,0)--(.6em,0)--(.37em,1.32ex)--(.23em,1.32ex)--cycle
        (.23em,1.32ex)--(.46em,0);
    }
  }{
    \tikz[scale=.7]{
      \draw[line width=.035em]
        (0,0)--(.6em,0)--(.37em,1.32ex)--(.23em,1.32ex)--cycle
        (.23em,1.32ex)--(.46em,0);
    }
  }{
    \tikz[scale=.55]{
      \draw[line width=.025em]
        (0,0)--(.6em,0)--(.37em,1.32ex)--(.23em,1.32ex)--cycle
        (.23em,1.32ex)--(.46em,0);
    }
  }
}}
\newcommand{\bigmid}{\mathrel{\big|}}
\newcommand{\biggmid}{\mathrel{\bigg|}}
\newcommand{\cat}[1]{\ensuremath\textup{\textsf{#1}}} % using textsf to retain kerning & ligature
\newcommand{\comma}{,}
\newcommand{\Ho}{\operatorname{Ho}}
\newcommand{\Hom}{\operatorname{Hom}}
\newcommand{\KD}{\operatorname{\reflectbox{$\mathrm{DK}$}}}
\newcommand{\Ndg}{\mathfrak{N}_{\mathrm{dg}}}
\newcommand{\op}{^{\mathrm{op}}}
\newcommand{\sHom}{\mathscr{H}\mkern-3mu\mathit{om}}
\newcommand{\simto}{\mathrel{\rlap{\raisebox{.8ex}{$\mkern2mu\sim$}}{\to}}}
\newcommand{\square}{\mathbin{◻}}

% text commands
\newcommand{\term}[1]{\textbf{\textup{#1}}}


\addbibresource{HMS.bib}
\nocite{*}

\begin{document}

\title{An Introduction to Gromov--Witten Invariants\\from the Symplectic Viewpoint}
\author{Ruitong Zhang\footnote{张睿桐,清华大学数学系数 72 班.}}
\headertitle{Gromov--Witten Invariants from the Symplectic Viewpoint}

\begin{abstract}
    In this article, we first present the preliminaries in symplectic geometry: 
    the moduli spaces of simple pseudo-holomorphic curves and the related evaluation maps. 
    Then we will define pseudo-cycles and introduce some basic propositions. 
    After the preparations, we will define Gromov--Witten invariant $\Phi$ 
    and introduce some examples and using a small conclusion, 
    we roughly prove the non-squeezing theorem. 
    Then we define the invariant $\Psi$. Finally, 
    we will introduce that the invariants we defined can be viewed 
    as special case of Gromov--Witten classes formulated by Kontsevich and Manin in 1994.
\end{abstract}

\section{Introduction}
The Gromov--Witten theory is a mathematical theory which corresponds to 
the A-model of closed topological string. 
And Gromov--Witten invariants play important role in this theory. 
Roughly speaking, Gromov--Witten invariants in certain situations, 
count pseudoholomorphic curves meeting prescribed conditions in a given symplectic manifold. 
Despite the use in string theory, 
these invariants have been used to distinguish symplectic manifolds that were previously indistinguishable.

In the class \emph{Topics in Mirror Symmetry} in Tsinghua 2019 Autumn, 
the invariants were defined in the algebraic setting, 
using Deligne--Mumford moduli space of curves. 
But in this passage, we will introduce the Gromov--Witten invariants from another perspective, 
we will define it in symplectic settings. 
We will first introduce the concept of pseudo-cycles, 
then define the invariants $\Phi$ and $\Psi$ by two different ways: 
by counting points in intersection theory, 
or by integration on the moduli spaces, but mainly focus on the first one. 
We will also see some basic examples, and as we calculated $\Phi_A(\pt)=1$ 
for $A=\left[\cp{1}\times\pt \right]\in H_2(\cp{1}\times V)$, 
where $\pi_2(V)=0$, we will prove the non-squeezing theorem. 
At last, we will roughly introduce the generalized Gromov--Witten classes.

\section{Symplectic Preliminaries and Notations}

We first introduce our notations.
For simplicity, we fix a Riemann surface $\Sigma$ with almost complex structure $j$ 
and local coordinates $s+it$, which we actually only consider it to be $\cp{1}$ 
in the main part of this introduction with the natural almost complex structure. 
In our setting, $M$ is a $2n$-dimensional symplectic manifold 
with symplectic structure $\omega$ and almost complex structure $J$. 

The moduli space $\maj$ consists of all simple $J$-holomorphic curves 
which represent a given homology class $A\in H_2(M)$. We have the following famous theorem.

\begin{theorem}
    \ 
    \begin{enumerate}
        \item If $J\in \mathscr{J}_{\mathrm{reg}}(A)$, 
        then the space $\maj$ is a smooth manifold of dimension $n(2-2g)+2c_1(A)$ and carries a natural orientation.
        \item The set $\mathscr{J}_{\mathrm{reg}}(A)$ has second category in $\mathscr{J}$. 
        That is, it contains the intersection of countably many open and dense subsets of $\mathscr{J}$.
    \end{enumerate}
\end{theorem}

We define the energy of a smooth map $u\colon \Sigma\rightarrow M$ 
to be the $L^2$-norm of the $1$-form $du\in \Omega^1(u^*TM)$
\[E(u)=\frac{1}{2}\int_{\Sigma}|du|^2_J\,d\mathrm{vol}_\Sigma.\]
If $J$ is $\omega$-tame then for all $J$-holomorphic curve $u$
\[ \begin{aligned}
    \frac{1}{2}|du|^2_J\,d\mathrm{vol}_\Sigma&=\frac{1}{2}\left(|\partial_su|_J^2+|\partial_tu|_J^2\right)ds\wedge dt \\
    &=\frac{1}{2}|\partial_su+J(u)\,\partial_tu|_J^2\,ds\wedge dt-\left\langle \partial_su,J(u)\,\partial_tu \right\rangle ds\wedge dt \\
    &=\frac{1}{2}\bigl(\omega(\partial_su,\partial_tu)+\omega(J(u)\,\partial_su,J(u)\,\partial_tu) \bigr)\,ds\wedge dt=u^*\omega.
\end{aligned} \]
Thus if $u\in \maj$,
\[E(u)=\omega(A).\]
For $K>0$, an $\omega$-compatible almost complex structure $J\in\mathscr{J}(M,\omega)$ 
is called \textbf{$K$-semi-positive} if every $J$-holomorphic sphere 
$u\colon \cp{1}\rightarrow M$ with energy $E(u)\leq K$ has nonnegative Chern number, 
i.e. 
\[\int u^*c_1\geq0,\]
and call $J$ \textbf{semi-positive} 
if it is $K$-semi-positive for all $K>0$. We denote the set as $\mathscr{J}_+(M,\omega,K)$ and 
\[\mathscr{J}_+(M,\omega)=\bigcap_{K>0}\mathscr{J}_+(M,\omega,K).\]
The manifold $(M,\omega)$ is called \textbf{weakly monotone} 
if for every spherical homology class $A\in H_2(M,\mathbb{Z})$,
\[\omega(A)>0,\quad c_1(A)\geq 3-n \quad \Longrightarrow \quad c_1(A)\geq 0.\]
It is called $\textbf{monotone}$ if there exists $\lambda>0$ such that
\[\omega(A)=\lambda c_1(A)\]
for all spherical homology class $A$.

Consider the reparametrization group $G=\mathrm{PSL}(2,\mathbb{C})$,
which acts on the space $\maj\times \cp{1}$ by  % added 'the space' to avoid an underfull box
\[\phi \cdot(u,z)=(u\circ \phi^{-1},\phi(z)). \]
And denote the quotient space as $\waj=\maj\times_G \cp{1}$, 
which is the domain of the evaluation map $\mathrm{ev}\colon \waj\rightarrow M$ as $e(u,z)=u(z)$. 
And the image is denoted by $X(A,J)=\mathrm{ev}(\waj))$.
We also need to consider $p$-fold evaluation maps, we now introduce two different evaluation maps: 
one way is the generalization of the above. 
Before that, we introduce the concept of effective framing class.
The \textbf{framed class}
\[D=\{A^1,A^2,\cdots,A^N,j_2,\cdots,j_N \}\]
where $A^i\in H_2(M,\mathbb{Z})$ and $j_n$ is an integer $1\leq j_n\leq n-1$. 
The homology classes $A^i$ are effective in the sense that 
they each have $J$-holomorphic representatives 
and can be the classes of the components of a simplified cusp-curve $C$ which represents the class $A$.

And from Gromov compactness theorem, 
there is only a finite number of possible framed classes $D$. 
Here $j_n$ indicates that the cusp-curve $u=(u^1,u^2,\cdots,u^N)$ must satisfy
\[u^n(\cp{1})\cap u^{j_n}(\cp{1})\neq \emptyset.\]

Now we consider the universal moduli space
\[\mathscr{M}(A^1,\cdots,A^N,\mathscr{J})\]
which consists of all $(N+1)$-tuples $(u^1,\cdots,u^N,J)$ such that $J\in\mathscr{J}$, $u^j\in \mathscr{M}(A^j,J)$ 
and $u^i\neq u^j\circ \phi$ for $\phi\in G$ and $i\neq j$.

Thus we can define the moduli space 
$\mdj\subset \mathscr{M}(A^1,\cdots,A^N,\mathscr{J})\times (\cp{1})^{2N-2}$ 
consisting of all points $(u,w,z)$ with $w=(w_2,\cdots,w_N)\in\cp{N-1}$, 
$z=(z_2,\cdots,z_N)\in\cp{N-1}$ and $u^j\in \mathscr{M}(A^j,J)$ such that 
$u=(u_1,\cdots,u^N)$ is a simple $J$-holomorphic curve cusp-curve with
\[u^j_n(w_n)=u^n(z_n).\]
We can show that there exists a set $\mathscr{J}_{\mathrm{reg}}(D)\subset \mathscr{J}$ 
of the second category such that $\mdj$ is a manifold of dimension $2\sum_{j=1}^{N}c_1(A^j)+2n+4(N-1)$, 
which is then equal to $2c_1(A)+2n+4(N-1)$.

We can define the space of simple unparametrized curves of type $D$
\[\mathscr{C}(D,J)=\mdj/G^N.\]

Fix $l\leq N$ and let $G_{l,N}$ be the subgroup
\[G_{l,N}=\{\phi=(\phi_1,\cdots,\phi_N) \mid \phi_l=\id  \}\subset G^N,\]
so $G$ acts on this space by acting on its $l$-th component in the usual way and we can thus define
\[\mathscr{W}_l(D,J)=\mathscr{C}_l(D,J)\times_G \cp{1}\]
as the quotient space, which is actually a manifold, and take the union
\[\wdj=\bigcup_l \mathscr{W}_l(D,J).\]
Then we have the evaluation map
\[\mathrm{ev}\colon \wdj\rightarrow M,\]
which is defined on $\mathscr{W}_l(D,J)$ as
\[(u^j,w_n,z_n,\zeta)\mapsto u^l(\zeta).\]

Now consider the pairs $(D,T)$ where $D$ is a framed class of 
simple cusp-curve and the label $T$ is a map from $\{1,2,\cdots,p \}$ to $\{1,2,\cdots,N \}$. 
For each pair $(D,T)$, we define 
\[X(D,T,J,p)=\{(x_1,\cdots,x_p)\mid x_j\in C_{T(j)},\ C\in\mathscr{C}(D,J)   \},\]
and set
\[\mathscr{W}(D,T,J,p)=\mdj\times_{G^N}(\cp{1})^p,\]
where the $n$-th factor of $\phi=(\phi_1,\cdots,\phi_N)\in G^N$ acts on $\mdj$ as before 
and acts on the $j$-th factor of $(\cp{1})^p$ iff $T(j)=n$, so there is an evaluation map
\[e_{D,T}\colon \mathscr{W}(D,T,J,p)\rightarrow M^p\]
defined by
\[(u,J,w,z,\zeta)=(u_{T(1)}(\zeta_1),\cdots,u_{T(p)}(\zeta_p)).\]
Then we define another type of $p$-fold evaluation by fixing a $p$-tuple 
\[\z=(z_1,\cdots,z_p)\in(\cp{1})^p\]
and the evaluation on $\maj$ is given by
\[ev_\z(u)=(u(z_1),\cdots,u(z_p)),\]
and we denote the image as 
\[Y(A,J,\z)=e_\z(\maj)\subset M^p.\]
To compactify it and ensure the boundary is a finite union of lower dimensional strata,
we introduce a concept of $J$-effective homology class: $A$ is called \textbf{$J$-effective} 
if it can be represented by a $J$-holomorphic sphere $u\colon \cp{1}\rightarrow M$.

And we need our almost complex structure $J\in\mathscr{J}(M,\omega)$ 
and the homology class $A$ to satisfy the following condition.

\begin{enumerate}[label=(\arabic*)]
    \item Every $J$-effective homology class $B\in H_2(M,\mathbb{Z})$ has Chern number
    \[c_1(B)\geq 2.\]
    Moreover, if $A=mB\in H_2(M,\mathbb{Z})$ is the $m$-fold multiple of a $J$-effective 
    homology class $B\in H_2(M,\mathbb{Z})$ with $m>1$, then either $c_1(B)\geq 3$ or $p\leq 2m$.
\end{enumerate}

To find the boundary (denote as $\mathscr{V}(D,T,J,\z)$), 
there are actually four cases for the possible limit behavior 
of the points $e_\z(u_\nu)$ for a sequence in $\maj$. 
These are that $u_\nu$ converges modulo bubbling

\begin{enumerate}[label=(\alph*)]
    \item to a simple $J$-holomorphic sphere,
    \item to an $m$-fold covering of a simple $J$-holomorphic sphere where $2m<p$,
    \item to an $m$-fold covering of a simple $J$-holomorphic sphere where $2m\geq p$,
    \item to a constant map.
\end{enumerate}

We first look at the first two cases. Let $M\subset\cp{1}$ be the finite set 
at which holomorphic spheres bubble off. 
Then $u_\nu$ converges on the complement of $X$ to a curve $v^0\circ \psi$ where
\[v^0\in\maj,\]
and $\psi\colon \cp{1}\rightarrow \cp{1}$ is a rational map of degree $m$. 
Denote the space of such maps by $\mathrm{Rat}_m$ and has real dimension $4m+2$. 
Suppose $l$ of the points $z_j$ lie in $X$ and WLOG assume to be $z_1,\cdots,z_l$. 
Then we must consider limit curves of the form $(\psi,v^0,\cdots,v^N)$ such that
\[v^\nu(0)=v^0(\psi(z_\nu)),\qquad \nu=1,\cdots,l.\]
The remaining intersection pattern of the limiting curve is unconstrained, and thus can be described by integers
$j_{l+1},\cdots,j_N$ with $0\leq j_\nu\leq \nu-1$ and
\[v^\nu(0)=v^{j_\nu}(w_\nu),\qquad \nu=l+1,\cdots,N.\]
for some points $w_\nu\in\cp{1}$. 
Thus the intersection pattern of this curve can be coded in a framing $D$ of the form
\[D=\{m,A^0,\cdots,A^n,j_{l+1},\cdots,j_N\},\]
where there are positive integers $m_0,\cdots,m_N$ such that $m_0=m$ and
\[\sum_{j=0}^{N}m_jA^j=A.\]
So we can consider the corresponding moduli space $\mathscr{M}(D,J,\z)$ 
which consists of all $2N+2-l$-tuples
\[(\psi,v^0,\cdots,v^N,w_{l+1},\cdots,w_{N})\]
satisfying the conditions and $v^j$ represents $A^j$. We have the following lemma:

\begin{lemma} \label{lem-2-2}
    For generic $J$, the moduli space $\mathscr{M}(D,J,\z)$ is a manifold of dimension
    \[\dim\mathscr{M}(D,J,\z)=2n+2N+4m+2c_1(D)-2l+2\]
\end{lemma}

Now choose a map $T\colon \{1,\cdots,l\}\rightarrow \{1,\cdots,N\}$ and define the space
\[\mathscr{V}(D,T,J,\z)=\frac{\mathscr{M}(D,J,\z)\times (\cp{1})^l}{G\times G^N_0},\]
where $G_0=\{\phi\in G \mid \phi(0)=0  \}$, and the group $G\times G^N_0$ acts by
\[\phi\cdot (\psi,v,w,\zeta)=(\phi_0\circ\psi,v^\nu\circ\phi_\nu^{-1},\phi_{j_\nu}(w_\nu),\phi_{T(\nu)}(\zeta_\nu)),\]
and the evaluation map
\[e_{D,T,\z}(\psi,v,w,\zeta)=(v^{T(1)}(\zeta_1),\cdots,v^{T(l)}(\zeta_l),v^0(\psi(z_{l+1}),\cdots,v^0(\psi(z_N)).\]
Now consider the case (c). The moduli space of corresponding simplified cusp-curves is dertermined by a framed class
\[D=\{A^0,\cdots,A^N,j_1,\cdots,j_N  \}\]
with $(N+1)$ components and Chern number
\[c_1(D)\leq c_1(A)-(m-1)\,c_1(A^0).\]
Since $p\leq 2m$, the reparametrization map $\psi$ acts transitively on $p$-tuples $\z$ of distinct points, 
so the set of points may accumulate anywhere on the limiting cusp-curve. 
Thus we define the corresponding space $\mathscr{V}(D,T,J,\z)$ 
to be the space $\mathscr{W}(D,T,J,p)$ considered before, which has dimension 
\[ \begin{aligned}
\dim\mathscr{W}(D,T,J,p)&=2n+2c_1(D)+2p-2N-6\\
&\leq 2n+2c_1(A)-2(m-1)c_1(A^0)+4m-2N-6\\
&=2n+2c_1(A)-2(m-1)(c_1(A^0)-2)-2N-2\\
&\leq 2n+2c_1(A)-2\\
&=\dim\maj-2
\end{aligned} \]
For case (d), the sequence $u_\nu(z)$ converges to a constant $J$-holomorphic curve 
on the complement of a finite set $X$, and take the special case in (a) and (b) such that 
$m=1$ and $A^0\in H_2(M,\mathbb{Z})$ is the zero homology class, 
where we need to further assume that $J$ is $K$-semi positive for some $K>\omega(A)$.

\section{Pseudo-cycles}

We now define the pseudo-cycle. Let $M$ be a smooth compact manifold. 
An arbitrary subset $B\subset M$ is said to be of \textbf{dimension at most} $m$ 
if it can be covered by the image of a map $g\colon W\rightarrow M$ 
which is defined on a manifold $W$ of dimension $m$. 
In this case we denoted as $\dim B\leq m$.

A $k$-dimensional \textbf{pseudo-cycle} in $M$ is a smooth map $f\colon V\rightarrow M$, 
defined on an oriented $k$-dimensional manifold $V$ such that 
\[\dim \Omega_f\leq \dim V-2,\]
where
\[\Omega_f=\bigcap_{K\subset V,\: K\: \mathrm{compact}}\overline{f(V-K)}.\]

Two $k$-dimensional pseudo-cycles $f_0\colon V_0\rightarrow M$ and $f_1\colon V_1\rightarrow M$ are called \textbf{bordant} 
if there exists a $(k+1)$-dimensional oriented manifold $W$ with $\partial W=V_1-V_0$ 
and a smooth map $F\colon W\rightarrow M$ such that 
\[F|_{V_0}=f_0, \qquad F|_{V_1}=f_1, \qquad \dim \Omega_F\leq k-1.\]

Pseudo-cycles form an abelian group with addition given by disjoint union, 
and the neutral element is the empty map defined on the empty manifold.

Two pseudo-cycles $e\colon U\rightarrow M$ and $f\colon V\rightarrow M$ are called \textbf{transverse} if
\[\Omega_e\cap \overline{f(V)}=\emptyset, \qquad \Omega_f\cap\overline{e(U)}=\emptyset,\]
and if
\[T_xM=\im de(u)+\im df(v)\]
whenever $e(u)=f(v)=x$. If $e$ and $f$ are transverse, then the set
\[\{(u,v)\in U\times V\mid e(u)=f(v)  \}\]
is a compact manifold of dimension $\dim U+\dim V-\dim M$.

We have the following lemma:

\begin{lemma}
Let $e\colon U\rightarrow M$ and $f\colon V\rightarrow M$ be pseudo-cycles of complementary dimension.
    \begin{enumerate}
        \item There exists a set $\mathrm{Diff}_{\mathrm{reg}}(M,e,f)\subset\mathrm{Diff}(M)$ of the second category 
        such that $e$ is transverse to $\phi\circ f$ for all $\phi\in\mathrm{Diff}_{\mathrm{reg}}(M,e,f)$
        \item If $e$ is transverse to $f$ then the set $\{(u,v)\in U\times V\mid e(u)=f(v)  \}$ is finite. 
        In this case define
        \[e\cdot f=\sum_{e(u)=f(v)}\nu(u,v)\]
        where $\nu(u,v)$ is the intersection number of $e(U)$ and $f(V)$ at point $x=e(u)=f(v)$.
        \item The intersection number $e\cdot f$ only depends on the bordism classes of $e$ and~$f$.
    \end{enumerate}
\end{lemma}

Given a class $\alpha\in H_d(M,\mathbb{Z})$, it can be represented by a map $f\colon P\rightarrow M$ 
defined on a $d$-dimensional finite oriented simplicial complex $P$ without boundary, 
i.e.\ the oriented faces of its top-dimensional simplices cancel each other out in pairs. 
So $\alpha=f_*[P]$. Now approximate $f$ by a map which is smooth on each simplex, 
and consider the union of the $d$ and $(d-1)$-dimensional faces of $P$ 
as a smooth $d$-dimensional manifold $V$ and approximate $f$ by a map 
which is smooth across the $(d-1)$-dimensional simplices. Thus in this way, 
we get a pseudo-cycle since the $\Omega_f$ part is the lower dimensional part which has dimension at most $d-2$.

Every $(m-d)$-dimensional pseudo-cycle $e\colon W\rightarrow M$ determines a homomorphism
\[\Phi_e\colon H_d(M,\mathbb{Z})\rightarrow \mathbb{Z}\]
as follows:
Represent the class $\beta\in H_d(M,\mathbb{Z})$ by a cycle $f\colon V\rightarrow M$, 
which can be chosen such that it is transverse to $e$ according to the above lemma
(Choose $\phi\in$Diff$_{\mathrm{reg}}(M,e,f)$ of degree 1, 
then replace $f$ with $\phi\circ f$), 
and since any two such representations are bordant and the intersection number
\[\Phi_e(\beta)=e\cdot f\]
is independent of the choice of the cycle $f$ representing $\beta$. 
And from the lemma above, $\Phi_e$ only depends on the bordism class of $e$.

An $(m-d)$-dimensional pseudo-cycle $e\colon U\rightarrow M$ 
is called a \textbf{weak representative} of the homology class $\alpha\in H_{m-d}(M)$ if 
\[e\cdot f=\alpha\cdot\beta\]
for all $\beta\in H_d(M,\mathbb{Z})$ and every cycle $f$ representing $\beta$. 
And we have the following lemma:

\begin{lemma}
Let $e\colon U\rightarrow M$ be an $(m-d)$-dimensional pseudo-cycle. 
If the $d$-dimensional pseudo-cycle $f\colon V\rightarrow M$ is 
a weak representative of the homology class $\beta\in H_d(M)$, then 
\[\Phi_e(\beta)=e\cdot f.\]
\end{lemma}

To define the Gromov--Witten invariants, we still need some theorems:

\begin{theorem}
Let $(M,\omega)$ be a compact symplectic manifold.
    \begin{enumerate}
        \item For every $J\in\mathscr{J}(M,\omega)$ there exists 
        a finite collection of evaluation maps $e_{D,T}\colon \mathscr{W}(D,T,J,p)\rightarrow M^p$ such that
        \[\bigcap_{K\subset \mathscr{W}(A,J,p),\: K\: \mathrm{compact}}\overline{e_p(\mathscr{W}(A,J,p)-K)}\subset \bigcup_{D,T}e_{D,T}(\mathscr{W}(D,T,J,p)).\]
        \item There exists a set 
        \[\mathscr{J}_{\mathrm{reg}}=\mathscr{J}_{\mathrm{reg}}(M,\omega,A,p)\subset \mathscr{J}(M,\omega)\]
        of the second category such that the set $\mathscr{W}(D,T,J,p)$ is a smooth oriented $\sigma$-compact manifold of dimension 
        \[\dim \mathscr{W}(D,T,J,p)=2n+2c_1(D)+2p-2N-4\]
        for $J\in\mathscr{J}_{\mathrm{reg}}$. Here $N$ is the number of components of class $D$.
        \item Assume that  $A$ is not a nontrivial multiple of a class 
        $B$ with $c_1(B)=0$. If $J\in\mathscr{J}_+(M,\omega,K)\cap \mathscr{J}_{\mathrm{reg}}$ for some $K>\omega(A)$, then
        \[\dim \mathscr{W}(D,T,J,p)\leq \dim\mathscr{W}(A,J,p)-2.\]
    \end{enumerate}
\end{theorem}

Using the language of pseudo-cycles, we can restate it as:

\begin{theorem}
    The evaluation map $e_{A,J,p}\colon \mathscr{W}(A,J,p)\rightarrow M^p$ defined by 
    \[e_p(u,z_1,\cdots,z_p)=(u(z_1),\cdots,u(z_p))\]
    determines a pseudo-cycle whenever $J\in\mathscr{J}_+(M,\omega,K)\cap \mathscr{J}_{\mathrm{reg}}$ 
    for some $K>\omega(A)$.
\end{theorem}

For the second case where we fix $\z$, we have the similar theorem

\begin{theorem}
    Let $(M,\omega)$ be a compact symplectic manifold and fix a homology class 
    $A\in H_2(M)$ and a $p$-tuple $\z=(z_1,\cdots.z_p)\in(\cp{1})^p$ of distinct points in $\cp{1}$. Then
    \begin{enumerate}
        \item For every $J\in\mathscr{J}(M,\omega)$ there exists a finite collection of evaluation maps
        \[d_{D,T,\z}\colon \mathscr{V}(D,T,J,\z)\rightarrow M^p,\]
        such that
        \[\bigcap_{K\subset \mathscr{M}(A,J),\: K\: \mathrm{compact}}\overline{e_\z(\maj-K)}\subset\bigcup_{D,T}e_{D,T,\z}(\mathscr{V}(D,T,J,\z)).\]
        \item There exists a set 
        $\mathscr{J}_{\mathrm{reg}}=\mathscr{J}_{\mathrm{reg}}(M,\omega,A,\z)\subset \mathscr{J}(M,\omega)$ 
        of the second category such that the set $\mathscr{V}(D,T,J,\z)$ is a 
        finite dimensional smooth oriented $\sigma$-compact manifold for every $J\in\mathscr{J}_{\mathrm{reg}}$
        \item  Assume $J\in \mathscr{J}_{\mathrm{reg}}$ and the pair $(J,A)$ satisfies \textup{(1)}, then
        \[\dim\mathscr{V}(D,T,J,\z)\leq\dim\maj-2\]
        for all $D\neq A$.
    \end{enumerate}
\end{theorem}

In the language of pseudo-cycles, it can be restated as

\begin{theorem}
The evaluation map $e_{A,J,\z}(u)=(u(z_1),\cdots,u(z_p)$ represents 
a pseudo-cycle whenever $J\in \mathscr{J}_{\mathrm{reg}}$ and the pair $(J,A)$ satisfies \textup{(1)}.
\end{theorem}

\section{The Invariant \texorpdfstring{$\Phi$}{Φ}}

In this section, we define the invariant $\Phi$. Roughly speaking, 
$\Phi_{A,p}(\alpha_1,\cdots,\alpha_p)$ counts the number of the set
\[\{[u,z_1,\cdots,z_p]\in\mathscr{W}(A,J,p)\mid u(z_i)\in \alpha_i    \}.\]
We will see that in the weakly monotone case, 
this is independent of the choice of $J$ and $\omega$ is not actually needed. 
We will give the precise definition below.

Let $(M,\omega)$ be a compact symplectic manifold and fix a class $A\in H_2(M,\mathbb{Z})$. 
We will assume that $A$ is not a nontrivial multiple 
of a class $B$ with $c_1(B)=0$ and the manifold is weakly monotone.

We have proved before that every $(m-d)$-dimensional pseudo-cycle $e\colon U\rightarrow M$ 
determines a homomorphism $\Phi_e\colon H_d(M,\mathbb{Z})\rightarrow \mathbb{Z}$ and hence 
is an element of $H^d(M)=H^d(M,\mathbb{Z})/\mathrm{torsion}=\homo(H_d(M,\mathbb{Z}),\mathbb{Z})$, 
which we denoted by $a_e=[\Phi_e]\in H^d(M)$.

From the previous discussion, $e_{A,J,p}$ is a pseudo-cycle, so it determines a homomorphism
\[\Phi_{A,J,p}\colon H_d(M^p,\mathbb{Z})\rightarrow \mathbb{Z},\]
with 
\begin{equation}\label{equ:dim}
    d=2(n-1)(p-1)+4-2c_1(A).
\end{equation}
For a homology class $\alpha\in H_d(M^p,\mathbb{Z})$, represent it by a cycle $f\colon V\rightarrow M^p$, 
and put it in general position so that it transverse to $e_{A,J,p}$, 
so the set $e_p(\mathscr{W}(A,J,p))\cap f(V)$ is finite and we can define
\[\Phi_{A,J,p}(\alpha)=e_{A,J,p}\cdot f\]
as in the previous Lemma. 
And the right hand side depends only on the bordism class of $f$ and hence only on the homology class of $\alpha$. 
Moreover, the equality holds for all pseudo-cycles $f$ 
which is a weak representative of $\alpha$ and the homomorphism $\Phi_{A,J,p}$ 
only depends on the bordism class of the evaluation map $e_{A,J,p}$.

The invariant $\Phi_{A,J,p}(\alpha)$ vanishes on all torsion elements $\alpha\in H_d(M^p,\mathbb{Z})$. 
Since the free part $H_d(M^p,\mathbb{Z})/\mathrm{torsion}$ is generated by 
product cycles $\alpha=(\alpha_1\times\alpha_2\times\cdots \alpha_p)$, 
we often restrict $\Phi_{A,J,p}$ to such cycles and write
\[\Phi_{A,J,p}(\alpha_1,\cdots,\alpha_p)=\Phi_{A,J,p}(\alpha_1\times\alpha_2\times\cdots \alpha_p).\]
It is convenient to set $\Phi_{A,J,p}=0$ when the dimension condition (\ref{equ:dim}) is not satisfied.

\begin{remark}
Actually, the invariant $\Phi_{A,J,p}(\alpha,\cdots,\alpha_p)$ can be defined as 
the intersection number of $e_{A,J,p}$ with a product cycle $f_1\times\cdots\times f_p$ 
where $f_j\colon V_j\rightarrow M$ are pseudo-cycles representing the homology class $\alpha_j$, respectively. 
This is a simple corollary of Lemma \ref{lem-2-2} by considering $f$ is a product cycle and $\phi$ a product diffeomorphism.
\end{remark}

It is an interesting fact that the invariant is actually independent of the choice of $J$ which used to define it.

\begin{proposition} \label{prop-4-2}
    Assume that $(M,\omega)$ is weakly monotone and 
    $A$ is not a nontrivial multiple of a class $B$ with $c_1(B)=0$. Then the homomorphism
    \[\Phi_{A,p}=\Phi_{A,J,p}\colon H_d(M^p,\mathbb{Z})\rightarrow \mathbb{Z}\]
    is independent of the choice of the regular $\omega$-tame 
    almost complex structure $J\in \mathscr{J}_{\mathrm{reg}}(M,\omega)$.
\end{proposition}

\begin{proof}
    Consider a homotopy $\{J_\lambda\}$ of regular almost complex structures 
    between $\{J_0\}$ and $\{J_1\}$, any path may be perturbed so that the sets
    \[\mathscr{W}(D,T,\{J_\lambda\}_\lambda,p)=\bigcup_{\lambda}\{\lambda\}\times\mathscr{W}(D,T,J_\lambda,p)\]
    are smooth manifolds for all $(D,T)$.
    
    By Gromov's compactness theorem, the closure of 
    the image of $\mathscr{W}(A,J_\lambda,p)$ under evaluation map 
    contain points in $M^p$ which lie on some cusp-curve 
    which represents class $A$. Thus, in order to compactify $X(A,J,p)$, 
    it suffices to add the points in $X(D,T,J,p)$ as $(D,T)$ ranges over some classes, 
    where $D$ should be effective, framed class which can represent 
    a simplified $J$-holomorphic cusp-curves 
    whose energy bounded by $\omega(A)$. And we can calculate the dimensions:
    \[ \begin{aligned}
        \dim\mathscr{W}(D,T,\{J_\lambda\}_\lambda,p)&= \dim \mdj+2p-6N+1\\
        &=2\sum_{j=1}^{N}c_1(A^j)+2n+4(N-1)+2p-6N+1\\
        &=2\sum_{j=1}^{N}c_1(A^j)+2n+2p-2N-3.
    \end{aligned} \]
    If $N=1$, then $A=mA^1$ for $m\geq 2$, and $c_1(A^1)\neq0$, 
    if $J_\lambda\in\mathscr{J}_+(M,\omega,K)$ for some $K>\omega(A)$, then $c_1(A^1)>0$, so
    \[1+\dim \maj\times_G(\cp{1})^p-\dim\mathscr{W}(D,T,\{J_\lambda\},p)\geq 2.\]
    If $N\geq 2$ then the above inequality also holds.
    Thus the evaluation maps
    \[e_{A,J_0,p}\colon \mathscr{W}(A,J_0,p)\rightarrow M^p,\qquad e_{A,J_1,p}\colon \mathscr{W}(A,J_1,p)\rightarrow M^p\]
    determine bordant pseudo-cycles and we are done. 
\end{proof}

A \textbf{deformation} of a symplectic form $\omega$ is 
a smooth 1-parameter family $\omega_t$,  $t\in[0,1]$ of forms starting from $\omega_0=\omega$. 
The difference between deformation and isotopy is that 
deformation do not require the cohomology class remain constant.

Since the taming condition is open, an almost complex structure $J$ 
which is tamed by $\omega$ is tamed by all sufficiently close symplectic forms. 
So the homomorphism $\Phi_{A,p}$ doesn't change under a deformation $\omega_t$ of $\omega$, 
provided that $(M,\omega_t)$ is weakly monotone for all $t$.

\begin{remark}
    There is an alternative way to express the invariant $\Phi_{A,p}$ in terms of 
    integrals of certain differential forms over the moduli space $\mathscr{W}(A,J,p)$. 
    Given a homology class $\alpha\in H_d(M^p)$ where $d$ satisfies the dimensional condition (\ref{equ:dim}), 
    we can define $\Phi_{A,J,p}(\alpha)$ as the integral
    \[\Phi_{A,J,p}(\alpha)=\int_{\mathscr{W}(A,J,p)}e_{A,J,p}^*\tau,\]
    where $\tau$ is closed and represents the Poincar\'e dual $a=\mathit{PD}([\alpha])\in H^{2np-d}(M^p)$, 
    then the degree of $\tau$ is
    \[ \begin{aligned}
        \deg \tau &=2np-d \\
        &=2np-2(np-n-p+1)-4+2c_1(A)\\
        &=2n+2p+2c_1(A)-6 \\
        &=\dim\mathscr{W}(A,J,p).
    \end{aligned} \]
    Thus $e_{A,J,p}^*\tau$ is a top-degree form on $\mathscr{W}(A,J,p)$. 
    If one uses this definition, there are some questions: 
    we must show that it is independent of differential form to represent the class, 
    and is independent of the almost complex structure $J$, and is finite.

    The representative question is because since 
    the pull-back of an exact form has integral 0 because the boundary is of codimension 2.
    The precise proof is nontrivial but in the case of intersection theory, 
    it can be avoided since Sard's theorem. 
    The finiteness problem can be solved by choosing a differential form 
    which is supported near the image of a generic pseudo-cycle $f\colon V\rightarrow M^p$ 
    which represents the Poincar\'e dual $a=\mathit{PD}(\alpha)$. 
    If $f$ is transverse to the evaluation map then the pull-back has compact support.
\end{remark}

In this sense, it is more similar to the definition in algebraic settings
\[\langle \phi_1,\cdots,\phi_p \rangle_{g,p,\beta}=\int_{\left[\overline{M}_{g,p}(X,\beta)\right]^{\mathrm{vir}}} \phi_1\cup\cdots\phi_p. \]

\section{Examples and Applications}

Let's look at some examples:

\begin{example}
    Let $(M,\omega)$ be a compact symplectic 4-manifold and $A$ be a spherical homology class 
    such that $c_1(A)=1$. Then if $a_p\in H_2(M)$, then
    \[\Phi_{A,p}(\alpha_1,\cdots,\alpha_p)=\Phi_{A,p-1}(\alpha_1,\cdots,\alpha_{p-1})\cdot \int_A \alpha_p.\]
    It can be easily seen through the definition using product cycles.
\end{example}

\begin{example}
    Let $(M,\omega)$ be a compact weakly monotone manifold and $A\neq 0$. Then
    \[\Phi_{A,p}(\alpha_1,\cdots,\alpha_{p-1},[M])=0.\]
\end{example}

Before starting the third example, we introduce a conclusion, 
in \cite{grothendieck} Grothendieck proved that 
any holomorphic bundle $E$ over $S^2=\cp{1}$ 
is holomorphically equivalent to a sum of holomorphic line bundles. 
And the splitting is unique up to the order of the summands. 
And we have the following lemma.

\begin{lemma}
    Assume $J$ is integrable and $u\colon \cp{1}\rightarrow M$ is a $J$-holomorphic curve. 
    Suppose that every summand of $u^*TM$ has Chern number $\geq -1$, 
    then the linearized Cauchy--Riemann operator $D_u$ is onto.
\end{lemma}

\begin{example}
    In $\cp{n}$, two points lie on a unique line. So this should be translated into
    \[\Phi_L(\pt,\pt)=1.\]
    Let us calculate it in our previous definition.
    
    In this case, $L=[\cp{1}]\in H_2(\cp{n},\mathbb{Z})$ and $\pt\in H_0(\cp{n},\mathbb{Z})$. 
    And the dimensional condition gives
    \[d=2(n-1)+4-2(n+1)=0\]
    from the above lemma, we can consider the standard complex structure $J_0$, which is integrable.
    
    Since $\cp{1}\subset\cp{2}\cdots\subset\cp{n}$, 
    the normal bundle to a complex line is a sum of line bundles each of Chern number 1. 
    Thus $J_0$ satisfies the regularity requirement needed for the definition for $\Phi_L$. 
    Since all the evaluation maps 
    \[e_D\colon \mathscr{M}(A_1,\cdots,A_N)\times(\cp{1})^{2N-2}\rightarrow M^{2N-2}\]
    are transverse to the diagonal $\Delta_N\subset M^{2N-2}$.
    Thus $\Phi_L(\pt,\pt)=1$.
\end{example}

We have the following lemma.

\begin{lemma}
    Let $\widetilde{M}$ be the product of $\cp{1}$ with a symplectic manifold 
    $(M,\omega)$ and $\widetilde{A}\in H_2(\widetilde{M},\mathbb{Z})$ 
    be the homology class represented by the spheres $S^2\times\pt  $. 
    Then for every $J\in\mathscr{J}(M,\omega)$, 
    the product almost complex structure $\widetilde{J}:=i\times J$ is regular for $\widetilde{A}$.
\end{lemma}

\begin{example}
    Consider the manifold $M=\cp{1}\times V$ where $\pi_2(V)=0$, 
    and consider the homology class $A=\left[\cp{1}\times \pt\right]\in H_2(M)$, then
    \[\Phi_{A}(\pt)=1.\]
\end{example}

\begin{proof}
    Since $c_1(A)=2$, thus $d=0$, so the dimensional condition is satisfied.  
    And $(M,\omega)$ is weakly monotone since $\pi_2(V)=0$. 
    Then we calculate the dimensions
    \[\dim\mathscr{W}(A,J)=2(n+2)=\dim M.\]
    Choose the almost complex structure $i\times J$, then notice that $u[\cp{1}]=[A]$, 
    so the evaluation map is surjective and thus a diffeomorphism. So $\Phi_{A}(\pt)=1$.
\end{proof} 

As a consequence of the fact that the evaluation map is of degree 1, 
Gromov used this fact in \cite{gromov} in 1985 to prove the famous non-squeezing theorem, 
which implies the rigidity of symplectic manifolds. We will give a sketch of proof here. 

\begin{lemma}
    Every non-parametrized $J$-holomorphic curve is absolutely area minimizing in its homology class.
\end{lemma}

\begin{theorem}
    If $\psi$ is a symplectic embedding of the ball $B^{2n}(r)$ into a cylinder $B^2(\lambda)\times V$ where $\pi_2(V)=0$, then $r\leq \lambda$.
\end{theorem}

\begin{proof}
    Embed the disc $B^2(\lambda)$ into a 2-sphere $\cp{1}$ of area $\pi\lambda^2+\epsilon$, 
    and let $\omega$ be the product symplectic structure on $\cp{1}\times V$. 
    Let $J'$ be an $\omega$-tame almost complex structure on $M$ which on the image of $\psi$, 
    equals to the push-forward of standard structure $J_0$ by $\psi$ of the ball $B^{2n}(r)$.
    
    Since the evaluation map which is independent of the choice of 
    almost complex structure and has degree 1, 
    there are $J'$-holomorphic curves through every point of $M$, 
    and in particular, one curve through $\psi(0)$. 
    This curve pulls back by $\psi$ to a $J_0$-holomorphic curve $C$ through 
    the center of the ball $B^{2n}(r)$. Since $J_0$ is standard, 
    this curve is holomorphic in the usual sense and is a minimal surface in $B^{2n}(r)$. 
    And from the result in minimal surface, the area is at least $\pi r^2$, thus
    \[\pi r^2\leq\int_C \omega_0=\int_{\psi^{-1}(C')}\psi^*\omega<\int_{C'}\omega=\omega(A)\pi\lambda^2+\epsilon,\]
    and we are done.
\end{proof}

\section{The Invariant \texorpdfstring{$\Psi$}{Ψ}}

As in the definition of $\Phi$, roughly speaking, this invariant $\Psi$ counts the number
\[\Psi_{A,p}(\alpha_1,\cdots,\alpha_p)=\#\{ u\in\maj\mid u(z_j)\in\alpha_j  \}.\]
The dimensional condition of $d$ will guarantee that this is indeed a finite set. 
And we will give the precise definition below.

As in the previous case, we consider the pseudo-cycle represented by $e_{A,J,\z}$ 
and it determines a homomorphism
\[\Psi_{A,J,\z}\colon H_d(M^p,\mathbb{Z})\rightarrow \mathbb{Z},\]
where
\[d=2n(p-1)-2c_1(A).\]
Represent a homology class $\alpha=\alpha_1\times\cdots\times\alpha_p\in H_d(M,\mathbb{Z})$ by 
\[f=f_1\times\cdots f_p\colon V_1\times\cdots\times V_p\rightarrow M^p.\]
Put this in general position so that it becomes transverse to the maps $e_{D,T,\z}$.
So we may define
\[\Psi_{A,J,\z}(\alpha_1,\cdots,\alpha_p)=e_{A,J,\z}\cdot f.\]
By the previous lemmas, this depends only on the bordism class of $f$ 
and hence only the homology class of $\alpha$, 
and depends only on the bordism class of the evaluation map $e_{A,J,\z}$.

Again, we mention some conditions that need to be added:

\begin{enumerate}
    \item If $A=mB$ is a nontrivial multiple of a homology class $B$ 
    with $m>1$ then either $c_1(B)\geq 3$ or $p\leq 2m$.
    \item For a generic almost complex structure $J\in\mathscr{J}(M,\omega)$, 
    every $J$-effective homology class $A\in H_2(M)$ has Chern number $c_1(A)\geq 2$.
\end{enumerate}

In this case, we have the following proposition.

\begin{proposition}
    Assume the above two conditions, the homomorphism
    \[\Psi_{A,p}=\Psi_{A,J,\z}\colon H_d(M^p,\mathbb{Z})\rightarrow \mathbb{Z}\]
    is independent of the choice of the regular $\omega$-tame almost complex structure 
    $J\in \mathscr{J}_{\mathrm{reg}}(M,\omega)$ and the $p$-tuple $\z\in (\cp{1})^p$ used to define it. 
    Hence it only depends on the deformation class of $\omega$.
\end{proposition}

\begin{proof}
    The proof is similar to the proof of Proposition \ref{prop-4-2}, 
    which only need to consider the regular homotopy $\{J_\lambda\}$ and $\{\z_\lambda\}$, 
    so that we need to consider the bordism manifold
    \[\mathscr{V}(D,T,\{J_\lambda\},\{\z_\lambda\})=\bigcup_{\lambda}{}\{\lambda\}\times\mathscr{V}(D,T,J_\lambda,\z_\lambda),\]
    and we need to show that this is a smooth manifold.
    
    Then following the similar argument we can show that the evaluation maps
    \[e_{A,J_0,\z_0}\colon \mathscr{M}(A,J_0)\rightarrow M^p, \qquad e_{A,J_1,\z_1}\colon \mathscr{M}(A,J_1)\rightarrow M^p\]
    determine bordant pseudo-cycles.
\end{proof}

In fact, this invariant can also be understood as integration over a moduli space, 
represent the cohomology classes $a_j=\mathit{PD}(\alpha_j)\in H^{2n-d_j}(M)$ 
by closed forms $\tau_j\in \Omega^{2n-d_j}(M)$ and define
\[\Psi_{A,J,\z}(\alpha_1,\cdots,\alpha_p)=\int_{\maj}e_1^*\tau_1\wedge\cdots e_p^*\tau_p,\]
where $e_j\colon \maj\rightarrow M$ denotes the evaluation map $e_j(u)=u(z_j)$. 
The dimension condition means that
\[\sum_{j=1}^p\deg \tau_j=2n+2c_1(A)=\dim\maj.\]
The difficulties have been mentioned in the case of $\Phi$.

Also, we can compare this to the definition in algebraic setting:
\[\langle \phi_1,\cdots,\phi_p \rangle'_{g,p,\beta}=\int_{\left[\overline{M}_{g,p}(X,\beta)\right]^{\mathrm{vir}}} \prod e_k^*\phi_k.\]

Actually, both $\Phi$ and $\Psi$ can be viewed as 
special cases of the Gromov--Witten classes as formulated by Kontsevich and Manin in \cite{kontsevich-manin}.
More precisely, let $\Sigma$ be an oriented Riemann surface of genus $g$, 
and denote $\mathscr{J}(\Sigma)$ the space of complex structures on $\Sigma$ 
which are compatible with the given orientation. 
A $(k+1)$-tuple $(j,z_1,\cdots,z_k)\subset \mathscr{J}(\Sigma)\times\Sigma^k$ is called \textbf{stable} 
if the only diffeomorphism $\phi\in$Diff$(\Sigma)$ which satisfies $\phi^*j=j$ 
and $\phi(z_i)=z_i$ is the identity map and denote 
$\mathscr{C}_{g,k}\subset\mathscr{J}(\Sigma)\times\Sigma^k $ such stable tuples.

Then consider the quotient space
\[\mathscr{M}_{g,k}=\frac{\mathscr{C}_{g,k}}{\mathrm{Diff}(\Sigma)},\]
which is a manifold of dimension $6g-6+2k$. Then consider the space
$\mathscr{C}_{g,k}(M,A)$ of all $(k+2)$-tuples $(u,j,z)$ where 
$(j,z)\in \mathscr{C}_{g,k}$ and $u\colon \Sigma\rightarrow M$ is a simple 
$(j,J)$-holomorphic curve representing the class $A$. 
The group Diff$(\Sigma)$ acts on this space by
\[\phi\cdot (u,j,z)=(u\circ\phi,\phi^*j,\phi^{-1}(z_1),\cdots,\phi^{-1}(z_p),\]
and, for a generic $J\in\mathscr{J}(M,\omega)$, the quotient space
\[\mathscr{M}_{g,k}(M,A)=\frac{\mathscr{C}_{g,k}(M,A)}{\mathrm{Diff}(\Sigma)}\]
is a manifold of dimension $(n-3)(2-2g)+2c_1(A)+2k$, which give rise to a linear map
\[\mathit{GW}_{A,g,k}\colon H^*(M^k)\rightarrow H^*(\mathscr{M}_{g,k}),\]
defined, heuristically by
\[\mathit{GW}_{A,g,k}(a_1,\cdots,a_k)=\mathit{PD}(\pi_*\mathit{PD}(e_1^*a_1\wedge\cdots e_k^*a_k))\]
for $a_i\in H^*(M)$ where $e_i$ is the evaluation map and $\pi\colon \mathscr{M}_{g,k}(M,A)\rightarrow \mathscr{M}_{g,k}$ is the projection. 
Note that 
\[\deg \mathit{GW}_{A,g,k}(a_1,\cdots,a_k)=n(2-2g)+2c_1(A)-\sum_{i=1}^{k}\deg a_i.\]
The invariant $\Psi$ corresponds to the case where this degree is zero and is given by 
evaluating the cohomology class at a point. 
And the invariant $\Phi$ correspond to the case where the degree is the dimension of $\mathscr{M}_{g,k}$ 
and is given by evaluating the cohomology class $\mathit{GW}_{A,g,k}(a_1,\cdots,a_k)$ 
on the fundamental cycle $\left[ \mathscr{M}_{g,k} \right]$.

\printbibliography

\end{document}