This section will provide preliminary
results that will be needed in the proofs of our main theorems.
The reader is suggested to skip to \S\ref{sec-proofs}
to read the proofs there,
and come back for the results when they are used.

\begin{lemma}\label{lem:compact-diffeo}
Let $M,N$ be boundaryless manifolds, and $K\subset M$ a compact subset.
If $f\:M\to N$, and $f|_K$ is an injective local diffeomorphism,
then there exists a neighbourhood $U$ of $K$ in $M$,
such that $f|_U$ is a diffeomorphism onto its image.
\end{lemma}

\begin{proof}
This is a standard exercise in mathematical analysis.
\end{proof}

\begin{lemma}\label{lem:diffeo}
Let $M$ be a Riemannian $n$-manifold without boundary,
and let $S$ be a compact $k$-submanifold without boundary.
Let $N(S)$ denote the normal bundle of $S$ in $M$,
and let $N_\epsilon(S)$ denote the $\epsilon$-neighbourhood of $S$ in $N(S)$.
There exists $\epsilon>0$ such that the map
\[ N_\epsilon(S)\ni(p,v)\mapsto \exp_p(v) \]
is a diffeomorphism of $N_\epsilon(S)$ onto its image.
\end{lemma}

\begin{proof}
By (\ref{lem:compact-diffeo}) and the inverse function theorem. 
\end{proof}

\begin{lemma}\label{lem:whitney-extension}
Let $M,N$ be manifolds with $\partial N=\emptyset$,
and let $A\subset M$ be a closed subset.
A smooth map $f\:A\to N$ can be extended to $M$,
if and only if it can be continuously extended to $M$.

This is also true if $\partial N\neq\emptyset$ and $f(A)\subset\partial N$.
In this case, if the continuous extension sends $M\setminus A$ into $N\setminus\partial N$,
then we may require the smooth extension to have the same property.
\end{lemma}

\begin{proof}
\cite[Corollary~6.27]{lee}. The last statement follows from
the construction in the proof of \cite[Theorem~6.21]{lee}.
\end{proof}

The next extension lemma will reduce our pain
dealing with boundaries.

\begin{lemma}\label{lem:extension}
Let $M\subset\widetilde M$ be $n$-manifolds,
$\partial\widetilde M=\emptyset$.
For every closed submanifold $S\subset M$,
there exists a submanifold $\widetilde S\subset\widetilde M$,
such that $\partial\widetilde S=\emptyset$,
$\dim S=\dim\widetilde S$, and $S\subset\widetilde S$.
If moreover $\partial S\subset\partial M$,
then we may require $S=\widetilde S\cap M$.
\end{lemma}

\begin{proof}
Let $S^+:=S\cup\partial S\times[0,+\infty)$, with their boundaries identified in the obvious way.
Choose a complete metric on $S^+$.
By (\ref{lem:whitney-extension}), the inclusion $S\hookrightarrow\widetilde M$
can be extended to a smooth map $f\:S^+\to\widetilde M$.

We embed $\widetilde M$ in some $\R^N$ as a closed submanifold. 
Let $B_r$ denote the open ball in $\R^N$ centred at $0$ and of radius $r$, or $\emptyset$ if $r\leq0$.
For each $i\in\N$, denote $K_i:=(\oline2B_i\setminus B_{i-1})\cap S$,
and $U_i:=(B_{i+1}\setminus \oline2B_{i-2})\cap\widetilde M$.
By (\ref{lem:compact-diffeo}), for each $i$ there exists $\epsilon_i>0$
so that $f$ is a diffeomorphism of the
$\epsilon_i$-neighbourhood of $K_{i-2}\cup\cdots\cup K_{i+2}$ in $S^+$ onto its image
(where $K_{-1}:=K_0:=\emptyset$).
We may assume that the images of the $\epsilon_i$-neighbourhoods of $K_{i-2},\dotsc,K_{i+2}$
are contained in $U_{i-2},\dotsc,U_{i+2}$ respectively.
Let $\subsup\epsilon i\prime:=\min(\epsilon_{i-2},\dotsc,\epsilon_{i+2})$
(where $\epsilon_{-1}:=\epsilon_0:=\epsilon_1$),
and let $\widetilde S:=\bigcup_i(\text{$\subsup\epsilon i\prime$-neighbourhood of $K_i$})\subset S^+$.
Then $f|_{\widetilde S}$ is an injective local diffeomorphism,
and thus a diffeomorphism onto its image.

For the last statement, it suffices to show that $f$ can be chosen so that $f^{-1}(M)=S$.
By the last statement of (\ref{lem:whitney-extension}),
we only need a continuous map $g\:\partial S\times[0,+\infty)\to\widetilde M$
such that $g|_{\partial S}$ is the inclusion and $g^{-1}(\partial S)=\partial S\times0$.
This can be constructed as follows.
Let $\epsilon\:\partial S\to\R_{>0}$ be a continuous function
such that for all $p\in\partial S$,
the exponential map $\exp_p(v)$ is defined for $|v|\leq\epsilon(p)$.
Now $g$ may be defined to be the map $(p,r)\mapsto\exp_p(\min(r,\epsilon(p))\,\eta_p)$,
where $\eta_p$ is the unit normal vector of $\partial M$ in $\T_p\widetilde M$,
pointing outwards from $M$.
\end{proof}

\begin{remark}
The same proof applies if $M$ and $S$ have corners,
with a modified construction of $S^+$.
The last statement remains true
if the corners of $M$ are convex. \varqed
\end{remark}

\begin{definition}
Let $M$ be a manifold, and let $S,T$ be submanifolds.
We say that $S$ and $T$ \term{intersect transversely},
or that $S$ is \term{transverse to} $T$,
if for all $p\in S\cap T$, we have $\T_pM=\T_pS+\T_pT$.
Note that $S,T$ does not necessarily intersect, and they are not necessarily boundaryless.
\end{definition}

\begin{proposition} \label{thm-transverse}
If $S$ is transverse to $T$,
then $S\cap T$ is a submanifold of $M$.
\qed
\end{proposition}

\begin{definition}
Let $S_1,\dotsc,S_k\subset M$ be submanifolds of $M$.
The set $\{S_1,\dotsc,S_k\}$ is said to be \term{transverse},
if when $k>0$, $\{S_1,\dotsc,S_{k-1}\}$ is transverse \emph{(defined inductively on $k$)},
and $S_k$ is transverse to everything in
$\bigl\{\,\bigcap_{i\in I}S_i\mid I\subset\{1,\dotsc,k-1\}\,\bigr\}$.
In this case, every possible intersection of the manifolds $S_i$ is a submanifold.
\end{definition}

The next lemma says that 
transverse submanifolds can be made orthogonal to each other.

\begin{lemma}\label{lem:orthogonal}
Let $S_1,\dotsc,S_k\subset M$ be transverse closed submanifolds.
Then there exists a Riemannian metric on $M$,
such that whenever $p\in S_i\cap S_j$,
the subspaces $\T_pS_i$ and $\T_pS_j$ of $\T_pM$ are orthogonal.
\end{lemma}

\begin{proof}
By (\ref{lem:extension}), we may assume that
$M$ and the $S_i$ are boundaryless.

For every $p\in M$, choose a neighbourhood $U_p$ that
does not intersect with any $S_i$ not containing $p$,
such that $U_p$ is diffeomorphic to $\R^n$, and under this diffeomorphism,
the $S_i$ containing $p$ (if any) are identified with some coordinate subspaces of $\R^n$.
This is possible since we may construct
smooth functions $f_1,\dotsc,f_n$ near $p$,
such that each of these $S_i$ is locally cut out by the zeros of some of these $f_j$
(each $f_j$ is used by at most one $S_i$),
and such that $\d f_1,\dotsc,\d f_n$ form a basis for $\Tstar pM$.
Using these $f_j$ as local coordinate functions,
we get a desired diffeomorphism by the inverse function theorem.

Let $g_p$ denote the metric on $U_p$ induced from $\R^n$.
Let $\{\rho_p\}$ be a partition of unity subordinate to the open cover $\{U_p\}$.
Then $\sum_p\rho_pg_p$ is a desired metric.
\end{proof}

One important result in intersection theory is the transversality theorem \cite[p.\,68]{gp}.
Here is one of its important corollaries.

\begin{lemma}\label{cor:transverse}
Let $M$ be a boundaryless $n$-manifold,
and let $S_1,\dotsc,S_k,T$ be submanifolds,
where $\partial T=\emptyset$, but the $S_i$ may have corners.
For any neighbourhood $U$ of $T$,
$T$ can be moved by an isotopy within $U$,
to become transverse to all $S_i$.

Moreover, the isotopy can be taken to be fixed on any closed subset of $T$
where $T$ is already transverse to all the $S_i$.
\end{lemma}

\begin{proof}
By (\ref{lem:extension}), we may assume that the $S_i$ are boundaryless.
See \cite[p.\,70]{gp} for a proof in this case.
\end{proof}

The next results are concerned with the existence of
functions with certain properties.
The gradient vector fields of these functions will be very useful.
We will use the flow of these vector fields on the boundary of a handlebody,
to manipulate the handles attached on it.

\begin{lemma}\label{lem:function}
Let $M$ be a closed $n$-manifold,
and let $S_1,\dotsc,S_k\subset M$ be transverse compact submanifolds without boundary.
%such that $\partial S_i\subset S_0\cup\cdots\cup S_{i-1}$ for all $i$,
%Let $T$ be another closed $(n-1)$-submanifold,
%which intersects each $S_i$ transversely.
Then there exists a smooth function $f\:M\to[0,1]$, such that
\begin{enum}
\item $f^{-1}(0)=S:=S_1\cup\cdots\cup S_k$.
\item $f$ has no critical values other than $0,1$.
%\item For every $t\in(0,\delta)$, $f^{-1}(t)$ and $T$ intersect transversely.
\end{enum}
\end{lemma}

\begin{proof}
By (\ref{lem:orthogonal}), there exists a Riemannian metric on $M$,
such that any two of $S_1,\dotsc,S_k$ intersect orthogonally.

We claim that for small enough $\epsilon>0$ and $i=1,\dotsc,k$,
there exists a smooth function $f_{i,\epsilon}\:M\to[0,1]$, such that
\begin{enum}
\item $\subsup f{i,\epsilon}{-1}(0)=S_i$.
\item Whenever $f_{i,\epsilon}(p)\in(0,1)$, we have
$\grad f_{i,\epsilon}(p)\neq0$,
and $\grad f_{i,\epsilon}$ points orthogonally outwards from $S_i$,
in the sense that its flow line through $p$
is the unique minimising geodesic from $p$ to $S_i$.
\end{enum}
Namely, by (\ref{lem:diffeo}), we may define a function $d_{i,\epsilon}$ to be $1/\epsilon$ times the distance to $S_i$,
and let $f_{i,\epsilon}=\rho\circ d_{i,\epsilon}$, where
$\rho\:[0,+\infty)\to[0,1]$ is a smooth function, such that $\rho(r)=1$ when $r\geq1$,
$\rho(0)=\rho'(0)=\cdots=0$, and $\rho'(r)>0$ if $0<r<1$.

Denote $U_{i,\epsilon}=\subsup f{i,\epsilon}{-1}([0,1))$.
Now we take $f=f_{1,\epsilon}\cdots f_{k,\epsilon}$,
where we choose $\delta>\epsilon>0$ small enough, so that
\begin{enum}
\item $f_{i,2\delta}$ is defined for all $i$.
\item For every path of length $\ell$ in $U_{i,2\delta}$,
its projection to $S_i$ has length $<2\ell$.
\item Whenever $p\in\bigcap_{i\in I}U_{i,\epsilon}$ for some $I\subset\{1,\dotsc,k\}$,
there exists a path from $p$ to $\bigcap_{i\in I}S_i$ with length $<\delta$.
\item Whenever a vector is parallel transported along a loop of length $<4\delta$,
its direction changes by an angle $<\uppi/8$.
\item Whenever a vector is parallel transported along a curve of length $<2\delta$ on $S_i$,
the angle between the vector and $S_i$ changes by $<\uppi/8$.
\end{enum}
(iii) is possible, since if no such $\epsilon$ exists, we would have a contradiction
against sequential compactness of $S$.
(iv) is possible since the Riemann curvature tensor of $M$ is bounded.
(v) is possible since the second fundamental forms of the $S_i$ are bounded.

We claim that for any $p$ that is $\epsilon$-near $S$, $p$ is not a critical point of $f$.
(For simplicity, two things are \term{$\epsilon$-near} if their distance is $<\epsilon$.)
Indeed, suppose $p$ is $\epsilon$-near $S_i$ for $i\in I$,
while not $\epsilon$-near the others.
Denote $S_I:=\bigcap_{i\in I}S_i$ and $U_{I,\epsilon}:=\bigcap_{i\in I}U_{i,\epsilon}$.
Note that 
\[\grad f(p)=\sum_{i\in I}\frac{f(p)}{f_{i,\epsilon}(p)}\grad f_{i,\epsilon}(p).\]


Let $p_i$ denote the projection of $p$ to $S_i$,
and let $q$ be a point in $S_I$ that is $\delta$-near $p$.
Let $\gamma$ be a path from $p$ to $q$ of length $<\delta$ (thus it falls in $U_{I,2\delta}$),
and let $\gamma_i$ be the projection of $\gamma$ to $S_i$,
which is a path from $p_i$ to $q$ of length $<2\delta$.
Let $v_i\in\T_{p_i^{}}M$ be the vector pointing to $p$, of length $|\grad f_{i,\epsilon}(p)|$.
Let $w_i\in\T_qM$ denote its parallel transport along $\gamma_i$.
Then the angle between $w_i$ and $\T_qS_i$ is $>3\uppi/8$.
Note that $w_i$ is obtained from $\grad f_{i,\epsilon}(p)\in\T_pM$
by parallel translation along the path $p$---$p_i$---$q$.
If we instead translate it directly along $\gamma$,
the resulting vector, say $u_i$,
would have a direction that differ from $w_i$ by $<\uppi/8$.
Thus the angle between $u_i$ and $\T_pS_i$ is $>\uppi/4$.
But the $S_i$ are transverse and orthogonal at $p$,
hence by a standard argument in linear algebra, the $u_i$ are linearly independent,
and so are the vectors $\grad f_{i,\epsilon}(p)$.
This implies that $\grad f(p)\neq0$. 
\end{proof}

\begin{remark}
This is not true for an arbitrary closed subset $S$.
For example, consider $S=$ Cantor set $\subset M=S^1$. \varqed
\end{remark}

We will need a slightly more general form of this lemma,
which follows from exactly the same construction
(involving distance functions to submanifolds with boundary,
which are not necessarily smooth near the boundary).

\begin{corollary}\label{cor:function}
Let $M$ be a closed $n$-manifold,
and let $S_1,\dotsc,S_k\subset M$ be transverse compact submanifolds,
possibly with boundary.
Let $U$ denote an open neighbourhood of $\partial S_1\cup\cdots\cup\partial S_k$.
Then there exists a smooth function $f\:M\setminus U\to[0,1]$, such that
\begin{enum}
\item $f^{-1}(0)=(S_1\cup\cdots\cup S_k)\setminus U$.
\item $f$ has no critical values other than $0,1$. \qed
\end{enum}
\end{corollary}

Finally, we mention two results concerning isotopies.

\begin{lemma}[Isotopy extension theorem]\label{cor:isotopy-ext}
If $\partial M=\emptyset$ and $S$ is compact,
then every compactly supported isotopy $h\:S\times\R\to M$
can be extended to a compactly supported diffeotopy on $M$.
\end{lemma}

\begin{proof}
    \cite[Theorem~8.1.3]{hirsch}.
\end{proof}

\begin{lemma}\label{prop:move-disk}
Let $M$ be a connected $n$-manifold.
If $M$ is orientable,
then two embeddings $D^n\hookrightarrow M$ are isotopic
if and only if they have the same orientation.
If $M$ is non-orientable, then all such embeddings are isotopic.
\end{lemma}

\begin{proof}
We first consider the special case $M=\R^n$.
We may assume the embedding $f\:D^n\to\R^n$ sends $0$ to $0$,
preserving orientation.
Using a smooth path in $\mathrm{GL}(n)$, i.e.\ a smooth family of linear transformations,
we may assume $\d f(0)=\id$.
We further shrink the disk linearly (w.r.t.\ the disk) by an isotopy,
and then enlarge it linearly (w.r.t.\ $\R^n$) to recover $\d f(0)=\id$,
so that if we denote $g(x):=f(x)-x$,
then the norm of $\d g(x)$ (supremum of $|\d g(x)v|/|v|$ for $v\neq0$) is $<1/2$ for all $x\in D^n$.
This will ensure that the linear homotopy $h_t$ from $f$ to the standard inclusion $D^n\hookrightarrow\R^n$
is injective for all $t$, and thus it is an isotopy.

For the general case, we may shrink $D^n$ into a coordinate chart,
and then transport it along a finite sequence of coordinate charts.
Doing this for both disks, we are reduced to the first case.
\end{proof}
