\usepackage{indentfirst}
\parindent=2em
\linespread{1.25}

% names in chinese
\def\abstractname{摘\quad 要}
\def\contentsname{目录}
\def\proofname{证明}

% bibliography
\DefineBibliographyStrings{english}{
    references = {参考文献},
}

\usepackage{setspace}
\AtBeginBibliography{
    \spacing{1}
}

% lengths
\AtBeginDocument{
    \addtolength{\abovedisplayskip}{-3pt}
    \addtolength{\belowdisplayskip}{-3pt}
    \addtolength{\belowdisplayshortskip}{-3pt}
}

% theorems & proofs
\newtheoremstyle{cjk-theorem}{}{}{\KaiTi}{}{\bfseries}{.}{.5em}{}
\newtheoremstyle{cjk-definition}{}{}{}{}{\bfseries}{.}{.5em}{}
\newtheoremstyle{cjk-remark}{}{}{}{}{\KaiTi}{.}{.5em}{}

\theoremstyle{cjk-theorem}
\renewtheorem{theorem}{定理}[section]
\renewtheorem{lemma}[theorem]{引理}
\renewtheorem{corollary}[theorem]{推论}
\renewtheorem{proposition}[theorem]{命题}
\renewtheorem{definition}[theorem]{定义}

\theoremstyle{cjk-definition}
\renewtheorem{remark}[theorem]{注}
\renewtheorem{example}[theorem]{例}
\theoremstyle{cjk-theorem}

\makeatletter
\renewenvironment{proof}[1][\proofname]{\par
    \pushQED{\qed}%
    \normalfont \topsep6\p@\@plus6\p@\relax
    \trivlist
    \item\relax{\bfseries#1}\hspace{1em}\ignorespaces
}{\popQED\endtrivlist\@endpefalse}
\makeatother
