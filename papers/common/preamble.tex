\usepackage{geometry}
\geometry{paperwidth=155mm,paperheight=235mm,outer=20mm,inner=20mm,top=25mm,bottom=20mm}

% fonts & unicode
\usepackage[PunctStyle=kaiming]{xeCJK}
\usepackage{amsmath}
\usepackage{unicode-math}
\setCJKmainfont[Path=../../fonts/,BoldFont=fzhtk.ttf,ItalicFont=fzktk.ttf]{fzssk.ttf}
\setCJKsansfont[Path=../../fonts/,BoldFont=SourceHanSansCN-Bold.otf,Scale=.97]{SourceHanSansCN-Normal.otf}
\setCJKmonofont[Path=../../fonts/,BoldFont=SourceHanSansCN-Bold.otf,Scale=.9]{SourceHanSansCN-Normal.otf}
\newCJKfontfamily{\KaiTi}[Path=../../fonts/,BoldFont=fzhtk.ttf,ItalicFont=fzssk.ttf]{fzktk.ttf}
\setmainfont[Path=../../fonts/,Extension=.otf,UprightFont=*-Regular,BoldFont=*-Bold,ItalicFont=*-Italic,BoldItalicFont=*-BoldItalic]{STIX2Text}
\setsansfont[Path=../../fonts/,Scale=MatchUppercase,Extension=.ttf,UprightFont=*-Regular,BoldFont=*-Bold,ItalicFont=*-Italic,BoldItalicFont=*-BoldItalic]{Lato}
\setmonofont[Path=../../fonts/,Scale=.9,Extension=.otf,UprightFont=*-Regular,BoldFont=*-Bold]{FiraMono}
\setmathfont[Path=../../fonts/,BoldFont=STIX2Math-Bold.otf]{STIX2Math.otf}
\setmathfont[Path=../../fonts/,Scale=.95,BoldFont=LatoMath.otf,version=sf]{LatoMath.otf}
\setmathfont[Path=../../fonts/,Scale=.95,BoldFont=LatoMath.otf,range={bb,sfup->up,sfit->it,bfsfup->bfup,bfsfit->bfit}]{LatoMath.otf}

\Umathcode`/ = "0 "0 "2215    % / -> U+2215 division slash

% patch 'text math' math alphabets in bold math
\setmathfontface\mathrm[Path=../../fonts/,version=bold]{STIX2Text-Bold.otf}
\setmathfontface\mathit[Path=../../fonts/,version=bold]{STIX2Text-BoldItalic.otf}
\setmathfontface\mathbf[Path=../../fonts/,version=bold]{STIX2Text-Bold.otf}
\setmathfontface\mathtt[Path=../../fonts/,Scale=.9,version=bold]{FiraMono-Bold.otf}

% bold math in bold text https://tex.stackexchange.com/q/41379
\makeatletter
\g@addto@macro\bfseries{\boldmath} 
\makeatother

% title & abstract
\def\title#1\author#2{\headertitle{#1}\vspace*{0mm}\begin{center}{\sf\LARGE#1\par}\vspace{10mm}{\large#2}\end{center}\vspace{10mm}}
\def\headertitle#1{\def\theheadertitle{#1}}

\def\abstractname{ABSTRACT}
\makeatletter
\let\endabstract@orig\endabstract
\def\endabstract{\endabstract@orig\vspace{5mm}}
\makeatother

% headers and footers
\usepackage{fancyhdr}
\fancyhf{}
\fancyhead[CE]{\sf\mathversion{sf}\theheadertitle}
\fancyhead[CO]{\sf\mathversion{sf}\nouppercase{\leftmark}}
\fancyhead[LE,RO]{\textbf{\textsf{\thepage}}}
\headsep=8mm
\headheight=6mm

\AtBeginDocument{
    \pagestyle{fancy}\thispagestyle{empty}
}

% spacing
\AtBeginDocument{
    \hfuzz=2pt
    \emergencystretch 2em
    \setlength{\belowdisplayshortskip}{\belowdisplayskip}
}

% environments
\usepackage{amsthm}
\newtheorem{theorem}{Theorem}[section]
\newtheorem{lemma}[theorem]{Lemma}
\newtheorem{corollary}[theorem]{Corollary}
\newtheorem{proposition}[theorem]{Proposition}
\newtheorem{definition}[theorem]{Definition}

\theoremstyle{definition}
\newtheorem{example}[theorem]{Example}
\newtheorem{remark}[theorem]{Remark}
\theoremstyle{plain}

\def\qedsymbol{$◻$}
\def\thmqedhere{\pushQED{\qed}\qedhere\popQED}

\numberwithin{equation}{theorem}

% renew theorem: https://tex.stackexchange.com/q/103013/
\makeatletter
\def\renewtheorem#1{%
  \expandafter\let\csname#1\endcsname\relax
  \expandafter\let\csname c@#1\endcsname\relax
  \gdef\renewtheorem@envname{#1}
  \renewtheorem@secpar
}
\def\renewtheorem@secpar{\@ifnextchar[{\renewtheorem@numberedlike}{\renewtheorem@nonumberedlike}}
\def\renewtheorem@numberedlike[#1]#2{\newtheorem{\renewtheorem@envname}[#1]{#2}}
\def\renewtheorem@nonumberedlike#1{  
\def\renewtheorem@caption{#1}
\edef\renewtheorem@nowithin{\noexpand\newtheorem{\renewtheorem@envname}{\renewtheorem@caption}}
\renewtheorem@thirdpar
}
\def\renewtheorem@thirdpar{\@ifnextchar[{\renewtheorem@within}{\renewtheorem@nowithin}}
\def\renewtheorem@within[#1]{\renewtheorem@nowithin[#1]}
\makeatother

% ref & biblatex
\usepackage[colorlinks,allcolors=black,bookmarksnumbered,linktoc=all]{hyperref}

\def\thesection{\arabic{section}\texorpdfstring{}{.}} % pdf bookmark numbering
\setcounter{secnumdepth}{1} % suppress subsection numbering

\usepackage[style=alphabetic,sorting=anyvt,useprefix=true]{biblatex}
\usepackage{xpatch}
\renewcommand*{\bibfont}{\small}
\DeclareFieldFormat{labelalpha}{\mkbibbold{#1}}
\DeclareFieldFormat{extraalpha}{\mkbibbold{\mknumalph{#1}}}
\DeclareFieldFormat[article]{volume}{\mkbibbold{#1}}
\DeclareFieldFormat[book,inbook]{number}{\mkbibbold{#1}}
\DeclareFieldFormat[article]{number}{(#1)}
\DeclareFieldFormat*{year}{(#1)}
\DeclareFieldFormat{pages}{#1}
\renewbibmacro{in:}{}
\renewbibmacro*{volume+number+eid}{%
  \printfield{volume}%
  \setunit*{\addnbspace}% originally: \setunit*{\adddot}
  \printfield{number}%
  \setunit{\addcomma\space}%
  \printfield{eid}%
}
\xapptobibmacro{author/editor+others/translator+others}{%
  \setunit{\space}%
  \printfield{year}%
  \clearfield{year}%
}{}{}
\xapptobibmacro{author/translator+others}{%
  \setunit{\space}%
  \printfield{year}%
  \clearfield{year}%
}{}{}
\renewbibmacro*{issue+date}{%
  \printfield{issue}%
  \newunit%
}
\AtBeginBibliography{
  \DeclareFieldFormat{labelalpha}{#1}
  \DeclareFieldFormat{extraalpha}{\mknumalph{#1}}
}
\AtEveryBibitem{
  \ifentrytype{online}{%
    \clearfield{year}%
  }{}
}

% tikz
\usepackage{tikz}
\usepackage{tikz-cd}
\tikzset{>=latex}
\tikzcdset{
  arrow style=tikz,
  arrows={/tikz/line width=.5pt},
  diagrams={>={Straight Barb[scale=0.8]}},
  nodes={inner xsep=3pt,inner ysep=3pt}
}
